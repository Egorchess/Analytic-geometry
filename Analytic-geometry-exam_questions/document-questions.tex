\documentclass[a4paper, 12pt]{article}
%\usepackage{mathtext}
\usepackage{cmap}
\usepackage[english, russian]{babel}
\usepackage[T2A]{fontenc}
\usepackage[utf8]{inputenc}
\usepackage[left=2cm, right=1.5cm, top=2cm, bottom=2cm]{geometry}
\usepackage{amsmath}
\usepackage{amssymb}
\usepackage{etoolbox}
\usepackage{amsthm}
\usepackage{amsfonts}
\usepackage{mathtools}
%\usepackage{indentfirst}
\usepackage{soul}
\usepackage{graphicx}
\usepackage{enumerate}


\usepackage{tikz,amstext}
\newlength{\tempheight}
\newcommand{\Let}[0]{%
	\mathbin{\text{\settoheight{\tempheight}{\mathstrut}\raisebox{0.5\pgflinewidth}{%
				\tikz[baseline,line cap=round,line join=round] \draw (0,0) --++ (0.4em,0) --++ (0,1.5ex) --++ (-0.4em,0);%
}}}}

\renewcommand{\phi}{\varphi}
\renewcommand{\epsilon}{\varepsilon}
\newcommand*\circled[1]{\tikz[baseline=(char.base)]{
            \node[shape=circle,draw,inner sep=2pt] (char) {#1};}}
\newcommand{\aug}{\fboxsep=-\fboxrule\!\!\!\fbox{\strut}\!\!\!}
\newcommand\tab[1][.5cm]{\hspace*{#1}}
\newcommand\undermat[2]{\makebox[0pt][l]{$\smash{\underbrace
			{\phantom{\begin{matrix}#2\end{matrix}}}_{\text{$#1$}}}$}#2}
\newcommand\overmat[2]{\makebox[0pt][l]{$\smash{\overbrace
			{\phantom{\begin{matrix}#2\end{matrix}}}^{\text{$#1$}}}$}#2}

\newcounter{lemcount}
\newcounter{sectcount}
\newcounter{thcount}
\theoremstyle{definition}
\newtheorem*{definition}{Определение}
\newtheorem*{theorem}{Теорема}
\newtheorem*{consequense}{Следствие}
\newtheorem*{lemma}{Лемма}
\newtheorem*{subtheorem}{Утверждение}
\newtheorem*{formula}{Вывод формулы}
\newtheorem*{formulas}{Вывод формул}
\newtheorem*{remark}{Замечание}
\newtheorem*{examples}{Примеры}
\newtheorem*{example}{Пример}
\newtheorem*{lalala}{Упражнение}
\newtheorem*{algorithm}{Алгоритм}
\newtheorem*{properties}{Свойства}
\newtheorem*{properties1}{Свойство}
\newtheorem{lemmanum}[lemcount]{Лемма}
\newtheorem{theoremnum}[thcount]{Теорема}
% \newtheorem{theoremL}{Теорема}[section]
\usepackage[T2A]{fontenc}
\usepackage[utf8]{inputenc}
\usepackage[russian]{babel}
\addto\captionsenglish{% Replace "english" with the language you use
	\renewcommand{\contentsname}%
	{Содержание}%
}

\usepackage{titlesec}
\titleformat{\section}{\LARGE \bfseries}{Билет \thesection.}{1em}{}
\titleformat{\subsection}{\Large\bfseries}{\thesubsection}{1em}{}
\titleformat{\subsubsection}{\large\bfseries}{\thesubsubsection}{1em}{}

\usepackage{hyperref}
\usepackage{xcolor}
% Цвета для гиперссылок
\definecolor{linkcolor}{HTML}{225ae2} % цвет ссылок
\definecolor{urlcolor}{HTML}{225ae2} % цвет гиперссылок
\hypersetup{
	pdfstartview=FitH, 
	linkcolor=linkcolor,
	urlcolor=urlcolor,
	colorlinks=true
}

\begin{document}
	\begin{titlepage}
		\newpage
		
		\begin{center}
		\end{center}
		
		\vspace{4em}
		
		\begin{center}
			\Large Механико-математический факультет  
		\end{center}
		
		\vspace{2em}
		
		\begin{center}
			\large{\textsc{\textbf{Аналитическая геометрия, 1 семестр, 2 поток}}}

			\vspace{6em}

			\large{\textsc{\textbf{Билеты}}}
		\end{center}
		
		\vspace{\fill}
		
		\begin{center}
			Москва \\2024 
		\end{center}
	\end{titlepage}
	\tableofcontents
	\fontsize{14pt}{20pt}\selectfont
	\newpage
	\fontsize{14pt}{20pt}\selectfont
	\section{Векторные пространства и множества}
	\subsection{Векторные пространства}
	Геометрические векторы в математике являются \bfseries свободными векторами \mdseries - классами эквивалентности направленных отрезков по уже известному нам отношению эквивалентности векторов.
	\begin{definition}
		Векторным (линейным) пространством (над полем $\mathbb{R}$) называется множество V с введенными на нем бинарными операциями "+": $V \times V \rightarrow V$ и "$*$": $\mathbb{R} \times V \rightarrow V$ , отвечающие следующим свойствам (аксиомам):
		
		$\forall \bar{a}, \bar{b}, \bar{c} \in V; \lambda, \mu \in \mathbb{R}:$
		\begin{enumerate}
			\item $\bar{a} + \bar{b} = \bar{b} + \bar{a}$ (коммутативность сложения);
			\item $(\bar{a} + \bar{b}) + \bar{c} = \bar{a} + (\bar{b} + \bar{c})$ (ассоциативность сложения);
			\item $\exists \bar{0} \in V: \bar{a} + \bar{0} = \bar{0} + \bar{a} = \bar{a}$ (существует нейтральный элемент по сложению - нулевой вектор);
			\item $\exists (-\bar{a}) \in V: \bar{a} + (-\bar{a}) = (-\bar{a}) + \bar{a} = \bar{0}$ (существует противоположный элемент по сложению);
			\item $\lambda(\mu\bar{a}) = (\lambda\mu)\bar{a}$ (ассоциативность умножения на числа);
			\item $(\lambda + \mu)\bar{a} = \lambda\bar{a}+ \mu\bar{a}$ (дистрибутивность по умножению);
			\item $\lambda(\bar{a} + \bar{b}) = \lambda\bar{a}+ \lambda\bar{b}$ (дистрибутивность по сложению);
			\item $1*\bar{a} = \bar{a}$.
		\end{enumerate}
	\end{definition}
	\begin{examples}
		Векторные пр-ва:
		\begin{itemize}
			\item $\mathbb{R}, \mathbb{R}^2, \mathbb{R}^n$:
			\item Функции;
			\item Многочлены;
			\item Многочлены степени $\leqslant n$.
		\end{itemize}
	\end{examples}
	\begin{remark}
		Св-ва векторных пространств:
		
		\begin{enumerate}
			
			\item $\bar{0}$ единственный.
			
			Пусть $\bar{0}_{1}, \bar{0}_{2}$ - нулевые векторы.
			
			Тогда $\bar{0}_{1} = \bar{0}_{1} + \bar{0}_{2} = \bar{0}_{2}$, ч.т.д.
			\item $-\bar{a}$ единственный.
			
			Пусть $-\bar{a}_{1}, -\bar{a}_{2}$ - противоположные к $\bar{a}$ векторы.
			
			Тогда $-\bar{a}_{1} = -\bar{a}_{1} + \bar{0} = -\bar{a}_{1} + (\bar{a} + -\bar{a}_{2}) = (-\bar{a}_{1} + \bar{a}) + (-\bar{a}_{2}) = \bar{0} + (-\bar{a}_{2}) = -\bar{a}_{2}$, ч.т.д.
			\item $\lambda * \bar{0} = \bar{0}$.
			
			$\lambda * \bar{0} = \lambda * (\bar{0} + \bar{0}) = \lambda * \bar{0} + \lambda * \bar{0}$
			
			Прибавив к обеим частям вектор, противоположный к $\lambda * \bar{0}$, получим $\lambda * \bar{0} = \bar{0}$, ч.т.д.
			\item $-(\lambda\bar{a}) = (-\lambda)\bar{a} = \lambda(-\bar{a})$.
			
			Нетрудно видеть, что все три вектора противоположны $\lambda\bar{a}$, а далее из п.2.
			\item $-\bar{a} = -1*\bar{a}$
			
			Следует из п.4.
			\item $\lambda\bar{a} = \bar{0} \Leftrightarrow$ либо $\lambda = 0$, либо $\bar{a} = \bar{0}$.
			
			Либо $\lambda = 0$, либо $\lambda \neq 0 \Rightarrow \frac{1}{\lambda}*\lambda*\bar{a} = \frac{1}{\lambda}\bar{0} = \bar{0} \Rightarrow \bar{a} = \bar{0}$, ч.т.д. 
		\end{enumerate}
	\end{remark}
	\subsection{Линейная комбинация векторов}
	\begin{definition}
		Сумма вида $\lambda_{1}\bar{x}_{1} + ... + \lambda_{n}\bar{x}_{n}$ называется линейной комбинацией векторов $\bar{x}_{1} ... \bar{x}_{n}$.
	\end{definition}
	\begin{definition}
		Если в линейной комбинации $\lambda_{1} = ... = \lambda_{n} = 0$, то она называется тривиальной, а иначе - нетривиальной.
	\end{definition}
	\begin{definition}
		Если вектор $\bar{x}$ равен линейной комбинации $\lambda_{1}\bar{x}_{1} + ... + \lambda_{n}\bar{x}_{n}$, то говорят, что он линейно выражается (раскладывается) через векторы $\bar{x}_{1}...\bar{x}_{n}$.
		(Сама линейная комбинация $\lambda_{1}\bar{x}_{1} + ... + \lambda_{n}\bar{x}_{n}$ называется выражением (разложением) вектора $\bar{x}$ через $\bar{x}_{1}...\bar{x}_{n}$)  
	\end{definition}
	\subsection{Линейно зависимые и линейно независимые множества и системы векторов}
	\begin{definition}
		Упорядоченное множество векторов называется системой векторов.
		(В системе векторов элементы могут повторяться)
	\end{definition}
	\begin{definition}
		Множество векторов называется линейно зависимым, если существует равная нулю нетривиальная линейная комбинация векторов из этого множества. В противном случае оно называется линейно независимым.
	\end{definition}
	\begin{example}
		Система из двух векторов $\bar{a}, \bar{b}$ линейно зависима $\Leftrightarrow \bar{a} = \lambda\bar{b}$.
	\end{example}
	\begin{remark}
		Множество векторов линейно зависимо $\Leftrightarrow$ один из векторов этого множества линейно выражается через некоторые другие векторы этого множества.
	\end{remark}
	\subsection{Полные множества и системы векторов}
	\begin{definition}
		Множество (система) векторов из векторного пространства V называется полным(полной) в $V$, если любой вектор $\bar{x}\in V$ линейно выражается через векторы этого множества.
	\end{definition}
	\begin{remark}
		$X \subset V$ полно в $V \Rightarrow \forall Y: X\subset Y$ полно в $V$.
	\end{remark}
	\begin{remark}
		$X \subset V$ линейно независимо в $V \Rightarrow \forall Y \subset X$ линейно независимо в $V$.
	\end{remark}
	\section{Базис и размерность векторного пространства}
	\subsection{Базис векторного пространства. Конечномерное векторное пространство.}
	\begin{definition}
		Множество векторов $E$ в векторном пространстве $V$ называется базисом $V$, если $E$ линейно независимо и полно в $V$.
	\end{definition}
	\begin{definition}
		Векторное пространство, в котором существует конечный (состоящий из конечного числа векторов) базис, называется конечномерным. В противном случае оно называется бесконечномерным. 
	\end{definition}
	\subsection{Лемма о количестве векторов в ЛНЗ системе (аналог ОЛЛЗ)}
	\begin{lemma}
		Если $X$ - конечное полное множество из $n$ векторов в векторном пространстве $V$ и $Y$ - линейно независимое множество векторов в $V$, то $Y$ конечно и число векторов в $Y \leqslant n$. 
	\end{lemma}
	\begin{proof}
		(пер.)
		Произвольно занумеруем векторы в $X: (\bar{x}_{1},...,\bar{x}_{n})$.
		Будем по одному добавлять в эту систему векторы из $Y$ и одновременно выкидывать векторы из $X$ так, чтобы система оставалась полной.
		
		Пусть за $k$ шагов ($0\leqslant k \leqslant n$) мы добавили некоторые $\bar{y}_{1},...,\bar{y}_{k}$ и выкинули какие-то $k$ векторов из $X$ - осталась система ($\bar{y_{1}},...,\bar{y_{k}},\bar{x}_{i_{1}},...,\bar{x}_{i_{n-k}}$).
		Возьмём $\bar{y}_{k+1}$ из $Y$ (если такого нет, то в $Y \leqslant n$ векторов, что нам и нужно), и добавим его в систему. Так как до этого система оставалась полной, $\bar{y}_{k+1}$ выражается через ($\bar{y_{1}},...,\bar{y_{k}},\bar{x}_{i_{1}},...,\bar{x}_{i_{n-k}}$), причём какой-то $\bar{x}_{i_{j}}$ входит в это разложение с коэффициентом, не равным нулю (иначе противоречие с линейной независимостью $Y$ - $\bar{y}_{k+1}$ выразился через $\bar{y}_{1},...,\bar{y}_{k}$).
		
		Тогда $\bar{x}_{i_{j}}$ выражается через другие векторы системы и $\bar{y}_{k+1}$ (в выражении $\bar{y}_{k+1}$ перенесём всё, кроме $\bar{x}_{i_{j}}$ в другую часть и разделим на коэффициент перед ним).
		А так как ($\bar{y_{1}},...,\bar{y_{k+1}},\bar{x}_{i_{1}},...,\bar{x}_{i_{n-k}}$) - полная, эта же система без $\bar{x}_{i_{j}}$. очевидно, останется полной.

		Пусть смогли проделать $n$ таких шагов. Тогда имеем систему ($\bar{y}_{1},...,\bar{y}_n$). Если в $Y$ есть ещё векторы, то они с одной стороны выражаются через векторы системы из её полноты, а с другой - не выражаются через них из линейной независимости $Y$. Противоречие, т.е. в $Y$ не может оказаться больше $n$ векторов, ч.т.д.   
	\end{proof}
	\subsection{Теорема о количестве векторов в базисе. Размерность векторного пространства.}
	\begin{theorem}
		Если в векторном пространстве есть конечный базис. то все базисы в нём конечны и содержат одинаковое количество векторов.
	\end{theorem}
	\begin{proof}
		Пусть $\bar{e}_{1},...,\bar{e}_{n}$ - конечный базис в $V$. Любой другой базис $V$ линейно независим, т.е. по лемме содержит $k \leqslant n$ векторов, а с другой стороны полон, т.е. первый базис по лемме содержит $n \leqslant k$ векторов. Отсюда $n=k$, ч.т.д.
	\end{proof}
	\begin{definition}
		Количество векторов в любом базисе векторного пространства $V$ называется размерностью $V$ и обозначается $dim V$.
	\end{definition}
	\begin{examples}
		$dim \ {\bar{0}} = 0, dim \ \pi (= dim \ \mathbb{R}^2) = 2, dim \ \mathbb{R}^3 = 3$.
	\end{examples}
	\section{Координаты в базисе}
	\subsection{Однозначность выражения вектора в конечномерном в. п. через базис}
	\begin{theorem}
		В конечномерном векторном пространстве выражение любого вектора через базис определяется однозначно.
	\end{theorem}
	\subsection{Координаты вектора в базисе}
	\begin{proof}
		Если $\bar{x} = \lambda_{1}\bar{e}_{1} + ... + \lambda_{n}\bar{e}_{n} = \lambda'_{1}\bar{e}_{1} + ... + \lambda'_{n}\bar{e}_{n}$, то $\bar{x} - \bar{x} = \bar{0} = (\lambda_{1} - \lambda'_{1})\bar{e}_{1} + ... + (\lambda_{n} - \lambda'_n)\bar{e}_{n}$. Если эти два разложения различны, то равная нулю линейная комбинация базисных векторов нетривиальна, что противоречит линейной независимости базиса. То есть двух различных разложений быть не может, ч.т.д. 
	\end{proof}
	\begin{definition}
		Пусть $V$ - конечномерное векторное пространство и $\bar{e}_{1},...,\bar{e}_{n}$ - базис в нём. Коэффициенты $\lambda_{1},...,\lambda_{n}$ в выражении любого вектора $x \in V$ через эти базисные векторы называются координатами вектора $x$ в базисе $\bar{e}_{1},...,\bar{e}_{n}$. ($\lambda_{k}$  называется $k$-й координатой)
	\end{definition}
	\begin{remark}
		Векторы в $n$-мерном векторном пространстве находятся во взаимно однозначном соответствии с упорядоченной строкой из $n$ чисел из $\mathbb{R}$ (например, векторы ассоциированного с евклидовой плоскостью векторного пространства соответствуют парам чисел)
		Таким образом можно задать операции сложения и умножения на число векторов плоскости через операции над числами, проводимыми покоординатно.
	\end{remark}
	\section{Аффинные пространства}
	\subsection{Аффинное пространство}
	Элементы плоскости (как множества) - точки, а не векторы, поэтому для работы непосредственно с плоскостью необходимо ввести данное определение.
	\begin{definition}
		Аффинное пространство - тройка ($X, V, +$) (обычно обозначается $\mathbb{A}$), где $X$ - множество (точек), $V$ - векторное пространство, а $+$ операция:  $X \times V \rightarrow X$, для которых выполнены аксиомы:
		\begin{enumerate}
			\item $\forall A \in X, \forall \bar{a}, \bar{b} \in V: A+(\bar{a}+\bar{b}) = (A+\bar{a})+\bar{b}$;
			\item $\forall A \in X: A + \bar{0} = A$;
			\item $\forall A, B \in X \ \ \exists! \ \bar{a} \in V: A + \bar{a} = B$. Обозначается $\bar{a} = \overrightarrow{AB}$.
		\end{enumerate}
	\end{definition}
	\subsection{Радиус-векторы и репер}
	Если зафиксировать какую-нибудь точку $O \in X$, возникает взаимно однозначное соответствие между точками $A$ и их радиус-векторами $\overrightarrow{OA}$.
	\begin{definition}
		Репер (система координат) в аффинном пространстве $(X, V, +)$ - пара $(O, E)$, где $O \in X$ и $E$ - базис в $V$. Точка $O$ называется началом координат (отсчёта). Координаты точки A в $(O, E)$ - координаты её радиус-вектора $\overrightarrow{OA}$ в базисе $E$.
	\end{definition}
	\begin{remark}
		Для аффинного пространства верно:
		\begin{enumerate}
			\item Если $A = (x_{1},...,x_{n}),  \bar{a} = (y_{1},...,y_{n})$, то $A + \bar{a} = (x_{1} + y_{1},...,x_{n} + y_{n})$.
			\item Если $A = (a_{1},...,a_{n}),  B = (b_{1},...,b_{n})$, то $\overrightarrow{AB} = (b_{1} - a_{1},...,b_{n} - a_{n})$.
		\end{enumerate}
		(Следует из сложения векторов)
	\end{remark}
	\subsection{Конечномерное аффинные пространства и их размерность}
	\begin{definition}
		Если $\mathbb{A} = (X, V, +)$ - аффинное пространство, то говорят, что $V$ - векторное пространство, ассоциированное с $\mathbb{A}$.    
	\end{definition}
	\begin{definition}
		$\mathbb{A}$ называется конечномерным, если ассоциированное с ним $V$ конечномерно. В этом случае $dim \mathbb{A}$ (размерность $\mathbb{A}$) равна $dim V$.
	\end{definition}
	Теперь точки аффинного пространства аналогично векторам можно ассоциировать с наборами чисел. Однако для ассоциирования евклидовой плоскости и её аксиом с двумерным аффинным пространством, необходимы отвечающие аксиомам понятия прямой и расстояния.
	\section{Подпространства}
	\subsection{Векторное подпространство}
	\begin{definition}
		Векторным подпространством векторного пространства $V$ называется непустое множество $V_{1} \subset V$ такое. что $\forall\bar{x}, \bar{y}\in V_{1}: \bar{x} + \bar{y} \in V_{1}, \lambda\bar{x} \in V_{1} (\forall\lambda \in\mathbb{R})$.
	\end{definition}
	\begin{remark}
		Определение эквивалентно следующему: множество $V_{1} \subset V$ - векторное подпространство $V$, если $V_{1}$ является векторным пространством относительно операций $+$ и $*$, определённых для $V$.
		(Доказательство осуществляется путём прямой проверки аксиом векторного пространства для $V_{1}$)
	\end{remark}
	\subsection{Аффинное подпространство}
	Введём несколько определений аффинного подпространства и докажем их эквивалентность.
	\begin{definition}
		Аффинным подпространством аффинного пространства $\mathbb{A} = (X, V, +)$ называется
		\begin{enumerate}
			\item его непустое подмножество вида $A + V_{1} = {A + \bar{a}:\bar{a} \in V_{1}}$, где $V_{1}$ - векторное подпространство $V$ и $A \in X$ - точка;
			\item тройка $(X_{1}\subset X, V_{1} \subset V, +_{1})$, где $V_{1}$ - векторное подпространство $V$ и операция $+_{1} = +$,  для которой $\forall A,B \in X_{1}, \forall \bar{a} \in V_{1}: A + \bar{a} \in X_{1}, \overrightarrow{AB} \in V_{1}$;
			\item тройка $(X_{1}\subset X, V_{1} \subset V, +_{1})$, где $V_{1}$ - векторное подпространство $V$ и операция $+_{1} = +$,  которая сама является аффинным пространством.
		\end{enumerate}
	\end{definition}
	\begin{subtheorem}
		Приведённые определения эквивалентны.
	\end{subtheorem}
	\begin{proof}
		Докажем следующие следствия:

		$\circled{1}\Rightarrow\circled{2}$  Пусть $P = A + \bar{a}, Q = A + \bar{b}$. Тогда $\overrightarrow{PQ} = \bar{b} - \bar{a}$ (в силу единственности такого вектора), т.е. $\overrightarrow{PQ} \in V_{1}$. Второе необходимое свойство $\circled{2}$ очевидно выполнено.

		$\circled{2}\Rightarrow\circled{1}$  Пусть $X_{1}, V_{1}$ удовлетворяют $\circled{2}$. Зафиксируем произвольную $A \in X_{1}$. $\forall B\in X_{1}$ имеем $B = A + \overrightarrow{AB}$, причём $A \in X_{1}, \overrightarrow{AB} \in V_{1} \Rightarrow B \in X_{1}$.
		
		Эквивалентность $\circled{2}\Leftrightarrow\circled{3}$ очевидна из определения аффинного пространства.
	\end{proof}
	\subsection{Прямая в аффинном пространстве}
	\begin{definition}
		Прямая в аффинном пространстве - его одномерное аффинное подпространство. \\Плоскость (двумерная) в аффинном пространстве - его двумерное аффинное подпространство.
	\end{definition}
	\begin{definition}
		Единственный вектор в любом базисе векторного пространства, ассоциированного с одномерным аффинным пространством, называется направляющим вектором этого аффинного пространства.
	\end{definition}
	\section{Скалярное произведение}
	\subsection{Скалярное произведение}
	\begin{definition}
		Пусть $V$ - векторное пространство. Скалярным произведением в $V$ называется функция $(\ ,\ ) : V \times V \rightarrow \mathbb{R}$ со свойствами:
		\begin{enumerate}
			\item $(\bar{x}, \bar{x}) \geqslant 0 \ \forall \bar{x} \in V$, причём $(\bar{x}, \bar{x}) = 0 \Leftrightarrow \bar{x} = \bar{0}$ (положительная определённость);
			\item $(\bar{x}, \bar{y}) = (\bar{y}, \bar{x}) \ \forall \bar{x}, \bar{y} \in V$ (коммутативность);
			\item $(\alpha\bar{x} + \beta\bar{y}, \bar{z}) = \alpha(\bar{x}, \bar{z}) + \beta(\bar{y}, \bar{z}) \ \forall \bar{x}, \bar{y}, \bar{z} \in V, \alpha,\beta \in \mathbb{R}$ (линейность по первому аргументу)
		\end{enumerate} 
	\end{definition}
	Из коммутативности выполнена и линейность по второму аргументу, т.е. скалярное произведение - билинейная функция.
	\subsection{Евклидово векторное и точечно-евклидово аффинное пространство}
	\begin{definition}
		Конечномерное аффинное (векторное) пространство вместе со скалярным произведением называется точечно-евклидовым (евклидовым) пространством. Двумерное точечно-евклидово пространство называется евклидовой плоскостью.
	\end{definition}
	\subsection{Длина вектора и расстояния между точками}
	\begin{definition}
		Длиной вектора называется величина $\sqrt{(\bar{x}, \bar{x})}$.
	\end{definition}
	\begin{definition}
		Расстоянием (евклидовым) между точками $A,B \in \mathbb{A}$ называется длина вектора $\overrightarrow{AB}$. Будем обозначать $d(A, B)$ как $|\overrightarrow{AB}|$. 
	\end{definition}
	\subsection{Выражение скалярного произведения через длины}
	\begin{remark}
		Зная длины всех векторов, скалярное произведение можно восстановить по формуле $(\bar{x}, \bar{y}) = \frac{1}{2}(|\bar{x} + \bar{y}|^2 - |\bar{x}|^2 - |\bar{y}|^2)$. Это несложно проверить: \\
		$\frac{1}{2}(|\bar{x} + \bar{y}|^2 - |\bar{x}|^2 - |\bar{y}|^2) = \frac{1}{2}((\bar{x} + \bar{y}, \bar{x} + \bar{y}) - (\bar{x}, \bar{x}) - (\bar{y}, \bar{y})) = \frac{1}{2}(2(\bar{x}, \bar{y})) = (\bar{x}, \bar{y})$.
	\end{remark}
	\section{Неравенство Коши-Буняковского}
	\subsection{Неравенство Коши-Буняковского}
	\begin{theorem}[Неравенство Коши-Буняковского]
		$\forall \bar{a}, \bar{b} \in V \ \ (\bar{a}, \bar{b}) \leqslant \sqrt{(\bar{a}, \bar{a})(\bar{b}, \bar{b})}$, причём равенство достигается только при $\bar{a} = \lambda\bar{b}$.
	\end{theorem}
	\begin{proof}
		Рассмотрим выражение $(\bar{a} + t\bar{b}, \bar{a} + t\bar{b})$. Оно равно нулю $\Leftrightarrow \bar{a} = -t\bar{b}$, т.е. может быть равно нулю не более чем при одном $t$. С другой стороны
		$(\bar{a} + t\bar{b}, \bar{a} + t\bar{b})$ = $(\bar{a}, \bar{a}) + 2(\bar{a}, \bar{b})t + (\bar{b}, \bar{b})t^2$ - квадратный трёхчлен относительно $t$. Его дискриминант равен $4(\bar{a}, \bar{b})^2 - 4(\bar{a}, \bar{a})(\bar{b}, \bar{b})$, а из первого рассуждения знаем, что дискриминант $\leqslant 0$, причём равенство достигается только в случае коллинеарности $\bar{a}$ и $\bar{b}$. Отсюда $(\bar{a}, \bar{b}) \leqslant \sqrt{(\bar{a}, \bar{a})(\bar{b}, \bar{b})}$, ч.т.д.
	\end{proof}
	\subsection{Величина угла и ортогональные векторы}
	\begin{definition}
		Величиной угла между ненулевыми векторами $\bar{a}, \bar{b}$  называется число $arccos\frac{(\bar{a}, \bar{b})}{|\bar{a}||\bar{b}|}$ (из н. Коши-Буняковского $|\frac{(\bar{a}, \bar{b})}{|\bar{a}||\bar{b}|}| \leqslant 1$).
	\end{definition}
	\begin{definition}
		Векторы $\bar{a}, \bar{b}$ называются ортогональными (перпендикулярными), если $(\bar{a}, \bar{b}) = 0$.
	\end{definition}
	\section{Прямоугольная система координат}
	\subsection{Ортонормированный базис и прямоугольная система координат}
	\begin{definition}
		Базис векторного пространства $V$ со скалярным произведением называется ортонормированным, если все его векторы попарно ортогональны и имеют длину 1.
	\end{definition}
	\begin{definition}
		Система координат в точечно-евклидовом пространстве называется прямоугольной, если её базис ортонормированный.
	\end{definition}
	\subsection{Выражение скалярного произведения через координаты векторов}
	\begin{subtheorem}
		В точечно-евклидовом пространстве верно следующее выражение скалярного произведения через координаты векторов: если в некотором базисе $ (\bar{e}_{1},...,\bar{e}_{n}) \ \bar{x} = \begin{pmatrix} x_{1} \\ \vdots \\ x_{n} \end{pmatrix}, \bar{y} = \begin{pmatrix} y_{1} \\ \vdots \\ y_{n} \end{pmatrix}$, то $(\bar{x}, \bar{y}) = \sum \limits_{i=1}^n x_{i} \cdot \sum \limits_{j=1}^n y_{j}(\bar{e}_{i}, \bar{e}_{j})$.
	\end{subtheorem}
	\begin{proof}
		$(\bar{x}, \bar{y}) = (x_{1}\bar{e}_{1}+...+x_{n}\bar{e}_{n}, y_{1}\bar{e}_{1}+...+y_{n}\bar{e}_{n}) = \sum \limits_{i=1}^{n}(x_{i}\bar{e}_{i},y_{1}\bar{e}_{1}+...+y_{n}\bar{e}_{n}) = \sum \limits_{i=1}^{n}x_{i}(\bar{e}_{i},y_{1}\bar{e}_{1}+...+y_{n}\bar{e}_{n}) = \sum \limits_{i=1}^n x_{i} \cdot \sum \limits_{j=1}^n y_{j}(\bar{e}_{i}, \bar{e}_{j})$
	\end{proof}
	\subsection{Выражение для прямоугольной системы координат}
	\begin{remark}
		В случае, когда базис ортонормированный, имеем $(e_{i}, e_{j}) = \delta_{ij}$, т.е. $(\bar{x}, \bar{y}) = x_{1}y_{1}+...+x_{n}y_{n}$. То есть в прямоугольной системе координат длина вектора вычисляется по формуле $|\bar{x}| = \sqrt{x_{1}^2+...+x_{n}^2}$, а расстояние между точками $P = (x_{1},...,x_{n}), Q = (y_{1},...,y_{n})$ выражается как $|PQ| = |\overrightarrow{PQ}| = \sqrt{(y_{1} - x_{1})^2+...+(y_{n}- x_{n})^2}$.
	\end{remark}
	\section{Проектирование}
	\begin{definition}
		Пусть задано два векторных подпространства $V_{1}, V_{2}$ векторного пространства $V$ такие, что $V_{1} \cap V_{2} = \{\bar{0}\}$ и $V_{1} + V_{2} = V$ (обозначается $V = V_{1} \oplus V_{2}$). Тогда сумма $\bar{x} = \bar{x}_{1} + \bar{x}_{2}$, где $\bar{x} \in V, \bar{x}_{1} \in V_{1}, \bar{x}_{2} \in V_{2}$, определена единственно. (Следует, например, из того, что в любом базисе $V$ каждый его вектор лежит либо в $V_{1}$, либо в $V_{2}$, тогда разложение в эту сумму соответствует единственному разложению по базису). Проекцией вектора $\bar{x} \in V$ на $V_{1}$ параллельно $V_{2}$ называется слагаемое $\bar{x_{1}}$ этой суммы. 
	\end{definition}
	\begin{definition}
		Пусть задано два аффинных подпространства $\mathbb{A}_{1} = (X_{1}, V_{1}, +),\\ \mathbb{A}_{2} = (X_{2}, V_{2}, +)$ аффинного пространства $\mathbb{A} = (X, V, +)$ такие, что $V = V_{1} \oplus V_{2}$. Проекцией точки $P \in \mathbb{A}$ на $\mathbb{A}_{1}$ параллельно $\mathbb{A}_{2}$ - точка $P_{1} = A_{1} + \bar{v}$, где $A_{1}$ - произвольная точка из $X_{1}$, а $\bar{v}$ - проекция $\overrightarrow{A_{1}P}$ на $V_{1}$ параллельно $V_{2}$.
		(Очевидно, что от выбора $A_{1}$ расположение проеции не зависит)
	\end{definition}
	\begin{example}
		Рассмотрим координаты точки евклидовой плоскости относительно прямоугольной системы координат. \\
		Найдём проекцию точки $A = (x, y)$ на прямую $Oy$ параллельно прямой $Ox$. По определению это точка (назовём её $A_{y}$), равная $O + \bar{v}$, где $\bar{v}$ - проекция $\overrightarrow{OP}$ на векторное пространство прямой $Oy$ параллельно $Ox$. $\overrightarrow{OP} = \{x, y\} = x\bar{e}_{1} + y\bar{e}_{2}$. Отсюда $\bar{v} = y\bar{e}_{2} = \{0, y\}$, то есть $A_{y} = (0, y)$. Аналогично $A_{x} = (x, 0)$. 
	\end{example}
	\section{Ортонормированный базис}
	Определение смотри в пункте 8.1
	Пусть $\mathbb{A}^n = (X, V^n, +)$ - $n$-мерное точечно-евклидово пространство.
	\subsection{Линейная независимость ортогональных векторов}
	\begin{subtheorem}
		В $V^n$ любая линейно независимая система из $n$ векторов образует базис.
	\end{subtheorem}
	\begin{proof}
		Предположим, что в $V^n$ существует неполная линейно независимая система из n векторов. Т.к. система не полная, существует вектор из $V^n$, не выражающийся через векторы этой системы, т.е. этот вектор можно добавить в систему без потери линейной независимости. Но по лемме-аналогу ОЛЛЗ линейно независимая система в $V^n$ не может иметь $> n$ векторов. Противоречие, т.е. любая линейно независимая система из $n$ векторов является полной, а значит и базисом, ч.т.д. 
	\end{proof}
	\begin{subtheorem}
		Если $\bar{e}_{1},...,\bar{e}_{n}$ - попарно ортогональные ненулевые векторы в евклидовом пространстве, то $\bar{e}_{1},...,\bar{e}_{n}$ линейно независимы.
	\end{subtheorem}
	\begin{proof}
		Предположим противное. Пусть один из векторов (без ограничения общности $\bar{e}_{n}$) линейно выражается через остальные: $\bar{e}_{n} = \lambda_{1}\bar{e}_{1} +...+ \lambda_{n-1}\bar{e}_{n-1}$. Тогда запишем квадрат его длины:
		$|\bar{e}_{n}|^2 = (\bar{e}_{n}, \bar{e}_{n}) = (\bar{e}_{n}, \lambda_{1}\bar{e}_{1} +...+ \lambda_{n-1}\bar{e}_{n-1}) = \sum \limits_{i=1}^{n-1} \lambda_{i}(\bar{e}_{n}, \bar{e}_{i}) = 0$ (т.к. $\bar{e}_{n}$ ортогонален всем остальным векторам). Отсюда $|\bar{e}_{n}| = 0$, и притом $\bar{e}_{n}$ ненулевой. Противоречие, т.е. никакой вектор системы не выражается через остальные, а значит система линейно независима. ч.т.д.
	\end{proof}
	\subsection{Теорема о существовании ортонормированного базиса}
	\begin{theorem}
		В любом евклидовом пространстве существует ортонормированный базис.
	\end{theorem}
	\begin{proof}
		(пер.)
		Индукция по $n$ - размерности пространства:

		База: $n = 1$ - очевидно, что существует вектор длины 1, который составляет ортонормированный базис одномерного пространства;

		Шаг: Пусть в любом $n$-мерном пространстве существует ортонормированный базис. Рассмотрим пространство $V$ размерности $n+1$ и выберем базис какого-то $n$-мерного подпространства $W$ (пусть $(\bar{e}_{1},...,\bar{e}_{n})$). Найдём вектор, ортогональный всем выбранным векторам. Так как базис $W$ не полон в $V$, к нему можно добавить ещё один вектор $x \in V$ без потери линейной независимости $\Rightarrow$ $(\bar{e}_{1},...,\bar{e}_{n}, \bar{x})$ - базис в $V$ (ЛНЗ система из $n+1$ векторов). \\
		Теперь необходимо представить $\bar{x}$ как следующую сумму: $\bar{x} = \lambda_{1}\bar{e}_{1} + ... + \lambda_{n}\bar{e}_{n} + \bar{e}_{n+1}$, где $\bar{e}_{n+1}$ ортогонален $\bar{e}_{1},...,\bar{e}_{n}$. Тогда $\bar{e}_{n+1} = \bar{x} - \lambda_{1}\bar{e}_{1} - ... - \lambda_{n}\bar{e}_{n}$. Рассмотрим $(\bar{e}_{n+1}, \bar{e}_{k}) = (\bar{x} - \lambda_{1}\bar{e}_{1} - ... - \lambda_{n}\bar{e}_{n}, \bar{e}_{k}) = (\bar{x}, \bar{e}_{k}) - \lambda_{1}(\bar{e}_{1}, \bar{e}_{k}) - ... - \lambda_{n}(\bar{e}_{n}, \bar{e}_{k})$. Так как $\bar{e}_{k}$ ортогонально всем этим векторам, кроме $\bar{e}_{k}$ и $\bar{x}$, это выражение равно $(\bar{x}, \bar{e}_{k}) - \lambda_{k}(\bar{e}_{k}, \bar{e}_{k})$. Отсюда при $\lambda_{k} = \frac{(\bar{x}, \bar{e}_{k})}{(\bar{e}_{k}, \bar{e}_{k})}$ векторы $\bar{e}_{n+1}$ и $\bar{e}_{k}$ ортогональны (зависит только от $\lambda_{k}$). Составив таким образом все $\lambda_{1},...,\lambda_{n}$, получим выражение вектора $\bar{e}_{n+1}$, ортогонального всем векторам базиса $W$.
		Таким образом, векторы полученной системы $\bar{e}_{1},...,\bar{e}_{n+1}$ попарно ортогональны (по предположению индукции) $\Rightarrow$ линейно независимы $\Rightarrow$ образуют базис в $V$. Разделив $\bar{e}_{n+1}$ на его длину, получим, что все векторы базиса попарно ортогональны и имеют длину 1 $\Rightarrow V$ имеет ортонормированный базис, ч.т.д.
	\end{proof}
	\begin{consequense}
		Любую систему ортогональных векторов длины 1 в векторном пространстве можно дополнить до ортонормированного базиса.
	\end{consequense}
	\section{Прямые и их уравнения}
	\subsection{Определения прямой и направляющего вектора}
	Смотри пункт 5.3
	\subsection{Уравнения прямой}
	\begin{formulas}[\bfseries уравнения прямой\mdseries]


		Пусть $l$ - прямая на плоскости: $l = \{X: \overrightarrow{OX} = \overrightarrow{OM} + t\bar{v}\}$, где $M$ - точка прямой, $\bar{v}$ - её направляющий вектор. Если $M=\begin{pmatrix}x_{0}\\y_{0}\end{pmatrix},\bar{v}=\begin{pmatrix}a\\b\end{pmatrix}$, то из совпадения координат совпадающих векторов $\overrightarrow{OX}$ и $(\overrightarrow{OM}+t\bar{v})$ для $X \in l$ верно: $\begin{cases}x = x_{0} + at \\ y = y_{0} + bt\end{cases}$ (\bfseries параметрические уравнения\mdseries)\\
		Выразим $t$ из первого уравнения и подставим во второе - получим: $\frac{x-x_{0}}{a} = \frac{y-y_{0}}{b}$ (\bfseries каноническое уравнение прямой\mdseries).
		(Заметим, что данное выражение не определено при нулевых $a$ или $b$, но очевидно, что они не равны нулю одновременно, а запись, где одна из дробей имеет знаменатель 0, иногда используется, поэтому здесь и далее случай равенства нулю знаменателя может не рассматриваться как отдельный и будет означать, что числитель должен равняться 0)\\
		Если известно, что прямой принадлежат $M = \begin{pmatrix}x_{0}\\y_{0}\end{pmatrix}, N = \begin{pmatrix}x_{1}\\y_{1}\end{pmatrix}$, то $\overrightarrow{MN} = \begin{pmatrix}x_{1} - x_{0}\\y_{1} - y_{0}\end{pmatrix}$ - направляющий вектор, т.е. каноническое уравнение прямых принимает вид $\frac{x-x_{0}}{x_{1}-x_{0}} = \frac{y-y_{0}}{y_{1}-y_{0}}$ (\bfseries уравнение прямой по двум точкам\mdseries).\\
		Домножим каноническое уравнение прямой на знаменатели: $\frac{x-x_{0}}{a} = \frac{y-y_{0}}{b} \Rightarrow bx-bx_{0}=ay-ay_{0} \Rightarrow bx - ay + (ay_{0}-bx_{0}) = 0$. Такое уравнение обычно называют \bfseries общим уравнением прямой\mdseries \ и записывают как $Ax + By + C = 0$.
	\end{formulas}
	\begin{remark}
		Для прямых в пространстве подобным образом выводятся параметрические и каноническое уравнения. 
	\end{remark}
	\begin{remark}
		Заметим также, что из итоговой формулы вывода общего уравнения ($bx - ay + (ay_{0}-bx_{0}) = 0 \Rightarrow Ax + By + C = 0$) следует, что для прямой $Ax + By +C = 0$ вектор ($B, -A$) (а соответственно и ($-B, A$)) является направляющим.
	\end{remark}
	\subsection{Критерий уравнения прямой (нет в билете, важно)}
	\begin{subtheorem}
		$Ax + By + C = 0$ является уравнением прямой $\Leftrightarrow A$ и $B$ не равны нулю одновременно.
	\end{subtheorem}
	\begin{proof}
		$\\ \Rightarrow$ Если $Ax + By + C = 0$, то её направляющий вектор ненулевой, а значит вектор $(-B, A)$ ненулевой, то есть одна из его координат $\neq 0$.
		$\\ \Leftarrow$ Пусть без ограничения общности $A \neq 0$. Тогда этому уравнению удовлетворяет точка $(x_{0}, y_{0}) = (-\frac{C}{A}, 0)$, а значит (нетрудно проверить) все удовлетворяющие ему точки имеют вид $(x_{0} + Bt, y_{0} - At)$, что соответствует прямой с такими параметрическими уравнениями.
	\end{proof}
	\section{Взаимное расположение прямых}
	\subsection{Случай общих уравнений}
	\begin{theorem}
		Прямые на плоскости параллельны (или совпадают) $\Leftrightarrow$ их направляющие векторы пропорциональны.
	\end{theorem}
	\begin{proof}
		Пусть $l_{1}: A_{1}x + B_{1}y + C_{1} = 0; l_{2}: A_{2}x + B_{2}y + C_{2} = 0$ - данные прямые.
		Рассмотрим систему уравнений, которой удовлетворяют координаты точек, принадлежащих обоим прямым: $\begin{cases}A_{1}x + B_{1}y = -C_{1}\\A_{2}x + B_{2}y = -C_{2}\end{cases}\\$ Из курса алгебры (форумла Крамера) известно, что система не является определённой $\Leftrightarrow det\begin{pmatrix} A_{1} \ B_{1}\\A_{2} \ B_{2}\end{pmatrix} = 0$. Таким образом, прямые параллельны или совпадают $\Leftrightarrow$ имеют 0 или бесконечно много общих точек $\Leftrightarrow A_{1}B_{2} - A_{2}B_{1} = 0 \Leftrightarrow (A_{1}\ B_{1})$ пропорционален $(A_{2}\ B_{2})$, ч.т.д. 
	\end{proof}
	\begin{remark}
		Из этого также видно, что прямые совпадают $\Leftrightarrow \frac{A_{1}}{A_{2}} = \frac{B_{1}}{B_{2}} = \frac{C_{1}}{C_{2}}$. 
	\end{remark}
	\subsection{Случай параметрических уравнений}
	\begin{consequense}
		Прямые $l_{1}: \begin{cases}x = x_{1} + a_{1}t \\ y = y_{1} + b_{1}t\end{cases}; l_{2}: \begin{cases}x = x_{2} + a_{2}t \\ y = y_{2} + b_{2}t\end{cases}$ пересекаются $\Leftrightarrow \frac{a_{1}}{a_{2}} \neq \frac{b_{1}}{b_{2}}$.
		Условие совпадения прямых также можно записать через параметрические уравнения (вектор $(x_{2} - x_{1} \ y_{2} - y_{1}) = \lambda(a, b)$). \\
		Из этого также следует, что через две различные точки проходит ровно одна прямая (все такие прямые совпадают).
	\end{consequense}
	\section{Пучки прямых}
	\subsection{Определение пучка прямых}
	\begin{definition}
		Собственным пучком прямых называется множество всех прямых, проходящих через данную точку, называемую центром пучка.\\
		Несобственным пучком прямых называется множество всех прямых, параллельных данной прямой.
	\end{definition}
	\subsection{Уравнение собственного пучка прямых}
	\begin{theorem}
		Пусть прямые $l_{1}: A_{1}x + B_{1}y + C_{1} = 0$ и $l_{2}: A_{2}x + B_{2}y + C_{2} = 0$ задают собственный пучок (т.е. содержатся в нём и не совпадают). Тогда прямая $l$ принадлежит пучку $\Leftrightarrow l$ задаётся уравнением $\lambda(A_{1}x + B_{1}y + C_{1}) + \mu(A_{2}x + B_{2}y + C_{2}) = 0 \ (*)$ для некоторых $\lambda, \mu \in \mathbb{R}$.
	\end{theorem}
	\begin{proof}
		$\\\Leftarrow$ Пусть $l$ задаётся уравнением $(*)$. Тогда, подставив в уравнение $l$ центр пучка ($x_{0}, y_{0}$), получим $\lambda(0) + \mu(0) = 0$ (т.к. центр удовлетворяет уравнениям $l_{1}, l_{2}$).\\
		$\Rightarrow$ Пусть $(x_{0}, y_{0}) \in l$. Возьмём произвольную точку $(x_{1}, y_{1}) \in l, (x_{1}, y_{1}) \neq (x_{0}, y_{0})$. Рассмотрим прямую вида $(*)$ с $\lambda = -(A_{2}x_{1} + B_{2}y_{1} + C_{2}), \ \mu = (A_{1}x_{1} + B_{1}y_{1} + C_{1}) : -(A_{2}x_{1} + B_{2}y_{1} + C_{2})(A_{1}x + B_{1}y + C_{1}) + (A_{1}x_{1} + B_{1}y_{1} + C_{1})(A_{2}x + B_{2}y + C_{2}) = 0$. Заметим, что это уравнение действительно задаёт прямую: в противном случае необходимы условия $\lambda A_{1} + \mu A_{2} = \lambda B_{1} + \mu B_{2} = 0$, но тогда $(A_{1}, B_{1})$ и $(A_{2}, B_{2})$ пропорциональны, а исходные прямые непараллельны. Такой прямой, очевидно, принадлежат точки $(x_{0}, y_{0})$ и $(x_{1}, y_{1})$. Так как через две различные точки проходит ровно одна прямая, любая прямая из собственного пучка имеет вид ($*$), ч.т.д.
	\end{proof}
	\subsection{Уравнение несобственного пучка прямых}
	\begin{theorem}
		Пусть прямые $l_{1}: A_{1}x + B_{1}y + C_{1} = 0$ и $l_{2}: A_{2}x + B_{2}y + C_{2} = 0$ задают несобственный пучок (т.е. содержатся в нём и не совпадают). Тогда прямая $l$ принадлежит пучку $\Leftrightarrow l$ задаётся уравнением $\lambda(A_{1}x + B_{1}y + C_{1}) + \mu(A_{2}x + B_{2}y + C_{2}) = 0 \ (*)$ для некоторых $\lambda, \mu \in \mathbb{R}$.
	\end{theorem}
	\begin{proof}
		$\\\Leftarrow$ Так как $l_{1} \parallel l_{2}$, $\frac{A_{1}}{B_{1}} = \frac{A_{2}}{B_{2}}$. Тогда если $l$ имеет вид ($*$), то $\frac{\lambda A_{1}+\mu A_{2}}{A_{1}} = \lambda + \frac{\mu A_{2}}{A_{1}} = \lambda + \frac{\mu B_{2}}{B_{1}} = \frac{\lambda B_{1}\mu B_{2}}{B_{1}} \Rightarrow l \parallel l_{1}$.\\
		$\Rightarrow$ Пусть $l$ принадлежит пучку. Так как направляющие векторы $l, l_{1}$ и $l_{2}$; пропорциональны, можем домножить уравнения на числа так, что коэффициенты перед переменными станут равны: пусть $l_{1}: Ax + By + C_{1} = 0; \ l_{2} = Ax + By + C_{2} = 0; \ l = Ax + By + C_{3} = 0$.Тогда возьмём $\lambda, \mu$ из следующей системы: $\begin{cases}C_{1}\lambda + C_{2}\mu = C_{3}\\\lambda + \mu = 1\end{cases} \Leftrightarrow \begin{cases}\lambda = \frac{C_{3}-C_{2}}{C_{1}-C_{2}}\\\mu = \frac{C_{1} - C_{3}}{C_{1} - C_{2}}\end{cases} (C_{1} \neq C_{2}$, иначе $l_{1}$ и $l_{2}$ совпадают). Очевидно, что для таких $\lambda, \mu$ уравнение $l$ имеет вид ($*$) (проверяется несложной подстановкой), ч.т.д.
	\end{proof}
	\section{Отрезки}
	\subsection{Отрезки на плоскости}
	\begin{definition}
		Пусть $l$ - прямая, $X_{1}(x_{1}, y_{1}), X_{2}(x_{2}, y_{2}) \in l$ и $X_{1} \neq X_{2}$. Отрезком с концами $X_{1}, X_{2}$ на плоскости называется множество всех точек, лежащих между $X_{1}$ и $X_{2}$ (на прямой $l$). Обозначается $[X_{1}, X_{2}]$. 
	\end{definition}
	\begin{formula}[Уравнение отрезка]
		Пусть $X \in [X_{1}, X_{2}]$. Тогда знаем, что $\begin{cases}x = x_{1} + t(x_{2} - x_{1}) \\ y = y_{1} + t(y_{2} - y_{1})\end{cases} (X \in l) \Rightarrow \begin{pmatrix} x-x_{1} \\ y-y_{1} \end{pmatrix} = t\begin{pmatrix} x_{2}-x_{1} \\ y_{2}-y_{1} \end{pmatrix} \Rightarrow \\ t\overrightarrow{X_{1}X_{2}} = \overrightarrow{X_{1}X} \Rightarrow \overrightarrow{XX_{2}} = (1-t)\overrightarrow{X_{1}X_{2}}$. Отсюда видно, что $|\overrightarrow{X_{1}X}|$ и $|\overrightarrow{XX_{2}}| < |\overrightarrow{X_{1}X_{2}}| \Leftrightarrow t \in [0, 1]$. Отсюда $X \in [X_{1}, X_{2}] \Leftrightarrow \begin{cases}x = x_{1} + t(x_{2} - x_{1}) \\ y = y_{1} + t(y_{2} - y_{1})\\t\in [0, 1]\end{cases}$
	\end{formula}
	\begin{subtheorem}
		
	\end{subtheorem}
	\bfseries Мораль в том, что дальше очев... (по Гейне, конечно) \mdseries
\end{document}