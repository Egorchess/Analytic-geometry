\documentclass[a4paper, 12pt]{article}
%\usepackage{mathtext}
\usepackage{cmap}
\usepackage[english, russian]{babel}
\usepackage[T2A]{fontenc}
\usepackage[utf8]{inputenc}
\usepackage[left=2cm, right=1.5cm, top=2cm, bottom=2cm]{geometry}
\usepackage{amsmath}
\usepackage{amssymb}
\usepackage{etoolbox}
\usepackage{amsthm}
\usepackage{amsfonts}
\usepackage{mathtools}
%\usepackage{indentfirst}
\usepackage{soul}
\usepackage{graphicx}
\usepackage{enumerate}
\usepackage{nicematrix}
\usepackage{booktabs}
\usepackage{tikz,amstext}
\newlength{\tempheight}
\newcommand{\Let}[0]{%
	\mathbin{\text{\settoheight{\tempheight}{\mathstrut}\raisebox{0.5\pgflinewidth}{%
				\tikz[baseline,line cap=round,line join=round] \draw (0,0) --++ (0.4em,0) --++ (0,1.5ex) --++ (-0.4em,0);%
}}}}

\renewcommand{\phi}{\varphi}
\renewcommand{\epsilon}{\varepsilon}
\newcommand*\circled[1]{\tikz[baseline=(char.base)]{
            \node[shape=circle,draw,inner sep=2pt] (char) {#1};}}
\newcommand{\aug}{\fboxsep=-\fboxrule\!\!\!\fbox{\strut}\!\!\!}
\newcommand\tab[1][.5cm]{\hspace*{#1}}
\newcommand\undermat[2]{\makebox[0pt][l]{$\smash{\underbrace
			{\phantom{\begin{matrix}#2\end{matrix}}}_{\text{$#1$}}}$}#2}
\newcommand\overmat[2]{\makebox[0pt][l]{$\smash{\overbrace
			{\phantom{\begin{matrix}#2\end{matrix}}}^{\text{$#1$}}}$}#2}

\newcounter{lemcount}
\newcounter{sectcount}
\newcounter{thcount}
\theoremstyle{definition}
\newtheorem*{definition}{Определение}
\newtheorem*{theorem}{Теорема}
\newtheorem*{consequense}{Следствие}
\newtheorem*{lemma}{Лемма}
\newtheorem*{subtheorem}{Утверждение}
\newtheorem*{formula}{Вывод формулы}
\newtheorem*{formulas}{Вывод формул}
\newtheorem*{remark}{Замечание}
\newtheorem*{examples}{Примеры}
\newtheorem*{example}{Пример}
\newtheorem*{lalala}{Упражнение}
\newtheorem*{algorithm}{Алгоритм}
\newtheorem*{properties}{Свойства}
\newtheorem*{properties1}{Свойство}
\newtheorem{lemmanum}[lemcount]{Лемма}
\newtheorem{theoremnum}[thcount]{Теорема}
% \newtheorem{theoremL}{Теорема}[section]
\usepackage[T2A]{fontenc}
\usepackage[utf8]{inputenc}
\usepackage[russian]{babel}
\addto\captionsenglish{% Replace "english" with the language you use
	\renewcommand{\contentsname}%
	{Содержание}%
}

\newsavebox{\boxedalignbox}
\newenvironment{boxedalign*}
  {\begin{equation*}\begin{lrbox}{\boxedalignbox}$\begin{aligned}}
  {\end{aligned}$\end{lrbox}\fbox{\usebox{\boxedalignbox}}\end{equation*}}
\usepackage{titlesec}
\titleformat{\section}{\LARGE \bfseries}{Билет \thesection.}{1em}{}
\titleformat{\subsection}{\Large\bfseries}{\thesubsection}{1em}{}
\titleformat{\subsubsection}{\large\bfseries}{\thesubsubsection}{1em}{}

\usepackage{hyperref}
\usepackage{xcolor}
% Цвета для гиперссылок
\definecolor{linkcolor}{HTML}{225ae2} % цвет ссылок
\definecolor{urlcolor}{HTML}{225ae2} % цвет гиперссылок
\hypersetup{
	pdfstartview=FitH, 
	linkcolor=linkcolor,
	urlcolor=urlcolor,
	colorlinks=true
}

\begin{document}
	\begin{titlepage}
		\newpage
		
		\begin{center}
		\end{center}
		
		\vspace{4em}
		
		\begin{center}
			\Large Механико-математический факультет  
		\end{center}
		
		\vspace{2em}
		
		\begin{center}
			\large{\textsc{\textbf{Аналитическая геометрия, 1 семестр, 2 поток}}}

			\vspace{6em}

			\large{\textsc{\textbf{Билеты}}}
		\end{center}
		
		\vspace{\fill}
		
		\begin{center}
			Москва \\2024 
		\end{center}
	\end{titlepage}
	\tableofcontents
	\fontsize{14pt}{20pt}\selectfont
	\newpage
	\fontsize{14pt}{20pt}\selectfont
	\section{Векторные пространства и множества}
	\subsection{Векторные пространства}
	Геометрические векторы в математике являются \bfseries свободными векторами \mdseries - классами эквивалентности направленных отрезков по уже известному нам отношению эквивалентности векторов.
	\begin{definition}
		Векторным (линейным) пространством (над полем $\mathbb{R}$) называется множество V с введенными на нем бинарными операциями "+": $V \times V \rightarrow V$ и "$*$": $\mathbb{R} \times V \rightarrow V$ , отвечающие следующим свойствам (аксиомам):
		
		$\forall \bar{a}, \bar{b}, \bar{c} \in V; \lambda, \mu \in \mathbb{R}:$
		\begin{enumerate}
			\item $\bar{a} + \bar{b} = \bar{b} + \bar{a}$ (коммутативность сложения);
			\item $(\bar{a} + \bar{b}) + \bar{c} = \bar{a} + (\bar{b} + \bar{c})$ (ассоциативность сложения);
			\item $\exists \bar{0} \in V: \bar{a} + \bar{0} = \bar{0} + \bar{a} = \bar{a}$ (существует нейтральный элемент по сложению - нулевой вектор);
			\item $\exists (-\bar{a}) \in V: \bar{a} + (-\bar{a}) = (-\bar{a}) + \bar{a} = \bar{0}$ (существует противоположный элемент по сложению);
			\item $\lambda(\mu\bar{a}) = (\lambda\mu)\bar{a}$ (ассоциативность умножения на числа);
			\item $(\lambda + \mu)\bar{a} = \lambda\bar{a}+ \mu\bar{a}$ (дистрибутивность по умножению);
			\item $\lambda(\bar{a} + \bar{b}) = \lambda\bar{a}+ \lambda\bar{b}$ (дистрибутивность по сложению);
			\item $1*\bar{a} = \bar{a}$.
		\end{enumerate}
	\end{definition}
	\begin{examples}
		Векторные пр-ва:
		\begin{itemize}
			\item $\mathbb{R}, \mathbb{R}^2, \mathbb{R}^n$:
			\item Функции;
			\item Многочлены;
			\item Многочлены степени $\leqslant n$.
		\end{itemize}
	\end{examples}
	\begin{remark}
		Св-ва векторных пространств:
		
		\begin{enumerate}
			
			\item $\bar{0}$ единственный.
			
			Пусть $\bar{0}_{1}, \bar{0}_{2}$ - нулевые векторы.
			
			Тогда $\bar{0}_{1} = \bar{0}_{1} + \bar{0}_{2} = \bar{0}_{2}$, ч.т.д.
			\item $-\bar{a}$ единственный.
			
			Пусть $-\bar{a}_{1}, -\bar{a}_{2}$ - противоположные к $\bar{a}$ векторы.
			
			Тогда $-\bar{a}_{1} = -\bar{a}_{1} + \bar{0} = -\bar{a}_{1} + (\bar{a} + -\bar{a}_{2}) = (-\bar{a}_{1} + \bar{a}) + (-\bar{a}_{2}) = \bar{0} + (-\bar{a}_{2}) = -\bar{a}_{2}$, ч.т.д.
			\item $\lambda * \bar{0} = \bar{0}$.
			
			$\lambda * \bar{0} = \lambda * (\bar{0} + \bar{0}) = \lambda * \bar{0} + \lambda * \bar{0}$
			
			Прибавив к обеим частям вектор, противоположный к $\lambda * \bar{0}$, получим $\lambda * \bar{0} = \bar{0}$, ч.т.д.
			\item $-(\lambda\bar{a}) = (-\lambda)\bar{a} = \lambda(-\bar{a})$.
			
			Нетрудно видеть, что все три вектора противоположны $\lambda\bar{a}$, а далее из п.2.
			\item $-\bar{a} = -1*\bar{a}$
			
			Следует из п.4.
			\item $\lambda\bar{a} = \bar{0} \Leftrightarrow$ либо $\lambda = 0$, либо $\bar{a} = \bar{0}$.
			
			Либо $\lambda = 0$, либо $\lambda \neq 0 \Rightarrow \frac{1}{\lambda}*\lambda*\bar{a} = \frac{1}{\lambda}\bar{0} = \bar{0} \Rightarrow \bar{a} = \bar{0}$, ч.т.д. 
		\end{enumerate}
	\end{remark}
	\subsection{Линейная комбинация векторов}
	\begin{definition}
		Сумма вида $\lambda_{1}\bar{x}_{1} + ... + \lambda_{n}\bar{x}_{n}$ называется линейной комбинацией векторов $\bar{x}_{1} ... \bar{x}_{n}$.
	\end{definition}
	\begin{definition}
		Если в линейной комбинации $\lambda_{1} = ... = \lambda_{n} = 0$, то она называется тривиальной, а иначе - нетривиальной.
	\end{definition}
	\begin{definition}
		Если вектор $\bar{x}$ равен линейной комбинации $\lambda_{1}\bar{x}_{1} + ... + \lambda_{n}\bar{x}_{n}$, то говорят, что он линейно выражается (раскладывается) через векторы $\bar{x}_{1}...\bar{x}_{n}$.
		(Сама линейная комбинация $\lambda_{1}\bar{x}_{1} + ... + \lambda_{n}\bar{x}_{n}$ называется выражением (разложением) вектора $\bar{x}$ через $\bar{x}_{1}...\bar{x}_{n}$)  
	\end{definition}
	\subsection{Линейно зависимые и линейно независимые множества и системы векторов}
	\begin{definition}
		Упорядоченное множество векторов называется системой векторов.
		(В системе векторов элементы могут повторяться)
	\end{definition}
	\begin{definition}
		Множество векторов называется линейно зависимым, если существует равная нулю нетривиальная линейная комбинация векторов из этого множества. В противном случае оно называется линейно независимым.
	\end{definition}
	\begin{example}
		Система из двух векторов $\bar{a}, \bar{b}$ линейно зависима $\Leftrightarrow \bar{a} = \lambda\bar{b}$.
	\end{example}
	\begin{remark}
		Множество векторов линейно зависимо $\Leftrightarrow$ один из векторов этого множества линейно выражается через некоторые другие векторы этого множества.
	\end{remark}
	\subsection{Полные множества и системы векторов}
	\begin{definition}
		Множество (система) векторов из векторного пространства V называется полным(полной) в $V$, если любой вектор $\bar{x}\in V$ линейно выражается через векторы этого множества.
	\end{definition}
	\begin{remark}
		$X \subset V$ полно в $V \Rightarrow \forall Y: X\subset Y$ полно в $V$.
	\end{remark}
	\begin{remark}
		$X \subset V$ линейно независимо в $V \Rightarrow \forall Y \subset X$ линейно независимо в $V$.
	\end{remark}
	\section{Базис и размерность векторного пространства}
	\subsection{Базис векторного пространства. Конечномерное векторное пространство.}
	\begin{definition}
		Множество векторов $E$ в векторном пространстве $V$ называется базисом $V$, если $E$ линейно независимо и полно в $V$.
	\end{definition}
	\begin{definition}
		Векторное пространство, в котором существует конечный (состоящий из конечного числа векторов) базис, называется конечномерным. В противном случае оно называется бесконечномерным. 
	\end{definition}
	\subsection{Лемма о количестве векторов в ЛНЗ системе (аналог ОЛЛЗ)}
	\begin{lemma}
		Если $X$ - конечное полное множество из $n$ векторов в векторном пространстве $V$ и $Y$ - линейно независимое множество векторов в $V$, то $Y$ конечно и число векторов в $Y \leqslant n$. 
	\end{lemma}
	\begin{proof}
		(пер.)
		Произвольно занумеруем векторы в $X: (\bar{x}_{1},...,\bar{x}_{n})$.
		Будем по одному добавлять в эту систему векторы из $Y$ и одновременно выкидывать векторы из $X$ так, чтобы система оставалась полной.
		
		Пусть за $k$ шагов ($0\leqslant k \leqslant n$) мы добавили некоторые $\bar{y}_{1},...,\bar{y}_{k}$ и выкинули какие-то $k$ векторов из $X$ - осталась система ($\bar{y_{1}},...,\bar{y_{k}},\bar{x}_{i_{1}},...,\bar{x}_{i_{n-k}}$).
		Возьмём $\bar{y}_{k+1}$ из $Y$ (если такого нет, то в $Y \leqslant n$ векторов, что нам и нужно), и добавим его в систему. Так как до этого система оставалась полной, $\bar{y}_{k+1}$ выражается через ($\bar{y_{1}},...,\bar{y_{k}},\bar{x}_{i_{1}},...,\bar{x}_{i_{n-k}}$), причём какой-то $\bar{x}_{i_{j}}$ входит в это разложение с коэффициентом, не равным нулю (иначе противоречие с линейной независимостью $Y$ - $\bar{y}_{k+1}$ выразился через $\bar{y}_{1},...,\bar{y}_{k}$).
		
		Тогда $\bar{x}_{i_{j}}$ выражается через другие векторы системы и $\bar{y}_{k+1}$ (в выражении $\bar{y}_{k+1}$ перенесём всё, кроме $\bar{x}_{i_{j}}$ в другую часть и разделим на коэффициент перед ним).
		А так как ($\bar{y_{1}},...,\bar{y_{k+1}},\bar{x}_{i_{1}},...,\bar{x}_{i_{n-k}}$) - полная, эта же система без $\bar{x}_{i_{j}}$. очевидно, останется полной.

		Пусть смогли проделать $n$ таких шагов. Тогда имеем систему ($\bar{y}_{1},...,\bar{y}_n$). Если в $Y$ есть ещё векторы, то они с одной стороны выражаются через векторы системы из её полноты, а с другой - не выражаются через них из линейной независимости $Y$. Противоречие, т.е. в $Y$ не может оказаться больше $n$ векторов, ч.т.д.   
	\end{proof}
	\subsection{Теорема о количестве векторов в базисе. Размерность векторного пространства.}
	\begin{theorem}
		Если в векторном пространстве есть конечный базис, то все базисы в нём конечны и содержат одинаковое количество векторов.
	\end{theorem}
	\begin{proof}
		Пусть $\bar{e}_{1},...,\bar{e}_{n}$ - конечный базис в $V$. Любой другой базис $V$ линейно независим, т.е. по лемме содержит $k \leqslant n$ векторов, а с другой стороны полон, т.е. первый базис по лемме содержит $n \leqslant k$ векторов. Отсюда $n=k$, ч.т.д.
	\end{proof}
	\begin{definition}
		Количество векторов в любом базисе векторного пространства $V$ называется размерностью $V$ и обозначается $dim V$.
	\end{definition}
	\begin{examples}
		$dim \ {\bar{0}} = 0, dim \ \pi (= dim \ \mathbb{R}^2) = 2, dim \ \mathbb{R}^3 = 3$.
	\end{examples}
	\section{Координаты в базисе}
	\begin{theorem}
		В конечномерном векторном пространстве выражение любого вектора через базис определяется однозначно.
	\end{theorem}
	\begin{proof}
		Если $\bar{x} = \lambda_{1}\bar{e}_{1} + ... + \lambda_{n}\bar{e}_{n} = \lambda'_{1}\bar{e}_{1} + ... + \lambda'_{n}\bar{e}_{n}$, то $\bar{x} - \bar{x} = \bar{0} = (\lambda_{1} - \lambda'_{1})\bar{e}_{1} + ... + (\lambda_{n} - \lambda'_n)\bar{e}_{n}$. Если эти два разложения различны, то равная нулю линейная комбинация базисных векторов нетривиальна, что противоречит линейной независимости базиса. То есть двух различных разложений быть не может, ч.т.д. 
	\end{proof}
	\begin{definition}
		Пусть $V$ - конечномерное векторное пространство и $\bar{e}_{1},...,\bar{e}_{n}$ - базис в нём. Коэффициенты $\lambda_{1},...,\lambda_{n}$ в выражении любого вектора $x \in V$ через эти базисные векторы называются координатами вектора $x$ в базисе $\bar{e}_{1},...,\bar{e}_{n}$. ($\lambda_{k}$  называется $k$-й координатой)
	\end{definition}
	\begin{remark}
		Векторы в $n$-мерном векторном пространстве находятся во взаимно однозначном соответствии с упорядоченной строкой из $n$ чисел из $\mathbb{R}$ (например, векторы ассоциированного с евклидовой плоскостью векторного пространства соответствуют парам чисел)
		Таким образом можно задать операции сложения и умножения на число векторов плоскости через операции над числами, проводимыми покоординатно.
	\end{remark}
	\section{Аффинные пространства}
	\subsection{Аффинное пространство}
	Элементы плоскости (как множества) - точки, а не векторы, поэтому для работы непосредственно с плоскостью необходимо ввести данное определение.
	\begin{definition}
		Аффинное пространство - тройка ($X, V, +$) (обычно обозначается $\mathbb{A}$), где $X$ - множество (точек), $V$ - векторное пространство, а $''+''$ операция:  $X \times V \rightarrow X$, для которых выполнены аксиомы:
		\begin{enumerate}
			\item $\forall A \in X, \forall \bar{a}, \bar{b} \in V: A+(\bar{a}+\bar{b}) = (A+\bar{a})+\bar{b}$;
			\item $\forall A \in X: A + \bar{0} = A$;
			\item $\forall A, B \in X \ \ \exists! \ \bar{a} \in V: A + \bar{a} = B$. Обозначается $\bar{a} = \overrightarrow{AB}$.
		\end{enumerate}
	\end{definition}
	\subsection{Радиус-векторы и репер}
	Если зафиксировать какую-нибудь точку $O \in X$, возникает взаимно однозначное соответствие между точками $A$ и их радиус-векторами $\overrightarrow{OA}$.
	\begin{definition}
		Репер (система координат) в аффинном пространстве $(X, V, +)$ - пара $(O, E)$, где $O \in X$ и $E$ - базис в $V$. Точка $O$ называется началом координат (отсчёта). Координаты точки A в $(O, E)$ - координаты её радиус-вектора $\overrightarrow{OA}$ в базисе $E$.
	\end{definition}
	\begin{remark}
		Для аффинного пространства верно:
		\begin{enumerate}
			\item Если $A = (x_{1},...,x_{n}),  \bar{a} = (y_{1},...,y_{n})$, то $A + \bar{a} = (x_{1} + y_{1},...,x_{n} + y_{n})$.
			\item Если $A = (a_{1},...,a_{n}),  B = (b_{1},...,b_{n})$, то $\overrightarrow{AB} = (b_{1} - a_{1},...,b_{n} - a_{n})$.
		\end{enumerate}
		(Следует из сложения векторов)
	\end{remark}
	\subsection{Конечномерное аффинные пространства и их размерность}
	\begin{definition}
		Если $\mathbb{A} = (X, V, +)$ - аффинное пространство, то говорят, что $V$ - векторное пространство, ассоциированное с $\mathbb{A}$.    
	\end{definition}
	\begin{definition}
		$\mathbb{A}$ называется конечномерным, если ассоциированное с ним $V$ конечномерно. В этом случае $dim \mathbb{A}$ (размерность $\mathbb{A}$) равна $dim V$.
	\end{definition}
	Теперь точки аффинного пространства аналогично векторам можно ассоциировать с наборами чисел. Однако для ассоциирования евклидовой плоскости и её аксиом с двумерным аффинным пространством, необходимы отвечающие аксиомам понятия прямой и расстояния.
	\section{Подпространства}
	\subsection{Векторное подпространство}
	\begin{definition}
		Векторным подпространством векторного пространства $V$ называется непустое множество $V_{1} \subset V$ такое. что $\forall\bar{x}, \bar{y}\in V_{1}: \bar{x} + \bar{y} \in V_{1}, \lambda\bar{x} \in V_{1} (\forall\lambda \in\mathbb{R})$.
	\end{definition}
	\begin{remark}
		Определение эквивалентно следующему: множество $V_{1} \subset V$ - векторное подпространство $V$, если $V_{1}$ является векторным пространством относительно операций $+$ и $*$, определённых для $V$.
		(Доказательство осуществляется путём прямой проверки аксиом векторного пространства для $V_{1}$)
	\end{remark}
	\subsection{Аффинное подпространство}
	Введём несколько определений аффинного подпространства и докажем их эквивалентность.
	\begin{definition}
		Аффинным подпространством аффинного пространства $\mathbb{A} = (X, V, +)$ называется
		\begin{enumerate}
			\item его непустое подмножество вида $A + V_{1} = {A + \bar{a}:\bar{a} \in V_{1}}$, где $V_{1}$ - векторное подпространство $V$ и $A \in X$ - точка;
			\item тройка $(X_{1}\subset X, V_{1} \subset V, +_{1})$, где $V_{1}$ - векторное подпространство $V$ и операция $+_{1} = +$,  для которой $\forall A,B \in X_{1}, \forall \bar{a} \in V_{1}: A + \bar{a} \in X_{1}, \overrightarrow{AB} \in V_{1}$;
			\item тройка $(X_{1}\subset X, V_{1} \subset V, +_{1})$, где $V_{1}$ - векторное подпространство $V$ и операция $+_{1} = +$,  которая сама является аффинным пространством.
		\end{enumerate}
	\end{definition}
	\begin{subtheorem}
		Приведённые определения эквивалентны.
	\end{subtheorem}
	\begin{proof}
		Докажем следующие следствия:

		$\circled{1}\Rightarrow\circled{2}$  Пусть $P = A + \bar{a}, Q = A + \bar{b}$. Тогда $\overrightarrow{PQ} = \bar{b} - \bar{a}$ (в силу единственности такого вектора), т.е. $\overrightarrow{PQ} \in V_{1}$. Второе необходимое свойство $\circled{2}$ очевидно выполнено.

		$\circled{2}\Rightarrow\circled{1}$  Пусть $X_{1}, V_{1}$ удовлетворяют $\circled{2}$. Зафиксируем произвольную $A \in X_{1}$. $\forall B\in X_{1}$ имеем $B = A + \overrightarrow{AB}$, причём $A \in X_{1}, \overrightarrow{AB} \in V_{1} \Rightarrow B \in X_{1}$.
		
		Эквивалентность $\circled{2}\Leftrightarrow\circled{3}$ очевидна из определения аффинного пространства.
	\end{proof}
	\subsection{Прямая в аффинном пространстве}
	\begin{definition}
		Прямая в аффинном пространстве - его одномерное аффинное подпространство. \\Плоскость (двумерная) в аффинном пространстве - его двумерное аффинное подпространство.
	\end{definition}
	\begin{definition}
		Единственный вектор в любом базисе векторного пространства, ассоциированного с одномерным аффинным пространством, называется направляющим вектором этого аффинного пространства.
	\end{definition}
	\section{Скалярное произведение}
	\subsection{Скалярное произведение}
	\begin{definition}
		Пусть $V$ - векторное пространство. Скалярным произведением в $V$ называется функция $(\ ,\ ) : V \times V \rightarrow \mathbb{R}$ со свойствами:
		\begin{enumerate}
			\item $(\bar{x}, \bar{x}) \geqslant 0 \ \forall \bar{x} \in V$, причём $(\bar{x}, \bar{x}) = 0 \Leftrightarrow \bar{x} = \bar{0}$ (положительная определённость);
			\item $(\bar{x}, \bar{y}) = (\bar{y}, \bar{x}) \ \forall \bar{x}, \bar{y} \in V$ (коммутативность);
			\item $(\alpha\bar{x} + \beta\bar{y}, \bar{z}) = \alpha(\bar{x}, \bar{z}) + \beta(\bar{y}, \bar{z}) \ \forall \bar{x}, \bar{y}, \bar{z} \in V, \alpha,\beta \in \mathbb{R}$ (линейность по первому аргументу)
		\end{enumerate} 
	\end{definition}
	Из коммутативности выполнена и линейность по второму аргументу, т.е. скалярное произведение - билинейная функция.
	\subsection{Евклидово векторное и точечно-евклидово аффинное пространство}
	\begin{definition}
		Конечномерное аффинное (векторное) пространство вместе со скалярным произведением называется точечно-евклидовым (евклидовым) пространством. Двумерное точечно-евклидово пространство называется евклидовой плоскостью.
	\end{definition}
	\subsection{Длина вектора и расстояния между точками}
	\begin{definition}
		Длиной вектора называется величина $\sqrt{(\bar{x}, \bar{x})}$.
	\end{definition}
	\begin{definition}
		Расстоянием (евклидовым) между точками $A,B \in \mathbb{A}$ называется длина вектора $\overrightarrow{AB}$. Будем обозначать $d(A, B)$ как $|\overrightarrow{AB}|$. 
	\end{definition}
	\subsection{Выражение скалярного произведения через длины}
	\begin{remark}
		Зная длины всех векторов, скалярное произведение можно восстановить по формуле $(\bar{x}, \bar{y}) = \frac{1}{2}(|\bar{x} + \bar{y}|^2 - |\bar{x}|^2 - |\bar{y}|^2)$. Это несложно проверить: \\
		$\frac{1}{2}(|\bar{x} + \bar{y}|^2 - |\bar{x}|^2 - |\bar{y}|^2) = \frac{1}{2}((\bar{x} + \bar{y}, \bar{x} + \bar{y}) - (\bar{x}, \bar{x}) - (\bar{y}, \bar{y})) = \frac{1}{2}(2(\bar{x}, \bar{y})) = (\bar{x}, \bar{y})$.
	\end{remark}
	\section{Неравенство Коши-Буняковского}
	\subsection{Неравенство Коши-Буняковского}
	\begin{theorem}[Неравенство Коши-Буняковского]
		$\forall \bar{a}, \bar{b} \in V \ \ (\bar{a}, \bar{b}) \leqslant \sqrt{(\bar{a}, \bar{a})(\bar{b}, \bar{b})}$, причём равенство достигается только при $\bar{a} = \lambda\bar{b}$.
	\end{theorem}
	\begin{proof}
		Рассмотрим выражение $(\bar{a} + t\bar{b}, \bar{a} + t\bar{b})$. Оно равно нулю $\Leftrightarrow \bar{a} = -t\bar{b}$, т.е. может быть равно нулю не более чем при одном $t$. С другой стороны
		$(\bar{a} + t\bar{b}, \bar{a} + t\bar{b})$ = $(\bar{a}, \bar{a}) + 2(\bar{a}, \bar{b})t + (\bar{b}, \bar{b})t^2$ - квадратный трёхчлен относительно $t$. Его дискриминант равен $4(\bar{a}, \bar{b})^2 - 4(\bar{a}, \bar{a})(\bar{b}, \bar{b})$, а из первого рассуждения знаем, что дискриминант $\leqslant 0$, причём равенство достигается только в случае коллинеарности $\bar{a}$ и $\bar{b}$. Отсюда $(\bar{a}, \bar{b}) \leqslant \sqrt{(\bar{a}, \bar{a})(\bar{b}, \bar{b})}$, ч.т.д.
	\end{proof}
	\subsection{Величина угла и ортогональные векторы}
	\begin{definition}
		Величиной угла между ненулевыми векторами $\bar{a}, \bar{b}$  называется число $arccos\frac{(\bar{a}, \bar{b})}{|\bar{a}||\bar{b}|}$ (из н. Коши-Буняковского $|\frac{(\bar{a}, \bar{b})}{|\bar{a}||\bar{b}|}| \leqslant 1$).
	\end{definition}
	\begin{definition}
		Векторы $\bar{a}, \bar{b}$ называются ортогональными (перпендикулярными), если $(\bar{a}, \bar{b}) = 0$.
	\end{definition}
	\section{Прямоугольная система координат}
	\subsection{Ортонормированный базис и прямоугольная система координат}
	\begin{definition}
		Базис векторного пространства $V$ со скалярным произведением называется ортонормированным, если все его векторы попарно ортогональны и имеют длину 1.
	\end{definition}
	\begin{definition}
		Система координат в точечно-евклидовом пространстве называется прямоугольной, если её базис ортонормированный.
	\end{definition}
	\subsection{Выражение скалярного произведения через координаты векторов}
	\begin{subtheorem}
		В точечно-евклидовом пространстве верно следующее выражение скалярного произведения через координаты векторов: если в некотором базисе $ (\bar{e}_{1},...,\bar{e}_{n}) \ \bar{x} = \begin{pmatrix} x_{1} \\ \vdots \\ x_{n} \end{pmatrix}, \bar{y} = \begin{pmatrix} y_{1} \\ \vdots \\ y_{n} \end{pmatrix}$, то $(\bar{x}, \bar{y}) = \sum \limits_{i=1}^n x_{i} \cdot \sum \limits_{j=1}^n y_{j}(\bar{e}_{i}, \bar{e}_{j})$.
	\end{subtheorem}
	\begin{proof}
		$(\bar{x}, \bar{y}) = (x_{1}\bar{e}_{1}+...+x_{n}\bar{e}_{n}, y_{1}\bar{e}_{1}+...+y_{n}\bar{e}_{n}) = \sum \limits_{i=1}^{n}(x_{i}\bar{e}_{i},y_{1}\bar{e}_{1}+...+y_{n}\bar{e}_{n}) = \sum \limits_{i=1}^{n}x_{i}(\bar{e}_{i},y_{1}\bar{e}_{1}+...+y_{n}\bar{e}_{n}) = \sum \limits_{i=1}^n x_{i} \cdot \sum \limits_{j=1}^n y_{j}(\bar{e}_{i}, \bar{e}_{j})$
	\end{proof}
	\subsection{Выражение для прямоугольной системы координат}
	\begin{remark}
		В случае, когда базис ортонормированный, имеем $(e_{i}, e_{j}) = \delta_{ij}$ (здесь и далее используется символ Кронекера: $\delta_{ij} = \begin{cases} 0, i \neq j \\ 1, i = j\end{cases}$), т.е. $(\bar{x}, \bar{y}) = x_{1}y_{1}+...+x_{n}y_{n}$. То есть в прямоугольной системе координат длина вектора вычисляется по формуле $|\bar{x}| = \sqrt{x_{1}^2+...+x_{n}^2}$, а расстояние между точками $P = (x_{1},...,x_{n}), Q = (y_{1},...,y_{n})$ выражается как \begin{boxedalign*}|PQ| = |\overrightarrow{PQ}| = \sqrt{(y_{1} - x_{1})^2+...+(y_{n}- x_{n})^2} \end{boxedalign*}
	\end{remark}
	\section{Проектирование}
	\begin{definition}
		Пусть задано два векторных подпространства $V_{1}, V_{2}$ векторного пространства $V$ такие, что $V_{1} \cap V_{2} = \{\bar{0}\}$ и $V_{1} + V_{2} = V$ (обозначается $V = V_{1} \oplus V_{2}$). Тогда сумма $\bar{x} = \bar{x}_{1} + \bar{x}_{2}$, где $\bar{x} \in V, \bar{x}_{1} \in V_{1}, \bar{x}_{2} \in V_{2}$, определена единственно. (Следует, например, из того, что в любом базисе $V$ каждый его вектор лежит либо в $V_{1}$, либо в $V_{2}$, тогда разложение в эту сумму соответствует единственному разложению по базису). Проекцией вектора $\bar{x} \in V$ на $V_{1}$ параллельно $V_{2}$ называется слагаемое $\bar{x_{1}}$ этой суммы. 
	\end{definition}
	\begin{definition}
		Пусть задано два аффинных подпространства $\mathbb{A}_{1} = (X_{1}, V_{1}, +),\\ \mathbb{A}_{2} = (X_{2}, V_{2}, +)$ аффинного пространства $\mathbb{A} = (X, V, +)$ такие, что $V = V_{1} \oplus V_{2}$. Проекция точки $P \in \mathbb{A}$ на $\mathbb{A}_{1}$ параллельно $\mathbb{A}_{2}$ - точка $P_{1} = A_{1} + \bar{v}$, где $A_{1}$ - произвольная точка из $X_{1}$, а $\bar{v}$ - проекция $\overrightarrow{A_{1}P}$ на $V_{1}$ параллельно $V_{2}$.
		(Очевидно, что от выбора $A_{1}$ расположение проеции не зависит)
	\end{definition}
	\begin{example}
		Рассмотрим координаты точки евклидовой плоскости относительно прямоугольной системы координат. \\
		Найдём проекцию точки $A = (x, y)$ на прямую $Oy$ параллельно прямой $Ox$. По определению это точка (назовём её $A_{y}$), равная $O + \bar{v}$, где $\bar{v}$ - проекция $\overrightarrow{OP}$ на векторное пространство прямой $Oy$ параллельно $Ox$. $\overrightarrow{OP} = \{x, y\} = x\bar{e}_{1} + y\bar{e}_{2}$. Отсюда $\bar{v} = y\bar{e}_{2} = \{0, y\}$, то есть $A_{y} = (0, y)$. Аналогично $A_{x} = (x, 0)$. 
	\end{example}
	\section{Ортонормированный базис}
	Определение смотри в пункте 8.1
	Пусть $\mathbb{A}^n = (X, V^n, +)$ - $n$-мерное точечно-евклидово пространство.
	\subsection{Линейная независимость ортогональных векторов}
	\begin{subtheorem}
		В $V^n$ любая линейно независимая система из $n$ векторов образует базис.
	\end{subtheorem}
	\begin{proof}
		Предположим, что в $V^n$ существует неполная линейно независимая система из n векторов. Т.к. система не полная, существует вектор из $V^n$, не выражающийся через векторы этой системы, т.е. этот вектор можно добавить в систему без потери линейной независимости. Но по лемме-аналогу ОЛЛЗ линейно независимая система в $V^n$ не может иметь $> n$ векторов. Противоречие, т.е. любая линейно независимая система из $n$ векторов является полной, а значит и базисом, ч.т.д. 
	\end{proof}
	\begin{subtheorem}
		Если $\bar{e}_{1},...,\bar{e}_{n}$ - попарно ортогональные ненулевые векторы в евклидовом пространстве, то $\bar{e}_{1},...,\bar{e}_{n}$ линейно независимы.
	\end{subtheorem}
	\begin{proof}
		Предположим противное. Пусть один из векторов (без ограничения общности $\bar{e}_{n}$) линейно выражается через остальные: $\bar{e}_{n} = \lambda_{1}\bar{e}_{1} +...+ \lambda_{n-1}\bar{e}_{n-1}$. Тогда запишем квадрат его длины:
		$|\bar{e}_{n}|^2 = (\bar{e}_{n}, \bar{e}_{n}) = (\bar{e}_{n}, \lambda_{1}\bar{e}_{1} +...+ \lambda_{n-1}\bar{e}_{n-1}) = \sum \limits_{i=1}^{n-1} \lambda_{i}(\bar{e}_{n}, \bar{e}_{i}) = 0$ (т.к. $\bar{e}_{n}$ ортогонален всем остальным векторам). Отсюда $|\bar{e}_{n}| = 0$, и притом $\bar{e}_{n}$ ненулевой. Противоречие, т.е. никакой вектор системы не выражается через остальные, а значит система линейно независима. ч.т.д.
	\end{proof}
	\subsection{Теорема о существовании ортонормированного базиса}
	\begin{theorem}
		В любом евклидовом пространстве существует ортонормированный базис.
	\end{theorem}
	\begin{proof}
		(пер.)
		Индукция по $n$ - размерности пространства:

		База: $n = 1$ - очевидно, что существует вектор длины 1, который составляет ортонормированный базис одномерного пространства;

		Шаг: Пусть в любом $n$-мерном пространстве существует ортонормированный базис. Рассмотрим пространство $V$ размерности $n+1$ и выберем базис какого-то $n$-мерного подпространства $W$ (пусть $(\bar{e}_{1},...,\bar{e}_{n})$). Найдём вектор, ортогональный всем выбранным векторам. Так как базис $W$ не полон в $V$, к нему можно добавить ещё один вектор $x \in V$ без потери линейной независимости $\Rightarrow$ $(\bar{e}_{1},...,\bar{e}_{n}, \bar{x})$ - базис в $V$ (ЛНЗ система из $n+1$ векторов). \\
		Теперь необходимо представить $\bar{x}$ как следующую сумму: $\bar{x} = \lambda_{1}\bar{e}_{1} + ... + \lambda_{n}\bar{e}_{n} + \bar{e}_{n+1}$, где $\bar{e}_{n+1}$ ортогонален $\bar{e}_{1},...,\bar{e}_{n}$. Тогда $\bar{e}_{n+1} = \bar{x} - \lambda_{1}\bar{e}_{1} - ... - \lambda_{n}\bar{e}_{n}$. Рассмотрим $(\bar{e}_{n+1}, \bar{e}_{k}) = (\bar{x} - \lambda_{1}\bar{e}_{1} - ... - \lambda_{n}\bar{e}_{n}, \bar{e}_{k}) = (\bar{x}, \bar{e}_{k}) - \lambda_{1}(\bar{e}_{1}, \bar{e}_{k}) - ... - \lambda_{n}(\bar{e}_{n}, \bar{e}_{k})$. Так как $\bar{e}_{k}$ ортогонально всем этим векторам, кроме $\bar{e}_{k}$ и $\bar{x}$, это выражение равно $(\bar{x}, \bar{e}_{k}) - \lambda_{k}(\bar{e}_{k}, \bar{e}_{k})$. Отсюда при $\lambda_{k} = \frac{(\bar{x}, \bar{e}_{k})}{(\bar{e}_{k}, \bar{e}_{k})}$ векторы $\bar{e}_{n+1}$ и $\bar{e}_{k}$ ортогональны (зависит только от $\lambda_{k}$). Составив таким образом все $\lambda_{1},...,\lambda_{n}$, получим выражение вектора $\bar{e}_{n+1}$, ортогонального всем векторам базиса $W$.
		Таким образом, векторы полученной системы $\bar{e}_{1},...,\bar{e}_{n+1}$ попарно ортогональны (по предположению индукции) $\Rightarrow$ линейно независимы $\Rightarrow$ образуют базис в $V$. Разделив $\bar{e}_{n+1}$ на его длину, получим, что все векторы базиса попарно ортогональны и имеют длину 1 $\Rightarrow V$ имеет ортонормированный базис, ч.т.д.
	\end{proof}
	\begin{consequense}
		Любую систему ортогональных векторов длины 1 в векторном пространстве можно дополнить до ортонормированного базиса.
	\end{consequense}
	\section{Прямые и их уравнения}
	\subsection{Определения прямой и направляющего вектора}
	Смотри пункт 5.3
	\subsection{Уравнения прямой}
	\begin{formulas}[\bfseries уравнения прямой\mdseries]
		Пусть $l$ - прямая на плоскости: $l = \{X: \overrightarrow{OX} = \overrightarrow{OM} + t\bar{v}\}$, где $M$ - точка прямой, $\bar{v}$ - её направляющий вектор. Если $M=\begin{pmatrix}x_{0}\\y_{0}\end{pmatrix},\bar{v}=\begin{pmatrix}a\\b\end{pmatrix}$, то из совпадения координат совпадающих векторов $\overrightarrow{OX}$ и $(\overrightarrow{OM}+t\bar{v})$ верно следующее: (\bfseries параметрические уравнения прямой\mdseries) \begin{boxedalign*}X \in l: \begin{cases}x = x_{0} + at \\ y = y_{0} + bt\end{cases}\end{boxedalign*}\\
		Выразим $t$ из первого уравнения и подставим во второе уравнение - получим \bfseries каноническое уравнение прямой\mdseries : \begin{boxedalign*}\frac{x-x_{0}}{a} = \frac{y-y_{0}}{b}\end{boxedalign*}
		(Заметим, что данное выражение не определено при нулевых $a$ или $b$, но очевидно, что они не равны нулю одновременно, а запись, где одна из дробей имеет знаменатель 0, иногда используется, поэтому здесь и далее случай равенства нулю знаменателя может не рассматриваться как отдельный и будет означать, что числитель должен равняться 0)\\
		Если известно, что прямой принадлежат $M = \begin{pmatrix}x_{0}\\y_{0}\end{pmatrix}, N = \begin{pmatrix}x_{1}\\y_{1}\end{pmatrix}$, то $\overrightarrow{MN} = \begin{pmatrix}x_{1} - x_{0}\\y_{1} - y_{0}\end{pmatrix}$ - направляющий вектор, т.е. каноническое уравнение имеет вид \begin{boxedalign*}\frac{x-x_{0}}{x_{1}-x_{0}} = \frac{y-y_{0}}{y_{1}-y_{0}}\end{boxedalign*}(\bfseries уравнение прямой по двум точкам\mdseries).\\
		Домножим каноническое уравнение прямой на знаменатели: $\frac{x-x_{0}}{a} = \frac{y-y_{0}}{b} \Rightarrow bx-bx_{0}=ay-ay_{0} \Rightarrow bx - ay + (ay_{0}-bx_{0}) = 0$. Такое уравнение обычно называют \bfseries общим уравнением прямой\mdseries \ и записывают как \begin{boxedalign*}Ax + By + C = 0\end{boxedalign*}
	\end{formulas}
	\begin{remark}
		Для прямых в пространстве подобным образом выводятся параметрические и каноническое уравнения. 
	\end{remark}
	\begin{remark}
		Заметим также, что из итоговой формулы вывода общего уравнения ($bx - ay + (ay_{0}-bx_{0}) = 0 \Rightarrow Ax + By + C = 0$) следует, что для прямой $Ax + By +C = 0$ вектор ($B, -A$) (а соответственно и ($-B, A$)) является направляющим.
	\end{remark}
	\subsection{Критерий уравнения прямой (нет в билете, важно)}
	\begin{subtheorem}
		$Ax + By + C = 0$ является уравнением прямой $\Leftrightarrow A$ и $B$ не равны нулю одновременно.
	\end{subtheorem}
	\begin{proof}
		$\\ \Rightarrow$ Если $Ax + By + C = 0$, то её направляющий вектор ненулевой, а значит вектор $(-B, A)$ ненулевой, то есть одна из его координат $\neq 0$.
		$\\ \Leftarrow$ Пусть без ограничения общности $A \neq 0$. Тогда этому уравнению удовлетворяет точка $(x_{0}, y_{0}) = (-\frac{C}{A}, 0)$, а значит (нетрудно проверить) все удовлетворяющие ему точки имеют вид $(x_{0} + Bt, y_{0} - At)$, что соответствует прямой с такими параметрическими уравнениями.
	\end{proof}
	\section{Взаимное расположение прямых}
	\subsection{Случай общих уравнений}
	\begin{theorem}
		Прямые на плоскости параллельны (или совпадают) $\Leftrightarrow$ их направляющие векторы пропорциональны.
	\end{theorem}
	\begin{proof}
		Пусть $l_{1}: A_{1}x + B_{1}y + C_{1} = 0; l_{2}: A_{2}x + B_{2}y + C_{2} = 0$ - данные прямые.
		Рассмотрим систему уравнений, которой удовлетворяют координаты точек, принадлежащих обоим прямым: $\begin{cases}A_{1}x + B_{1}y = -C_{1}\\A_{2}x + B_{2}y = -C_{2}\end{cases}\\$ Из курса алгебры (форумла Крамера) известно, что система не является определённой $\Leftrightarrow det\begin{pmatrix} A_{1} \ B_{1}\\A_{2} \ B_{2}\end{pmatrix} = 0$. Таким образом, прямые параллельны или совпадают $\Leftrightarrow$ имеют 0 или бесконечно много общих точек $\Leftrightarrow A_{1}B_{2} - A_{2}B_{1} = 0 \Leftrightarrow \begin{pmatrix} A_1 \\ B_1 \end{pmatrix}$ пропорционален $\begin{pmatrix} A_2 \\ B_2 \end{pmatrix}$, ч.т.д. 
	\end{proof}
	\begin{remark}
		Из этого также видно, что прямые совпадают $\Leftrightarrow \frac{A_{1}}{A_{2}} = \frac{B_{1}}{B_{2}} = \frac{C_{1}}{C_{2}}$. 
	\end{remark}
	\subsection{Случай параметрических уравнений}
	\begin{consequense}
		Прямые $l_{1}: \begin{cases}x = x_{1} + a_{1}t \\ y = y_{1} + b_{1}t\end{cases}; l_{2}: \begin{cases}x = x_{2} + a_{2}t \\ y = y_{2} + b_{2}t\end{cases}$ пересекаются $\Leftrightarrow \frac{a_{1}}{a_{2}} \neq \frac{b_{1}}{b_{2}}$.
		Условие совпадения прямых также можно записать через параметрические уравнения (вектор $(x_{2} - x_{1}, y_{2} - y_{1}) = \lambda(a, b)$). \\
		Из этого также следует, что через две различные точки проходит ровно одна прямая (все такие прямые совпадают).
	\end{consequense}
	\section{Пучки прямых}
	\subsection{Определение пучка прямых}
	\begin{definition}
		Собственным пучком прямых называется множество всех прямых, проходящих через данную точку, называемую центром пучка.\\
		Несобственным пучком прямых называется множество всех прямых, параллельных данной прямой.
	\end{definition}
	\subsection{Уравнение собственного пучка прямых}
	\begin{theorem}
		Пусть прямые $l_{1}: A_{1}x + B_{1}y + C_{1} = 0$ и $l_{2}: A_{2}x + B_{2}y + C_{2} = 0$ задают собственный пучок (т.е. содержатся в нём и не совпадают). Тогда прямая $l$ принадлежит пучку $\Leftrightarrow l$ задаётся уравнением $\lambda(A_{1}x + B_{1}y + C_{1}) + \mu(A_{2}x + B_{2}y + C_{2}) = 0 \ (*)$ для некоторых $\lambda, \mu \in \mathbb{R}$.
	\end{theorem}
	\begin{proof}
		$\\\Leftarrow$ Пусть $l$ задаётся уравнением $(*)$. Тогда, подставив в уравнение $l$ центр пучка ($x_{0}, y_{0}$), получим $\lambda(0) + \mu(0) = 0$ (т.к. центр удовлетворяет уравнениям $l_{1}, l_{2}$).\\
		$\Rightarrow$ Пусть $(x_{0}, y_{0}) \in l$. Возьмём произвольную точку $(x_{1}, y_{1}) \in l, (x_{1}, y_{1}) \neq (x_{0}, y_{0})$. Рассмотрим прямую вида $(*)$ с $\lambda = -(A_{2}x_{1} + B_{2}y_{1} + C_{2}), \ \mu = (A_{1}x_{1} + B_{1}y_{1} + C_{1}) : -(A_{2}x_{1} + B_{2}y_{1} + C_{2})(A_{1}x + B_{1}y + C_{1}) + (A_{1}x_{1} + B_{1}y_{1} + C_{1})(A_{2}x + B_{2}y + C_{2}) = 0$. Заметим, что это уравнение действительно задаёт прямую: в противном случае необходимы условия $\lambda A_{1} + \mu A_{2} = \lambda B_{1} + \mu B_{2} = 0$, но тогда $(A_{1}, B_{1})$ и $(A_{2}, B_{2})$ пропорциональны, а исходные прямые непараллельны. Такой прямой, очевидно, принадлежат точки $(x_{0}, y_{0})$ и $(x_{1}, y_{1})$. Так как через две различные точки проходит ровно одна прямая, любая прямая из собственного пучка имеет вид ($*$), ч.т.д.
	\end{proof}
	\subsection{Уравнение несобственного пучка прямых}
	\begin{theorem}
		Пусть прямые $l_{1}: A_{1}x + B_{1}y + C_{1} = 0$ и $l_{2}: A_{2}x + B_{2}y + C_{2} = 0$ задают несобственный пучок (т.е. содержатся в нём и не совпадают). Тогда прямая $l$ принадлежит пучку $\Leftrightarrow l$ задаётся уравнением $\lambda(A_{1}x + B_{1}y + C_{1}) + \mu(A_{2}x + B_{2}y + C_{2}) = 0 \ (*)$ для некоторых $\lambda, \mu \in \mathbb{R}$.
	\end{theorem}
	\begin{proof}
		$\\\Leftarrow$ Так как $l_{1} \parallel l_{2}$, $\frac{A_{1}}{B_{1}} = \frac{A_{2}}{B_{2}}$. Тогда если $l$ имеет вид ($*$), то $\frac{\lambda A_{1}+\mu A_{2}}{A_{1}} = \lambda + \frac{\mu A_{2}}{A_{1}} = \lambda + \frac{\mu B_{2}}{B_{1}} = \frac{\lambda B_{1}+\mu B_{2}}{B_{1}} \Rightarrow l \parallel l_{1}$.\\
		$\Rightarrow$ Пусть $l$ принадлежит пучку. Так как направляющие векторы $l, l_{1}$ и $l_{2}$ пропорциональны, можем домножить уравнения на числа так, что коэффициенты перед переменными станут равны: пусть $l_{1}: Ax + By + C_{1} = 0; \ l_{2} = Ax + By + C_{2} = 0; \ l = Ax + By + C_{3} = 0$. \\Тогда возьмём $\lambda, \mu$ из следующей системы (получена из равенства коэффициентов $\lambda l_1 + \mu l_2$ и $l$): $\begin{cases}C_{1}\lambda + C_{2}\mu = C_{3}\\\lambda + \mu = 1\end{cases} \Leftrightarrow \begin{cases}\lambda = \frac{C_{3}-C_{2}}{C_{1}-C_{2}}\\\mu = \frac{C_{1} - C_{3}}{C_{1} - C_{2}}\end{cases} (C_{1} \neq C_{2}$, иначе $l_{1}$ и $l_{2}$ совпадают). Очевидно, что для таких $\lambda, \mu$ уравнение $l$ имеет вид ($*$) (проверяется несложной подстановкой), ч.т.д.
	\end{proof}
	\section{Отрезки}
	\subsection{Отрезки на плоскости}
	\begin{definition}
		Пусть $l$ - прямая, $X_{1}(x_{1}, y_{1}), X_{2}(x_{2}, y_{2}) \in l$ и $X_{1} \neq X_{2}$. Отрезком с концами $X_{1}, X_{2}$ на плоскости называется множество всех точек, лежащих между $X_{1}$ и $X_{2}$ (на прямой $l$). Обозначается $[X_{1}, X_{2}]$. 
	\end{definition}
	\begin{formula}[Уравнение отрезка]
		Пусть $X \in [X_{1}, X_{2}]$. Тогда знаем, что $\begin{cases}x = x_{1} + t(x_{2} - x_{1}) \\ y = y_{1} + t(y_{2} - y_{1})\end{cases} (X \in l) \Rightarrow \begin{pmatrix} x-x_{1} \\ y-y_{1} \end{pmatrix} = t\begin{pmatrix} x_{2}-x_{1} \\ y_{2}-y_{1} \end{pmatrix} \Rightarrow \\ t\overrightarrow{X_{1}X_{2}} = \overrightarrow{X_{1}X} \Rightarrow \overrightarrow{XX_{2}} = (1-t)\overrightarrow{X_{1}X_{2}}$. Отсюда видно, что $|\overrightarrow{X_{1}X}|$ и $|\overrightarrow{XX_{2}}| < |\overrightarrow{X_{1}X_{2}}| \Leftrightarrow t \in [0, 1]$. Отсюда \begin{boxedalign*}X \in [X_{1}, X_{2}] \Leftrightarrow \begin{cases}x = x_{1} + t(x_{2} - x_{1}) \\ y = y_{1} + t(y_{2} - y_{1})\\t\in [0, 1]\end{cases}\end{boxedalign*}
	\end{formula}
	\begin{definition}
		Отрезком в произвольном аффинном пространстве называется множество точек $[X_{1}, X_{2}] = \{X_{1} + t\overrightarrow{X_{1}X_{2}}, t\in [0, 1]\}$
	\end{definition}
	\section{Полуплоскости}
	\subsection{Выпуклые множества}
	\begin{definition}
		Множество $X$ в произвольном аффинном пространстве называется выпуклым, если $\forall \ X_{1},X_{2} \in X \ \ [X_{1}, X_{2}] \subset X$. 
	\end{definition}
	\subsection{Полуплоскости как выпуклые множества}
	\begin{definition}
		Пусть в аффинной системе координат $l: Ax + By + C = 0$. Множества $\Pi_{0}^{+} = \{X(x, y): Ax + By + C \geqslant 0\}$ и $\Pi_{0}^{-} = \{X(x, y): Ax + By + C \leqslant 0\}$ называются замкнутыми полуплоскостями, а множества $\Pi^{+} = \{X(x, y): Ax + By + C > 0\}$ и $\Pi^{-} = \{X(x, y): Ax + By + C < 0\}$ - открытыми полуплоскостями.
	\end{definition}
	\begin{theorem}
		Для любой прямой $l: Ax + By + C = 0$ множества $\Pi_{0}^{+}, \Pi_{0}^{-}, \Pi^{+}, \Pi^{-}$ выпуклы.
	\end{theorem}
	\begin{proof}
		Рассмотрим $\Pi_{0}^{+}$ (остальные аналогично).\\
		Пусть $X_{1}(x_{1}, y_{1}), X_{2}(x_{2}, y_{2}) \in \Pi_{0}^{+}$. Знаем, что любая точка $X \in [X_{1}, X_{2}]$ имеет координаты $(tx_{1}+(1-t)x_{2}, ty_{1}+(1-t)y_{2}), 0 \leqslant t \leqslant 1$. Тогда:\\
		$\begin{cases}Ax_{1} + By_{1} + C \geqslant 0 \\ Ax_{2} + By_{2} + C \geqslant 0\end{cases} \Rightarrow \begin{cases}tAx_{1} + tBy_{1} + tC \geqslant 0 \\ (1-t)Ax_{2} + (1-t)By_{2} + (1-t)C \geqslant 0\end{cases} \Rightarrow \\ \Rightarrow A(tx_{1}+(1-t)x_{2}) + B(ty_{1}+(1-t)y_{2}) + C \geqslant 0 \Rightarrow X \in \Pi_{0}^{+}$, ч.т.д.
	\end{proof}
	\section{Углы между прямыми}
	\subsection{Определение угла}
	\begin{definition}
		Пусть $l$ - прямая, $O \in l, \bar{v}$ - любой направляющий вектор $l$. Множества $l^{+} = \{O + \lambda\bar{v}, \lambda \geqslant 0\}$ и $l^{-} = \{O + \lambda\bar{v}, \lambda \leqslant 0\}$ называются лучами, на которые $O$ делит $l$. 
	\end{definition}
	\begin{remark}
		Если даны два луча с общим началом (обозначим их $l_{1}^{+}, l_{2}^{+}$), то они являются подмножествами однозначно определённых прямых $l_{1}, l_{2}$, для которых можно выбрать направляющие векторы так, чтобы лучи соответствовали формуле из определения (назовём эти векторы $\bar{v}_{1}, \bar{v}_{2}$).
	\end{remark}
	\begin{definition}
		Углом на плоскости называется объединение двух лучей с общим началом. Величиной угла $l_{1}^{+}\cup l_{2}^{+}$ называется величина угла между векторами $\bar{v}_{1}, \bar{v}_{2}$, определёнными как в замечании. Говорят, что один угол меньше другого, если величина первого угла меньше величины второго.
	\end{definition}
	\begin{definition}
		Угол называется прямым, если его величина равна $\frac{\pi}{2} (\Leftrightarrow$ направляющие векторы лучей ортогональны). Угол называется развёрнутым, если его величина равна $\pi (\Leftrightarrow$ образующие угол лучи дополняют друг друга до прямой).
	\end{definition}
	\begin{remark}
		В дальнейшем будем иногда называть углом величину угла. Также аналогичным образом можно говорить о величине угла между вектором и прямой
	\end{remark}
	\subsection{Определение угла между двумя прямыми}
	\begin{definition}
		Углом между двумя прямыми называется наименьший из углов, образованных лучами с началом в их точке пересечения этих прямых и лежащих на этих прямых, если прямые пересекаются. Если прямые параллельны или совпадают, то угол между ними равен нулю. Обозначается $\angle({l_{1}}{l_{2}})$.
	\end{definition}
	\begin{definition}
		Прямые называются перпендикулярными, если угол между ними равен $\frac{\pi}{2}$.
	\end{definition}
	\begin{definition}
		Перпендикуляром, опущенным из данной точки на данную прямую, называется прямая, проходящая через точку и перпендикулярная прямой, либо отрезок этой прямой с концами в данной точке и точке пересечения прямой с данной. 
	\end{definition}
	\subsection{Угол между прямыми в прямоугольной системе координат}
	\begin{formula}[Угол между прямыми в прямоугольной с.к.]
		Пусть $l_{1}: A_{1}x + B_{1}y + C_{1} = 0, l_{2}: A_{2}x + B_{2}y + C_{2} = 0$ - прямые в прямоугольной системе координат. Тогда их направляющие векторы равны $\begin{pmatrix} -B_{1} \\ A_{1} \end{pmatrix} = \bar{v}_{1}, \begin{pmatrix} -B_{2} \\ A_{2} \end{pmatrix} = \bar{v}_{2}$. Отсюда (выражение скалярного произведения):$\ \cos\angle(l_{1}, l_{2}) = |\cos\angle(\bar{v}_{1}, \bar{v}_{2})| \Rightarrow$ \begin{boxedalign*} \cos\angle(l_{1}, l_{2}) = \frac{|A_{1}A_{2} + B_{1}B_{2}|}{\sqrt{A_{1}^2 + B_{1}^2}\cdot\sqrt{A_{2}^2 + B_{2}^2}} \end{boxedalign*}(модуль в косинусе, а соответственно и в числителе формулы, позволяет сразу взять меньший угол).
	\end{formula}
	\subsection{Условие перпендикулярности. Нормаль}
	\begin{consequense}
		Прямые $l_{1}: A_{1}x + B_{1}y + C_{1} = 0$ и $l_{2}: A_{2}x + B_{2}y + C_{2} = 0$ перпендикулярны $\Leftrightarrow A_{1}A_{2} + B_{1}B_{2} = 0$.
	\end{consequense}
	\begin{definition}
		Нетрудно проверить, что в прямоугольной системе координат вектор $\bar{n} = \begin{pmatrix} A \\ B \end{pmatrix}$ перпендикулярен вектору $\begin{pmatrix} -B \\ A \end{pmatrix}$, а значит и прямой $l: Ax + By + C = 0$. Вектор $\bar{n}$ называется нормалью (нормальным вектором) прямой $l$ (в прямоугольной с.к.)
	\end{definition}
	\begin{remark}
		Любой вектор, коллинеарный нормали, также является нормалью, так как уравнения $Ax + By + C = 0$ и $\lambda Ax + \lambda By + \lambda C = 0$ задают одну и ту же прямую.
	\end{remark}
	\section{Расстояние от точки до прямой}
	\subsection{Определение расстояния между множествами точек}
	\begin{definition}
		Пусть $A, B$  - множества в точечно-евклидовом пространстве. Расстоянием от $A$ до $B$ называется число $\inf\{|XY|, X\in A, Y\in B\}$ Расстояние от точки до прямой определяется аналогично. когда $A = \{X\}$ (его часто обозначают $d(X, l)$).
	\end{definition}
	\begin{remark}
		У множества из определения существует нижняя грань, т.к. оно является ограниченным снизу подмножеством действительных чисел (принцип полноты Вейерштрасса из курса математического анализа). 
	\end{remark}
	\subsection{Расстояние от точки до прямой}
	\begin{theorem}
		Расстояние от точки до прямой равно длине перпендикуляра, опущенного из этой точки на прямую.
	\end{theorem}
	\begin{proof}
		Пусть заданы $l: Ax + By + C = 0$ и $X_{0}$ - произвольные прямая и точка. Выберем на $l$ произвольную точку $X_{1}$. Проведём через $X_{0}$ прямую $l'$ с направляющим вектором $\begin{pmatrix} A \\ B \end{pmatrix}$  - она будет перпендикулярна $l$, а значит имеет с ней единственную общую точку - назовём её $X_{2}$. Имеем: $|\overrightarrow{X_{1}X_{0}}|^2 = (\overrightarrow{X_{1}X_{0}}, \overrightarrow{X_{1}X_{0}}) = (\overrightarrow{X_{1}X_{2}} + \overrightarrow{X_{2}X_{0}}, \overrightarrow{X_{1}X_{2}} + \overrightarrow{X_{2}X_{0}}) = |\overrightarrow{X_{1}X_{2}}|^2 + |\overrightarrow{X_{2}X_{0}}|^2 \geqslant |\overrightarrow{X_{2}X_{0}}|^2$. Отсюда $|\overrightarrow{X_{2}X_{0}}|^2$ - минимум всех расстояний между $X$ и точкой прямой, т.е. он достигается в точке пересечения $l$ и $l'$, ч.т.д.
	\end{proof}
	\begin{remark}
		Расстояние между двумя прямыми на плоскости $\neq 0 \Leftrightarrow$ они параллельны и не совпадают (у них нет общих точек). В этом случае расстояние между ними равно расстоянию от любой точки одной прямой до другой. 
	\end{remark}
	\begin{formula}[Расстояние от точки до прямой в прямоугольной с.к.]
		Пусть $X_{0} = (x_{0}, y_{0}), l: Ax + By + C = 0$. Посчитаем $d(X_{0}, l)$. Проведём через $X_{0}$ перпендикуляр $l'$ к прямой $l$. Направляющий вектор $l'$ - $\begin{pmatrix} A \\ B \end{pmatrix}$, то есть его параметрические уравнения - $\begin{cases} x = x_{0} + At \\ y = y_{0} + Bt \end{cases}$. Найдём $t_{1}$, удовлетворяющее точке пересечения: имеем $A(x_{0} + At_{1}) + B(y_{0} + Bt_{1}) + C = 0 \Rightarrow t_{1} = -\frac{Ax_{0} + By_{0} + C}{A^2+B^2}$.\\
		Отсюда точка пересечения $X_{1}$ имеет координаты $(x_{0} + At_{1}, y_{0} + Bt_{1})$. Тогда $\overrightarrow{X_{0}X_{1}} = \begin{pmatrix} At_{1} \\ Bt_{1} \end{pmatrix} \Rightarrow |\overrightarrow{X_{0}X_{1}}|^2 = A^2t_{1}^2 + B^2t_{1}^2 = (A^2 + B^2)t_{1}^2 = \frac{(Ax_{0} + By_{0} + C)^2}{A^2 + B^2}$, а значит 
		\begin{boxedalign*} d(X, l) = \frac{|Ax_{0} + By_{0} + C|}{\sqrt{A^2 + B^2}} \end{boxedalign*}
	\end{formula}
	\section{Преобразования координат векторов. Матрица перехода}
	\begin{formula}[Матрица перехода]
		Пусть $E = (\bar{e}_{1},...,\bar{e}_{n})$ и $E' = (\bar{e}_{1}',...,\bar{e}_{n}')$ - два базиса в одном и том же евклидовом пространстве $V$. Будем называть $E$ старым базисом, а $E'$ - новым. Получим формулу для координат в старом базисе вектора $\bar{x}\in V$, заданного в новом базисе координатами $(x_{1}',...,x_{n}')$.\\
		Итак, пусть в старом базисе $\bar{x}$ имеет координаты $(x_{1},...,x_{n})$. Установим связь между базисами, выразив векторы нового базиса через старый: пусть \\ $\bar{e}_{i}' = c_{1i}\bar{e}_{1} + ... + c_{ni}\bar{e}_{n}$, т.е. $\bar{e}_{i}' = \begin{pmatrix} c_{1i} \\ \vdots \\ c_{ni} \end{pmatrix}$. Подставляя эти выражения, получим:\\ $\bar{x} = x_{1}'\bar{e}_{1}' + ... + x_{n}'\bar{e}_{n}' = x_{1}'(c_{11}\bar{e}_{1} +...+ c_{n1}\bar{e}_{n}) +...+ x_{n}'(c_{1n}\bar{e}_{1} +...+ c_{nn}\bar{e}_{n}) = \\=(c_{11}x_{1}'+...+c_{1n}x_{n}')\bar{e}_{1} +...+(c_{n1}x_{1}'+...+c_{nn}x_{n}')\bar{e}_{n} \Rightarrow \ $(из равенства координат)  \begin{boxedalign*}\begin{pmatrix} x_{1} \\ \vdots \\ x_{n} \end{pmatrix} = \begin{pmatrix} c_{11}&\dots&c_{1n} \\ \vdots&\null&\vdots \\ c_{n1} & ... & c_{nn} \end{pmatrix} \begin{pmatrix} x_{1}' \\ \vdots \\ x_{n}' \end{pmatrix} \Leftrightarrow \begin{pmatrix} x_{1} \\ \vdots \\ x_{n} \end{pmatrix} = C\begin{pmatrix} x_{1}' \\ \vdots \\ x_{n}' \end{pmatrix}\end{boxedalign*} 
	\end{formula}
	\begin{definition}
		Такая матрица $C$ называется матрицей перехода от базиса $E$ к базису $E'$.
	\end{definition}
	\begin{remark}
		Столбцы матрицы $C$ являются координатами базисных векторов $E'$ в базисе $E \Rightarrow$ столбцы $C$ линейно независимы $\Rightarrow C$ невырожденная (из курса алгебры).
	\end{remark}
	\section{Матрица Грама. Формула скалярного произведения.}
	\begin{formula}
		Рассмотрим также скалярное произведение векторов в случае, когда базис $E$ ортонормированный. Если $\bar{x} = \begin{pmatrix} x_{1} \\ \vdots \\ x_{n}  \end{pmatrix}, \bar{y} = \begin{pmatrix} y_{1} \\ \vdots \\ y_{n}  \end{pmatrix}$ (в базисе $E$), имеем: $(\bar{x}, \bar{y}) = \begin{pmatrix} x_{1}&\dots&x_{n} \end{pmatrix} \begin{pmatrix} y_{1} \\ \vdots \\ y_{n} \end{pmatrix} = \begin{pmatrix} x_{1}'&\dots&x_{n}' \end{pmatrix}C^{T}C\begin{pmatrix} y_{1}' \\ \vdots \\ y_{n}' \end{pmatrix}$ (так как $\begin{pmatrix} x_{1}&\dots&x_{n}\end{pmatrix} = \begin{pmatrix} C \begin{pmatrix} x_{1}' \\ \vdots \\x_{n}' \end{pmatrix}\end{pmatrix}^{T}$). Нетрудно видеть, что произведение матриц $C^{T}C$ имеет вид $G = \begin{pmatrix} (\bar{e}_{1}', \bar{e}_{1}')&\dots&(\bar{e}_{1}', \bar{e}_{n}') \\ \vdots&\null&\vdots \\ (\bar{e}_{n}', \bar{e}_{1}') & \dots & (\bar{e}_{n}', \bar{e}_{n}') \end{pmatrix}$ (строки $C^{T}$ и столбцы $C$ - координаты векторов базиса $E'$ в базисе $E$) \\
		Такая матрица называется матрицей Грама (матрицей скалярных произведений) для базиса $E'$. Так как матрица $G$ не зависит от базиса $E$, получаем формулу для скалярного произведения в произвольном базисе: \begin{boxedalign*}(\bar{x}, \bar{y}) = \begin{pmatrix} x_{1}' & \dots & x_{n}' \end{pmatrix}G\begin{pmatrix} y_{1}' \\ \vdots \\ y_{n}' \end{pmatrix} \end{boxedalign*}
	\end{formula}
	\section{Выражения матриц перехода}
	\begin{consequense}[1]
		Если $C$ и $C_{1}$ - матрицы перехода от базиса $E$ к $E'$ и от $E'$ к $E''$ соответственно, то матрица перехода от базиса $E$ к $E''$ равна $CC_{1}$
	\end{consequense}
	\begin{consequense}[2]
		Если $C$ - матрица перехода от базиса $E$ к $E'$, то матрица перехода от базиса $E'$ к $E$ равна $C^{-1}$.
	\end{consequense}
	\section{Критерий матрицы перехода}
	\begin{remark}
		Пусть $\bar{e}_{1},...,\bar{e}_{n}$ - базис векторного пространства $V$. Тогда векторы $\bar{x}_{1},...,\bar{x}_{n} \in V$ линейно независимы $\Leftrightarrow$ столбцы их координат линейно независимы. Это очевидно следует из представления линейной комбинации $\lambda_{1}\bar{x}_{1} + ... + \lambda_{n}\bar{x}_{n} = 0$ через координаты. 
	\end{remark}
	\begin{theorem}
		Для произвольного данного базиса матрица $C$ является матрицей перехода к некоторому другому базису $\Leftrightarrow \det C \neq 0$. 
	\end{theorem}
	\begin{proof}
		Следует из утверждения из курса алгебры о том, что матрица невырожденна $\Leftrightarrow$ её столбцы линейно независимы ($\Leftrightarrow$ векторы-столбцы образуют базис). 
	\end{proof}
	\section{Преобразования координат точек}
	\begin{formula}[Координаты точки при перемене с.к.]
		Пусть заданы два репера $O\bar{e}_{1}...\bar{e}_{n}$ и $O'\bar{e}_{1}'...\bar{e}_{n}'$. Для этого необходимо задать новый репер через старый: пусть $C$ - матрица перехода от $\bar{e}_{1},...,\bar{e}_{n}$ к $\bar{e}_{1}'...\bar{e}_{n}'$, а вектор $\overrightarrow{OO'} = \begin{pmatrix} x_{01} \\ \vdots \\ x_{0n} \end{pmatrix}$.
		Пусть $X$ - произвольная точка с координатами $(x_{1},...,x_{n})$ в старой системе координат и $(x_{1}',...,x_{n}')$ в новой. Тогда $\overrightarrow{OX} = \begin{pmatrix} x_{1} \\ \vdots \\ x_{n} \end{pmatrix}$. Также $\overrightarrow{O'X} = \begin{pmatrix} x_{1}' \\ \vdots \\ x_{n}' \end{pmatrix}$ в новой с.к. $\Rightarrow C\begin{pmatrix} x_{1}' \\ \vdots \\ x_{n}' \end{pmatrix}$ в старой. Осталось заметить, что $\overrightarrow{OX} = \overrightarrow{OO'} + \overrightarrow{O'X}$, а отсюда (через старую с.к.)
		\begin{boxedalign*}\begin{pmatrix} x_{1} \\ \vdots \\ x_{n} \end{pmatrix} = \begin{pmatrix} x_{01} \\ \vdots \\ x_{0n} \end{pmatrix} + C \begin{pmatrix} x_{1}' \\ \vdots \\ x_{n}' \end{pmatrix} \end{boxedalign*}
		Выразим новые координаты, умножив слева на $C^{-1}$:  
		\begin{boxedalign*}\begin{pmatrix} x_{1}' \\ \vdots \\ x_{n}' \end{pmatrix} = - C^{-1}\begin{pmatrix} x_{01} \\ \vdots \\ x_{0n} \end{pmatrix} + C^{-1}\begin{pmatrix} x_{1} \\ \vdots \\ x_{n} \end{pmatrix} \end{boxedalign*}
		(отметим также, что $- C^{-1}\begin{pmatrix} x_{01} \\ \vdots \\ x_{0n} \end{pmatrix}$ - новые координаты вектора $O'O$)
	\end{formula}
	\section{Ортогональные матрицы}
	\subsection{Определение. Критерии ортогональности}
	\begin{definition}
		Матрица перехода от одного ортонормированного базиса к другому называется ортогональной.
	\end{definition}
	\begin{theorem}
		Для $C = \begin{pmatrix} c_{11}&\dots&c_{1n} \\ \vdots&\null&\vdots \\ c_{n1}&\dots&c_{nn} \end{pmatrix}$ следующие утверждения равносильны:
		\begin{enumerate}
			\item $C$ - ортогональная;
			\item $\sum \limits_{k=1}^{n} c_{ki}c_{kj} = \delta_{ij}$ ("скалярное произведение" столбцов равно $\delta_{ij}$);
			\item $\sum \limits_{k=1}^{n} c_{ik}c_{jk} = \delta_{ij}$ ("скалярное произведение" строк равно $\delta_{ij}$);
			\item $C C^T = E$;
			\item $C^T C = E$;
			\item $C^T = C^{-1}$;
		\end{enumerate}
	\end{theorem}
	\begin{proof}(в конспекте указан "прямой подсчёт")\\
		Для начала заметим равносильность утверждений $\circled{4}, \circled{5}, \circled{6}$: $\circled{4} \Leftrightarrow \circled{5}$ (получаются друг из друга транспонированием обоих частей равенства), а $\circled{6}$ равносильно им по определению обратной матрицы.\\
		$\circled{1} \Rightarrow \circled{2}$: Столбцы матрицы C - координаты векторов ортонормированного базиса $E$ в другом ортонормированном базисе $E'$. Рассмотрим скалярное произведение $(\bar{e}_{i}', \bar{e}_{j}')$: с одной стороны оно равно $\sum \limits_{k=1}^{n} c_{ki}c_{kj}$ (так как базис $E$ ортонормирован, можем применять формулу с матрицей Грама, причём $G = E$ из ортонормированности $E_{1}$), а с другой стороны - $\delta_{ij}$ (из ортонормированности $E_{1}$).\\
		$\circled{1} \Leftarrow \circled{2}$: Из $\circled{2}$ знаем, что векторы с координатами в столбцах попарно ортогональны и имеют длину 1 в базисе, в котором записаны эти координаты. Значит, применив такое преобразование к векторам ортонормированного базиса, получим также попарно ортогональные векторы длины 1, т.е. $C$ - ортогональная.\\
		$\circled{2} \Leftrightarrow \circled{5}$: Оба условия равносильны тому, что элемент $C^T C$ на позиции $ij$ равен $\delta_{ij}$.\\
		Аналогично $\circled{3} \Leftrightarrow \circled{4}$.

		Итого $\circled{1} \Leftrightarrow \circled{2} \Leftrightarrow \circled{5} \Leftrightarrow \circled{6} \Leftrightarrow \circled{4} \Leftrightarrow \circled{3}$.
	\end{proof}
	\begin{consequense}
		Для ортогональной $C: |C^T| = |C|, \ |C^T C| = |E| = 1 \Rightarrow |C| = \pm 1$
	\end{consequense}
	\begin{consequense}(Из определения ортогональной матрицы)\\
		Произведение ортогональных матриц - ортогональная матрица.\\
		Матрица, обратная ортогональной, ортогональна.
	\end{consequense}
	\subsection{Двумерный случай}
	\begin{formula}(Двумерный случай ортогональной матрицы)\\
		$C = \begin{pmatrix} a&b\\c&d \end{pmatrix}$ ортогональна $\Rightarrow a^2 + c^2 = 1 \Rightarrow \exists \phi: a = \cos\phi, c = \sin\phi$.\\
		Из теоремы $a^2 + b^2 = 1, \ c^2 + d^2 = 1 \Rightarrow b = \pm\sin\phi, \ d = \pm\cos\phi$.\\
		Из ортогональности столбцов следует, что $ab + cd = 0$, поэтому остаются следующие случаи: \begin{boxedalign*}C = \begin{pmatrix} \cos\phi&-\sin\phi\\\sin\phi&\cos\phi\end{pmatrix}; \ \ C = \begin{pmatrix} \cos\phi&\sin\phi\\\sin\phi&\-cos\phi\end{pmatrix}\end{boxedalign*}
		т.е. для любой ортогональной $C$ найдётся $\phi$ такой, что $C$ имеет один из видов выше. Несложно также убедиться, что любая матрица такого вида ортогональна: первая производит поворот на угол $\phi$, а вторая - поворот с отражением.
	\end{formula}
	\section{Ориентация векторных пространств}
	\begin{definition}
		Два базиса в конечномерном векторном пространстве называются одинаково ориентированными (одноимёнными), если матрица перехода от одного базиса к другому имеет положительный определитель. В противном случае базисы называются противоположно ориентированными (разноимёнными).
	\end{definition}
	\begin{theorem}
		Отношение одноимённости является отношением эквивалентности на множестве базисов.
	\end{theorem}
	\begin{proof}
		По определению отношения эквивалентности:
		\begin{itemize}
			\item рефлексивность: матрица перехода от базиса в себя равна $E$, $|E| = 1$;
			\item симметричность: $C: E \rightarrow E' \Rightarrow C^{-1}: E' \rightarrow E$, причём $|C C^{-1}| = 1$, т.е. $|C|$ и $|C^{-1}|$ одного знака;
			\item транзитивность: $C: E \rightarrow E', \ C': E' \rightarrow E'', C'': E \rightarrow E'' \Rightarrow C'' = CC'$, т.е. из одноимённостей $E$ с $E'$ и $E'$ с $E''$ следует одноимённость $E$ с $E''$.
		\end{itemize}
	\end{proof}
	\begin{definition}
		Ориентацией векторного пространства, а также любого аффинного пространства, с которым оно ассоциировано, называется выбор любого из двух классов эквивалентности по отношению одноимённости и объявления базисов в нём положительными (а остальных - отрицательными).\\ 
		(Достаточно выбрать один базис и объявить его положительным)
	\end{definition}
	\section{Ориентация пар векторов. Углы с учётом ориентации}
	\subsection{Ориентация на плоскости}
	\begin{example} Ориентация на плоскости:
		Достаточно выбрать пару неколлинеарных векторов (базис), и объявить его положительным. Заметим, что пары $\bar{a}, \bar{b}$ и $\bar{b}, \bar{a}$ разноимённы ($C = \begin{pmatrix} 0&1\\1&0\end{pmatrix})$.

		Посмотрим, когда пары $\bar{a}, \bar{b}$ и $\bar{a}, \bar{b}'$ одноимённые. Рассмотрим произвольную прямую $l$ с направляющим вектором $\bar{a}$ и любую точку $O \in l$. В репере $O\bar{a}\bar{b}$ уравнение $l$ имеет вид $y = 0$. Подставив в это уравнение координаты $\bar{b} = \begin{pmatrix} 0\\1\end{pmatrix}$, получим $1$.\\
		Пусть $\bar{b}' = \lambda\bar{a} + \mu\bar{b} \Rightarrow$ матрица перехода между парами $C$ равна $\begin{pmatrix} 1&\lambda\\0&\mu\end{pmatrix}$. $|C| > 0 \Leftrightarrow \mu > 0 \Leftrightarrow$ при подстановке координат $\bar{b}'$ в уравнение $l$ результат будет $> 0 \Leftrightarrow$ точки $B, B'$ с радиус-векторами $\bar{b}, \bar{b}'$ лежат в одной полуплоскости относительно $l$.\\
		Получили \bfseries критерий одноимённости пар $\bar{a}, \bar{b}$ и $\bar{a}, \bar{b}'$: они одноимённы $\Leftrightarrow$ точки, полученные прибавлением векторов $\bar{b}$ и $\bar{b}'$ к произвольной точке произвольной прямой с направляющим вектором $\bar{a}$, лежат относительно неё в одной полуплоскости. \mdseries 
	\end{example}
	\subsection{Угол от вектора до вектора}
	\begin{definition}
		Углом от вектора $\bar{a}$ до вектора $\bar{b}$ в ориентированном двумерном евклидовом пространстве называется угол между $\bar{a}$ и $\bar{b}$, взятый со знаком плюс, если пара $\bar{a}, \bar{b}$ положительно ориентирована, и со знаком минус иначе.
	\end{definition}
	\subsection{Угол от прямой до прямой}
	\begin{definition}
		Положительным углом от $\bar{a}$ до $\bar{b}$ называется угол $\phi$ от $\bar{a}$ до $\bar{b}$, если $\phi > 0$, и угол $2\pi + \phi$, если $\phi < 0$.
	\end{definition}
	\begin{definition}
		Углом от прямой $l_{1}$ до прямой $l_{2}$ в ориентированном двумерном евклидовом пространстве называется наименьший положительный угол от направляющего вектора $l_{1}$ до направляющего вектора $l_{2}$.
	\end{definition}
	\subsection{Угол наклона}
	\begin{formula}(Известное ранее уравнение прямой. Угол наклона)\\
		Пусть на плоскости задана прямоугольная система координат $O\bar{e}_{1}\bar{e}_{2}$ и в ней прямая $l$ определена уравнением $y = kx + b \Leftrightarrow kx - y + b = 0$. Прямая с направляющим вектором $e_{1}$ называется \bfseries осью абсцисс\mdseries, а угол от этой прямой до прямой $l$ - \bfseries углом наклона \mdseries прямой $l$.\\
		Подсчитаем угол наклона $l$. Направляющие векторы $l$ имеют вид $\begin{pmatrix}\lambda\\ \lambda k\end{pmatrix}$, причём $\phi$, очевидно, зависит только от знака $\lambda$ (так как углы при $\lambda > 0$ и  $\lambda < 0$ различаются на $\pi$). Угол от $\begin{pmatrix} 1\\0 \end{pmatrix}$ до $\begin{pmatrix} \pm 1\\\pm k\end{pmatrix}$ равен углу от $\begin{pmatrix} -1\\0 \end{pmatrix}$ до $\begin{pmatrix} \mp 1\\\mp k\end{pmatrix}$, поэтому выбирать наименьший можем только из положительных углов от вектора $\begin{pmatrix} 1\\0 \end{pmatrix}$ - наименьшим, как нетрудно видеть, будет угол до $\begin{pmatrix} 1\\k \end{pmatrix}$ при $k>0$ и до $\begin{pmatrix} -1 \\ -k \end{pmatrix}$ иначе. Обозначив нужный вектор за $\bar{k}$, имеем: 
		\begin{align*}
			\cos\phi = \cos\angle(\bar{e}_{1}, \bar{k}) = \pm \frac{1}{\sqrt{1 + k^2}} \Rightarrow \sin\angle(\bar{e}_{1}, \bar{k}) = \frac{|k|}{\sqrt{1 + k^2}} \Rightarrow \tg\phi = \pm k 
		\end{align*}
		(синус > 0, т.к. угол от 0 до $\pi$).\\
		Заметим, что при $k > 0$ необходимый угол должен иметь тангенс, больший нуля, а при $k < 0$ - меньший нуля(очевидно из расположения соответствующих векторов в координатных четвертях). Отсюда видно, что вне зависимости от нужного нам вектора верно: \begin{boxedalign*} \tg\phi = k\end{boxedalign*}
	\end{formula}
	\section{Площади}
	\subsection{Определение и следствия}
	\begin{definition}
		Фигуры $\Phi_{1}, \Phi_{2}$ называются конгруэнтными, если $\exists$ отображение $f: \Pi \rightarrow \Pi$, сохраняющее расстояния ($|f(A)f(B)| = |AB| \ \forall A, B$) такое, что $f(\Phi_{1}) = \Phi_{2}$.
	\end{definition}
	\begin{definition}
		Площадь(мера) на евклидовой плоскости - численная величина (обозначается $S_{\Phi}, \Phi$ - фигура), соответствующая следующим свойствам:
		\begin{enumerate}
			\item Площадь квадрата со стороной 1 равна 1;
			\item Площади конгруэнтных фигур равны;
			\item Если $\Phi_{1}, \Phi_{2}$ - фигуры, $\exists \ S_{\Phi_{1}}, S_{\Phi_{2}}$ и $\Phi_{1} \cap \Phi_{2} = \varnothing$, то $\exists S_{\Phi_{1}\cup\Phi_{2}} = S_{\Phi_{1}} + S_{\Phi_{2}}$; 
			\item Если $\Phi$ - фигура и существуют такие последовательности фигур $\{\phi_{n}\}_{n=1}^{\infty}$ и $\{\Phi_{n}\}_{n=1}^{\infty}$, что $\phi_{n} \subset \phi_{n+1} \subset \Phi \subset \Phi_{n+1} \subset \Phi{n} \ \forall n \geqslant 1$, причём $\lim \limits_{n\rightarrow\infty}S_{\phi_{n}}$ и $\lim \limits_{n\rightarrow\infty}S_{\Phi_{n}}$ существуют и равны, то $S_{\Phi} = \lim \limits_{n\rightarrow\infty}S_{\phi_{n}}$ = $\lim \limits_{n\rightarrow\infty}S_{\Phi_{n}}$
		\end{enumerate} 
	\end{definition}
	\begin{consequense}(Площади некоторых фигур:)
		\begin{itemize}
			\item Площадь отрезка равна 0.
			\begin{proof}
				Из предельного перехода (Площадь отрезка с длиной 1 равна 0, иначе площадь квадрата со стороной 1 не была бы конечной, а из этого можно получить длину любого отрезка).
			\end{proof}
			\item Площадь квадрата со стороной $a$ равна $a^2$.
			\begin{proof}
				Для $a = \frac{1}{n}$ разбиваем квадрат со стороной 1 на $n^2$ квадратиков, для $a \in \mathbb{Q}$ складываем квадрат со стороной $\frac{m}{n}$ из предыдущих, для $a \notin \mathbb{Q}$ приближаем квадрат сверху и снизу квадратами с рациональными сторонами и переходим к пределам (по 4 пункту определения).
			\end{proof}
			\item Площадь прямоугольника со сторонами $a, b$ равна $ab$.
			\begin{proof}
				Рассмотрим квадрат со стороной $a+b$. Его можно разбить на квадрат со стороной $a$, квадрат со стороной $b$ и два нужных нам прямоугольника. Отсюда $2S = (a+b)^2 - a^2 -b^2 = 2ab \Rightarrow S = ab$.
			\end{proof} 
		\end{itemize}
		Отсюда также выводятся формулы площади параллелограмма (перестановкой треугольника из параллелограмма получается прямоугольник) и треугольника (как половины параллелограмма).
	\end{consequense}
	\subsection{Площадь параллелограмма}
	\begin{definition}
		Говорят, что параллелограмм натянут на векторы $\bar{a}, \bar{b}$, если для одной из вершин $A$ параллелограмма точки $A + \bar{a}$ и $A + \bar{b}$ также являются его вершинами. Площадь параллелограмма, натянутого на векторы $\bar{a}, \bar{b}$, обозначается $S_{\bar{a},\bar{b}}$.
	\end{definition}
	\begin{formula}(Площадь параллелограмма в прямоугольной с.к.)\\
		Нам уже известно, что $S_{\bar{a},\bar{b}} = |\bar{a}||\bar{b}|\sin\phi$ (из выражения высоты параллелограмма через угол). В прямоугольной системе координат:\begin{align*}
			S_{\bar{a},\bar{b}}^2 = |\bar{a}|^2|\bar{b}|^2\sin^2\phi = |\bar{a}|^2|\bar{b}|^2(1 - \cos^2\phi) = |\bar{a}|^2|\bar{b}|^2 - (\bar{a}, \bar{b})^2
		\end{align*}
		Если $\bar{a} = \begin{pmatrix} x_1 \\ y_1 \end{pmatrix}, \ \bar{b} = \begin{pmatrix} x_2 \\ y_2 \end{pmatrix}$, то $S_{\bar{a},\bar{b}}^2 = (x_1^2 + y_1^2)(x_2^2 + y_2^2) - (x_1x_2 + y_1y_2) = (x_1y_2 - x_2y_1)^2 \Rightarrow$ \begin{boxedalign*}S_{\bar{a},\bar{b}} = |\begin{vmatrix} x_1&y_1\\x_2&y_2 \end{vmatrix}|\end{boxedalign*}
		При этом определитель $>0 \Leftrightarrow \bar{a}, \bar{b}$ одноимённа с базисом.
	\end{formula}
	\subsection{Ориентированная площадь}
	\begin{definition}
		\ Ориентированной площадью параллелограмма, натянутого на $\bar{a},\bar{b}$, на ориентированной плоскости называется число, равное $S_{\bar{a},\bar{b}}$, если пара $\bar{a},\bar{b}$ положительно ориентирована, и $-S_{\bar{a},\bar{b}}$ иначе. Обозначается $\langle\bar{a}, \bar{b}\rangle$.
	\end{definition}
	\begin{consequense}
		В любом положительно ориентированном базисе $\langle\bar{a}, \bar{b}\rangle = \begin{vmatrix} x_1&y_1\\x_2&y_2 \end{vmatrix}$
	\end{consequense}
	\begin{subtheorem} Свойства ориентированной площади:
		\begin{enumerate}
			\item $\langle\bar{a}, \bar{b}\rangle = -\langle\bar{b}, \bar{a}\rangle$
			\item $\langle\bar{\lambda a}, \bar{b}\rangle = \lambda\langle\bar{a}, \bar{b}\rangle = \langle\bar{a}, \bar{\lambda b}\rangle$
			\item $\langle\bar{a} + \bar{b}, \bar{c}\rangle = \langle\bar{a}, \bar{c}\rangle + \langle\bar{b}, \bar{c}\rangle$\\
			($\langle\bar{a}, \bar{b} + \bar{c}\rangle = \langle\bar{a}, \bar{b}\rangle + \langle\bar{a}, \bar{c}\rangle; \langle\bar{a} - \bar{b}, \bar{c}\rangle = \langle\bar{a}, \bar{c}\rangle-\langle\bar{b}, \bar{c}\rangle$)
		\end{enumerate}
	\end{subtheorem}
	\begin{proof}
		Следует из свойств определителя.
	\end{proof}
	\section{Плоскость в пространстве}
	\begin{definition}
		Плоскость в (трёхмерном) аффинном пространстве - его двумерное аффинное подпространство.
	\end{definition}
	\begin{formula}[Уравнения плоскости]
		Плоскость $\pi$ - это множество $X_0 + V^2$, где $X_0$ - точка и $V^2$ - двумерное векторное подпространство пространства $V$, ассоциированного с трёхмерным аффинным пространством.
		Так как $\dim V^2 = 2 \Rightarrow$ в нём есть базис $\bar{a}, \bar{b} \Rightarrow \pi = \{X_0 + u\bar{a} + v\bar{b}: u, v\in\mathbb{R}\}$. Такие векторы $\bar{a}, \bar{b}$ называются \bfseries направляющими векторами плоскости\mdseries.
		В произвольной системе координат: если $X_0 = (x_0, y_0, z_0), \bar{a} = \begin{pmatrix} a_1 \\ a_2 \\ a_3 \end{pmatrix}, \bar{b} = \begin{pmatrix}  b_1 \\ b_2 \\ b_3 \end{pmatrix}$, то $\pi$ - мн-во точек с координатами: (\bfseries параметрические уравнения плоскости\mdseries)
		\begin{boxedalign*}\begin{cases} x = x_0 + ua_1 + vb_1 \\ y = y_0 + ua_2 + vb_2 \\ z = z_0 + ua_3 + vb_3\end{cases}\end{boxedalign*}
		Векторы $\bar{a}, \bar{b}$ называются направляющими векторами плоскости $\pi$.\\
		Выражая параметры $u, v$, получим \bfseries общее уравнение плоскости\mdseries :
		\begin{boxedalign*}Ax + By + Cz + D = 0 \ \ \ (|A| + |B| + |C| \neq 0)\end{boxedalign*}
	\end{formula}
	\section{Взаимное расположение плоскостей}
	\begin{definition}
		Говорят, что вектор $\bar{a}$ параллелен плоскости $\pi$ (обозначается $\bar{a} \parallel \pi$), если он выражается через направляющие векторы этой плоскости ($\Leftrightarrow$ через любой базис ассоциированного с плоскостью векторного пространства).
	\end{definition}
	\begin{remark}
		Ясно, что $X\in \pi \Leftrightarrow \overrightarrow{X_0X} \parallel \pi$ (для любой $X_0 \in \pi$). То же верно и для любой отличной от $X_0$ точки плоскости: $\overrightarrow{X_1X} = \overrightarrow{X_1X_0} + \overrightarrow{X_0X}$, т.е. $\overrightarrow{X_1X}$ выражается через направляющие векторы $\Leftrightarrow \overrightarrow{X_0X}$ выражается через направляющие векторы.
	\end{remark}
	\subsection{Взаимное расположение плоскостей}
	\begin{formula}
		Рассмотрим две плоскости, заданные уравнениями $A_{1}x + B_{1}y + C_{1}z + D_{1} = 0$ и $A_{2}x + B_{2}y + C_{2}z + D_{2} = 0$. Эти плоскости пересекаются $\Leftrightarrow \begin{cases}A_{1}x + B_{1}y + C_{1}z + D_{1} = 0\\A_{2}x + B_{2}y + C_{2}z + D_{2} = 0\end{cases}$ имеет решения. Матрица коэффициентов системы $A = \begin{pmatrix} A_{1}&B_{1}&C_{1}\\A_{2}&B_{2}&C_{2}\end{pmatrix}$, а её расширенная матрица $(A|B)= \begin{pmatrix} A_{1}&B_{1}&C_{1}&\vrule&-D_{1}\\A_{2}&B_{2}&C_{2}&\vrule&-D_{2} \end{pmatrix}$.
		По теореме Кронекера-Капелли из курса алгебры СЛУ совместна $\Leftrightarrow rk A = rk (A|B)$. Так как $1 \leqslant rk A \leqslant 2$, плоскости не пересекаются $\Leftrightarrow$ строки $A$ линейно зависимы ($rk A = 1$), а строки $(A|B)$ - нет. Таким образом, плоскости не пересекаются $\Leftrightarrow$\begin{align*}A_1:A_2 = B_1:B_2 = C_1:C_2 \neq D_1:D_2 \end{align*}
		Очевидно, плоскости совпадают $\Leftarrow$\begin{align*}A_1:A_2 = B_1:B_2 = C_1:C_2 = D_1:D_2\end{align*}
		Докажем, что плоскости совпадают только в этом случае. Пусть плоскости совпадают. Приведём матрицу $(A|B)$ к ступенчатому виду: $\begin{pmatrix} A_{1}&B_{1}&C_{1}&\vrule&-D_{1}\\0&a_{22}&a_{23}&\vrule&a_{24} \end{pmatrix}$ (хотя бы один из коэффициентов первой плоскости $\neq 0$, без ограничения общности - $A_1$). Так как плоскости совпадают, любая точка, принадлежащая первой плоскости, принадлежит и второй. Тогда для $a_{24}$ из принадлежности плоскости $(-\frac{D_1}{A_1}, 0, 0)$ следует $a_{24} = 0$, для $a_{22}$ аналогично при $B \neq 0$, а при $B = 0$ - из принадлежности точки $(-\frac{D_1}{A_1}, 1, 0)$ (для $a_{23}$ и $C$ аналогично). Отсюда вторая строка ступенчатой матрицы нулевая, то есть строки $(A|B)$ пропорциональны, а отсюда плоскости совпадают $\Leftrightarrow$ \begin{align*}A_1:A_2 = B_1:B_2 = C_1:C_2 = D_1:D_2\end{align*}
		Предположим, что плоскости пересекаются, но не совпадают. Тогда $rk A = 2$. Из курса алгебры знаем, что решение такой системы имеет вид $X_0 + V^1$ (частное решение СЛУ + одномерное пространство решений ОСЛУ).
		Таким образом:
		\begin{center}
			Плоскости совпадают $\Leftrightarrow$ их уравнения (со свободными коэффициентами) пропорциональны;

			Плоскости параллельны $\Leftrightarrow$ их уравнения пропорциональны, а свободные члены - нет;

			Плоскости пересекаются и не совпадают (их уравнения не пропорциональны) $\Leftrightarrow$ пересечением плоскостей является множество вида $X_0 + V^1$, где $X_0$ - точка и $V^1 = \{\lambda \bar{a}: \lambda \in \mathbb{R}\}$ ($\bar{a}$ - решение ОСЛУ $AX=0$), т.е. их пересечением является прямая.
		\end{center}
	\end{formula}
	\section{Пучки плоскостей}
	\subsection{Определения}
	\begin{definition}
		Собственным пучком плоскостей называется множество всех плоскостей, проходящих через данную прямую, называемую центром пучка.\\
		Несобственным пучком плоскостей называется множество всех плоскостей, параллельных данной плоскости.
	\end{definition}
	\subsection{Собственные пучки}
	\begin{theorem}
		Пусть плоскости $\pi_{1}: A_{1}x + B_{1}y + C_{1}z + D_1 = 0$ и $\pi_{2}: A_{2}x + B_{2}y + C_{2}z + D_2 = 0$ задают собственный пучок (т.е. содержатся в нём и не совпадают). Тогда плоскость $\pi$ принадлежит пучку $\Leftrightarrow \pi$ задаётся уравнением $\lambda(A_{1}x + B_{1}y + C_{1}z + D_1) + \mu(A_{2}x + B_{2}y + C_{2}z + D_2) = 0 \ (*)$ для некоторых $\lambda, \mu \in \mathbb{R}$.
	\end{theorem}
	\begin{proof}
		$\\\Leftarrow$ Пусть $\pi$ задаётся уравнением $(*)$. Тогда, подставив в уравнение $\pi$ любую точку прямой - центра пучка (назовём её $l$), получим $\lambda(0) + \mu(0) = 0$ (т.к. центр удовлетворяет уравнениям $\pi_{1}, \pi_{2}$).\\
		$\Rightarrow$ Пусть $l \subset \pi$. Возьмём произвольную точку $(x_{1}, y_{1}, z_1) \in \pi, (x_{1}, y_{1}, z_1) \notin l$. Рассмотрим плоскость вида $(*)$ с $\lambda = -(A_{2}x_{1} + B_{2}y_{1} + C_{2}z_1 + D_2), \ \mu = (A_{1}x_{1} + B_{1}y_{1} + C_{1}z_1 + D_1) : -(A_{2}x_{1} + B_{2}y_{1} + C_{2}z_{1} + D_2)(A_{1}x + B_{1}y + C_{1}z + D_1) + (A_{1}x_{1} + B_{1}y_{1} + C_{1}z_1 + D_1)(A_{2}x + B_{2}y + C_{2}z + D_2) = 0$. Заметим, что это уравнение действительно задаёт плоскость: в противном случае необходимы условия $\lambda A_{1} + \mu A_{2} = \lambda B_{1} + \mu B_{2} = \lambda C_{1} + \mu C_{2} = 0$, но тогда $(A_{1}, B_{1}, C_1)$ и $(A_{2}, B_{2}, C_2)$ пропорциональны, а исходные плоскости непараллельны. Такой плоскости, очевидно, принадлежат точки $l$ и $(x_{1}, y_{1}, z_1)$. Так как через прямую и не лежащую на ней точку проходит ровно одна плоскость (она имеет вид $\{X_1 + \bar{v} = X_1 + \lambda\overrightarrow{X_1X_2} + \mu\overrightarrow{X_1X_3}\}$, где $X_2, X_3$ - произвольные точки на $l$), любая плоскость из собственного пучка имеет вид ($*$), ч.т.д.
	\end{proof}
	\subsection{Несобственные пучки}
	\begin{theorem}
		Пусть плоскости $\pi_{1}: A_{1}x + B_{1}y + C_{1}z + D_1 = 0$ и $\pi_{2}: A_{2}x + B_{2}y + C_{2}z + D_2 = 0$ задают несобственный пучок (т.е. содержатся в нём и не совпадают). Тогда плоскость $\pi$ принадлежит пучку $\Leftrightarrow \pi$ задаётся уравнением $\lambda(A_{1}x + B_{1}y + C_{1}z + D_1) + \mu(A_{2}x + B_{2}y + C_{2}z + D_2) = 0 \ (*)$ для некоторых $\lambda, \mu \in \mathbb{R}$.
	\end{theorem}
	\begin{proof}
		$\\\Leftarrow$ Так как $\pi_{1} \parallel \pi_{2}$, $\frac{A_{1}}{A_{2}} = \frac{B_{1}}{B_{2}} = \frac{C_1}{C_2}$. Тогда если $\pi$ имеет вид ($*$), то $\frac{\lambda A_{1}+\mu A_{2}}{A_{1}} = \lambda + \frac{\mu A_{2}}{A_{1}} = \lambda + \frac{\mu B_{2}}{B_{1}} = \frac{\lambda B_{1} +\mu B_{2}}{B_{1}} = \lambda + \frac{\mu B_{2}}{B_{1}} = \lambda + \frac{\mu C_{2}}{C_{1}} = \frac{\lambda C_{1} +\mu C_{2}}{C_{1}} \Rightarrow \pi \parallel \pi_{1}$.\\
		$\Rightarrow$ Пусть $\pi$ принадлежит пучку. Так как уравнения $\pi, \pi_{1}$ и $\pi_{2}$; пропорциональны (без свободных коэффициентов), можем домножить их на числа так, что коэффициенты перед переменными станут равны: пусть $\pi_{1}: Ax + By + Cz + D_1' = 0; \ \pi_{2} = Ax + By + Cz + D_2' = 0; \ \pi = Ax + By + Cz + D_3' = 0$.Тогда возьмём $\lambda, \mu$ из следующей системы: $\begin{cases}D_{1}'\lambda + D_{2}'\mu = D_{3}'\\\lambda + \mu = 1\end{cases} \Leftrightarrow \begin{cases}\lambda = \frac{D_{3}'-D_{2}'}{D_{1}'-D_{2}'}\\\mu = \frac{D_{1}' - D_{3}'}{D_{1}' - D_{2}'}\end{cases} (D_{1}' \neq D_{2}'$, иначе $\pi_{1}$ и $\pi_{2}$ совпадают). Очевидно, что для таких $\lambda, \mu$ уравнение $\pi$ имеет вид ($*$) (проверяется несложной подстановкой), ч.т.д.
	\end{proof}
	\section{Полупространства}
	\begin{definition}
		Аналогично прямой на плоскости, каждая плоскость $\pi$  в пространстве $V^3$ разбивает множество всех не принадлежащих ей точек пространства на два выпуклых подмножества $V_1, V_2: V_1 \cup \pi \cup V_2 = V^3, V_1 \cap V_2 = V_1 \cap \pi = V_2 \cap \pi = \varnothing$. Такие подмножества называются полупространствами, ограниченными $\pi$ и определяются однозначно с точностью до обозначения.
	\end{definition}
	Также можем определить полупространства следующим образом: возьмём произвольную точку $X \notin \pi$ и скажем, что $V_1 = \{Y \in V^3: [X, Y]\cap\pi = \varnothing\}$, а затем выберем $X' \notin \pi \cup V_1$ (такая точка всегда будет существовать) и определим $V_2 = \{Y' \in V^3: [X', Y']\cap\pi = \varnothing\}$.
	\begin{subtheorem}
		Так определённые множества являются полупространствами и не зависят от выбора точек $X, X'$.
	\end{subtheorem}
	\begin{proof}
		(Аналогично случаю прямой) Введём любую систему координат. Тогда полупространства $V^{\pm} = \{X(x, y, z): Ax + By + Cz + D \gtrless 0\}$. Рассмотрим $V^{+}$ (остальные аналогично).\\
		Пусть $X_{1}(x_{1}, y_{1}, z_1), X_{2}(x_{2}, y_{2}, z_2) \in V^{+}$. Знаем, что любая точка $X \in [X_{1}, X_{2}]$ имеет координаты $(tx_{1}+(1-t)x_{2}, ty_{1}+(1-t)y_{2}, tz_1+(1-t)z_2), 0 \leqslant t \leqslant 1$. Тогда:\\
		$\begin{cases}Ax_{1} + By_{1} + Cz_1 + D \geqslant 0 \\ Ax_{2} + By_{2} + Cz_2 + D \geqslant 0\end{cases} \Rightarrow \begin{cases}tAx_{1} + tBy_{1} + tCz_1 + D \geqslant 0 \\ (1-t)(Ax_{2} + By_{2} + Cz_2 + D) \geqslant 0\end{cases} \Rightarrow \\ \Rightarrow A(tx_{1}+(1-t)x_{2}) + B(ty_{1}+(1-t)y_{2}) + C(tz_1+(1-t)z_2) + D \geqslant 0 \Rightarrow X \in V^{+}$.\\
		Таким образом, для $X_1, X_2 \in V^{+} \ \ [X_1, X_2] \subset V^{+}$ из доказанной выше выпуклости, а для точек $X_1 \in V^{+}, X_{2} \in V^{-}$ точку пересечения $[X_1, X_2]$ и $\pi$ можно найти явно, но её существование очевидно.  
	\end{proof}
	\section{Прямая в пространстве}
	\begin{formulas}(Уравнения прямой)
		Пусть $l$ - прямая в пространстве: $l = \{X_0 + t\bar{c}\}$, где $X_0$ - точка прямой, $\bar{c}$ - её направляющий вектор. Если $X_0=\begin{pmatrix}x_{0}\\y_{0}\\z_0\end{pmatrix},\bar{v}=\begin{pmatrix}c_1\\c_2\\c_3\end{pmatrix}$, то координаты точек на прямой выражаются следующим образом: (\bfseries параметрические уравнения прямой\mdseries) \begin{boxedalign*}X \in l: \begin{cases}x = x_{0} + tc_1 \\ y = y_{0} + tc_2 \\ z = z_0 + tc_3\end{cases}\end{boxedalign*}\\
		Выразим $t$ из всех уравнений и приравняем - получим \bfseries каноническое уравнение прямой\mdseries : \begin{boxedalign*}\frac{x-x_{0}}{c_1} = \frac{y-y_{0}}{c_2} = \frac{z-z_0}{c_3}\end{boxedalign*}\\
		Если известно, что прямой принадлежат $X_0 = (x_{0}, y_{0}, z_0), X_1 = (x_{1}, y_{1}, z_1)$, то $\overrightarrow{X_0X_1} = \begin{pmatrix}x_{1} - x_{0}\\y_{1} - y_{0}\\z_1-z_0\end{pmatrix}$ - направляющий вектор, т.е. каноническое уравнение имеет следующий вид (\bfseries уравнение прямой по двум точкам\mdseries)\begin{boxedalign*}\frac{x-x_{0}}{x_{1}-x_{0}} = \frac{y-y_{0}}{y_{1}-y_{0}} = \frac{z-z_{0}}{z_{1}-z_{0}}\end{boxedalign*}
	\end{formulas}
	\begin{consequense}
		Любая прямая является пересечением двух плоскостей. (Видно из канонического уравнения: например, плоскостей $\frac{x-x_{0}}{x_{1}-x_{0}} = \frac{y-y_{0}}{y_{1}-y_{0}}$ и $\frac{y-y_{0}}{y_{1}-y_{0}} = \frac{z-z_{0}}{z_{1}-z_{0}}$) 
	\end{consequense}
	\section{Взаимное расположение плоскости и прямой}
	\begin{formulas}(Случаи расположения)
		Пусть $l = \{X_0 + t\bar{c}: t \in \mathbb{R}\}, \pi = \{X_1 + u\bar{a} + v\bar{b}\}$. Уже знаем, что для любых других точек прямой и плоскости верны те же выражения множеств точек. \\
		Предположим, что прямая и плоскость пересекаются хотя бы в двух точках: пусть $X_0, X_1$ - их различные общие точки. Имеем $\overrightarrow{X_0X_1} = t_0\bar{c} = u_0\bar{a} + v_0\bar{b}$ (для некоторых $t_0 \neq 0, u_0, v_0$, из принадлежности точек и прямой, и плоскости). Отсюда $\bar{c}$ выражается через $\bar{a}. \bar{b}$, а значит для любой $X \in l: X = X_0 + t\bar{c} = X_0 + t(\frac{u_0}{t_0}\bar{a} + \frac{v_0}{t_0}\bar{b}) \in \pi$. Таким образом, 
		\begin{center}\underline{
			$l,\pi$ имеют 2 точки пересечения $\Leftrightarrow l \subset \pi \Leftrightarrow l \cap \pi \neq \varnothing$, $\bar{c}$ выражается через $\bar{a}, \bar{b}$}
		\end{center}
		Заметим, что $X \in l \cap \pi \Leftrightarrow X = X_0 + t_0\bar{c} = X_1 + u_0\bar{a} + v_0\bar{b} \Rightarrow \overrightarrow{X_0X_1} = u_0\bar{a} + v_0\bar{b} - t_0\bar{c}$. В случае некомпланарности $\bar{a}, \bar{b}, \bar{c}$ вектор $\overrightarrow{X_0X_1}$ выражается через них единственным образом, то есть точка пересечения есть и единственная, т.е.
		\begin{center}\underline{
		$l, \pi$ имеют одну т. пересечения $\Leftrightarrow \bar{a}, \bar{b}, \bar{c}$ линейно независимы}
		\end{center}
		(В случае, если $\bar{c}$ выражается через $\bar{a}, \bar{b}$: $l, \pi$ имеют общую точку $\Rightarrow l \subset \pi$, поэтому т.пересечения одна только в этом случае)
	\end{formulas}
	\begin{consequense}
		\underline{$\bar{c}$ выражается через $\bar{a}, \bar{b} \Leftrightarrow$ либо $l\subset\pi$, либо $l\cap\pi=\varnothing$} (в этом случае говорят, что \bfseries прямая параллельна плоскости\mdseries)\\
		\underline{$\bar{c}$ не выражается через $\bar{a}, \bar{b} \Leftrightarrow$ $l\cap\pi$ - одна точка.}
	\end{consequense}
	\section{Взаимное расположение двух прямых}
	\begin{definition}
		Прямые $l_1, l_2$ в пространстве называются параллельными, если они либо совпадают, либо лежат в одной плоскости и не пересекаются. 
	\end{definition}
	\begin{definition}
		Прямые $l_1, l_2$ в пространстве скрещиваются, если они не пересекаются и не параллельны. 
	\end{definition}
	\begin{remark}
		Если прямые $l_1, l_2$ с направляющими векторами $\bar{c}_1, \bar{c}_2$ пересекаются в точке $X_0$ и не совпадают, то $\bar{c}_1 \nparallel \bar{c}_2$ и $l_1, l_2 \subset \pi = \{X_0 + u\bar{c}_1 + v\bar{c}_2: u, v \in\mathbb{R}\}$. Отсюда прямые скрещиваются $\Leftrightarrow$ они не лежат в одной плоскости.\\
		Если направляющие векторы $\bar{c}_1, \bar{c}_2$ прямых $l_1, l_2$ коллинеарны, то прямые либо параллельны, либо совпадают. 
	\end{remark}
	\begin{consequense}
		Прямые $l_1, l_2$ скрещиваются $\Leftrightarrow$ они не пересекаются и $c_1 \nparallel c_2$.\\
		$l_1 \parallel l_2 \Leftrightarrow c_1 \parallel c_2$ (иначе прямые либо скрещиваются, либо пересекаются).
	\end{consequense}
	\section{Плоскость в точечно-евклидовом пр-ве.}
	\subsection{Нормаль к плоскости}
	\begin{definition}
		Пусть $\pi: Ax + By + Cz + D = 0$ в прямоугольной с.к. Вектор $\bar{n} = \begin{pmatrix} A  \\ B \\ C \end{pmatrix}$ называется нормалью (нормальным вектором) к плоскости $\pi$.
	\end{definition}
	\begin{consequense}
		Вектор нормали $\bar{n}$ к плоскости $\pi$ ортогонален любому вектору из $\pi$.
	\end{consequense}
	\begin{proof}
		Пусть $X_0 = (x_0, y_0, z_0), X_1 = (x_1, y_1, z_1), X_0, X_1 \in \pi \Rightarrow A(x_1-x_0) + B(y_1-y_0) + C(z_1 - z_0) = 0$ (вектор $\overrightarrow{X_0X_1} \parallel \pi$). Данное выражение также означает, что $\begin{pmatrix} x_1-x_0&y_1-y_0&z_1-z_0\end{pmatrix}\begin{pmatrix} A  \\ B \\ C \end{pmatrix} = 0 \Rightarrow \bar{n}\perp\overrightarrow{X_0X_1}$.
	\end{proof}
	\begin{consequense}
		Зная вектор нормали, направляющие векторы плоскости можно найти как любые два неколлинеарных вектора, ортогональных нормали.
	\end{consequense}
	\subsection{Расстояние от точки до плоскости}
	\begin{formula}(Расстояние от точки до плоскости)\\
		$d(X_0, \pi) = \inf\{|\overrightarrow{X_0Y}|:Y\in\pi\}$. Из теоремы Пифагора (аналогично расстоянию от точки до прямой) следует, что это расстояние - длина отрезка перпендикуляра к $\pi$, проходящего через $X_0$, заключённого между $X_0$ и $\pi$ (только в прямоугольной с.к.!)
		Найдём это расстояние. Перпендикуляр к $\pi$ через точку $X_0$ имеет направляющий вектор $\begin{pmatrix} A  \\ B \\ C \end{pmatrix}$ и проходит через $(x_0, y_0, z_0)$, т.е. $l:\begin{cases}x = x_{0} + tA \\ y = y_{0} + tB \\ z = z_0 + tC\end{cases}$.
		Точка пересечения $l$ и $\pi$: \begin{align*}
			A(x_0+tA) + B(y_0+tB) + C(z_0+tC) + D = 0 \Rightarrow t = \frac{Ax_0 + By_0 + Cz_0 + D}{A^2 + B^2 + C^2},
		\end{align*}
		квадрат расстояния от $X_0$ до неё равен $(tA)^2 + (tB)^2 +(tC)^2 = (\frac{(Ax_0 + By_0 + Cz_0 + D)^2}{A^2 + B^2 + C^2})$\begin{boxedalign*}
			\Rightarrow d(X_0, \pi) = \frac{|Ax_0 + By_0 + Cz_0 + D|}{\sqrt{A^2 + B^2 + C^2}}
		\end{boxedalign*}
	\end{formula}
	\subsection{Иные формулы расстояния (нет в б.)}
	Величина расстояния от прямой $l$ до плоскости $\pi$ в пространстве имеет смысл только при $l \parallel \pi$. Пусть $\bar{a}$ - направляющий вектор $l$ и $\bar{n}$ - нормаль к $\pi$. Проведём плоскость $\pi'$ через $l$ с направляющими векторами $\bar{a}, \bar{n}$. Тогда все перпендикуляры, опущенные из точки прямой $l$ на $\pi$, будут лежать в $\pi'$. Кроме того, их основания будут лежать на $l' = \pi \cap \pi'$, а отсюда расстояния от всех точек $l$ до $\pi$ одинаковы. Поэтому \bfseries расстояние от прямой до плоскости, параллельной ей, равно расстоянию от любой точки прямой до плоскости.\mdseries 
	\begin{formula}(Расстояние от точки до прямой)\\
		Пусть в прямоугольной с.к. даны $X_1 = (x_1, y_1, z_1)$ - точка и $l$ - прямая, проходящая через $X_0 = (x_0, y_0, z_0)$ с направляющим вектором $\begin{pmatrix}a\\b\\c\end{pmatrix}$.\\
		(ВОЗМОЖНО, ЗДЕСЬ БУДЕТ ЧЕРТЁЖ)\\
		$d(X_1, l) = |X_0X_1|\cdot\sin\phi$ (угол между $\overrightarrow{X_0X_1}$ и $l$) = $|X_0X_1|\cdot\sqrt{1 - \cos^2\phi} \Rightarrow d(X_1, l)^2 = $ (после раскрытия скобок)\begin{boxedalign*}
			(x_1-x_0)^2 + (y_1-y_0)^2 + (z_1-z_0)^2 - \frac{(a(x_1-x_0) + b(y_1-y_0) + c(z_1-z_0))^2}{a^2 + b^2 + c^2} 
		\end{boxedalign*}
	\end{formula} 
	\begin{formula}(Расстояние между скрещивающимися прямыми)\\
		Пусть $l_1, l_2$ - скрещивающиеся прямые с направляющими векторами $\bar{a}, \bar{b}$ соотв., $X_1 \in l_1, X_2 \in l_2$. Из уже известных нам соображений расстояние между $X_1$ и $X_2$ минимально в случае, если $\overrightarrow{X_1X_2} \perp l_1, l_2$.\\Покажем существование таких $X_1, X_2$: построим плоскость $\pi$ через $l_1$ с направляющими векторами $\bar{a}, \bar{b}$ и построим перпендикуляры из точек $l_2$ на $\pi$. Их основания лежат на некоторой прямой $l \parallel l_2$, причём $l_1 \nparallel l$, потому что иначе $l_1 \parallel l_2$. Значит, существует точка пересечения $l_1$ и $l$, а следовательно существует перпендикуляр из точки $l_2$ на $\pi$, основание которого лежит на $l_1$, т.е. этот перпендикуляр общий для $l_1, l_2$.\\
		Тогда для произвольных $X_1, X_2$ на прямых $d(l_1, l_2) = d(l_2, \pi)$ - длина перпендикуляра, опущенного на $\pi$ из любой точки $l_2 =$ высота параллелепипеда, построенного на векторах $\bar{a}, \bar{b}, \overrightarrow{X_1X_2} \Rightarrow$
		\begin{boxedalign*}
			d(l_1, l_2) = \frac{V_{\bar{a}, \bar{b}, \overrightarrow{X_1X_2}}}{S_{\bar{a}, \bar{b}}}
		\end{boxedalign*}
		Осталось только получить формулы для объёма.
	\end{formula}
	\section{Векторное, смешанное произведение. Объём}
	\subsection{Векторное произведение}
	\begin{definition}
		Векторным произведением неколлинеарных векторов $\bar{a}, \bar{b}$ в ориентированном трёхмерном точечно-евклидовом пространстве называется вектор $\bar{v}$ такой, что\begin{enumerate}
			\item $|\bar{v}| = S_{\bar{a},\bar{b}} = \langle\bar{a}, \bar{b}\rangle;$
			\item $\bar{v} \perp \bar{a}, \bar{v} \perp \bar{b};$
			\item Тройка $\bar{a}, \bar{b}, \bar{v}$ имеет положительную ориентацию.
		\end{enumerate}
		Обозначается $[\bar{a}, \bar{b}]$.\\
		Для $\bar{a} \parallel \bar{b} \ \ [\bar{a}, \bar{b}] = 0$. 
	\end{definition}
	\subsection{Объём и ориентированный объём}
	Обычный объём определяется полностью аналогично площади на плоскости.
	Из свойств выводится, что объём параллелепипеда, натянутого на векторы $\bar{a}, \bar{b}, \bar{c}$ (т.е. для его вершины $X_0$ его вершинами также являются $X_0 + \bar{a}, X_0 + \bar{b}, X_0 + \bar{c}$), равен $\langle\bar{a}, \bar{b}\rangle \cdot \ h$, где $h = \bar{c}\cdot\sin\phi$ - высота параллелепипеда $\phi$ - угол между $\bar{c}$ и плоскостью $X_0 + u\bar{a} + v\bar{b}$. Запишем $\phi$ как $\frac{\pi}{2} - \psi$, где $\psi$ - угол между $\bar{c}$ и нормалью к плоскости $X_0 + u\bar{a} + v\bar{b}$. Отсюда:
	\begin{align*}
		h = |\bar{c}\cdot\cos\psi| = \ \vline\frac{(\bar{c}, [\bar{a}, \bar{b}])}{|\bar{c}||[\bar{a}, \bar{b}]|}\cdot \bar{c}\ \vline = \frac{|(\bar{c}, [\bar{a}, \bar{b}])|}{|[\bar{a}, \bar{b}]|} = \frac{|(\bar{c}, [\bar{a}, \bar{b}])|}{\langle\bar{a}, \bar{b}\rangle} \Rightarrow V_{\bar{a}, \bar{b}, \bar{c}} = |([\bar{a}, \bar{b}], \bar{c})|
	\end{align*}
	Определим ориентацию тройки $\bar{a}, \bar{b}, \bar{c}$: рассмотрим базис $\bar{a}, \bar{b}, \bar{n}$. В нём $\bar{c}$ имеет координаты $\begin{pmatrix} c_1  \\ c_2 \\ (\cos\psi)|\bar{c}| \end{pmatrix} (c_1, c_2$ неважны), т.е. матрица перехода от этого базиса к $\bar{a}, \bar{b}, \bar{c}$ равна $\begin{pmatrix} 1&0&c_1\\0&1&c_2\\0&0&(\cos\psi)|c|\end{pmatrix}$. Её определитель $>0 \Leftrightarrow \cos\psi > 0 \Leftrightarrow ([\bar{a}, \bar{b}], \bar{c}) > 0$. Отсюда \underline{$([\bar{a}, \bar{b}], \bar{c}) > 0 \Leftrightarrow$ тройка $\bar{a}, \bar{b}, \bar{c}$ положительно ориентирована}.
	\begin{definition}
		Ориентированным объёмом параллелепипеда, натянутого на векторы $\bar{a}, \bar{b}, \bar{c}$, называется число $\langle \bar{a}, \bar{b}, \bar{c}\rangle = ([\bar{a}, \bar{b}], \bar{c})$. Также это число называется смешанным произведением векторов $\bar{a}, \bar{b}, \bar{c}$.
	\end{definition}
	\begin{consequense}
		$V_{\bar{a}, \bar{b}, \bar{c}} = |\langle\bar{a}, \bar{b}, \bar{c}\rangle|;\\\langle\bar{a}, \bar{b}, \bar{c}\rangle = V_{\bar{a}, \bar{b}, \bar{c}} \Leftrightarrow \bar{a}, \bar{b}, \bar{c}$ положительно ориентирована.   
	\end{consequense}
	\subsection{Свойства смешанного произведения}
	\begin{subtheorem}
		\begin{enumerate}
			\item $\langle \bar{a}, \bar{b}, \bar{c}\rangle = -\langle \bar{b}, \bar{a}, \bar{c}\rangle = -\langle \bar{a}, \bar{c}, \bar{b}\rangle = -\langle \bar{c}, \bar{b}, \bar{a}\rangle$ (очев);
			\item $\langle \bar{a}, \bar{b}, \bar{c}_1 + \bar{c}_2 \rangle = \langle \bar{a}, \bar{b}, \bar{c}_1\rangle + \langle \bar{a}, \bar{b}, \bar{c}_2\rangle$;
			\item $\langle \bar{a}, \bar{b}, \lambda\bar{c}\rangle = \lambda\langle \bar{a}, \bar{b}, \bar{c}\rangle$;
			\item $\langle \bar{a}, \bar{b}_1 + \bar{b}_2, \bar{c}\rangle = \langle \bar{a}, \bar{b}_1, \bar{c}\rangle + \langle \bar{a}, \bar{b}_2, \bar{c}\rangle$;
			\item $\langle \bar{a}, \lambda\bar{b}, \bar{c}\rangle = \lambda\langle \bar{a}, \bar{b}, \bar{c}\rangle$;
			\item $\langle \bar{a}_1 + \bar{a}_2, \bar{b}, \bar{c}\rangle = \langle \bar{a}_1, \bar{b}, \bar{c}\rangle + \langle \bar{a}_2, \bar{b}, \bar{c}\rangle$;
			\item $\langle \lambda\bar{a}, \bar{b}, \bar{c}\rangle = \lambda\langle \bar{a}, \bar{b}, \bar{c}\rangle$;
			\item $\langle \bar{e}_{1}, \bar{e}_{2}, \bar{e}_{3}\rangle = 1 (\bar{e}_{1}, \bar{e}_{2}, \bar{e}_{3}$ - ортонормированный базис)
		\end{enumerate}
		$\circled{1}, \circled{8}$ - очев, $\circled{2}, \circled{3}$ - из свойств скалярного произведения, $\circled{4} - \circled{7}$ получаются из $\circled{2}, \circled{3}$ с помощью $\circled{1}$.
	\end{subtheorem}
	\section{Объёмы в ортонормированном базисе}
	Пусть $\bar{e}_{1}, \bar{e}_{2}, \bar{e}_{3}$ - ортонормированный положительно ориентированный базис.
	В курсе алгебры доказывается, что единственная функция от строк (столбцов) $\bar{a} = (a_1, a_2, a_3), \bar{b} = (b_1, b_2, b_3), \bar{c} = (c_1, c_2, c_3)$ со свойствами $\circled{1} - \circled{8}$ (полилинейность, кососимметричность, единица в $\bar{e}_{1}, \bar{e}_{2}, \bar{e}_{3}$) - определитель матрицы $\begin{pmatrix} a_1&a_2&a_3 \\ b_1&b_2&b_3 \\ c_1&c_2&c_3 \end{pmatrix}$ (или $\begin{pmatrix} a_1&b_1&c_1 \\ a_2&b_2&c_2 \\ a_3&b_3&c_3 \end{pmatrix}$)\\
	Следовательно, $\langle \bar{a}, \bar{b}, \bar{c}\rangle = \begin{vmatrix} a_1&b_1&c_1 \\ a_2&b_2&c_2 \\ a_3&b_3&c_3 \end{vmatrix}, V_{\bar{a}, \bar{b}, \bar{c}} = \vline \begin{vmatrix} a_1&b_1&c_1 \\ a_2&b_2&c_2 \\ a_3&b_3&c_3 \end{vmatrix} \ \vline$ (модуль).\\
	(далее в курсе приведено доказательство утверждения про определитель, более полное доказательство ищите в конспекте алгебры у \href{https://github.com/Viacheslavik122333/Halgebra1sem/blob/main/lecture.pdf}{Viacheslavik122333})
	\section{Векторное произведение и расстояния в ортонормированном базисе}
	\subsection{Выражение векторного произведения}
	\begin{formula}(Векторное произведение в прямоугольной с.к.)		
		Знаем, что в ортонормированном базисе вектор $\bar{x}$ имеет координаты\begin{align*}
			x = (\bar{x}, \bar{e}_1), \ \ y = (\bar{x}, \bar{e}_2), \ \ z = (\bar{x}, \bar{e}_3)
		\end{align*}
		Вычислим: $([\bar{a}, \bar{b}], \bar{e}_1) = \begin{vmatrix} a_1&a_2&a_3 \\ b_1&b_2&b_3 \\ 1&0&0 \end{vmatrix} = \begin{vmatrix} a_2&a_3\\b_2&b_3 \end{vmatrix}$.\\
		Аналогично $([\bar{a}, \bar{b}], \bar{e}_2) = -\begin{vmatrix} a_1&a_3\\b_1&b_3 \end{vmatrix}; \ \ ([\bar{a}, \bar{b}], \bar{e}_3) = \begin{vmatrix} a_1&a_2\\b_1&b_2 \end{vmatrix}$
		Данную формулу обычно записывают в виде определителя:\begin{boxedalign*}
			[\bar{a}, \bar{b}] = \begin{vmatrix}\bar{e}_1&\bar{e}_2&\bar{e}_3\\a_1&a_2&a_3 \\ b_1&b_2&b_3\end{vmatrix}
		\end{boxedalign*}
		(не является определителем по сути, но считается по той же формуле)
	\end{formula}
	\begin{remark}
		Прямые $l_1 = X_1 + t\bar{a}, l_2 = X_2 + t\bar{b}$ скрещиваются $\Leftrightarrow \\\langle \overrightarrow{X_1X_2}, \bar{a}, \bar{b}\rangle \neq 0$.
	\end{remark}
	\subsection{Расстояния через векторное произведение}
	Расстояние от точки $X_1$ до прямой $l: X = X_0 +t\bar{a}$ - высота параллелограмма, натянутого на $\bar{a}, \overrightarrow{X_0X_1} = \frac{|S_{\bar{a}, \overrightarrow{X_0X_1}}|}{|\bar{a}|} \Rightarrow$\begin{boxedalign*}
		d(X, l) = \frac{|[\overrightarrow{X_0X_1}, \bar{a}]|}{|\bar{a}|}
	\end{boxedalign*}
	Расстояние от точки $X_1$ до плоскости $\pi: X = X_0 + u\bar{a}+v\bar{b}$ - высота параллелепипеда, натянутого на $\bar{a}, \bar{b}, \overrightarrow{X_0X_1} = \frac{|V_{\bar{a}, \bar{b}, \overrightarrow{X_0X_1}}|}{[\bar{a}, \bar{b}]} \Rightarrow$\begin{boxedalign*}
		d(X, \pi) = \frac{|\langle\overrightarrow{X_0X_1}, \bar{a}, \bar{b}\rangle|}{|[\bar{a}, \bar{b}]|}
	\end{boxedalign*}
	Расстояние между скрещивающимися прямыми $l_1 = X_1 + t\bar{a},\ l_2 = X_2 + t\bar{b}$: \begin{boxedalign*}
		d(l_1, l_2) = \frac{|\langle\overrightarrow{X_1X_2}, \bar{a}, \bar{b}\rangle|}{|[\bar{a}, \bar{b}]|}
	\end{boxedalign*}
	(так как равно расстоянию от точки первой прямой до плоскости, параллельной первой прямой и проходящей через вторую)
	\section{Линии второго порядка}
	\subsection{Определения}
	Далее рассматриваем аффинную систему координат на плоскости (как на аффинном или точечно-евклидовом пространстве).
	\begin{definition}
		Линией первого порядка называется множество точек $\{(x, y): Ax + By + C =0\}$, где $A, B, C$ - вещественные числа и хотя бы одно из чисел $A, B$ не равно нулю. Другими словами, это множество точек, координаты которых удовлетворяют фиксированному уравнению первой степени.
	\end{definition}
	(Это прямая)
	\begin{definition}
		Линией второго порядка (кривой второго порядка) называется множество точек, координаты которых (в некоторой аффинной системе координат) удовлетворяют уравнению $F(x, y) = 0$, где \begin{align*}
			F(x, y) = a_{11}x^2 + 2a_{12}xy + a_{22}y^2 + 2a_{1}x + 2a_{2}y + a_{0}
		\end{align*}
		где $a_{11}, a_{12}, a_{22} , a_1, a_2, a_0$ - некоторые фиксированные числа и хотя бы одно из чисел $a_11, a_12, a_22$. Выражение для $F(x, y)$ называется многочленом второй степени от переменных $x, y$. Уравнение $F(x, y) = 0$ называется общим уравнением линии второго порядка.
	\end{definition}
	\begin{definition}
		Многочлен $F(x, y)$ ставит в соответствие каждой паре чисел $(x, y)$ некоторое вещественное число. С ним связано отображение $f: \pi \rightarrow \mathbb{R}$ (вместо пары чисел $f$ сопоставляет точку плоскости с такими координатами в заданной с.к.) Отображение $f$ называется квадратичным отображением, представленным многочленом $F$.
	\end{definition}
	\begin{remark}
		Соответствие $F \leftrightarrow f$ взаимно однозначно.\\
		(Каждая точка однозначно определяется своими координатами $\Rightarrow f$ однозначно определяется $F$; подставив в общую формулу для $F(x, y)$ координаты шести различных точек, однозначно определим коэффициенты $\Rightarrow F$ однозначно определяется $f$).
	\end{remark}
	\begin{definition}
		Квадратичная часть $F_1(x, y) = a_{11}x^2 + 2a_{12}xy + a_{22}y^2$ многочлена $F(x, y)$ называется его квадратичной формой.
	\end{definition}
	\begin{definition}
		Линия второго порядка в заданной системе координат однозначно определяется матрицей коэффициентов $A = \begin{pmatrix}a_{11}&a_{12}&a_{1}\\a_{12}&a_{22}&a_{2}\\a_{1}&a_2&a_{0}\end{pmatrix}$. Она называется большой матрицей линии второго порядка.
	\end{definition}
	\begin{definition}
		Квадратичная форма линии второго порядка в заданной системе координат однозначно определяется матрицей коэффициентов $A_1 =\\\begin{pmatrix}a_{11}&a_{12}\\a_{12}&a_{22}\end{pmatrix}$. Она называется малой матрицей линии второго порядка.
	\end{definition}
	\subsection{Формы записи}
	Имеют место равенства (несложно проверить):
	\begin{boxedalign*}
		F(x, y) = \begin{pmatrix}x&y&1\end{pmatrix}A\begin{pmatrix} x \\ y \\ 1 \end{pmatrix}; \ \ \ 
		F_1(x, y) = \begin{pmatrix}x&y\end{pmatrix}A_1\begin{pmatrix} x \\ y \end{pmatrix}
	\end{boxedalign*}
	В прямоугольной системе координат на евклидовой плоскости также имеет место следующее равенство (из выражения скалярного произведения):
	\begin{boxedalign*}
		F(x, y) = \begin{pmatrix}x&y\end{pmatrix}A_1\begin{pmatrix} x \\ y \end{pmatrix} + 2\begin{pmatrix}x&y\end{pmatrix}\begin{pmatrix}a_1\\a_2\end{pmatrix} + a_0
	\end{boxedalign*}
	\subsection{Связь уравнений в разных системах координат}
	Далее работаем в прямоугольной системе координат. Цель - избавиться от слагаемого с $xy$, т.е. перейти в такую прямоугольную систему координат, что $A_1$ в ней - диагональная.\\
	(Для произвольных аффинных систем координат задача упрощения уравнения простая - можно выделить полный квадрат в квадратичной форме и заменой избавиться сначала от члена с $xy$, а затем и от линейных членов за исключением нескольких простых случаев)

	Сначала будем изменять только базис. Пусть старая система координат задана репером $O\bar{e}_{1}\bar{e}_{2}$, новая - $O\bar{e}_{1}'\bar{e}_{2}'$ и $C$ - матрица перехода от старого базиса к новому. Тогда старые координаты радиус-вектора точки $X = (x, y)$ выражаются через новые $(x', y')$ так:\begin{align*}\begin{pmatrix} x \\ y \end{pmatrix} = C\begin{pmatrix} x'\\ y' \end{pmatrix} \ \ \Rightarrow \begin{pmatrix} x&y \end{pmatrix} = \begin{pmatrix} x'&y' \end{pmatrix}C^T\end{align*}
	Очевидно, многочлен линии второго порядка изменится, хоть линия и не меняется как множество точек. Назовём новый многочлен, полученный подстановкой новых координат в старый многочлен, $F'$. Тогда:
	\begin{align*}
		F'(x', y') = \begin{pmatrix} x'&y' \end{pmatrix}C^T \cdot A_1 \cdot C\begin{pmatrix} x'\\ y' \end{pmatrix} + 2\begin{pmatrix} x'&y' \end{pmatrix}C^T\begin{pmatrix}a_1\\a_2\end{pmatrix} + a_0 = \\ = \begin{pmatrix} x'&y' \end{pmatrix}\cdot A_1' \cdot\begin{pmatrix} x'\\ y' \end{pmatrix} + 2\begin{pmatrix} x'&y' \end{pmatrix}\begin{pmatrix}a_1'\\a_2'\end{pmatrix} + a_0 \ \ \ \ \ \ \ \ \ \ \ \ \,
	\end{align*}
	где $A_1' = C^TAC$ и $\begin{pmatrix}a_1'\\a_2'\end{pmatrix} = C^T\begin{pmatrix}a_1\\a_2\end{pmatrix}$.\\
	Далее прямой подсчет показывает, что большая матрица изменяется так: \begin{align*}
		A' = \begin{pNiceMatrix}\Block{2-2}<\Huge>{C}&&0\\&&0\\0&0&1\end{pNiceMatrix}^T \cdot A \cdot \begin{pNiceMatrix}\Block{2-2}<\Huge>{C}&&0\\&&0\\0&0&1\end{pNiceMatrix}
	\end{align*} 
	Теперь изменим ещё и начало координат. Тогда:\begin{align*}
		\begin{pmatrix} x \\ y \\ 1 \end{pmatrix} = D \begin{pmatrix} x' \\ y' \\ 1 \end{pmatrix}, \text{где} \ \ D = \begin{pNiceMatrix}\Block{2-2}<\Huge>{C}&&x_0\\&&y_0\\0&0&1\end{pNiceMatrix} 
	\end{align*}
	Подставляя в выражение $F(x, y)$, получим: \begin{align*}
		F'(x, y) = \begin{pmatrix}x'&y'&1\end{pmatrix}D^TAD\begin{pmatrix} x' \\ y' \\ 1 \end{pmatrix}, \text{т.е.}\ \  A' = D^TAD.
	\end{align*}
	Несложно проверить, что при такой замене также верно равенство $A_1' = C^TAC$.
	\section{Канонические уравнения линий второго порядка}
	\subsection{Собственные векторы и значения}
	Итак, найдём необходимую матрицу $C$. Она должна являться матрицей перехода от одного ортонормированного базиса к другому, т.е. она по определению ортогональна. Таким образом, необходимо найти такие векторы $\bar{e}_{1}', \bar{e}_{2}'$ (новый базис), что: \begin{enumerate}
		\item $\bar{e}_{1}' \perp \bar{e}_{2}';$
		\item $|\bar{e}_{1}'| = |\bar{e}_{2}'| = 1;$
		\item Если $\bar{e}_{1}' = \begin{pmatrix} c_{11} \\ c_{21} \end{pmatrix}, \bar{e}_{2}' = \begin{pmatrix} c_{12} \\ c_{22} \end{pmatrix}$, то матрица $\begin{pmatrix} c_{11}&c_{12} \\ c_{21}&c_{22} \end{pmatrix}$ - искомая матрица $C$. Она уже будет ортогональной, т.е. достаточно условия, что $A_1' = C^TA_1C$ - диагональная (пусть равна $\begin{pmatrix} \lambda_1&0 \\ 0&\lambda_2 \end{pmatrix}$).
	\end{enumerate}
	\begin{align*}
		A_1' = C^{-1}A_1C \Leftrightarrow A_1C = CA_1' \Leftrightarrow A_1\begin{pmatrix} c_{11}&c_{12} \\ c_{21}&c_{22} \end{pmatrix} = \begin{pmatrix} c_{11}&c_{12} \\ c_{21}&c_{22} \end{pmatrix}\begin{pmatrix} \lambda_1&0 \\ 0&\lambda_2 \end{pmatrix} \Leftrightarrow\\\Leftrightarrow
		\begin{cases}
			A_1 \begin{pmatrix} c_{11} \\ c_{21} \end{pmatrix} = \lambda_1 \begin{pmatrix} c_{11} \\ c_{21} \end{pmatrix}\\
			A_1 \begin{pmatrix} c_{12} \\ c_{22} \end{pmatrix} = \lambda_2 \begin{pmatrix} c_{12} \\ c_{22} \end{pmatrix}
		\end{cases}, \ \text{т.е.} \ \ A_1\bar{e}_1 = \lambda_1 \bar{e}_1 \ \text{и} \ \ A_1\bar{e}_2 = \lambda_2 \bar{e}_2.
	\end{align*}
	Найдём ненулевые $\bar{x}$, для которых существует $\lambda$: $A_1 \bar{x} = \lambda \bar{x}$, т.е. $(A_1 - \lambda E)\bar{x} = \bar{0}$. Это уравнение относительно $\bar{x}$ имеет нетривиальное решение $\Leftrightarrow |A_1 - \lambda E| = 0$ (из курса алгебры). Это квадратное уравнение относительно $\lambda$ (оно называется \bfseries характеристическим многочленом линии второго порядка\mdseries) :\begin{align*}
	\begin{vmatrix} a_{11} - \lambda&a_{12}\\a_{12}&a_{22}-\lambda \end{vmatrix} = 0 \Leftrightarrow \lambda^2 - (a_{11} + a_{22})\lambda + a_{11}a_{22}-a_{12}^2 = 0\\
	\mathcal{D} = (a_{11} + a_{22})^2 - 4(a_{11}a_{22}-a_{12}^2) = (a_{11} - a_{22})^2 + 4a_{12}^2 \geqslant 0,
	\end{align*}
	то есть уравнение всегда имеет решения.

	Пусть $\lambda_1, \lambda_2$ - корни этого уравнения (возможно, совпадающие). Занумеруем их так: если они разных знаков, то $\lambda_1$ - положительный корень, а если одного знака, то $\lambda_1$ - меньший по модулю корень.\\
	Из построения $\lambda_1, \lambda_2$ следует, что существуют $\bar{a}, \bar{b}$ такие, что $A_1 \bar{a} = \lambda_1 \bar{a}, A_1 \bar{b} = \lambda_2 \bar{b}$ (они называются собственными векторами матрицы $A_1$, соответствующими собственным значениям $\lambda_1$ и $\lambda_2$ соответственно)
  	\subsection{Переход к канонической системе координат}
	Рассмотрим случаи: \begin{enumerate}
		\item [\LARGE 1.] {$\lambda_1 \neq \lambda_2$}.
		
		Пусть $\bar{a} = \begin{pmatrix} a_1 \\ a_2 \end{pmatrix}, \bar{b} = \begin{pmatrix} b_1 \\ b_2 \end{pmatrix}$. Из того, что матрица $A$ симметричная и изначальная система координат прямоугольная, получим:\begin{align*}
		\lambda_1(\bar{a}, \bar{b}) = (\lambda_1 \bar{a}, \bar{b}) = (A_1 \bar{a}, \bar{b}) = (A_1 \begin{pmatrix} a_1 \\ a_2 \end{pmatrix})^T \begin{pmatrix} b_1 \\ b_2 \end{pmatrix} = \begin{pmatrix} a_1 & a_2 \end{pmatrix}A_1^T \begin{pmatrix} b_1 \\ b_2 \end{pmatrix} =\\= \begin{pmatrix} a_1 & a_2 \end{pmatrix}A_1 \begin{pmatrix} b_1 \\ b_2 \end{pmatrix} = (\bar{a}, A_1 \bar{b}) = (\bar{a}, \lambda_2 \bar{b}) = \lambda_2(\bar{a}, \bar{b})
		\end{align*}
		Притом $\lambda_1 \neq \lambda_2$, а значит $(\bar{a}, \bar{b}) = 0 \Leftrightarrow \bar{a} \perp \bar{b}$. Тогда система координат $O\bar{e}_1'\bar{e}_2'$, где $\bar{e}_{1} = \frac{\bar{a}}{|\bar{a}|}, \bar{e}_{2} = \pm\frac{\bar{b}}{|\bar{b}|}$ является прямоугольной, причём $A_1'$ в ней - диагональная (= $\begin{pmatrix} \lambda_1&0 \\ 0&\lambda_2 \end{pmatrix}$), что мы и хотели получить (можем выбрать знаки так, что ориентация положительна).
		\item [\LARGE 2.] $\lambda_1 = \lambda_2 \ \ (= \lambda)$. 
		
		Найдём $\bar{a}$ такой, что $A_1 \bar{a} = \lambda \bar{a}$. Возьмём ненулевой вектор $\bar{b}$, ортогональный $\bar{a}$. Пусть $\bar{a} = \begin{pmatrix} a_1 \\ a_2 \end{pmatrix}, \bar{b} = \begin{pmatrix} b_1 \\ b_2 \end{pmatrix}$. Тогда: \begin{align*}
		(\bar{a}, \bar{b}) = 0 \Rightarrow (A_1 \bar{a}, \bar{b}) = \lambda(\bar{a}, \bar{b}) = 0
		\end{align*}
		С другой стороны, \begin{align*}
			(A_1 \bar{a}, \bar{b}) = \begin{pmatrix} a_1 & a_2 \end{pmatrix}A_1^T \begin{pmatrix} b_1 \\ b_2 \end{pmatrix} = \begin{pmatrix} a_1 & a_2 \end{pmatrix}A_1 \begin{pmatrix} b_1 \\ b_2 \end{pmatrix} = (\bar{a}, A_1 \bar{b})\Rightarrow \bar{a} \perp A_1 \bar{b} 
		\end{align*} 
		Отсюда $A_1 \bar{b}$ пропорционален $\bar{b}$, т.е. $A_1 \bar{b} = \lambda' \bar{b}$. А так как уже знаем, что такое уравнение имеет решение только при $\lambda' = \lambda$, получаем, что для любого $\bar{b} \perp \bar{a}$ верно $A_1 \bar{b} = \lambda\bar{b}$. Тогда подойдёт система координат $O\bar{e}_1'\bar{e}_2'$, где $\bar{e}_{1} = \frac{\bar{a}}{|\bar{a}|}, \bar{e}_{2} = \pm\frac{\bar{b}}{|\bar{b}|}$ (знак для положительной ориентации).
	\end{enumerate}

	Итак, для любого случая нашли прямоугольную систему координат $O\bar{e}_1'\bar{e}_2'$, в которой уравнение линии имеет вид: \begin{align*}
		F'(x', y') = \lambda_1{x'}^2 + \lambda_2{y'}^2 + 2a_1'x' + 2a_2'y' + a_0 = 0
	\end{align*}
	Вновь рассмотрим случаи:
	\begin{enumerate}
		\item [\LARGE 1.] $\lambda_1 \neq 0$
		
		Тогда из нашей нумерации и $\lambda_2 \neq 0$. Можем выделить полные квадраты: $x'' = x' + \frac{a_1'}{\lambda_1}, y'' = y' + \frac{a_2'}{\lambda_2}$ и перейти к системе координат $O''\bar{e}_{1}'\bar{e}_{2}'$, где $O'' = (\frac{a_1'}{\lambda_1}, \frac{a_2'}{\lambda_2})$ в системе координат $O\bar{e}_{1}'\bar{e}_{2}'$.
		Получим уравнение:\begin{align*}
			\lambda_1(x' + \frac{a_1'}{\lambda_1})^2 + \lambda_2(y' + \frac{a_2'}{\lambda_2}) + a_0 - \frac{{a_1'}^2}{\lambda_1} - \frac{{a_2'}^2}{\lambda_2} = 0
		\end{align*}
		то есть в новейших координатах уравнение имеет вид $\lambda_1{x''}^2 + \lambda_2{y''}^2 = C$.\\
		Если $\lambda_1 > 0, C>0$, разделим на $C$ и получим уравнение вида \begin{boxedalign*}
			\frac{{x''}^2}{a^2} \pm \frac{{y''}^2}{b^2} = 1 \text{  - эллипс (если +) или гипербола (если -)}
		\end{boxedalign*}
		Если $\lambda_1 > 0, C < 0$, посмотрим на $\lambda_2$:\\
			Если $\lambda_2 > 0$, разделим на $-C$ и получим \begin{boxedalign*}
				\frac{{x''}^2}{a^2} + \frac{{y''}^2}{b^2} = -1 \text{  - пустое множество (уравнение мнимого эллипса)}
			\end{boxedalign*}\\
			Если $\lambda_2 < 0$, разделим на $-C$ и ещё раз заменим координаты, поменяв местами $x$ и $y$ (также надо развернуть одну из осей, чтобы сохранить ориентацию, поэтому матрица перехода имеет вид $\begin{pmatrix} 0&-1 \\ 1&0 \end{pmatrix}$). Получим снова \begin{boxedalign*}
				\frac{{x'''}^2}{a^2} - \frac{{y'''}^2}{b^2} = 1 \text{  - гипербола}
			\end{boxedalign*}\\
		Случаи с $\lambda_1 < 0, C \neq 0$ рассматриваются аналогично - либо гипербола, либо эллипс, либо мнимый эллипс.

		Если $C = 0$ и $\lambda_1, \lambda_2$ одного знака, то уравнение имеет вид (с точностью до домножения на -1):\begin{boxedalign*}
			\frac{{x''}^2}{a^2} + \frac{{y''}^2}{b^2} = 0 \text{  - одна точка (пара мнимых пересекающихся прямых)}
		\end{boxedalign*}
		Если $C = 0$ и $\lambda_1, \lambda_2$ разных знаков, то уравнение имеет вид (с точностью до домножения на -1):\begin{boxedalign*}
			\frac{{x''}^2}{a^2} - \frac{{y''}^2}{b^2} = 0 \text{  - пара пересекающихся прямых}
		\end{boxedalign*}
	\item [\LARGE 2.] $\lambda_1 = 0$.
	Тогда $\lambda_2 \neq 0$ (иначе уравнение не второго порядка).
	\begin{align*}
		F'(x', y') = \lambda_2{y'}^2 + 2a_1'x' + 2a_2'y' + a_0 = 0
	\end{align*}
	Если $a_1' \neq 0$, то выделением полного квадрата избавимся от члена с $y$ и сдвигом по оси $Ox$ избавимся от константы: $x'' = x' + \frac{a_0}{2a_1'} - \frac{{a_2'}^2}{2a_1'\lambda_2^2}, \ y'' = y' + \frac{a_2'}{\lambda_2}$. Получим уравнение $\lambda_2{y''}^2 + 2a_1'x'' = 0$, т.е. $\lambda_2{y''}^2 = -2a_1'x''$. Можем сделать $\lambda_2$ и $a_1'$ разного знака: если одного, заменим $x$ на $-x$  и $y$ на $-y$. Поделив на $\lambda_2$, получим \begin{boxedalign*}
		{y''}^2 = 2px (p > 0) \text{  - парабола}
	\end{boxedalign*} 
	Если $a_1' = 0$, то $F'(x', y') = \lambda_2{y'}^2 + 2a_2'y' + a_0 = 0$, т.е. $\lambda_2{y''}^2 + C = 0$. Поделив на $\lambda_2$, получим один из следующих случаев:
	\begin{boxedalign*}
		{y''}^2 + a^2 = 0 \text{  - пара мнимых параллельных прямых}\\
		{y''}^2 - a^2 = 0 \text{  - пара параллельных прямых}\\
		{y''}^2 = 0 \text{  - пара совпадающих прямых}
	\end{boxedalign*}
	\end{enumerate}
	\subsection{Классификация линий второго порядка}
	Сформулируем то, что получили:
	\begin{theorem} (Классификация линий второго порядка)\\
		Для любой линии второго порядка прямоугольная система координат (она называется канонической системой координат) такая, что в ней линия имеет один из следующих видов:	
		\begin{enumerate}
			\item $\frac{x^2}{a^2} + \frac{y^2}{b^2} = 1$ (эллипс);
			\item $\frac{x^2}{a^2} + \frac{y^2}{b^2} = -1$ (мнимый эллипс);
			\item $\frac{x^2}{a^2} + \frac{y^2}{b^2} = 0$ (пара мнимых пересекающихся прямых);
			\item $\frac{x^2}{a^2} - \frac{y^2}{b^2} = 1$ (гипербола);
			\item $\frac{x^2}{a^2} - \frac{y^2}{b^2} = 0$ (пара пересекающихся прямых);
			\item $y^2 = 2px$ (парабола);
			\item $y^2 - a^2 = 0$ (пара параллельных прямых);
			\item $y^2 + a^2 = 0$ (пара мнимых параллельных прямых);
			\item $y^2 = 0$ (пара совпадающих прямых).
		\end{enumerate}
	\end{theorem}
	\section{Ортогональные инварианты}
	\subsection{Основные инварианты}
	Пусть линия второго порядка задана в прямоугольной системе координат формулой $F(x, y) = a_{11}x^2 + 2a_{12}xy + a_{22}y^2 + 2a_{1}x + 2a_{2}y + a_{0}, A$ - её большая матрица, $A_1$ - малая матрица.
	При переходе к новой прямоугольной системе координат:\begin{align*}
		A' = D^TAD, D = \begin{pNiceMatrix}\Block{2-2}<\Huge>{C}&&x_0\\&&y_0\\0&0&1\end{pNiceMatrix}; \ \ 
		A_1' = C^TAC \\ \text{(С - матрица перехода, } x_0, y_0 - \text{новое начало отсчёта)}
	\end{align*}
	При таком преобразовании не меняются:
	\begin{enumerate}
		\item $\Delta = |A|$ (т.к. $|D| = |D^T| = |D^{-1}| = \pm 1$);
		\item $\delta = |A_1|$ (аналогично);
		\item $|A_1 - \lambda E|$;
		\item $S = a_{11}+ a_{22}$. (нетрудно проверить)
	\end{enumerate}
	Заметим, что инварианты изменяются при домножении уравнения на число, но не меняются знаки $\delta$ и $\Delta\cdot S$.
	\subsection{Классификация}
	Запишем большие матрицы канонических уравнений:
	\begin{enumerate}
		\item $\begin{pmatrix} \frac{1}{a^2}&0&0 \\ 0&\frac{1}{b^2}&0 \\ 0&0&-1 \end{pmatrix}$ (эллипс) - $\delta > 0, \Delta S < 0$;
		\item $\begin{pmatrix} \frac{1}{a^2}&0&0 \\ 0&\frac{1}{b^2}&0 \\ 0&0&1 \end{pmatrix}$ (мнимый эллипс) - $\delta > 0, \Delta S > 0$;
		\item $\begin{pmatrix} \frac{1}{a^2}&0&0 \\ 0&\frac{1}{b^2}&0 \\ 0&0&0 \end{pmatrix}$ (пара мнимых пересекающихся прямых) - $\delta > 0, \Delta S = 0$;
		\item $\begin{pmatrix} \frac{1}{a^2}&0&0 \\ 0&-\frac{1}{b^2}&0 \\ 0&0&-1 \end{pmatrix}$ (гипербола) - $\delta < 0, \Delta S \neq 0$;
		\item $\begin{pmatrix} \frac{1}{a^2}&0&0 \\ 0&-\frac{1}{b^2}&0 \\ 0&0&0 \end{pmatrix}$ (пара пересекающихся прямых) - $\delta < 0, \Delta S = 0$;
		\item $\begin{pmatrix} 0&0&-2p \\ 0&1&0 \\ -2p&0&0 \end{pmatrix}$ (парабола) - $\delta = 0, \Delta S \neq 0$;
		\item $\begin{pmatrix} 0&0&0 \\ 0&1&0 \\ 0&0&-a^2 \end{pmatrix}$ (пара параллельных прямых) - $\delta = \Delta = 0$;
		\item $\begin{pmatrix} 0&0&0 \\ 0&1&0 \\ 0&0&a^2 \end{pmatrix}$ (пара мнимых параллельных прямых) - $\delta = \Delta = 0$;
		\item $\begin{pmatrix} 0&0&0 \\ 0&1&0 \\ 0&0&0 \end{pmatrix}$ (пара совпадающих прямых) - $\delta = \Delta = 0$;
	\end{enumerate}
	
	Итак, разберём случаи знаков ортогональных инвариантов:
	\begin{enumerate}
		\item [\LARGE 1.] $\delta \neq 0$ - центральный случай (кривые 1-5):
		$\delta > 0 \Rightarrow$ либо эллипс ($\Delta S < 0$), либо мнимый эллипс ($\Delta S > 0$), либо пара мнимых пересекающихся прямых ($\Delta S = 0$);\\
		$\delta < 0 \Rightarrow$ либо гипербола ($\Delta \neq 0$), либо пара пересекающихся прямых ($\Delta = 0$); 
		\item [\LARGE 2.] $\delta = 0$ - параболический случай (кривые 6-9):
		$\Delta \neq 0$ - парабола;
		$\Delta = 0$ - кривые 7-9.
	\end{enumerate} 
	\subsection{Семиинвариант (нет в б., но полезно)}
	Для последнего случая необходим семиинвариант: $K = \begin{vmatrix} a_{22}&a_2 \\ a_2&a_0 \end{vmatrix} + \begin{vmatrix} a_{11}&a_1 \\ a_1&a_0 \end{vmatrix}$.
	\begin{subtheorem}
		$K$ не изменяется при замене базиса (без сдвига)
	\end{subtheorem}
	\begin{proof}
		\begin{align*}
			K = (a_{11} + a_{22})a_0 - a_1^2 - a_2^2 = S\cdot a_0 - |\bar{a}|^2
		\end{align*}
		При переходе к новому ортонормированному базису $a_0$ не изменяется (очев.), $S$ не изменяется (инвариант), а $\bar{a}$ умножается на транспонированную матрицу перехода, которую можно представить в виде матрицы поворота на некоторый угол (см. Билет 23), т.е. $|\bar{a}|$ не изменяется. 
	\end{proof}
	\begin{subtheorem}
		В случае $\delta = \Delta = 0 \ \ K$ не изменяется и при сдвиге.
	\end{subtheorem}
	\begin{proof}
		Пусть уравнение кривой $a_{11}x^2 + 2a_{12}xy + a_{22}y^2 + 2a_{1}x + 2a_{2}y + a_{0}$ задано в прямоугольной системе координат. Доказали, что существует прямоугольная система координат с тем же началом, в которой данная линия задаётся уравнением $\lambda_1{x'}^2 + \lambda_2{y'}^2 + a_{1}'x' + a_{2}'y' + a_{0}$, и притом $K$ не изменяется при этом переходе. Из условия $\delta = 0$ следует, что $\lambda_1\lambda_2 = 0$. Пусть $\lambda_1 = 0$ (в силу нашей нумерации). Тогда большая матрица имеет вид $\begin{pmatrix}0&0&a_{1}'\\0&\lambda_2&a_{2}'\\ a_{1}'&a_2'&a_{0}\end{pmatrix}$. Из условия $\Delta = 0$ имеем $\lambda_2{a_1'}^2 = 0$. Причём знаем, что $\lambda_2 \neq 0$ (иначе линия не является линией второго порядка), то есть $a_1' = 0$. Тогда подстановкой получим, что $K = Sa_0 - |\bar{a}|^2 = \lambda_2a_0 - {a_2'}^2$. Заменим начало координат: $x' = x'' + x_0, y' = y'' + y_0$. Подставим в уравнение линии: \begin{align*}\lambda_2(y'' + y_0)^2 + 2a_{2}'(y'' + y_0) + a_{0} = \lambda_2{y''}^2 + 2(\lambda_2y_0 +a_{2}')y'' + \lambda_2y_0^2 +2a_2'y_0 + a_{0} \Rightarrow \\K' = \lambda_2(\lambda_2y_0^2 +2a_2'y_0 + a_{0}) - (\lambda_2y_0 +a_{2}')^2 = \\\lambda_2^2y_0^2 +2\lambda_2a_2'y_0 + \lambda_2a_{0} - \lambda_2^2y_0^2 +2\lambda_2a_2'y_0 + {a_2'}^2 = \lambda_2a_0 - {a_2'}^2 = K
		\end{align*}
		Отсюда $K' = K$, ч.т.д.
	\end{proof}
	Тогда в случае $\delta = \Delta = 0$:
	\begin{enumerate}
		\item $\begin{pmatrix} 0&0&0 \\ 0&1&0 \\ 0&0&-a^2 \end{pmatrix}$ (пара параллельных прямых) - $K < 0$;
		\item $\begin{pmatrix} 0&0&0 \\ 0&1&0 \\ 0&0&a^2 \end{pmatrix}$ (пара мнимых параллельных прямых) - $K>0$;
		\item $\begin{pmatrix} 0&0&0 \\ 0&1&0 \\ 0&0&0 \end{pmatrix}$ (пара совпадающих прямых) - $K = 0$;
	\end{enumerate}
	\subsection{Полная классификация}
	Отсюда получаем во-первых единственность канонического уравнения линии второго порядка (из инвариантности однозначно определяющих её величин), а также алгоритм определения линии по значениям инвариантов в её большой матрице:
	\begin{table}[htbp]
		\begin{tabular}{@{}ccccccccc@{}}
			\toprule
		\multicolumn{5}{c}{$\delta \neq 0$} & \multicolumn{4}{c}{$\delta = 0$} \\ \cmidrule(lr){1-5}\cmidrule(lr){6-9}
		\multicolumn{3}{c}{$\delta > 0$} & \multicolumn{2}{c}{$\delta \neq 0$} & $\Delta \neq 0$ & \multicolumn{3}{c}{$\Delta = 0$} \\ \cmidrule(lr){1-3}\cmidrule(lr){4-5}\cmidrule{7-9}
		$\Delta S < 0$ & $\Delta S > 0$ & $\Delta S = 0$ & $\Delta \neq 0$ & $\Delta = 0$ &  & $K < 0$ & $K > 0$ & $K = 0$ \\ \cmidrule(lr){1-1}\cmidrule(lr){2-2}\cmidrule{3-3}\cmidrule(lr){4-4}\cmidrule(lr){5-5}\cmidrule{6-6}\cmidrule(lr){7-7}\cmidrule(lr){8-8}\cmidrule{9-9}
		эллипс&мн. эллипс&мн. $\times$&гипербола&$\times$&парабола&$\parallel$&мн. $\parallel$& совп. $|$ \\\bottomrule
		\end{tabular}
	\end{table} 
	\section{Центр линии второго порядка}
	\subsection{Центр симметрии}
	\begin{definition}
		Точка $X_1$ называется симметричной точке $X_2$ относительно точки $O$, если $O$ - середина отрезка $[X_1 X_2]$
	\end{definition}
	\begin{definition}
		Точка $O$ называется центром симметрии множества точек $M$ (на плоскости), если для любой точки $X$ из $M$ точка, симметричная $X$ относительно $O$, также принадлежит $M$. 
	\end{definition}
	\subsection{Уравнение центра}
	\begin{theorem}
		Пусть непустая линия второго порядка задана в некоторой системе координат (не обязательно прямоугольной) уравнением $a_{11}x^2 + 2a_{12}xy + a_{22}y^2 + 2a_{1}x + 2a_{2}y + a_{0} = 0$. Точка $O = (x_0, y_0)$ является центром симметрии этой линии $\Leftrightarrow \begin{cases}
			a_{11}x_0+a_{12}y_0+a_1=0\\
			a_{12}x_0+a_{22}y_0+a_2=0
		\end{cases}$ 
	\end{theorem}
	\begin{proof} 
		\begin{lemma}
			Координаты точки $X(x_1, y_1)$ удовлетворяют или не удовлетворяют системе $\begin{cases}a_{11}x_0+a_{12}y_0+a_1=0\\a_{12}x_0+a_{22}y_0+a_2=0 \end{cases} (*)$ независимо от того, в какой аффинной системе координат записано уравнение и координаты.
		\end{lemma}
		\begin{proof}
			$(x_1, y_1)$ удовлетворяет $(*) \Rightarrow A \begin{pmatrix} x_1  \\ y_1 \\ 1 \end{pmatrix} = \begin{pmatrix} 0  \\ 0 \\ \alpha \end{pmatrix},$ где $\alpha = a_1x_1 + a_2y_1 + a_0$. Тогда в другой системе координат, переход к которой задан матрицей $D = \begin{pNiceMatrix}\Block{2-2}<\Huge>{C}&&x_0\\&&y_0\\0&0&1\end{pNiceMatrix}$, имеем $\begin{pmatrix} x_1\\y_1\\1\end{pmatrix} = D\begin{pmatrix} x_1'\\y_1'\\1\end{pmatrix};$ $A' = D^TAD \Rightarrow A'\begin{pmatrix} x_1'\\y_1'\\1\end{pmatrix} = D^TAD\begin{pmatrix} x_1'\\y_1'\\1\end{pmatrix} = D^TA\begin{pmatrix} x_1\\y_1\\1\end{pmatrix} = D^T\begin{pmatrix} 0\\0\\\alpha\end{pmatrix} = \begin{pNiceMatrix}\Block{2-2}<\Huge>{C}&&0\\&&0\\x_0&y_0&1\end{pNiceMatrix}\begin{pmatrix} 0\\0\\\alpha\end{pmatrix} = \begin{pmatrix} 0\\0\\\alpha\end{pmatrix}$. Остюда в новой системе координат для координат точки $X$ эта система уравнений также выполнена.
		\end{proof}
		$\\\Rightarrow \ \ $ Сдвинем начало координат в точку $O: x = x' + x_0, y = y' + y_0$ и получим новое уравнение:
		\begin{align*}
			a_{11}'{x'}^2 + 2a_{12}'x'y' + a_{22}'{y'}^2 + 2a_{1}'x' + 2a_{2}'y' + a_{0}' = 0
		\end{align*} 
		Так как для этой системы координат точка $(0, 0)$ в этой системе координат является центром симметрии линии, если ей принадлежит точка $(x_1', y_1')$, то принадлежит и точка $(-x_1', -y_1')$. Подставим точку $(-x_1', -y_1')$:
		\begin{align*}
			a_{11}'{x'}^2 + 2a_{12}'x'y' + a_{22}'{y'}^2 - 2a_{1}'x' - 2a_{2}'y' + a_{0}' = 0
		\end{align*}
		Вычитая это уравнение из уравнения для точки $(x_1', y_1')$, получим:
		\begin{align*}
			4(a_{1}'x' + a_{2}'y') = 0 \Rightarrow a_{1}'x' + a_{2}'y' = 0
		\end{align*} 
		для любой точки $(x_1, y_1)$ линии. Получаем следующее:\begin{center}
		Либо $a_1' = a_2' = 0$ (тогда уравнение верно всегда), либо все точки линии удовлетворяют уравнению прямой $a_{1}'x' + a_{2}'y' = 0$. \bfseries(*)\mdseries
		\end{center}
		При этом можем выразить $a_1'$ и $a_2'$ через коэффициенты старого уравнения - получим $\begin{cases}
			a_1' = a_{11}x_0+a_{12}y_0+a_1\\
			a_2' = a_{12}x_0+a_{22}y_0+a_2
		\end{cases}$
		. Из рассуждений выше либо $a_1' = a_2' = 0$, то есть теорема верна, либо все точки линии лежат на одной прямой - для непустых линий второго порядка это возможно только в случае точки (пара мнимых пересекающихся прямых) или совпадающих прямых, а у этих линий каждая точка является центром симметрии.
		$\\\Leftarrow \ \ $ Сдвинем начало координат в точку $O: x = x' + x_0, y = y' + y_0$. В новой системе получим новое уравнение:
	    \begin{align*}
			F'(x', y') = a_{11}{x'}^2 + 2a_{12}x'y' + a_{22}{y'}^2 + C_0 = 0
		\end{align*}
		так как коэффициенты перед $x', y'$ получаются равными нулю из системы. Очевидно, что это уравнение симметрично относительно начала координат.
	\end{proof}
	\begin{consequense}
		В центральном случае для непустой линии второго порядка (не мнимый эллипс) центр симметрии единственный. 
	\end{consequense}
	\begin{proof}
		Центральный случай $\Rightarrow \delta \neq 0 \Rightarrow$ система имеет единственное решение.\\
		(Решение единственно и в случае мнимого эллипса, но у него нет действительных точек) 
	\end{proof}
	Отдельно рассмотрим случаи из утверждения (*). В первом случае имеем $A_1 = \begin{pmatrix}a_{11}&a_{12}\\a_{12}&a_{22}\end{pmatrix}, A = \begin{pmatrix}a_{11}&a_{12}&0\\a_{12}&a_{22}&0\\ 0&0&a_{0}\end{pmatrix}$. После перехода к базису канонической системы координат получим уравнение вида $\lambda_1{x''}^2 + \lambda_2{y''}^2 + a_0 = 0$ - таким уравнением можно представить всё, кроме параболы. Справедливо и обратное утверждение: если линия задаётся уравнением вида $\lambda_1{x''}^2 + \lambda_2{y''}^2 + a_0 = 0$, то точка с координатами $(0, 0)$ в этой с.к. - её центр симметрии.

	Во втором случае все точки линии второго порядка лежат на одной прямой. Покажем, когда такое возможно: $a_1'x' + a_2'y' = 0 \Leftrightarrow (a_1'x' + a_2'y')^2 = 0 \Leftrightarrow {a_1'}^2{x'}^2 + a_1'a_2'x'y' + {a_2'}^2{y'}^2 = 0$ - такое уравнение задаёт либо пару мнимых пересекающихся прямых, либо пару совпадающих прямых (либо пустое множество, для которого наши рассуждения о принадлежности всех точек одной прямой неприменимы, так что рассматриваем непустые линии) - то есть наша линия является либо точкой, либо прямой.\\
	\underline{У любой линии второго порядка, кроме параболы, есть центр симметрии}.

	\subsection{Определение центра}
	Итак, теперь можем определить центр линии второго порядка.
	\begin{definition}
		Точка $O(x_0, y_0)$ называется центром линии второго порядка $a_{11}x^2 + 2a_{12}xy + a_{22}y^2 + 2a_{1}x + 2a_{2}y + a_{0}$, если её координаты удовлетворяют системе уравнений $\begin{cases}
			a_{11}x_0+a_{12}y_0+a_1=0\\
			a_{12}x_0+a_{22}y_0+a_2=0
		\end{cases}$
	\end{definition}
	\begin{remark}
		Из доказанного выше: $O$ - центр линии $\Leftrightarrow$ уравнение линии симметрично относительно этой точки в любой с.к.\\
		Для всех прямых, кроме параболы, центр канонической с.к. является её центром (несложно проверить).
	\end{remark}
	\section{Сопряжённые направления}
	\subsection{Определение}
	Попробуем найти произвольный базис (не обязательно ортогональный), в котором матрица $A_1$ линии второго порядка станет диагональной. Пусть $\bar{e}_1 = \begin{pmatrix} \alpha_1 \\ \beta_1 \end{pmatrix}, \bar{e}_2 = \begin{pmatrix} \alpha_2 \\ \beta_2 \end{pmatrix}$ - координаты новых базисных векторов в старом базисе. Их новые координаты - $\begin{pmatrix} 1 \\ 0 \end{pmatrix}$ и $\begin{pmatrix} 0 \\ 1 \end{pmatrix}$. Так как $A_1' = \begin{pmatrix}a_{11}'&a_{12}'\\a_{12}'&a_{22}'\end{pmatrix}$ - диагональная, $a_{22}' = 0$. Выразим $a_{22}'$ через малую матрицу $: a_{22}' = \begin{pmatrix} 1 & 0 \end{pmatrix}A_1'\begin{pmatrix} 0 \\ 1 \end{pmatrix}$. Также если $C$ - матрица перехода к новому базису, то известны равенства: $A_1' = C^TA_1C, \begin{pmatrix} \alpha_1 \\ \beta_1 \end{pmatrix} = C \begin{pmatrix} 1 \\ 0 \end{pmatrix}, \begin{pmatrix} \alpha_2 \\ \beta_2 \end{pmatrix} = C \begin{pmatrix} 0 \\ 1 \end{pmatrix}$. Отсюда: \begin{align*}
		a_{22}' = \begin{pmatrix} 1 & 0 \end{pmatrix}A_1'\begin{pmatrix} 0 \\ 1 \end{pmatrix} = a_{22}' = \begin{pmatrix} 1 & 0 \end{pmatrix}C^TA_1C\begin{pmatrix} 0 \\ 1 \end{pmatrix} = \begin{pmatrix} \alpha_1 & \beta_1 \end{pmatrix}A_1\begin{pmatrix} \alpha_2 \\ \beta_2 \end{pmatrix}
	\end{align*}
	Отсюда $A_1'$ будет диагональной в случае $\begin{pmatrix} \alpha_1 & \beta_1 \end{pmatrix}A_1\begin{pmatrix} \alpha_2 \\ \beta_2 \end{pmatrix} = 0$.
	\begin{definition}
		Направления ненулевых векторов $\begin{pmatrix} \alpha_1 & \beta_1 \end{pmatrix}, \begin{pmatrix} \alpha_2 \\ \beta_2 \end{pmatrix}$ и сами эти векторы называются сопряжёнными относительно линии второго порядка, заданной в некоторой с.к. уравнением с малой матрицей $A_1$, если выполнено $\begin{pmatrix} \alpha_1 & \beta_1 \end{pmatrix}A_1\begin{pmatrix} \alpha_2 \\ \beta_2 \end{pmatrix} = 0$.
	\end{definition}
	\begin{remark}
		Направления или векторы $\begin{pmatrix} \alpha_1 & \beta_1 \end{pmatrix}$ и $\begin{pmatrix} \alpha_2 \\ \beta_2 \end{pmatrix}$ сопряжены $\Leftrightarrow \\\begin{pmatrix} \alpha_2 & \beta_2 \end{pmatrix}$ и $\begin{pmatrix} \alpha_1 \\ \beta_1 \end{pmatrix}$ сопряжены.
	\end{remark}
	\begin{subtheorem}
		Направления являются или не являются сопряжёнными вне зависимости от системы координат.
	\end{subtheorem}
	\begin{proof}
		Из размышлений до определения видно, что спряжённость не зависит от базиса, а также в уравнении линии второго порядка она зависит только от малой матрицы, т.е. не зависит от начала координат. А значит сопряженность не зависит от системы координат в целом.
	\end{proof}
	\subsection{Асимптотические направления}
	\begin{definition}
		Направление (= ненулевой вектор) $\begin{pmatrix} \alpha \\ \beta \end{pmatrix}$, сопряжённое самому себе, т.е. $\begin{pmatrix} \alpha & \beta \end{pmatrix}A_1\begin{pmatrix} \alpha \\ \beta \end{pmatrix} = 0$, называется асимптотическим.
	\end{definition}
	Далее во всех связанных с асимптотическими направлениями рассуждениях можем считать, что линия задана уравнением вида $\lambda_1x^2 + \lambda_2y^2 + c = 0$ или $\lambda_2y^2 + 2cx = 0$ - здесь и далее такй вид уравнения линии будем называть \bfseries простейшим\mdseries.
	\subsection{Пересечения линии и прямой}
	\begin{theorem}
		1. Если прямая $l$ имеет асимптотическое направление относительно линии второго порядка $L$, то либо $l$ имеет с $L$ не более одной общей точки, либо $l$ содержится в $L$.\\
		2. Прямая $l$ неасимптотического направления имеет с $L$ не более двух общих точек (причём в случае ровно одной точки её обычно считают двумя совпадающими точками).
	\end{theorem}
	\begin{proof}
		Пусть $\begin{pmatrix} \alpha \\ \beta \end{pmatrix}$ - любое направление. Рассмотрим прямую $l$ с направляющим вектором $\begin{pmatrix} \alpha \\ \beta \end{pmatrix}: \begin{cases}
		x = x_0 + \alpha t\\
		y = y_0 + \beta t
		\end{cases}$
		и подставим в простейшее уравнение линии $F(x, y) = \lambda_1x^2 + \lambda_2y^2 + c$ или $\lambda_2y^2 + 2cx = 0$. После приведения подобных слагаемых получим не более чем квадратное уравнение относительно $t$, в котором коэффициент перед $t^2$ будет равен $\lambda_1\alpha^2 + \lambda_2\beta^2$. Тогда уравнение не может иметь больше двух корней в случае, если этот коэффициент ненулевой, то есть при $\lambda_1\alpha^2 + \lambda_2\beta^2 = \begin{pmatrix} \alpha & \beta \end{pmatrix}A_1\begin{pmatrix} \alpha \\ \beta \end{pmatrix} \neq 0$ - в случае неасимптотического направления. В случае асимптотического направления наше уравнение линейно, а значит оно либо не имеет решений, либо решение единственно, либо любое $t$ является решением, т.е. $l$ целиком содержится в $L$.
 	\end{proof}
	\begin{consequense}
		Никакие три точки эллипса, параболы или гиперболы не лежат на одной прямой.
	\end{consequense}
	\begin{proof}
		Предположим противное. Тогда по доказанной выше теореме имеем, что линия второго порядка содержит прямую, на которой лежат данные точки. Докажем, что это невозможно:
		\begin{itemize}
			\item Эллипс: в канонической системе координат для его точек выполнены неравенства $|x| \leqslant a, |y| \leqslant b$, т.к. $\frac{x^2}{a^2} + \frac{y^2}{b^2} = 1$, поэтому эллипс ограничен и не может содержать прямую;
			\item Парабола: в канонической системе координат для её точек выполнено неравенство $x \geqslant 0$, т.к. $y^2 = 2px$. Отсюда парабола не может содержать никакую прямую, кроме прямой вида $x = c \geqslant 0$, но уравнение $y^2 = 2pc$ не может иметь более двух решений относительно y;
			\item Гипербола: в канонической системе координат для её точек выполнено неравенство $|x| \geqslant a$, т.к. $\frac{x^2}{a^2} - \frac{y^2}{b^2} = 1$, то есть она не может содержать точек прямой $x = 0$. Тогда она не может содержать никакую прямую, кроме прямой вида $x = c \neq 0$, но уравнение $\frac{c^2}{a^2} - \frac{y^2}{b^2} = 1$ не может иметь более двух решений относительно y.
		\end{itemize}
	\end{proof}
	\begin{definition}
		Из следствия видно, что прямая может содержаться целиком только в линии, представляющей собой пару пересекающихся прямых, пару параллельных прямых или пару совпадающих прямых. Такие линии называются распадающимися, так как задающие их многочлены $F(x, y)$ распадаются в произведение двух линейных функций.
		(это свойство очевидно для этих линий в каноническом виде, а далее можем проделать обратные к приведению к каноническому виду операции, чтобы получить тот же результат для произвольного вида). 
	\end{definition}
	\bfseries Мораль в том, что пупупу...\mdseries
\end{document}