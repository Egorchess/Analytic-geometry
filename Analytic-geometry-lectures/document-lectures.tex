\documentclass[a4paper, 12pt]{article}
%\usepackage{mathtext}
\usepackage{cmap}
\usepackage[english, russian]{babel}
\usepackage[T2A]{fontenc}
\usepackage[utf8]{inputenc}
\usepackage[left=2cm, right=1.5cm, top=2cm, bottom=2cm]{geometry}
\usepackage{amsmath}
\usepackage{amssymb}
\usepackage{etoolbox}
\usepackage{amsthm}
\usepackage{amsfonts}
\usepackage{mathtools}
%\usepackage{indentfirst}
\usepackage{soul}
\usepackage{graphicx}
\usepackage{enumerate}


\usepackage{tikz,amstext}
\newlength{\tempheight}
\newcommand{\Let}[0]{%
	\mathbin{\text{\settoheight{\tempheight}{\mathstrut}\raisebox{0.5\pgflinewidth}{%
				\tikz[baseline,line cap=round,line join=round] \draw (0,0) --++ (0.4em,0) --++ (0,1.5ex) --++ (-0.4em,0);%
}}}}

\renewcommand{\phi}{\varphi}
\renewcommand{\epsilon}{\varepsilon}
\newcommand*\circled[1]{\tikz[baseline=(char.base)]{
            \node[shape=circle,draw,inner sep=2pt] (char) {#1};}}
\newcommand{\aug}{\fboxsep=-\fboxrule\!\!\!\fbox{\strut}\!\!\!}
\newcommand\tab[1][.5cm]{\hspace*{#1}}
\newcommand\undermat[2]{\makebox[0pt][l]{$\smash{\underbrace
			{\phantom{\begin{matrix}#2\end{matrix}}}_{\text{$#1$}}}$}#2}
\newcommand\overmat[2]{\makebox[0pt][l]{$\smash{\overbrace
			{\phantom{\begin{matrix}#2\end{matrix}}}^{\text{$#1$}}}$}#2}

\newcounter{lemcount}
\newcounter{lemcount2}
\newcounter{thcount}
\theoremstyle{definition}
\newtheorem*{definition}{Определение}
\newtheorem*{theorem}{Теорема}
\newtheorem*{consequense}{Следствие}
\newtheorem*{lemma}{Лемма}
\newtheorem*{subtheorem}{Утверждение}
\newtheorem*{formula}{Вывод формулы}
\newtheorem*{formulas}{Вывод формул}
\newtheorem*{remark}{Замечание}
\newtheorem*{examples}{Примеры}
\newtheorem*{example}{Пример}
\newtheorem*{lalala}{Упражнение}
\newtheorem*{algorithm}{Алгоритм}
\newtheorem*{properties}{Свойства}
\newtheorem*{properties1}{Свойство}
\newtheorem{lemmanum}[lemcount]{Лемма}
\newtheorem{lemmanum2}[lemcount2]{Лемма}
\newtheorem{theoremnum}[thcount]{Теорема}
% \newtheorem{theoremL}{Теорема}[section]
\usepackage[T2A]{fontenc}
\usepackage[utf8]{inputenc}
\usepackage[russian]{babel}
\addto\captionsenglish{% Replace "english" with the language you use
	\renewcommand{\contentsname}%
	{Содержание}%
}

\usepackage{titlesec}
\titleformat{\section}{\LARGE \bfseries}{\thesection}{1em}{}
\titleformat{\subsection}{\Large\bfseries}{\thesubsection}{1em}{}
\titleformat{\subsubsection}{\large\bfseries}{\thesubsubsection}{1em}{}

\usepackage{hyperref}
\usepackage{xcolor}
% Цвета для гиперссылок
\definecolor{linkcolor}{HTML}{225ae2} % цвет ссылок
\definecolor{urlcolor}{HTML}{225ae2} % цвет гиперссылок
\hypersetup{
	pdfstartview=FitH, 
	linkcolor=linkcolor,
	urlcolor=urlcolor,
	colorlinks=true
}

\begin{document}
	\begin{titlepage}
		\newpage
		
		\begin{center}
		\end{center}
		
		\vspace{4em}
		
		\begin{center}
			\Large Механико-математический факультет  
		\end{center}
		
		\vspace{2em}
		
		\begin{center}
			\large{\textsc{\textbf{Аналитическая геометрия, 1 семестр, 2 поток}}}
		\end{center}
		
		\vspace{6em}
		
		\vspace{\fill}
		
		\begin{center}
			Москва \\2024 
		\end{center}
	\end{titlepage}
	\tableofcontents
	\fontsize{14pt}{20pt}\selectfont
	\newpage
	\fontsize{14pt}{20pt}\selectfont
	\section{Общие понятия (нет в билетах)}
	Уже известные нам понятия:
	\begin{itemize}
	\item Точка - то, что не имеет частей;
	\item Линия - длина без ширины;
	\item Прямая - линия, равно расположенная по отношению к точкам на ней;
	\item Поверхность - то. что имеет длину и ширину;
	\item \dots
	\end{itemize}
	(Эти определения, как и последующие постулаты и аксиомы, были даны Евклидом.)
	
	И определяемые понятия - аналогично школьным учебникам. 
	\subsection{Постулаты}
	\begin{enumerate}
	\item От всякой точки до всякой можно провести прямую.
	
	\item Ограниченную часть прямой можно непрерывно продолжать.
	
	\item Из всякой точки можно описать окружность со всяким радиусом.
	
	\item Все прямые углы равны между собой.
	
	\item Если прямая $l$ образует с двумя другими прямыми $l_{1}$ и $l_{2}$ внутренние и по одну сторону углы, меньшие прямых, то $l_{1}$ и $l_{2}$ пересекутся с той стороны от $l$, где углы меньше прямых.   
	\end{enumerate}
	
	\subsection{Аксиомы (общие понятия) по Евклиду}
	\begin{enumerate}
		\item Равные одному и тому же равны между собой.
		\item Если к равным прибавляются равные, то суммы равны.
		\item Если из равных вычитаются равные, то разности равны.
		\item Если к неравным прибавляются равные, то суммы неравны.
		\item Удвоенные равные равны.
		\item Половины равных равны.
		\item Совмещающиеся друг с другом равны между собой.
		\item Целое больше части.
		\item Две прямые не образуют пространство.
	\end{enumerate}
	Сейчас используется более строгая аксиоматика (Гильберт)
	
	\subsection{Современная аксиоматика}
	\begin{definition}
		Плоскость (по Колмогорову) - тройка ($X, L, d$), где $X$ - множество (точек), $L$ - выделенная совокупность его подмножеств (прямых) и $d$ - отображение. сопоставляющее паре точек $x, y\in X$ неотрицательное $d(x, y) \in \mathbb{R}$ - расстояние от $x$ до $y$. 
	\end{definition} 
	Аксиомы делятся на 5 групп:
	\begin{enumerate}[I.]
		\item \bfseries Аксиомы принадлежности: \mdseries
		\begin{enumerate}[1.]
			\item Каждая прямая есть множество точек.
			\item Через любые две различные точки проходит прямая, и притом только одна.
			\item Существует хотя бы одна прямая.
		    \item Каждой прямой принадлежит хотя бы одна точка.
		    \item Вне каждой прямой существует хотя бы одна точка.
		\end{enumerate}
		\item \bfseries Аксиомы расстояния: \mdseries
		\begin{enumerate}[1.]
			\item $d(x, y) = 0 \Leftrightarrow x = y$
			\item $\forall x,y\in X \ \ d(x, y) = d(y, x)$
			\item (Неравенство треугольника) $d(x, z) \leqslant d(x, y) + d(y, z)$ 
		\end{enumerate}
		Пространство $X$ с выполненными аксиомами принадлежности и расстояния называется метрическим.
		\begin{definition}
			Точка $y$ лежит между $x$ и $z$, если $d(x, z) = d(x, y) + d(y, z)$. Множество точек между $x$ и $z$ называется отрезком $[x, z]$, а $d(x, z)$ - его длиной. 
		\end{definition}
		
		\item \bfseries Аксиомы порядка: \mdseries
		\begin{enumerate}[1.]
			\item Три точки лежат на одной прямой $\Leftrightarrow$ одна из них лежит между двумя другими.
			\item Любая точка $x$ прямой $l$ разбивает множество отличных от $x$ точек, лежащих на $l$, на два непустых подмножества так, что $x$ лежит между любыми двумя точками из разных подмножеств.
		\end{enumerate}
		\begin{definition}
			Одно из таких подмножеств точек прямой, взятое вместе с точкой $x$, называется лучом с началом в точке $x$.
		\end{definition}
		3. $\forall a\geqslant0 \ \exists!\ y$ на луче с началом в $x$ такая, что $d(x, y) = a$.
		\begin{definition}
		    Множество точек А плоскости называется выпуклым, если $\forall x, y\in A \ \ \ [x, y]\in A$.
		\end{definition}
		4. Любая прямая $l$ разбивает множество не принадлежащих ей точек плоскости на два непустых выпуклых подмножества.
		\begin{definition}
			Одно из таких подмножеств точек плоскости, взятое вместе с прямой $l$, называется полуплоскостью. Прямая $l$ называется граничной для этой полуплоскости. Если $l$ содержит луч, то эта полуплоскость примыкает к данному лучу.  
		\end{definition}
		\item \bfseries Аксиомы подвижности: \mdseries
		\begin{definition}
			Взаимно однозначное отображение плоскости на себя, сохраняющее расстояния, называется движением (изометрией).
		\end{definition}
		\begin{enumerate}[1.]
			\item Для любой пары лучей $l_{1}, l_{2}$ и примыкающих к ним полуплоскостей $\pi_{1}, \pi_{2}$ существует единственное движение $\phi$ такое, что $\phi(l_{1}) = l_{2}, \phi(\pi_{1}) = \pi_{2} $.
			\item  Для любой пары отрезков $[a_{1}, b_{1}], [a_{2}, b_{2}]$ равной ненулевой длины существуют ровно два движения $\phi_{1}, \phi_{2}$ таких, что $\phi_{1,2}([a_{1}, b_{1}]) = [a_{2}, b_{2}]$.
		\end{enumerate}
		\item \bfseries Аксиома параллельных прямых: \mdseries
		
		Через любую точку вне прямой $l$ можно провести ровно одну прямую, не пересекающуюся с $l$ (такая прямая называется параллельной к $l$).
		
		Данная аксиома разделяет евклидову и неевклидову геометрии.
	\end{enumerate}
	
	Методы аналитической геометрии основаны на том, что, задавая точки парами чисел (через координаты), мы можем получать разнообразные зависимости (например, уравнения множеств точек). Причем точку можно переопределить как пару чисел вместо абстрактного объекта, и все аксиомы останутся выполнены. То есть, рассматривая точки как арифметический объект, мы получим неотличимую по свойствам от абстрактной геометрии геометрию арифметическую, с которой уже можно работать по известным нам принципам.
	
	\section{Векторы и операции над ними}
	\subsection{Векторные пространства}
	Геометрические векторы в математике являются \bfseries свободными векторами \mdseries - классами эквивалентности направленных отрезков по уже известному нам отношению эквивалентности векторов.
	\begin{definition}
		Векторным (линейным) пространством (над полем $\mathbb{R}$) называется множество V с введенными на нем бинарными операциями "+": $V \times V \rightarrow V$ и "$*$": $\mathbb{R} \times V \rightarrow V$ , отвечающие следующим свойствам (аксиомам):
		
		$\forall \bar{a}, \bar{b}, \bar{c} \in V; \lambda, \mu \in \mathbb{R}:$
		\begin{enumerate}
			\item $\bar{a} + \bar{b} = \bar{b} + \bar{a}$ (коммутативность сложения);
			\item $(\bar{a} + \bar{b}) + \bar{c} = \bar{a} + (\bar{b} + \bar{c})$ (ассоциативность сложения);
			\item $\exists \bar{0} \in V: \bar{a} + \bar{0} = \bar{0} + \bar{a} = \bar{a}$ (существует нейтральный элемент по сложению - нулевой вектор);
			\item $\exists (-\bar{a}) \in V: \bar{a} + (-\bar{a}) = (-\bar{a}) + \bar{a} = \bar{0}$ (существует противоположный элемент по сложению);
			\item $\lambda(\mu\bar{a}) = (\lambda\mu)\bar{a}$ (ассоциативность умножения на числа);
			\item $(\lambda + \mu)\bar{a} = \lambda\bar{a}+ \mu\bar{a}$ (дистрибутивность по умножению);
			\item $\lambda(\bar{a} + \bar{b}) = \lambda\bar{a}+ \lambda\bar{b}$ (дистрибутивность по сложению);
			\item $1*\bar{a} = \bar{a}$.
		\end{enumerate}
	\end{definition}
	\begin{examples}
		Векторные пр-ва:
		\begin{itemize}
			\item $\mathbb{R}, \mathbb{R}^2, \mathbb{R}^n$:
			\item Функции;
			\item Многочлены;
			\item Многочлены степени $\leqslant n$.
		\end{itemize}
	\end{examples}
	\begin{remark}
		Св-ва векторных пространств:
		
		\begin{enumerate}
			
			\item $\bar{0}$ единственный.
			
			Пусть $\bar{0}_{1}, \bar{0}_{2}$ - нулевые векторы.
			
			Тогда $\bar{0}_{1} = \bar{0}_{1} + \bar{0}_{2} = \bar{0}_{2}$, ч.т.д.
			\item $-\bar{a}$ единственный.
			
			Пусть $-\bar{a}_{1}, -\bar{a}_{2}$ - противоположные к $\bar{a}$ векторы.
			
			Тогда $-\bar{a}_{1} = -\bar{a}_{1} + \bar{0} = -\bar{a}_{1} + (\bar{a} + -\bar{a}_{2}) = (-\bar{a}_{1} + \bar{a}) + (-\bar{a}_{2}) = \bar{0} + (-\bar{a}_{2}) = -\bar{a}_{2}$, ч.т.д.
			\item $\lambda * \bar{0} = \bar{0}$.
			
			$\lambda * \bar{0} = \lambda * (\bar{0} + \bar{0}) = \lambda * \bar{0} + \lambda * \bar{0}$
			
			Прибавив к обеим частям вектор, противоположный к $\lambda * \bar{0}$, получим $\lambda * \bar{0} = \bar{0}$, ч.т.д.
			\item $-(\lambda\bar{a}) = (-\lambda)\bar{a} = \lambda(-\bar{a})$.
			
			Нетрудно видеть, что все три вектора противоположны $\lambda\bar{a}$, а далее из п.2.
			\item $-\bar{a} = -1*\bar{a}$
			
			Следует из п.4.
			\item $\lambda\bar{a} = \bar{0} \Leftrightarrow$ либо $\lambda = 0$, либо $\bar{a} = \bar{0}$.
			
			Либо $\lambda = 0$, либо $\lambda \neq 0 \Rightarrow \frac{1}{\lambda}*\lambda*\bar{a} = \frac{1}{\lambda}\bar{0} = \bar{0} \Rightarrow \bar{a} = \bar{0}$, ч.т.д. 
		\end{enumerate}
	\end{remark}
	\subsection{Линейная зависимость векторов}
	\begin{definition}
		Сумма вида $\lambda_{1}\bar{x}_{1} + ... + \lambda_{n}\bar{x}_{n}$ называется линейной комбинацией векторов $\bar{x}_{1} ... \bar{x}_{n}$.
	\end{definition}
	\begin{definition}
		Если в линейной комбинации $\lambda_{1} = ... = \lambda_{n} = 0$, то она называется тривиальной, а иначе - нетривиальной.
	\end{definition}
	\begin{definition}
		Если вектор $\bar{x}$ равен линейной комбинации $\lambda_{1}\bar{x}_{1} + ... + \lambda_{n}\bar{x}_{n}$, то говорят, что он линейно выражается (раскладывается) через векторы $\bar{x}_{1}...\bar{x}_{n}$.
		(Сама линейная комбинация $\lambda_{1}\bar{x}_{1} + ... + \lambda_{n}\bar{x}_{n}$ называется выражением (разложением) вектора $\bar{x}$ через $\bar{x}_{1}...\bar{x}_{n}$)  
	\end{definition}
	\begin{definition}
		Множество векторов называется линейно зависимым, если существует равная нулю нетривиальная линейная комбинация векторов из этого множества. В противном случае оно называется линейно независимым.
	\end{definition}
	\begin{example}
		Система из двух векторов $\bar{a}, \bar{b}$ линейно зависима $\Leftrightarrow \bar{a} = \lambda\bar{b}$.
	\end{example}
	\begin{remark}
		Множество векторов линейно зависимо $\Leftrightarrow$ один из векторов этого множества линейно выражается через некоторые другие векторы этого множества.
	\end{remark}
	\begin{definition}
		Упорядоченное множество векторов называется системой векторов.
		(В системе векторов элементы могут повторяться)
	\end{definition}
	\begin{definition}
		Множество (система) векторов из векторного пространства V называется полным(полной) в $V$, если любой вектор $\bar{x}\in V$ линейно выражается через векторы этого множества.
	\end{definition}
	\begin{remark}
		$X \subset V$ полно в $V \Rightarrow \forall Y: X\subset Y$ полно в $V$.
	\end{remark}
	\begin{remark}
		$X \subset V$ линейно независимо в $V \Rightarrow \forall Y \subset X$ линейно независимо в $V$.
	\end{remark}
	\subsection{Базис векторного пространства}
	\begin{definition}
		Множество векторов $E$ в векторном пространстве $V$ называется базисом $V$, если $E$ линейно независимо и полно в $V$.
	\end{definition}
	\begin{definition}
		Векторное пространство, в котором существует конечный (состоящий из конечного числа векторов) базис, называется конечномерным. В противном случае оно называется бесконечномерным. 
	\end{definition}
	\begin{lemma}
		Если $X$ - конечное полное множество из $n$ векторов в векторном пространстве $V$ и $Y$ - линейно независимое множество векторов в $V$, то $Y$ конечно и число векторов в $Y \leqslant n$. 
	\end{lemma}
	\begin{proof}
		(пер.)
		Произвольно занумеруем векторы в $X: (\bar{x}_{1},...,\bar{x}_{n})$.
		Будем по одному добавлять в эту систему векторы из $Y$ и одновременно выкидывать векторы из $X$ так, чтобы система оставалась полной.
		
		Пусть за $k$ шагов ($0\leqslant k \leqslant n$) мы добавили некоторые $\bar{y}_{1},...,\bar{y}_{k}$ и выкинули какие-то $k$ векторов из $X$ - осталась система ($\bar{y_{1}},...,\bar{y_{k}},\bar{x}_{i_{1}},...,\bar{x}_{i_{n-k}}$).
		Возьмём $\bar{y}_{k+1}$ из $Y$ (если такого нет, то в $Y \leqslant n$ векторов, что нам и нужно), и добавим его в систему. Так как до этого система оставалась полной, $\bar{y}_{k+1}$ выражается через ($\bar{y_{1}},...,\bar{y_{k}},\bar{x}_{i_{1}},...,\bar{x}_{i_{n-k}}$), причём какой-то $\bar{x}_{i_{j}}$ входит в это разложение с коэффициентом, не равным нулю (иначе противоречие с линейной независимостью $Y$ - $\bar{y}_{k+1}$ выразился через $\bar{y}_{1},...,\bar{y}_{k}$).
		
		Тогда $\bar{x}_{i_{j}}$ выражается через другие векторы системы и $\bar{y}_{k+1}$ (в выражении $\bar{y}_{k+1}$ перенесём всё, кроме $\bar{x}_{i_{j}}$ в другую часть и разделим на коэффициент перед ним).
		А так как ($\bar{y_{1}},...,\bar{y_{k+1}},\bar{x}_{i_{1}},...,\bar{x}_{i_{n-k}}$) - полная, эта же система без $\bar{x}_{i_{j}}$. очевидно, останется полной.

		Пусть смогли проделать $n$ таких шагов. Тогда имеем систему ($\bar{y}_{1},...,\bar{y}_n$). Если в $Y$ есть ещё векторы, то они с одной стороны выражаются через векторы системы из её полноты, а с другой - не выражаются через них из линейной независимости $Y$. Противоречие, т.е. в $Y$ не может оказаться больше $n$ векторов, ч.т.д.   
	\end{proof}
	\begin{theorem}
		Если в векторном пространстве есть конечный базис. то все базисы в нём конечны и содержат одинаковое количество векторов.
	\end{theorem}
	\begin{proof}
		Пусть $\bar{e}_{1},...,\bar{e}_{n}$ - конечный базис в $V$. Любой другой базис $V$ линейно независим, т.е. по лемме содержит $k \leqslant n$ векторов, а с другой стороны полон, т.е. первый базис по лемме содержит $n \leqslant k$ векторов. Отсюда $n=k$, ч.т.д.
	\end{proof}
	\begin{definition}
		Количество векторов в любом базисе векторного пространства $V$ называется размерностью $V$ и обозначается $dim V$.
	\end{definition}
	\begin{examples}
		$dim \ {\bar{0}} = 0, dim \ \pi (= dim \ \mathbb{R}^2) = 2, dim \ \mathbb{R}^3 = 3$.
	\end{examples}
	\begin{theorem}
		В конечномерном векторном пространстве выражение любого вектора через базис определяется однозначно.
	\end{theorem}
	\begin{proof}
		Если $\bar{x} = \lambda_{1}\bar{e}_{1} + ... + \lambda_{n}\bar{e}_{n} = \lambda'_{1}\bar{e}_{1} + ... + \lambda'_{n}\bar{e}_{n}$, то $\bar{x} - \bar{x} = \bar{0} = (\lambda_{1} - \lambda'_{1})\bar{e}_{1} + ... + (\lambda_{n} - \lambda'_n)\bar{e}_{n}$. Если эти два разложения различны, то равная нулю линейная комбинация базисных векторов нетривиальна, что противоречит линейной независимости базиса. То есть двух различных разложений быть не может, ч.т.д. 
	\end{proof}
	\begin{definition}
		Пусть $V$ - конечномерное векторное пространство и $\bar{e}_{1},...,\bar{e}_{n}$ - базис в нём. Коэффициенты $\lambda_{1},...,\lambda_{n}$ в выражении любого вектора $x \in V$ через эти базисные векторы называются координатами вектора $x$ в базисе $\bar{e}_{1},...,\bar{e}_{n}$. ($\lambda_{k}$  называется $k$-й координатой)
	\end{definition}
	\begin{remark}
		Векторы в $n$-мерном векторном пространстве находятся во взаимно однозначном соответствии с упорядоченной строкой из $n$ чисел из $\mathbb{R}$ (например, векторы ассоциированного с евклидовой плоскостью векторного пространства соответствуют парам чисел)
		Таким образом можно задать операции сложения и умножения на число векторов плоскости через операции над числами, проводимыми покоординатно.
	\end{remark}
	Однако элементы плоскости (как множества) - точки, а не векторы, поэтому для работы непосредственно с плоскостью необходимо ввести ещё одно определение.
	\subsection{Аффинные пространства}
	\begin{definition}
		Аффинное пространство - тройка ($X, V, +$) (обычно обозначается $\mathbb{A}$), где $X$ - множество (точек), $V$ - векторное пространство, а $+$ операция:  $X \times V \rightarrow X$, для которых выполнены аксиомы:
		\begin{enumerate}
			\item $\forall A \in X, \forall \bar{a}, \bar{b} \in V: A+(\bar{a}+\bar{b}) = (A+\bar{a})+\bar{b}$;
			\item $\forall A \in X: A + \bar{0} = A$;
			\item $\forall A, B \in X \ \ \exists! \ \bar{a} \in V: A + \bar{a} = B$. Обозначается $\bar{a} = \overrightarrow{AB}$.
		\end{enumerate}
	\end{definition}
	Если зафиксировать какую-нибудь точку $O \in X$, возникает взаимно однозначное соответствие между точками $A$ и их радиус-векторами $\overrightarrow{OA}$.
	\begin{definition}
		Репер (система координат) в аффинном пространстве $(X, V, +)$ - пара $(O, E)$, где $O \in X$ и $E$ - базис в $V$. Точка $O$ называется началом координат (отсчёта). Координаты точки A в $(O, E)$ - координаты её радиус-вектора $\overrightarrow{OA}$ в базисе $E$.
	\end{definition}
	\begin{remark}
		Для аффинного пространства верно:
		\begin{enumerate}
			\item Если $A = (x_{1},...,x_{n}),  \bar{a} = (y_{1},...,y_{n})$, то $A + \bar{a} = (x_{1} + y_{1},...,x_{n} + y_{n})$.
			\item Если $A = (a_{1},...,a_{n}),  B = (b_{1},...,b_{n})$, то $\overrightarrow{AB} = (b_{1} - a_{1},...,b_{n} - a_{n})$.
		\end{enumerate}
		(Следует из сложения векторов)
	\end{remark}
	\begin{definition}
		Если $\mathbb{A} = (X, V, +)$ - аффинное пространство, то говорят, что $V$ - векторное пространство, ассоциированное с $\mathbb{A}$.    
	\end{definition}
	\begin{definition}
		$\mathbb{A}$ называется конечномерным, если ассоциированное с ним $V$ конечномерно. В этом случае $dim \mathbb{A}$ (размерность $\mathbb{A}$) равна $dim V$.
	\end{definition}
	Теперь точки аффинного пространства аналогично векторам можно ассоциировать с наборами чисел. Однако для ассоциирования евклидовой плоскости и её аксиом с двумерным аффинным пространством, необходимы отвечающие аксиомам понятия прямой и расстояния.
	\subsection{Подпространства}
	\begin{definition}
		Векторным подпространством векторного пространства $V$ называется непустое множество $V_{1} \subset V$ такое. что $\forall\bar{x}, \bar{y}\in V_{1}: \bar{x} + \bar{y} \in V_{1}, \lambda\bar{x} \in V_{1} (\forall\lambda \in\mathbb{R})$.
	\end{definition}
	\begin{remark}
		Определение эквивалентно следующему: множество $V_{1} \subset V$ - векторное подпространство $V$, если $V_{1}$ является векторным пространством относительно операций $+$ и $*$, определённых для $V$.
		(Доказательство осуществляется путём прямой проверки аксиом векторного пространства для $V_{1}$)
	\end{remark}
	Введём несколько определений аффинного подпространства и докажем их эквивалентность.
	\begin{definition}
		Аффинным подпространством аффинного пространства $\mathbb{A} = (X, V, +)$ называется
		\begin{enumerate}
			\item его непустое подмножество вида $A + V_{1} = {A + \bar{a}:\bar{a} \in V_{1}}$, где $V_{1}$ - векторное подпространство $V$ и $A \in X$ - точка;
			\item тройка $(X_{1}\subset X, V_{1} \subset V, +_{1})$, где $V_{1}$ - векторное подпространство $V$ и операция $+_{1} = +$,  для которой $\forall A,B \in X_{1}, \forall \bar{a} \in V_{1}: A + \bar{a} \in X_{1}, \overrightarrow{AB} \in V_{1}$;
			\item тройка $(X_{1}\subset X, V_{1} \subset V, +_{1})$, где $V_{1}$ - векторное подпространство $V$ и операция $+_{1} = +$,  которая сама является аффинным пространством.
		\end{enumerate}
	\end{definition}
	\begin{subtheorem}
		Приведённые определения эквивалентны.
	\end{subtheorem}
	\begin{proof}
		Докажем следующие следствия:

		$\circled{1}\Rightarrow\circled{2}$  Пусть $P = A + \bar{a}, Q = A + \bar{b}$. Тогда $\overrightarrow{PQ} = \bar{b} - \bar{a}$ (в силу единственности такого вектора), т.е. $\overrightarrow{PQ} \in V_{1}$. Второе необходимое свойство $\circled{2}$ очевидно выполнено.

		$\circled{2}\Rightarrow\circled{1}$  Пусть $X_{1}, V_{1}$ удовлетворяют $\circled{2}$. Зафиксируем произвольную $A \in X_{1}$. $\forall B\in X_{1}$ имеем $B = A + \overrightarrow{AB}$, причём $A \in X_{1}, \overrightarrow{AB} \in V_{1} \Rightarrow B \in X_{1}$.
		
		Эквивалентность $\circled{2}\Leftrightarrow\circled{3}$ очевидна из определения аффинного пространства.
	\end{proof}
	\begin{definition}
		Прямая в аффинном пространстве - его одномерное аффинное подпространство. \\Плоскость (двумерная) в аффинном пространстве - его двумерное аффинное подпространство.
	\end{definition}
	\begin{definition}
		Единственный вектор в любом базисе векторного пространства, ассоциированного с одномерным аффинным пространством, называется направляющим вектором этого аффинного пространства.
	\end{definition}
	\begin{remark}
		У любой прямой множество точек имеет вид $A + V^1$, где $A$ - фиксированная точка.
	\end{remark}
	\begin{proof}
		Направляющий вектор $\bar{e}$ любой прямой ненулевой, т.к. базис одномерного векторного пространства линейно независим.\\
		Так как любой вектор $V^1$ выражается через базис, $\forall\bar{x} \in V^1 \ \exists\lambda\in\mathbb{R}: \bar{x} = \lambda\bar{e}$. \\
		Также $\forall\lambda\in\mathbb{R} \ \lambda\bar{e} \in V^1$, так как $V^1$ - векторное пространство. \\
		Отсюда любое ассоциированное с прямой векторное пространство имеет вид $\{\lambda\bar{e}:\lambda \in \mathbb{R}\}$, где $\bar{e}$ - фиксированный ненулевой вектор, а тогда, взяв любую точку одномерного аффинного пространства, получим, что всё его множество точек имеет необходимый вид.
	\end{proof}
	\begin{subtheorem}
		Для прямых в двумерном аффинном пространстве выполнены евклидовы аксиомы принадлежности.
	\end{subtheorem}
	\begin{proof}
		Пусть $\mathbb{A}^2 = (X^2, V^2, +)$ - данное двумерное аффинное пространство, $\bar{e}_{1}, \bar{e}_{2}$ - некоторый базис $V^2. l = \{O + \bar{e}_{1}\}$ - некоторая прямая (из предыдущего замечания). Тогда в $\mathbb{A}^2$ существует прямая, являющаяся множеством точек, на которой есть хотя бы одна точка - $O$, а вне её есть хотя бы одна точка - $O + \bar{e}_{2}$. (не на прямой, т.к. иначе $\bar{e}_{2} = \lambda\bar{e}_{1}$)\\
		Через любые две точки можно провести прямую ($\forall A,B \in X^2$ возьмём $\{A + \overrightarrow{AB}\}$ - прямую, проходящую через $A, B$), и притом только одну: если $l = \{O +\lambda\bar{x}: \lambda \in \mathbb{R}\}$ - прямая и $A,B \in l$, то $A = O + \lambda\bar{x}, B = O + \mu\bar{x} \Rightarrow B = A + (\mu - \lambda)\bar{x} \Rightarrow \overrightarrow{AB} = (\mu - \lambda)\bar{x}$. Тогда для любой $C \in l$ имеем $C = O + \eta\bar{x} = A + \overrightarrow{AO} + \eta\bar{x} = A + (\eta - \lambda)\bar{x} = A + \frac{\eta - \lambda}{\mu - \lambda}\overrightarrow{AB}$, т.е. любая проходящая через две точки прямая совпадает с этой.\\
	\end{proof}
	\begin{theorem}
		Любые два непропорциональных вектора $\bar{a}, \bar{b} \in V^2$ образуют базис двумерного векторного пространства $V_{2}$.
	\end{theorem}
	\begin{proof}
		Пусть $\bar{e}_{1}, \bar{e}_{2}$ - произвольный базис $V^2$. Тогда $\bar{a} = x\bar{e}_{1} + y\bar{e}_{2}, b = x'\bar{e}_{1} + y'\bar{e}_{2}$. Пусть без ограничения общности $x \neq 0$ (т.к. вектор $\bar{a} \neq \bar{0}$, хотя бы один коэффициент в его разложении $\neq0$). Тогда $\bar{e}_{1} = \frac{1}{x}\bar{a} - \frac{y}{x}\bar{e}_{2}$. Подставим во второе выражение: $\bar{b} = x'(\frac{1}{x}\bar{a} - \frac{y}{x}\bar{e}_{2}) + y'\bar{e}_{2} \Rightarrow \bar{e}_{2} = \frac{x}{xy'-yx'}\bar{b} - \frac{x'}{xy'-yx'}\bar{a}$. Подставив это в первое выражение, получим, что $\bar{e}_{1}$ также выражается через $\bar{a}, \bar{b}$. Тогда система $\bar{a}, \bar{b}$ - полная (из полноты базиса), а тогда $\bar{a}, \bar{b}$ - базис по определению, ч.т.д. 
	\end{proof}
	\begin{subtheorem}
		Для прямых в двумерном аффинном пространстве выполнена аксиома параллельных прямых.
	\end{subtheorem}
	\begin{proof}
		Если $l = \{O + \lambda\bar{x}: \lambda\in\mathbb{R}\}$ - прямая и $O_{1}\notin l$, то докажем, что $l_{1} = \{O_{1} + \lambda\bar{x}: \lambda\in\mathbb{R}\}$ - единственная прямая, проходящая через $O_{1}$ и параллельная $l$.
		Предположим, что у них есть общая точка: пусть $X = O + \lambda\bar{x} = O_{1} + \mu\bar{x}$. Тогда $O_{1} = O + (\lambda - \mu)\bar{x} \Rightarrow O_{1} \in l$. Противоречие, т.е. прямая $l_{1}$ действительно параллельна $l$.\\
		Пусть $l_{2}$ - другая прямая, проходящая через $O_{1}$. Тогда $l_{2} = \{O_{1} + \lambda\bar{y}: \lambda\in\mathbb{R}\}$, причём $\bar{y}$ не пропорционален $\bar{x}$ (иначе прямые совпадают). Тогда $\bar{x}, \bar{y}$ - базис $V^2$, то есть $\exists\alpha,\beta \in \mathbb{R}: \overrightarrow{OO_{1}} = \alpha\bar{x} + \beta\bar{y}$. Отсюда $O_{1} = O + \alpha\bar{x} + \beta\bar{y} \Rightarrow O + \alpha\bar{x} = O_{1} - \beta\bar{y}$, причём $O + \alpha\bar{x} \in l$ и $O_{1} - \beta\bar{y} \in l_{2}$, т.е. $l$ и $l_{2}$ имеют общую точку. Отсюда параллельная $l$ прямая, проходящая через $O_{1}$, единственная, ч.т.д. 
	\end{proof}
	\subsection{Скалярное произведение. Расстояния и углы}
	\begin{definition}
		Пусть $V$ - векторное пространство. Скалярным произведением в $V$ называется функция $(\ ,\ ) : V \times V \rightarrow \mathbb{R}$ со свойствами:
		\begin{enumerate}
			\item $(\bar{x}, \bar{x}) \geqslant 0 \ \forall \bar{x} \in V$, причём $(\bar{x}, \bar{x}) = 0 \Leftrightarrow \bar{x} = \bar{0}$ (положительная определённость);
			\item $(\bar{x}, \bar{y}) = (\bar{y}, \bar{x}) \ \forall \bar{x}, \bar{y} \in V$ (коммутативность);
			\item $(\alpha\bar{x} + \beta\bar{y}, \bar{z}) = \alpha(\bar{x}, \bar{z}) + \beta(\bar{y}, \bar{z}) \ \forall \bar{x}, \bar{y}, \bar{z} \in V, \alpha,\beta \in \mathbb{R}$ (линейность по первому аргументу)
		\end{enumerate} 
	\end{definition}
	Из коммутативности выполнена и линейность по второму аргументу, т.е. скалярное произведение - билинейная функция.
	\begin{definition}
		Длиной вектора называется величина $\sqrt{(\bar{x}, \bar{x})}$.
	\end{definition}
	\begin{definition}
		Расстоянием (евклидовым) между точками $A,B \in \mathbb{A}$ называется длина вектора $\overrightarrow{AB}$. Будем обозначать $d(A, B)$ как $|\overrightarrow{AB}|$. 
	\end{definition}
	\begin{remark}
		Зная длины всех векторов, скалярное произведение можно восстановить по формуле $(\bar{x}, \bar{y}) = \frac{1}{2}(|\bar{x} + \bar{y}|^2 - |\bar{x}|^2 - |\bar{y}|^2)$. Это несложно проверить: \\
		$\frac{1}{2}(|\bar{x} + \bar{y}|^2 - |\bar{x}|^2 - |\bar{y}|^2) = \frac{1}{2}((\bar{x} + \bar{y}, \bar{x} + \bar{y}) - (\bar{x}, \bar{x}) - (\bar{y}, \bar{y})) = \frac{1}{2}(2(\bar{x}, \bar{y})) = (\bar{x}, \bar{y})$.
	\end{remark}
	\begin{theorem}[Неравенство Коши-Буняковского]
		$\forall \bar{a}, \bar{b} \in V (\bar{a}, \bar{b}) \leqslant \sqrt{(\bar{a}, \bar{a})(\bar{b}, \bar{b})}$, причём равенство достигается только при $\bar{a} = \lambda\bar{b}$.
	\end{theorem}
	\begin{proof}
		Рассмотрим выражение $(\bar{a} + t\bar{b}, \bar{a} + t\bar{b})$. Оно равно нулю $\Leftrightarrow \bar{a} = -t\bar{b}$, т.е. может быть равно нулю не более чем при одном $t$. С другой стороны
		$(\bar{a} + t\bar{b}, \bar{a} + t\bar{b})$ = $(\bar{a}, \bar{a}) + 2(\bar{a}, \bar{b})t + (\bar{b}, \bar{b})t^2$ - квадратный трёхчлен относительно $t$. Его дискриминант равен $4(\bar{a}, \bar{b}) - 4(\bar{a}, \bar{a})(\bar{b}, \bar{b})$, а из первого рассуждения знаем, что дискриминант $\leqslant 0$, причём равенство достигается только в случае коллинеарности $\bar{a}$ и $\bar{b}$. Отсюда $(\bar{a}, \bar{b}) \leqslant \sqrt{(\bar{a}, \bar{a})(\bar{b}, \bar{b})}$, ч.т.д.
	\end{proof}
	\begin{subtheorem}
		Точки $A, B, C$ лежат на одной прямой $\Leftrightarrow$ одна из них лежит между двумя другими.
	\end{subtheorem}
	\begin{proof} $\\$
		$(\Rightarrow)$ Пусть $A, B, C$ лежат на одной прямой. Тогда $\exists O \in X^2, \bar{v} \in V^2, \alpha,\beta,\gamma \in \mathbb{R}: A = O + \alpha\bar{v}, B = O + \beta\bar{v}, C = O + \gamma\bar{v}$. Пусть без ограничения общности $\alpha \leqslant \beta \leqslant \gamma$. Тогда $d(A, B) = |\overrightarrow{AB}| = (\beta - \alpha)|\bar{v}|, d(B, C) = |\overrightarrow{BC}| = (\gamma - \beta)|\bar{v}|, d(A, C) = |\overrightarrow{AC}| = (\gamma - \alpha)|\bar{v}|$. Отсюда видно, что $d(A, C) = d(A, B) + d(B, C).\\$
		$(\Leftarrow)$ Пусть $d(A, C) = d(A, B) + d(B, C)$. Обозначим $\bar{a} = \overrightarrow{AB}, \bar{b} = \overrightarrow{BC}, \bar{c} = \overrightarrow{AC}$. Тогда $\sqrt{(\bar{c}, \bar{c})} = \sqrt{(\bar{a} + \bar{b}, \bar{a} + \bar{b})} = \sqrt{(\bar{a}, \bar{a})} + \sqrt{(\bar{b}, \bar{b})} \Rightarrow (\bar{a} + \bar{b}, \bar{a} + \bar{b}) = (\bar{a}, \bar{a}) + 2\sqrt{(\bar{a}, \bar{a})(\bar{b}, \bar{b})} + (\bar{b}, \bar{b}) \Rightarrow 2(\bar{a}, \bar{b}) = 2\sqrt{(\bar{a}, \bar{a})(\bar{b}, \bar{b})}$. Из неравенства Коши-Буняковского знаем, что равенство достигается при $\bar{a} = \lambda\bar{b}$, т.е. $\overrightarrow{AB}$ коллинеарен $\overrightarrow{BC}$, то есть $A, B, C$ лежат на одной прямой.  
	\end{proof}
	\begin{definition}
		Величиной угла между ненулевыми векторами $\bar{a}, \bar{b}$  называется число $arccos\frac{(\bar{a}, \bar{b})}{|\bar{a}||\bar{b}|}$ (из н. Коши-Буняковского $|\frac{(\bar{a}, \bar{b})}{|\bar{a}||\bar{b}|}| \leqslant 1$).
	\end{definition}
	\begin{definition}
		Конечномерное аффинное (векторное) пространство вместе со скалярным произведением называется точечно-евклидовым (евклидовым) пространством. Двумерное точечно-евклидово пространство называется евклидовой плоскостью.
	\end{definition}
	\subsection{Проектирование точек и векторов}
	\begin{definition}
		Пусть задано два векторных подпространства $V_{1}, V_{2}$ векторного пространства $V$ такие, что $V_{1} \cap V_{2} = \bar{0}$ и $V_{1} + V_{2} = V$ (обозначается $V = V_{1} \oplus V_{2}$). Тогда сумма $\bar{x} = \bar{x}_{1} + \bar{x}_{2}$, где $\bar{x} \in V, \bar{x}_{1} \in V_{1}, \bar{x}_{2} \in V_{2}$, определена единственно. (Следует, например, из того, что в любом базисе $V$ каждый его вектор лежит либо в $V_{1}$, либо в $V_{2}$, тогда разложение в эту сумму соответствует единственному разложению по базису). Проекцией вектора $\bar{x} \in V$ на $V_{1}$ параллельно $V_{2}$ называется слагаемое $\bar{x_{1}}$ этой суммы. 
	\end{definition}
	\begin{definition}
		Пусть задано два аффинных подпространства $\mathbb{A}_{1} = (X_{1}, V_{1}, +),\\ \mathbb{A}_{2} = (X_{2}, V_{2}, +)$ аффинного пространства $\mathbb{A} = (X, V, +)$ такие, что $V = V_{1} \oplus V_{2}$. Проекцией точки $P \in \mathbb{A}$ на $\mathbb{A}_{1}$ параллельно $\mathbb{A}_{2}$ - точка $P_{1} = A_{1} + \bar{v}$, где $A_{1}$ - произвольная точка из $X_{1}$, а $\bar{v}$ - проекция $\overrightarrow{A_{1}P}$ на $V_{1}$ параллельно $V_{2}$.
		(Очевидно, что от выбора $A_{1}$ расположение проеции не зависит)
	\end{definition}
	\begin{example}
		Рассмотрим координаты точки евклидовой плоскости относительно прямоугольной системы координат. \\
		Найдём проекцию точки $A = (x, y)$ на прямую $Oy$ параллельно прямой $Ox$. По определению это точка (назовём её $A_{y}$), равная $O + \bar{v}$, где $\bar{v}$ - проекция $\overrightarrow{OP}$ на векторное пространство прямой $Oy$ параллельно $Ox$. $\overrightarrow{OP} = \{x, y\} = x\bar{e}_{1} + y\bar{e}_{2}$. Отсюда $\bar{v} = y\bar{e}_{2} = \{0, y\}$, то есть $A_{y} = (0, y)$. Аналогично $A_{x} = (x, 0)$. 
	\end{example}
	\subsection{Ортонормированный базис и прямоугольная система координат}
	\begin{definition}
		Векторы $\bar{a}, \bar{b}$ называются ортогональными (перпендикулярными), если $(\bar{a}, \bar{b}) = 0$.
	\end{definition}
	\begin{definition}
		Базис векторного пространсва $V$ со скалярным произведением называется ортонормированным, если все его векторы попарно ортогональны и имеют длину 1.
	\end{definition}
	\begin{definition}
		Система координат в точечно-евклидовом пространстве называется прямоугольной, если её базис ортонормированный.
	\end{definition}
	\begin{subtheorem}
		В точечно-евклидовом пространстве верно следующее выражение скалярного произведения через координаты векторов: если в некотором базисе $ (\bar{e}_{1},...,\bar{e}_{n}) \ \bar{x} = \begin{pmatrix} x_{1} \\ \vdots \\ x_{n} \end{pmatrix}, \bar{y} = \begin{pmatrix} y_{1} \\ \vdots \\ y_{n} \end{pmatrix}$, то $(\bar{x}, \bar{y}) = \sum \limits_{i=1}^n x_{i} \cdot \sum \limits_{j=1}^n y_{j}(\bar{e}_{i}, \bar{e}_{j})$.
	\end{subtheorem}
	\begin{proof}
		$(\bar{x}, \bar{y}) = (x_{1}\bar{e}_{1}+...+x_{n}\bar{e}_{n}, y_{1}\bar{e}_{1}+...+y_{n}\bar{e}_{n}) = \sum \limits_{i=1}^{n}(x_{i}\bar{e}_{i},y_{1}\bar{e}_{1}+...+y_{n}\bar{e}_{n}) = \sum \limits_{i=1}^{n}x_{i}(\bar{e}_{i},y_{1}\bar{e}_{1}+...+y_{n}\bar{e}_{n}) = \sum \limits_{i=1}^n x_{i} \cdot \sum \limits_{j=1}^n y_{j}(\bar{e}_{i}, \bar{e}_{j})$
	\end{proof}
	\begin{remark}
		В случае, когда базис ортонормированный, имеем $(e_{i}, e_{j}) = \delta_{ij}$, т.е. $(\bar{x}, \bar{y}) = x_{1}y_{1}+...+x_{n}y_{n}$. То есть в прямоугольной системе координат длина вектора вычисляется по формуле $|\bar{x}| = \sqrt{x_{1}^2+...+x_{n}^2}$, а расстояние между точками $P = (x_{1},...,x_{n}), Q = (y_{1},...,y_{n})$ выражается как $|PQ| = |\overrightarrow{PQ}| = \sqrt{(y_{1} - x_{1})^2+...+(y_{n}- x_{n})^2}$.
	\end{remark}
	Пусть $\mathbb{A}^n = (X, V^n, +)$ - $n$-мерное точечно-евклидово пространство.
	\begin{subtheorem}
		В $V^n$ любая линейно независимая система из $n$ векторов образует базис.
	\end{subtheorem}
	\begin{proof}
		Предположим, что в $V^n$ существует неполная линейно независимая система из n векторов. Т.к. система не полная, существует вектор из $V^n$, не выражающийся через векторы этой системы, т.е. этот вектор можно добавить в систему без потери линейной независимости. Но по лемме-аналогу ОЛЛЗ линейно независимая система в $V^n$ не может иметь $> n$ векторов. Противоречие, т.е. любая линейно независимая система из $n$ векторов является полной, а значит и базисом, ч.т.д. 
	\end{proof}
	\begin{subtheorem}
		Если $\bar{e}_{1},...,\bar{e}_{n}$ - попарно ортогональные ненулевые векторы в евклидовом пространстве, то $\bar{e}_{1},...,\bar{e}_{n}$ линейно независимы.
	\end{subtheorem}
	\begin{proof}
		Предположим противное. Пусть один из векторов (без ограничения общности $\bar{e}_{n}$) линейно выражается через остальные: $\bar{e}_{n} = \lambda_{1}\bar{e}_{1} +...+ \lambda_{n-1}\bar{e}_{n-1}$. Тогда запишем квадрат его длины:
		$|\bar{e}_{n}|^2 = (\bar{e}_{n}, \bar{e}_{n}) = (\bar{e}_{n}, \lambda_{1}\bar{e}_{1} +...+ \lambda_{n-1}\bar{e}_{n-1}) = \sum \limits_{i=1}^{n-1} \lambda_{i}(\bar{e}_{n}, \bar{e}_{i}) = 0$ (т.к. $\bar{e}_{n}$ ортогонален всем остальным векторам). Отсюда $|\bar{e}_{n}| = 0$, и притом $\bar{e}_{n}$ ненулевой. Противоречие, т.е. никакой вектор системы не выражается через остальные, а значит система линейно независима. ч.т.д.
	\end{proof}
	\begin{theorem}
		В любом евклидовом пространстве существует ортонормированный базис.
	\end{theorem}
	\begin{proof}
		(пер.)
		Индукция по $n$ - размерности пространства:

		База: $n = 1$ - очевидно, что существует вектор длины 1, который составляет ортонормированный базис одномерного пространства;

		Шаг: Пусть в любом $n$-мерном пространстве существует ортонормированный базис. Рассмотрим пространство $V$ размерности $n+1$ и выберем базис какого-то $n$-мерного подпространства $W$ (пусть $(\bar{e}_{1},...,\bar{e}_{n})$). Найдём вектор, ортогональный всем выбранным векторам. Так как базис $W$ не полон в $V$, к нему можно добавить ещё один вектор $x \in V$ без потери линейной независимости $\Rightarrow$ $(\bar{e}_{1},...,\bar{e}_{n}, \bar{x})$ - базис в $V$ (ЛНЗ система из $n+1$ векторов). \\
		Теперь необходимо представить $\bar{x}$ как следующую сумму: $\bar{x} = \lambda_{1}\bar{e}_{1} + ... + \lambda_{n}\bar{e}_{n} + \bar{e}_{n+1}$, где $\bar{e}_{n+1}$ ортогонален $\bar{e}_{1},...,\bar{e}_{n}$. Тогда $\bar{e}_{n+1} = \bar{x} - \lambda_{1}\bar{e}_{1} - ... - \lambda_{n}\bar{e}_{n}$. Рассмотрим $(\bar{e}_{n+1}, \bar{e}_{k}) = (\bar{x} - \lambda_{1}\bar{e}_{1} - ... - \lambda_{n}\bar{e}_{n}, \bar{e}_{k}) = (\bar{x}, \bar{e}_{k}) - \lambda_{1}(\bar{e}_{1}, \bar{e}_{k}) - ... - \lambda_{n}(\bar{e}_{n}, \bar{e}_{k})$. Так как $\bar{e}_{k}$ ортогонально всем этим векторам, кроме $\bar{e}_{k}$ и $\bar{x}$, это выражение равно $(\bar{x}, \bar{e}_{k}) - \lambda_{k}(\bar{e}_{k}, \bar{e}_{k})$. Отсюда при $\lambda_{k} = \frac{(\bar{x}, \bar{e}_{k})}{(\bar{e}_{k}, \bar{e}_{k})}$ векторы $\bar{e}_{n+1}$ и $\bar{e}_{k}$ ортогональны (зависит только от $\lambda_{k}$). Составив таким образом все $\lambda_{1},...,\lambda_{n}$, получим выражение вектора $\bar{e}_{n+1}$, ортогональный всем векторам базиса $W$.
		Таким образом, векторы полученной системы $\bar{e}_{1},...,\bar{e}_{n+1}$ попарно ортогональны (по предположению индукции) $\Rightarrow$ линейно независимы $\Rightarrow$ образуют базис в $V$. Разделив $\bar{e}_{n+1}$ на его длину, получим, что все векторы базиса попарно ортогональны и имеют длину 1 $\Rightarrow V$ имеет ортонормированный базис, ч.т.д.
	\end{proof}
	\begin{consequense}
		Любую систему ортогональных векторов длины 1 в векторном пространстве можно дополнить до ортонормированного базиса.
	\end{consequense}
	\section{Прямые}
	\subsection{Уравнение прямой}
	\begin{definition}
		Уравнением (либо уравнениями) множества точек будем называть уравнение со следующим свойством: точка принадлежит множеству тогда и только тогда, когда её координаты удовлетворяют уравнению. 
	\end{definition}
	\begin{formulas}[\bfseries уравнения прямой\mdseries]


		Пусть $l$ - прямая на плоскости: $l = \{X: \overrightarrow{OX} = \overrightarrow{OM} + t\bar{v}\}$, где $M$ - точка прямой, $\bar{v}$ - её направляющий вектор. Если $M=\begin{pmatrix}x_{0}\\y_{0}\end{pmatrix},\bar{v}=\begin{pmatrix}a\\b\end{pmatrix}$, то из совпадения координат совпадающих векторов $\overrightarrow{OX}$ и $(\overrightarrow{OM}+t\bar{v})$ для $X \in l$ верно: $\begin{cases}x = x_{0} + at \\ y = y_{0} + bt\end{cases}$ (\bfseries параметрические уравнения\mdseries)\\
		Выразим $t$ из первого уравнения и подставим во второе - получим: $\frac{x-x_{0}}{a} = \frac{y-y_{0}}{b}$ (\bfseries каноническое уравнение прямой\mdseries).
		(Заметим, что данное выражение не определено при нулевых $a$ или $b$, но очевидно, что они не равны нулю одновременно, а запись, где одна из дробей имеет знаменатель 0, иногда используется, поэтому здесь и далее случай равенства нулю знаменателя может не рассматриваться как отдельный и будет означать, что числитель должен равняться 0)\\
		Если известно, что прямой принадлежат $M = \begin{pmatrix}x_{0}\\y_{0}\end{pmatrix}, N = \begin{pmatrix}x_{1}\\y_{1}\end{pmatrix}$, то $\overrightarrow{MN} = \begin{pmatrix}x_{1} - x_{0}\\y_{1} - y_{0}\end{pmatrix}$ - направляющий вектор, т.е. каноническое уравнение прямых принимает вид $\frac{x-x_{0}}{x_{1}-x_{0}} = \frac{y-y_{0}}{y_{1}-y_{0}}$ (\bfseries уравнение прямой по двум точкам\mdseries).\\
		Домножим каноническое уравнение прямой на знаменатели: $\frac{x-x_{0}}{a} = \frac{y-y_{0}}{b} \Rightarrow bx-bx_{0}=ay-ay_{0} \Rightarrow bx - ay + (ay_{0}-bx_{0}) = 0$. Такое уравнение обычно называют \bfseries общим уравнением прямой\mdseries \ и записывают как $Ax + By + C = 0$.
	\end{formulas}
	\begin{remark}
		Для прямых в пространстве подобным образом выводятся параметрические и каноническое уравнения. 
	\end{remark}
	\begin{remark}
		Заметим также, что из итоговой формулы вывода общего уравнения ($bx - ay + (ay_{0}-bx_{0}) = 0 \Rightarrow Ax + By + C = 0$) следует, что для прямой $Ax + By +C = 0$ вектор ($B, -A$) (а соответственно и ($-B, A$)) является направляющим.
	\end{remark}
	\begin{subtheorem}
		$Ax + By + C = 0$ является уравнением прямой $\Leftrightarrow A$ и $B$ не равны нулю одновременно.
	\end{subtheorem}
	\begin{proof}
		$\\ \Rightarrow$ Если $Ax + By + C = 0$, то её направляющий вектор ненулевой, а значит вектор $(-B, A)$ ненулевой, то есть одна из его координат $\neq 0$.
		$\\ \Leftarrow$ Пусть без ограничения общности $A \neq 0$. Тогда этому уравнению удовлетворяет точка $(x_{0}, y_{0}) = (-\frac{C}{A}, 0)$, а значит (нетрудно проверить) все удовлетворяющие ему точки имеют вид $(x_{0} + Bt, y_{0} - At)$, что соответствует прямой с такими параметрическими уравнениями.
	\end{proof}
	\subsection{Взаимное расположение прямых}
	\begin{theorem}
		Прямые на плоскости параллельны (или совпадают) $\Leftrightarrow$ их направляющие векторы пропорциональны.
	\end{theorem}
	\begin{proof}
		Пусть $l_{1}: A_{1}x + B_{1}y + C_{1} = 0; l_{2}: A_{2}x + B_{2}y + C_{2} = 0$ - данные прямые.
		Рассмотрим систему уравнений, которой удовлетворяют координаты точек, принадлежащих обоим прямым: $\begin{cases}A_{1}x + B_{1}y = -C_{1}\\A_{2}x + B_{2}y = -C_{2}\end{cases}\\$ Из курса алгебры (форумла Крамера) известно, что система не является определённой $\Leftrightarrow det\begin{pmatrix} A_{1} \ B_{1}\\A_{2} \ B_{2}\end{pmatrix} = 0$. Таким образом, прямые параллельны или совпадают $\Leftrightarrow$ имеют 0 или бесконечно много общих точек $\Leftrightarrow A_{1}B_{2} - A_{2}B_{1} = 0 \Leftrightarrow (A_{1}\ B_{1})$ пропорционален $(A_{2}\ B_{2})$, ч.т.д. 
	\end{proof}
	\begin{remark}
		Из этого также видно, что прямые совпадают $\Leftrightarrow \frac{A_{1}}{A_{2}} = \frac{B_{1}}{B_{2}} = \frac{C_{1}}{C_{2}}$. 
	\end{remark}
	\begin{consequense}
		Прямые $l_{1}: \begin{cases}x = x_{1} + a_{1}t \\ y = y_{1} + b_{1}t\end{cases}; l_{2}: \begin{cases}x = x_{2} + a_{2}t \\ y = y_{2} + b_{2}t\end{cases}$ пересекаются $\Leftrightarrow \frac{a_{1}}{a_{2}} \neq \frac{b_{1}}{b_{2}}$.
		Условие совпадения прямых также можно записать через параметрические уравнения (вектор $(x_{2} - x_{1} \ y_{2} - y_{1}) = \lambda(a, b)$). \\
		Из этого также следует, что через две различные точки проходит ровно одна прямая (все такие прямые совпадают).
	\end{consequense}
	\subsection{Пучки прямых}
	\begin{definition}
		Собственным пучком прямых называется множество всех прямых, проходящих через данную точку, называемую центром пучка.\\
		Несобственным пучком прямых называется множество всех прямых, параллельных данной прямой.
	\end{definition}
	\begin{theorem}
		Пусть прямые $l_{1}: A_{1}x + B_{1}y + C_{1} = 0$ и $l_{2}: A_{2}x + B_{2}y + C_{2} = 0$ задают собственный пучок (т.е. содержатся в нём и не совпадают). Тогда прямая $l$ принадлежит пучку $\Leftrightarrow l$ задаётся уравнением $\lambda(A_{1}x + B_{1}y + C_{1}) + \mu(A_{2}x + B_{2}y + C_{2}) = 0 \ (*)$ для некоторых $\lambda, \mu \in \mathbb{R}$.
	\end{theorem}
	\begin{proof}
		$\\\Leftarrow$ Пусть $l$ задаётся уравнением $(*)$. Тогда, подставив в уравнение $l$ центр пучка ($x_{0}, y_{0}$), получим $\lambda(0) + \mu(0) = 0$ (т.к. центр удовлетворяет уравнениям $l_{1}, l_{2}$).\\
		$\Rightarrow$ Пусть $(x_{0}, y_{0}) \in l$. Возьмём произвольную точку $(x_{1}, y_{1}) \in l, (x_{1}, y_{1}) \neq (x_{0}, y_{0})$. Рассмотрим прямую вида $(*)$ с $\lambda = -(A_{2}x_{1} + B_{2}y_{1} + C_{2}), \ \mu = (A_{1}x_{1} + B_{1}y_{1} + C_{1}) : -(A_{2}x_{1} + B_{2}y_{1} + C_{2})(A_{1}x + B_{1}y + C_{1}) + (A_{1}x_{1} + B_{1}y_{1} + C_{1})(A_{2}x + B_{2}y + C_{2}) = 0$. Заметим, что это уравнение действительно задаёт прямую: в противном случае необходимы условия $\lambda A_{1} + \mu A_{2} = \lambda B_{1} + \mu B_{2} = 0$, но тогда $(A_{1}, B_{1})$ и $(A_{2}, B_{2})$ пропорциональны, а исходные прямые непараллельны. Такой прямой, очевидно, принадлежат точки $(x_{0}, y_{0})$ и $(x_{1}, y_{1})$. Так как через две различные точки проходит ровно одна прямая, любая прямая из собственного пучка имеет вид ($*$), ч.т.д.
	\end{proof}
	\begin{theorem}
		Пусть прямые $l_{1}: A_{1}x + B_{1}y + C_{1} = 0$ и $l_{2}: A_{2}x + B_{2}y + C_{2} = 0$ задают несобственный пучок (т.е. содержатся в нём и не совпадают). Тогда прямая $l$ принадлежит пучку $\Leftrightarrow l$ задаётся уравнением $\lambda(A_{1}x + B_{1}y + C_{1}) + \mu(A_{2}x + B_{2}y + C_{2}) = 0 \ (*)$ для некоторых $\lambda, \mu \in \mathbb{R}$.
	\end{theorem}
	\begin{proof}
		$\\\Leftarrow$ Так как $l_{1} \parallel l_{2}$, $\frac{A_{1}}{B_{1}} = \frac{A_{2}}{B_{2}}$. Тогда если $l$ имеет вид ($*$), то $\frac{\lambda A_{1}+\mu A_{2}}{A_{1}} = \lambda + \frac{\mu A_{2}}{A_{1}} = \lambda + \frac{\mu B_{2}}{B_{1}} = \frac{\lambda B_{1}\mu B_{2}}{B_{1}} \Rightarrow l \parallel l_{1}$.\\
		$\Rightarrow$ Пусть $l$ принадлежит пучку. Так как направляющие векторы $l, l_{1}$ и $l_{2}$; пропорциональны, можем домножить уравнения на числа так, что коэффициенты перед переменными станут равны: пусть $l_{1}: Ax + By + C_{1} = 0; \ l_{2} = Ax + By + C_{2} = 0; \ l = Ax + By + C_{3} = 0$.Тогда возьмём $\lambda, \mu$ из следующей системы: $\begin{cases}C_{1}\lambda + C_{2}\mu = C_{3}\\\lambda + \mu = 1\end{cases} \Leftrightarrow \begin{cases}\lambda = \frac{C_{3}-C_{2}}{C_{1}-C_{2}}\\\mu = \frac{C_{1} - C_{3}}{C_{1} - C_{2}}\end{cases} (C_{1} \neq C_{2}$, иначе $l_{1}$ и $l_{2}$ совпадают). Очевидно, что для таких $\lambda, \mu$ уравнение $l$ имеет вид ($*$) (проверяется несложной подстановкой), ч.т.д.
	\end{proof}
	\subsection{Отрезки}
	\begin{definition}
		Пусть $l$ - прямая, $X_{1}(x_{1}, y_{1}), X_{2}(x_{2}, y_{2}) \in l$ и $X_{1} \neq X_{2}$. Отрезком с концами $X_{1}, X_{2}$ на плоскости называется множество всех точек, лежащих между $X_{1}$ и $X_{2}$ (на прямой $l$). Обозначается $[X_{1}, X_{2}]$. 
	\end{definition}
	\begin{formula}[Уравнение отрезка]
		Пусть $X \in [X_{1}, X_{2}]$. Тогда знаем, что $\begin{cases}x = x_{1} + t(x_{2} - x_{1}) \\ y = y_{1} + t(y_{2} - y_{1})\end{cases} (X \in l) \Rightarrow \begin{pmatrix} x-x_{1} \\ y-y_{1} \end{pmatrix} = t\begin{pmatrix} x_{2}-x_{1} \\ y_{2}-y_{1} \end{pmatrix} \Rightarrow \\ t\overrightarrow{X_{1}X_{2}} = \overrightarrow{X_{1}X} \Rightarrow \overrightarrow{XX_{2}} = (1-t)\overrightarrow{X_{1}X_{2}}$. Отсюда видно, что $|\overrightarrow{X_{1}X}|$ и $|\overrightarrow{XX_{2}}| < |\overrightarrow{X_{1}X_{2}}| \Leftrightarrow t \in [0, 1]$. Отсюда $X \in [X_{1}, X_{2}] \Leftrightarrow \begin{cases}x = x_{1} + t(x_{2} - x_{1}) \\ y = y_{1} + t(y_{2} - y_{1})\\t\in [0, 1]\end{cases}$
	\end{formula}
	\begin{subtheorem}
		
	\end{subtheorem}
	\bfseries Мораль в том, что дальше очев... (по Гейне, конечно) \mdseries
\end{document}