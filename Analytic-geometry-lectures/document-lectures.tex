\documentclass[a4paper, 12pt]{article}
%\usepackage{mathtext}
\usepackage{cmap}
\usepackage[english, russian]{babel}
\usepackage[T2A]{fontenc}
\usepackage[utf8]{inputenc}
\usepackage[left=2cm, right=1.5cm, top=2cm, bottom=2cm]{geometry}
\usepackage{amsmath}
\usepackage{etoolbox}
\usepackage{amsthm}
\usepackage{amsfonts}
%\usepackage{indentfirst}
\usepackage{soul}
\usepackage{graphicx}
\usepackage{enumerate}

\usepackage{mathtools,amssymb}

\usepackage{tikz,amstext}
\newlength{\tempheight}
\newcommand{\Let}[0]{%
	\mathbin{\text{\settoheight{\tempheight}{\mathstrut}\raisebox{0.5\pgflinewidth}{%
				\tikz[baseline,line cap=round,line join=round] \draw (0,0) --++ (0.4em,0) --++ (0,1.5ex) --++ (-0.4em,0);%
}}}}

\renewcommand{\phi}{\varphi}
\renewcommand{\epsilon}{\varepsilon}
\newcommand*\circled[1]{\tikz[baseline=(char.base)]{
            \node[shape=circle,draw,inner sep=2pt] (char) {#1};}}
\newcommand{\aug}{\fboxsep=-\fboxrule\!\!\!\fbox{\strut}\!\!\!}
\newcommand\tab[1][.5cm]{\hspace*{#1}}
\newcommand\undermat[2]{\makebox[0pt][l]{$\smash{\underbrace
			{\phantom{\begin{matrix}#2\end{matrix}}}_{\text{$#1$}}}$}#2}
\newcommand\overmat[2]{\makebox[0pt][l]{$\smash{\overbrace
			{\phantom{\begin{matrix}#2\end{matrix}}}^{\text{$#1$}}}$}#2}

\newcounter{lemcount}
\newcounter{lemcount2}
\newcounter{thcount}
\theoremstyle{definition}
\newtheorem*{definition}{Определение}
\newtheorem*{theorem}{Теорема}
\newtheorem*{consequense}{Следствие}
\newtheorem*{lemma}{Лемма}
\newtheorem*{subtheorem}{Утверждение}
\newtheorem*{formula}{Вывод формулы}
\newtheorem*{formulas}{Вывод формул}
\newtheorem*{remark}{Замечание}
\newtheorem*{examples}{Примеры}
\newtheorem*{example}{Пример}
\newtheorem*{lalala}{Упражнение}
\newtheorem*{algorithm}{Алгоритм}
\newtheorem*{properties}{Свойства}
\newtheorem*{properties1}{Свойство}
\newtheorem{lemmanum}[lemcount]{Лемма}
\newtheorem{lemmanum2}[lemcount2]{Лемма}
\newtheorem{theoremnum}[thcount]{Теорема}
% \newtheorem{theoremL}{Теорема}[section]
\usepackage[T2A]{fontenc}
\usepackage[utf8]{inputenc}
\usepackage[russian]{babel}
\addto\captionsenglish{% Replace "english" with the language you use
	\renewcommand{\contentsname}%
	{Содержание}%
}

\newsavebox{\boxedalignbox}
\newenvironment{boxedalign*}
  {\begin{equation*}\begin{lrbox}{\boxedalignbox}$\begin{aligned}}
  {\end{aligned}$\end{lrbox}\fbox{\usebox{\boxedalignbox}}\end{equation*}}

\usepackage{titlesec}
\titleformat{\section}{\LARGE \bfseries}{\thesection}{1em}{}
\titleformat{\subsection}{\Large\bfseries}{\thesubsection}{1em}{}
\titleformat{\subsubsection}{\large\bfseries}{\thesubsubsection}{1em}{}

\usepackage{hyperref}
\usepackage{xcolor}
% Цвета для гиперссылок
\definecolor{linkcolor}{HTML}{225ae2} % цвет ссылок
\definecolor{urlcolor}{HTML}{225ae2} % цвет гиперссылок
\hypersetup{
	pdfstartview=FitH, 
	linkcolor=linkcolor,
	urlcolor=urlcolor,
	colorlinks=true
}

\begin{document}
	\begin{titlepage}
		\newpage
		
		\begin{center}
		\end{center}
		
		\vspace{4em}
		
		\begin{center}
			\Large Механико-математический факультет  
		\end{center}
		
		\vspace{2em}
		
		\begin{center}
			\large{\textsc{\textbf{Аналитическая геометрия, 1 семестр, 2 поток}}}
		\end{center}
		
		\vspace{6em}
		
		\vspace{\fill}
		
		\begin{center}
			Москва \\2024 
		\end{center}
	\end{titlepage}
	\tableofcontents
	\fontsize{14pt}{20pt}\selectfont
	\newpage
	\fontsize{14pt}{20pt}\selectfont
	\section{Общие понятия (нет в билетах)}
	Уже известные нам понятия:
	\begin{itemize}
	\item Точка - то, что не имеет частей;
	\item Линия - длина без ширины;
	\item Прямая - линия, равно расположенная по отношению к точкам на ней;
	\item Поверхность - то. что имеет длину и ширину;
	\item \dots
	\end{itemize}
	(Эти определения, как и последующие постулаты и аксиомы, были даны Евклидом.)
	
	И определяемые понятия - аналогично школьным учебникам. 
	\subsection{Постулаты}
	\begin{enumerate}
	\item От всякой точки до всякой можно провести прямую.
	
	\item Ограниченную часть прямой можно непрерывно продолжать.
	
	\item Из всякой точки можно описать окружность со всяким радиусом.
	
	\item Все прямые углы равны между собой.
	
	\item Если прямая $l$ образует с двумя другими прямыми $l_{1}$ и $l_{2}$ внутренние и лежащие по одну сторону углы, меньшие прямых, то $l_{1}$ и $l_{2}$ пересекутся с той стороны от $l$, где углы меньше прямых.   
	\end{enumerate}
	
	\subsection{Аксиомы (общие понятия) по Евклиду}
	\begin{enumerate}
		\item Равные одному и тому же равны между собой.
		\item Если к равным прибавляются равные, то суммы равны.
		\item Если из равных вычитаются равные, то разности равны.
		\item Если к неравным прибавляются равные, то суммы неравны.
		\item Удвоенные равные равны.
		\item Половины равных равны.
		\item Совмещающиеся друг с другом равны между собой.
		\item Целое больше части.
		\item Две прямые не образуют пространство.
	\end{enumerate}
	Сейчас используется более строгая аксиоматика (Гильберт)
	
	\subsection{Современная аксиоматика}
	\begin{definition}
		Плоскость (по Колмогорову) - тройка ($X, L, d$), где $X$ - множество (точек), $L$ - выделенная совокупность его подмножеств (прямых) и $d$ - отображение. сопоставляющее паре точек $x, y\in X$ неотрицательное $d(x, y) \in \mathbb{R}$ - расстояние от $x$ до $y$. 
	\end{definition} 
	Аксиомы делятся на 5 групп:
	\begin{enumerate}[I.]
		\item \bfseries Аксиомы принадлежности: \mdseries
		\begin{enumerate}[1.]
			\item Каждая прямая есть множество точек.
			\item Через любые две различные точки проходит прямая, и притом только одна.
			\item Существует хотя бы одна прямая.
		    \item Каждой прямой принадлежит хотя бы одна точка.
		    \item Вне каждой прямой существует хотя бы одна точка.
		\end{enumerate}
		\item \bfseries Аксиомы расстояния: \mdseries
		\begin{enumerate}[1.]
			\item $d(x, y) = 0 \Leftrightarrow x = y$
			\item $\forall x,y\in X \ \ d(x, y) = d(y, x)$
			\item (Неравенство треугольника) $d(x, z) \leqslant d(x, y) + d(y, z)$ 
		\end{enumerate}
		Пространство $X$ с выполненными аксиомами принадлежности и расстояния называется метрическим.
		\begin{definition}
			Точка $y$ лежит между $x$ и $z$, если $d(x, z) = d(x, y) + d(y, z)$. Множество точек между $x$ и $z$ называется отрезком $[x, z]$, а $d(x, z)$ - его длиной. 
		\end{definition}
		
		\item \bfseries Аксиомы порядка: \mdseries
		\begin{enumerate}[1.]
			\item Три точки лежат на одной прямой $\Leftrightarrow$ одна из них лежит между двумя другими.
			\item Любая точка $x$ прямой $l$ разбивает множество отличных от $x$ точек, лежащих на $l$, на два непустых подмножества так, что $x$ лежит между любыми двумя точками из разных подмножеств.
		\end{enumerate}
		\begin{definition}
			Одно из таких подмножеств точек прямой, взятое вместе с точкой $x$, называется лучом с началом в точке $x$.
		\end{definition}
		3. $\forall a\geqslant0 \ \exists!\ y$ на луче с началом в $x$ такая, что $d(x, y) = a$.
		\begin{definition}
		    Множество точек А плоскости называется выпуклым, если $\forall x, y\in A \ \ \ [x, y]\in A$.
		\end{definition}
		4. Любая прямая $l$ разбивает множество не принадлежащих ей точек плоскости на два непустых выпуклых подмножества.
		\begin{definition}
			Одно из таких подмножеств точек плоскости, взятое вместе с прямой $l$, называется полуплоскостью. Прямая $l$ называется граничной для этой полуплоскости. Если $l$ содержит луч, то эта полуплоскость примыкает к данному лучу.  
		\end{definition}
		\item \bfseries Аксиомы подвижности: \mdseries
		\begin{definition}
			Взаимно однозначное отображение плоскости на себя, сохраняющее расстояния, называется движением (изометрией).
		\end{definition}
		\begin{enumerate}[1.]
			\item Для любой пары лучей $l_{1}, l_{2}$ и примыкающих к ним полуплоскостей $\pi_{1}, \pi_{2}$ существует единственное движение $\phi$ такое, что $\phi(l_{1}) = l_{2}, \phi(\pi_{1}) = \pi_{2} $.
			\item  Для любой пары отрезков $[a_{1}, b_{1}], [a_{2}, b_{2}]$ равной ненулевой длины существуют ровно два движения $\phi_{1}, \phi_{2}$ таких, что $\phi_{1,2}([a_{1}, b_{1}]) = [a_{2}, b_{2}]$.
		\end{enumerate}
		\item \bfseries Аксиома параллельных прямых: \mdseries
		
		Через любую точку вне прямой $l$ можно провести ровно одну прямую, не пересекающуюся с $l$ (такая прямая называется параллельной к $l$).
		
		Данная аксиома разделяет евклидову и неевклидову геометрии.
	\end{enumerate}
	
	Методы аналитической геометрии основаны на том, что, задавая точки парами чисел (через координаты), мы можем получать разнообразные зависимости (например, уравнения множеств точек). Причем точку можно переопределить как пару чисел вместо абстрактного объекта, и все аксиомы останутся выполнены. То есть, рассматривая точки как арифметический объект, мы получим неотличимую по свойствам от абстрактной геометрии геометрию арифметическую, с которой уже можно работать по известным нам принципам.
	
	\section{Векторы и операции над ними}
	\subsection{Векторные пространства}
	Геометрические векторы в математике являются \bfseries свободными векторами \mdseries - классами эквивалентности направленных отрезков по уже известному нам отношению эквивалентности векторов.
	\begin{definition}
		Векторным (линейным) пространством (над полем $\mathbb{R}$) называется множество V с введенными на нем бинарными операциями "+": $V \times V \rightarrow V$ и "$*$": $\mathbb{R} \times V \rightarrow V$ , отвечающие следующим свойствам (аксиомам):
		
		$\forall \bar{a}, \bar{b}, \bar{c} \in V; \lambda, \mu \in \mathbb{R}:$
		\begin{enumerate}
			\item $\bar{a} + \bar{b} = \bar{b} + \bar{a}$ (коммутативность сложения);
			\item $(\bar{a} + \bar{b}) + \bar{c} = \bar{a} + (\bar{b} + \bar{c})$ (ассоциативность сложения);
			\item $\exists \bar{0} \in V: \bar{a} + \bar{0} = \bar{0} + \bar{a} = \bar{a}$ (существует нейтральный элемент по сложению - нулевой вектор);
			\item $\exists (-\bar{a}) \in V: \bar{a} + (-\bar{a}) = (-\bar{a}) + \bar{a} = \bar{0}$ (существует противоположный элемент по сложению);
			\item $\lambda(\mu\bar{a}) = (\lambda\mu)\bar{a}$ (ассоциативность умножения на числа);
			\item $(\lambda + \mu)\bar{a} = \lambda\bar{a}+ \mu\bar{a}$ (дистрибутивность по умножению);
			\item $\lambda(\bar{a} + \bar{b}) = \lambda\bar{a}+ \lambda\bar{b}$ (дистрибутивность по сложению);
			\item $1*\bar{a} = \bar{a}$.
		\end{enumerate}
	\end{definition}
	\begin{examples}
		Векторные пр-ва:
		\begin{itemize}
			\item $\mathbb{R}, \mathbb{R}^2, \mathbb{R}^n$:
			\item Функции;
			\item Многочлены;
			\item Многочлены степени $\leqslant n$.
		\end{itemize}
	\end{examples}
	\begin{remark}
		Св-ва векторных пространств:
		
		\begin{enumerate}
			
			\item $\bar{0}$ единственный.
			
			Пусть $\bar{0}_{1}, \bar{0}_{2}$ - нулевые векторы.
			
			Тогда $\bar{0}_{1} = \bar{0}_{1} + \bar{0}_{2} = \bar{0}_{2}$, ч.т.д.
			\item $-\bar{a}$ единственный.
			
			Пусть $-\bar{a}_{1}, -\bar{a}_{2}$ - противоположные к $\bar{a}$ векторы.
			
			Тогда $-\bar{a}_{1} = -\bar{a}_{1} + \bar{0} = -\bar{a}_{1} + (\bar{a} + -\bar{a}_{2}) = (-\bar{a}_{1} + \bar{a}) + (-\bar{a}_{2}) = \bar{0} + (-\bar{a}_{2}) = -\bar{a}_{2}$, ч.т.д.
			\item $\lambda * \bar{0} = \bar{0}$.
			
			$\lambda * \bar{0} = \lambda * (\bar{0} + \bar{0}) = \lambda * \bar{0} + \lambda * \bar{0}$
			
			Прибавив к обеим частям вектор, противоположный к $\lambda * \bar{0}$, получим $\lambda * \bar{0} = \bar{0}$, ч.т.д.
			\item $-(\lambda\bar{a}) = (-\lambda)\bar{a} = \lambda(-\bar{a})$.
			
			Нетрудно видеть, что все три вектора противоположны $\lambda\bar{a}$, а далее из п.2.
			\item $-\bar{a} = -1*\bar{a}$
			
			Следует из п.4.
			\item $\lambda\bar{a} = \bar{0} \Leftrightarrow$ либо $\lambda = 0$, либо $\bar{a} = \bar{0}$.
			
			Либо $\lambda = 0$, либо $\lambda \neq 0 \Rightarrow \frac{1}{\lambda}*\lambda*\bar{a} = \frac{1}{\lambda}\bar{0} = \bar{0} \Rightarrow \bar{a} = \bar{0}$, ч.т.д. 
		\end{enumerate}
	\end{remark}
	\subsection{Линейная зависимость векторов}
	\begin{definition}
		Сумма вида $\lambda_{1}\bar{x}_{1} + ... + \lambda_{n}\bar{x}_{n}$ называется линейной комбинацией векторов $\bar{x}_{1} ... \bar{x}_{n}$.
	\end{definition}
	\begin{definition}
		Если в линейной комбинации $\lambda_{1} = ... = \lambda_{n} = 0$, то она называется тривиальной, а иначе - нетривиальной.
	\end{definition}
	\begin{definition}
		Если вектор $\bar{x}$ равен линейной комбинации $\lambda_{1}\bar{x}_{1} + ... + \lambda_{n}\bar{x}_{n}$, то говорят, что он линейно выражается (раскладывается) через векторы $\bar{x}_{1}...\bar{x}_{n}$.
		(Сама линейная комбинация $\lambda_{1}\bar{x}_{1} + ... + \lambda_{n}\bar{x}_{n}$ называется выражением (разложением) вектора $\bar{x}$ через $\bar{x}_{1}...\bar{x}_{n}$)  
	\end{definition}
	\begin{definition}
		Множество векторов называется линейно зависимым, если существует равная нулю нетривиальная линейная комбинация векторов из этого множества. В противном случае оно называется линейно независимым.
	\end{definition}
	\begin{example}
		Система из двух векторов $\bar{a}, \bar{b}$ линейно зависима $\Leftrightarrow \bar{a} = \lambda\bar{b}$.
	\end{example}
	\begin{remark}
		Множество векторов линейно зависимо $\Leftrightarrow$ один из векторов этого множества линейно выражается через некоторые другие векторы этого множества.
	\end{remark}
	\begin{definition}
		Упорядоченное множество векторов называется системой векторов.
		(В системе векторов элементы могут повторяться)
	\end{definition}
	\begin{definition}
		Множество (система) векторов из векторного пространства V называется полным(полной) в $V$, если любой вектор $\bar{x}\in V$ линейно выражается через векторы этого множества.
	\end{definition}
	\begin{remark}
		$X \subset V$ полно в $V \Rightarrow \forall Y: X\subset Y$ полно в $V$.
	\end{remark}
	\begin{remark}
		$X \subset V$ линейно независимо в $V \Rightarrow \forall Y \subset X$ линейно независимо в $V$.
	\end{remark}
	\subsection{Базис векторного пространства}
	\begin{definition}
		Множество векторов $E$ в векторном пространстве $V$ называется базисом $V$, если $E$ линейно независимо и полно в $V$.
	\end{definition}
	\begin{definition}
		Векторное пространство, в котором существует конечный (состоящий из конечного числа векторов) базис, называется конечномерным. В противном случае оно называется бесконечномерным. 
	\end{definition}
	\begin{lemma}
		Если $X$ - конечное полное множество из $n$ векторов в векторном пространстве $V$ и $Y$ - линейно независимое множество векторов в $V$, то $Y$ конечно и число векторов в $Y \leqslant n$. 
	\end{lemma}
	\begin{proof}
		(пер.)
		Произвольно занумеруем векторы в $X: (\bar{x}_{1},...,\bar{x}_{n})$.
		Будем по одному добавлять в эту систему векторы из $Y$ и одновременно выкидывать векторы из $X$ так, чтобы система оставалась полной.
		
		Пусть за $k$ шагов ($0\leqslant k \leqslant n$) мы добавили некоторые $\bar{y}_{1},...,\bar{y}_{k}$ и выкинули какие-то $k$ векторов из $X$ - осталась система ($\bar{y_{1}},...,\bar{y_{k}},\bar{x}_{i_{1}},...,\bar{x}_{i_{n-k}}$).
		Возьмём $\bar{y}_{k+1}$ из $Y$ (если такого нет, то в $Y \leqslant n$ векторов, что нам и нужно), и добавим его в систему. Так как до этого система оставалась полной, $\bar{y}_{k+1}$ выражается через ($\bar{y_{1}},...,\bar{y_{k}},\bar{x}_{i_{1}},...,\bar{x}_{i_{n-k}}$), причём какой-то $\bar{x}_{i_{j}}$ входит в это разложение с коэффициентом, не равным нулю (иначе противоречие с линейной независимостью $Y$ - $\bar{y}_{k+1}$ выразился через $\bar{y}_{1},...,\bar{y}_{k}$).
		
		Тогда $\bar{x}_{i_{j}}$ выражается через другие векторы системы и $\bar{y}_{k+1}$ (в выражении $\bar{y}_{k+1}$ перенесём всё, кроме $\bar{x}_{i_{j}}$ в другую часть и разделим на коэффициент перед ним).
		А так как ($\bar{y_{1}},...,\bar{y_{k+1}},\bar{x}_{i_{1}},...,\bar{x}_{i_{n-k}}$) - полная, эта же система без $\bar{x}_{i_{j}}$. очевидно, останется полной.

		Пусть смогли проделать $n$ таких шагов. Тогда имеем систему ($\bar{y}_{1},...,\bar{y}_n$). Если в $Y$ есть ещё векторы, то они с одной стороны выражаются через векторы системы из её полноты, а с другой - не выражаются через них из линейной независимости $Y$. Противоречие, т.е. в $Y$ не может оказаться больше $n$ векторов, ч.т.д.   
	\end{proof}
	\begin{theorem}
		Если в векторном пространстве есть конечный базис, то все базисы в нём конечны и содержат одинаковое количество векторов.
	\end{theorem}
	\begin{proof}
		Пусть $\bar{e}_{1},...,\bar{e}_{n}$ - конечный базис в $V$. Любой другой базис $V$ линейно независим, т.е. по лемме содержит $k \leqslant n$ векторов, а с другой стороны полон, т.е. первый базис по лемме содержит $n \leqslant k$ векторов. Отсюда $n=k$, ч.т.д.
	\end{proof}
	\begin{definition}
		Количество векторов в любом базисе векторного пространства $V$ называется размерностью $V$ и обозначается $dim V$.
	\end{definition}
	\begin{examples}
		$dim \ {\bar{0}} = 0, dim \ \pi (= dim \ \mathbb{R}^2) = 2, dim \ \mathbb{R}^3 = 3$.
	\end{examples}
	\begin{theorem}
		В конечномерном векторном пространстве выражение любого вектора через базис определяется однозначно.
	\end{theorem}
	\begin{proof}
		Если $\bar{x} = \lambda_{1}\bar{e}_{1} + ... + \lambda_{n}\bar{e}_{n} = \lambda'_{1}\bar{e}_{1} + ... + \lambda'_{n}\bar{e}_{n}$, то $\bar{x} - \bar{x} = \bar{0} = (\lambda_{1} - \lambda'_{1})\bar{e}_{1} + ... + (\lambda_{n} - \lambda'_n)\bar{e}_{n}$. Если эти два разложения различны, то равная нулю линейная комбинация базисных векторов нетривиальна, что противоречит линейной независимости базиса. То есть двух различных разложений быть не может, ч.т.д. 
	\end{proof}
	\begin{definition}
		Пусть $V$ - конечномерное векторное пространство и $\bar{e}_{1},...,\bar{e}_{n}$ - базис в нём. Коэффициенты $\lambda_{1},...,\lambda_{n}$ в выражении любого вектора $x \in V$ через эти базисные векторы называются координатами вектора $x$ в базисе $\bar{e}_{1},...,\bar{e}_{n}$. ($\lambda_{k}$  называется $k$-й координатой)
	\end{definition}
	\begin{remark}
		Векторы в $n$-мерном векторном пространстве находятся во взаимно однозначном соответствии с упорядоченной строкой из $n$ чисел из $\mathbb{R}$ (например, векторы ассоциированного с евклидовой плоскостью векторного пространства соответствуют парам чисел)
		Таким образом можно задать операции сложения и умножения на число векторов плоскости через операции над числами, проводимыми покоординатно.
	\end{remark}
	Однако элементы плоскости (как множества) - точки, а не векторы, поэтому для работы непосредственно с плоскостью необходимо ввести ещё одно определение.
	\subsection{Аффинные пространства}
	\begin{definition}
		Аффинное пространство - тройка ($X, V, +$) (обычно обозначается $\mathbb{A}$), где $X$ - множество (точек), $V$ - векторное пространство, а $''+''$ - операция:  $X \times V \rightarrow X$, для которых выполнены аксиомы:
		\begin{enumerate}
			\item $\forall A \in X, \forall \bar{a}, \bar{b} \in V: A+(\bar{a}+\bar{b}) = (A+\bar{a})+\bar{b}$;
			\item $\forall A \in X: A + \bar{0} = A$;
			\item $\forall A, B \in X \ \ \exists! \ \bar{a} \in V: A + \bar{a} = B$. Обозначается $\bar{a} = \overrightarrow{AB}$.
		\end{enumerate}
	\end{definition}
	Если зафиксировать какую-нибудь точку $O \in X$, возникает взаимно однозначное соответствие между точками $A$ и их радиус-векторами $\overrightarrow{OA}$.
	\begin{definition}
		Репер (система координат) в аффинном пространстве $(X, V, +)$ - пара $(O, E)$, где $O \in X$ и $E$ - базис в $V$. Точка $O$ называется началом координат (отсчёта). Координаты точки A в $(O, E)$ - координаты её радиус-вектора $\overrightarrow{OA}$ в базисе $E$.
	\end{definition}
	\begin{remark}
		Для аффинного пространства верно:
		\begin{enumerate}
			\item Если $A = (x_{1},...,x_{n}),  \bar{a} = (y_{1},...,y_{n})$, то $A + \bar{a} = (x_{1} + y_{1},...,x_{n} + y_{n})$.
			\item Если $A = (a_{1},...,a_{n}),  B = (b_{1},...,b_{n})$, то $\overrightarrow{AB} = (b_{1} - a_{1},...,b_{n} - a_{n})$.
		\end{enumerate}
		(Следует из сложения векторов)
	\end{remark}
	\begin{definition}
		Если $\mathbb{A} = (X, V, +)$ - аффинное пространство, то говорят, что $V$ - векторное пространство, ассоциированное с $\mathbb{A}$.    
	\end{definition}
	\begin{definition}
		$\mathbb{A}$ называется конечномерным, если ассоциированное с ним $V$ конечномерно. В этом случае $dim \mathbb{A}$ (размерность $\mathbb{A}$) равна $dim V$.
	\end{definition}
	Теперь точки аффинного пространства аналогично векторам можно ассоциировать с наборами чисел. Однако для ассоциирования евклидовой плоскости и её аксиом с двумерным аффинным пространством, необходимы отвечающие аксиомам понятия прямой и расстояния.
	\subsection{Подпространства}
	\begin{definition}
		Векторным подпространством векторного пространства $V$ называется непустое множество $V_{1} \subset V$ такое. что $\forall\bar{x}, \bar{y}\in V_{1}: \bar{x} + \bar{y} \in V_{1}, \lambda\bar{x} \in V_{1} (\forall\lambda \in\mathbb{R})$.
	\end{definition}
	\begin{remark}
		Определение эквивалентно следующему: множество $V_{1} \subset V$ - векторное подпространство $V$, если $V_{1}$ является векторным пространством относительно операций $+$ и $*$, определённых для $V$.
		(Доказательство осуществляется путём прямой проверки аксиом векторного пространства для $V_{1}$)
	\end{remark}
	Введём несколько определений аффинного подпространства и докажем их эквивалентность.
	\begin{definition}
		Аффинным подпространством аффинного пространства $\mathbb{A} = (X, V, +)$ называется
		\begin{enumerate}
			\item его непустое подмножество вида $A + V_{1} = {A + \bar{a}:\bar{a} \in V_{1}}$, где $V_{1}$ - векторное подпространство $V$ и $A \in X$ - точка;
			\item тройка $(X_{1}\subset X, V_{1} \subset V, +_{1})$, где $V_{1}$ - векторное подпространство $V$ и операция $+_{1} = +$,  для которой $\forall A,B \in X_{1}, \forall \bar{a} \in V_{1}: A + \bar{a} \in X_{1}, \overrightarrow{AB} \in V_{1}$;
			\item тройка $(X_{1}\subset X, V_{1} \subset V, +_{1})$, где $V_{1}$ - векторное подпространство $V$ и операция $+_{1} = +$,  которая сама является аффинным пространством.
		\end{enumerate}
	\end{definition}
	\begin{subtheorem}
		Приведённые определения эквивалентны.
	\end{subtheorem}
	\begin{proof}
		Докажем следующие следствия:

		$\circled{1}\Rightarrow\circled{2}$  Пусть $P = A + \bar{a}, Q = A + \bar{b}$. Тогда $\overrightarrow{PQ} = \bar{b} - \bar{a}$ (в силу единственности такого вектора), т.е. $\overrightarrow{PQ} \in V_{1}$. Второе необходимое свойство $\circled{2}$ очевидно выполнено.

		$\circled{2}\Rightarrow\circled{1}$  Пусть $X_{1}, V_{1}$ удовлетворяют $\circled{2}$. Зафиксируем произвольную $A \in X_{1}$. $\forall B\in X_{1}$ имеем $B = A + \overrightarrow{AB}$, причём $A \in X_{1}, \overrightarrow{AB} \in V_{1} \Rightarrow B \in X_{1}$.
		
		Эквивалентность $\circled{2}\Leftrightarrow\circled{3}$ очевидна из определения аффинного пространства.
	\end{proof}
	\begin{definition}
		Прямая в аффинном пространстве - его одномерное аффинное подпространство. \\Плоскость (двумерная) в аффинном пространстве - его двумерное аффинное подпространство.
	\end{definition}
	\begin{definition}
		Единственный вектор в любом базисе векторного пространства, ассоциированного с одномерным аффинным пространством, называется направляющим вектором этого аффинного пространства.
	\end{definition}
	\begin{remark}
		У любой прямой множество точек имеет вид $A + V^1$, где $A$ - фиксированная точка.
	\end{remark}
	\begin{proof}
		Направляющий вектор $\bar{e}$ любой прямой ненулевой, т.к. базис одномерного векторного пространства линейно независим.\\
		Так как любой вектор $V^1$ выражается через базис, $\forall\bar{x} \in V^1 \ \exists\lambda\in\mathbb{R}: \bar{x} = \lambda\bar{e}$. \\
		Также $\forall\lambda\in\mathbb{R} \ \lambda\bar{e} \in V^1$, так как $V^1$ - векторное пространство. \\
		Отсюда любое ассоциированное с прямой векторное пространство имеет вид $\{\lambda\bar{e}:\lambda \in \mathbb{R}\}$, где $\bar{e}$ - фиксированный ненулевой вектор, а тогда, взяв любую точку одномерного аффинного пространства, получим, что всё его множество точек имеет необходимый вид.
	\end{proof}
	\begin{subtheorem}
		Для прямых в двумерном аффинном пространстве выполнены евклидовы аксиомы принадлежности.
	\end{subtheorem}
	\begin{proof}
		Пусть $\mathbb{A}^2 = (X^2, V^2, +)$ - данное двумерное аффинное пространство, $\bar{e}_{1}, \bar{e}_{2}$ - некоторый базис $V^2. l = \{O + \bar{e}_{1}\}$ - некоторая прямая (из предыдущего замечания). Тогда в $\mathbb{A}^2$ существует прямая, являющаяся множеством точек, на которой есть хотя бы одна точка - $O$, а вне её есть хотя бы одна точка - $O + \bar{e}_{2}$. (не на прямой, т.к. иначе $\bar{e}_{2} = \lambda\bar{e}_{1}$)\\
		Через любые две точки можно провести прямую ($\forall A,B \in X^2$ возьмём $\{A + \overrightarrow{AB}\}$ - прямую, проходящую через $A, B$), и притом только одну: если $l = \{O +\lambda\bar{x}: \lambda \in \mathbb{R}\}$ - прямая и $A,B \in l$, то $A = O + \lambda\bar{x}, B = O + \mu\bar{x} \Rightarrow B = A + (\mu - \lambda)\bar{x} \Rightarrow \overrightarrow{AB} = (\mu - \lambda)\bar{x}$. Тогда для любой $C \in l$ имеем $C = O + \eta\bar{x} = A + \overrightarrow{AO} + \eta\bar{x} = A + (\eta - \lambda)\bar{x} = A + \frac{\eta - \lambda}{\mu - \lambda}\overrightarrow{AB}$, т.е. любая проходящая через две точки прямая совпадает с этой.\\
	\end{proof}
	\begin{theorem}
		Любые два непропорциональных вектора $\bar{a}, \bar{b} \in V^2$ образуют базис двумерного векторного пространства $V_{2}$.
	\end{theorem}
	\begin{proof}
		Пусть $\bar{e}_{1}, \bar{e}_{2}$ - произвольный базис $V^2$. Тогда $\bar{a} = x\bar{e}_{1} + y\bar{e}_{2}, b = x'\bar{e}_{1} + y'\bar{e}_{2}$. Пусть без ограничения общности $x \neq 0$ (т.к. вектор $\bar{a} \neq \bar{0}$, хотя бы один коэффициент в его разложении $\neq0$). Тогда $\bar{e}_{1} = \frac{1}{x}\bar{a} - \frac{y}{x}\bar{e}_{2}$. Подставим во второе выражение: $\bar{b} = x'(\frac{1}{x}\bar{a} - \frac{y}{x}\bar{e}_{2}) + y'\bar{e}_{2} \Rightarrow \bar{e}_{2} = \frac{x}{xy'-yx'}\bar{b} - \frac{x'}{xy'-yx'}\bar{a}$. Подставив это в первое выражение, получим, что $\bar{e}_{1}$ также выражается через $\bar{a}, \bar{b}$. Тогда система $\bar{a}, \bar{b}$ - полная (из полноты базиса), а тогда $\bar{a}, \bar{b}$ - базис по определению, ч.т.д. 
	\end{proof}
	\begin{subtheorem}
		Для прямых в двумерном аффинном пространстве выполнена аксиома параллельных прямых.
	\end{subtheorem}
	\begin{proof}
		Если $l = \{O + \lambda\bar{x}: \lambda\in\mathbb{R}\}$ - прямая и $O_{1}\notin l$, то докажем, что $l_{1} = \{O_{1} + \lambda\bar{x}: \lambda\in\mathbb{R}\}$ - единственная прямая, проходящая через $O_{1}$ и параллельная $l$.
		Предположим, что у них есть общая точка: пусть $X = O + \lambda\bar{x} = O_{1} + \mu\bar{x}$. Тогда $O_{1} = O + (\lambda - \mu)\bar{x} \Rightarrow O_{1} \in l$. Противоречие, т.е. прямая $l_{1}$ действительно параллельна $l$.\\
		Пусть $l_{2}$ - другая прямая, проходящая через $O_{1}$. Тогда $l_{2} = \{O_{1} + \lambda\bar{y}: \lambda\in\mathbb{R}\}$, причём $\bar{y}$ не пропорционален $\bar{x}$ (иначе прямые совпадают). Тогда $\bar{x}, \bar{y}$ - базис $V^2$, то есть $\exists\alpha,\beta \in \mathbb{R}: \overrightarrow{OO_{1}} = \alpha\bar{x} + \beta\bar{y}$. Отсюда $O_{1} = O + \alpha\bar{x} + \beta\bar{y} \Rightarrow O + \alpha\bar{x} = O_{1} - \beta\bar{y}$, причём $O + \alpha\bar{x} \in l$ и $O_{1} - \beta\bar{y} \in l_{2}$, т.е. $l$ и $l_{2}$ имеют общую точку. Отсюда параллельная $l$ прямая, проходящая через $O_{1}$, единственная, ч.т.д. 
	\end{proof}
	\subsection{Скалярное произведение. Расстояния и углы}
	\begin{definition}
		Пусть $V$ - векторное пространство. Скалярным произведением в $V$ называется функция $(\ ,\ ) : V \times V \rightarrow \mathbb{R}$ со свойствами:
		\begin{enumerate}
			\item $(\bar{x}, \bar{x}) \geqslant 0 \ \forall \bar{x} \in V$, причём $(\bar{x}, \bar{x}) = 0 \Leftrightarrow \bar{x} = \bar{0}$ (положительная определённость);
			\item $(\bar{x}, \bar{y}) = (\bar{y}, \bar{x}) \ \forall \bar{x}, \bar{y} \in V$ (коммутативность);
			\item $(\alpha\bar{x} + \beta\bar{y}, \bar{z}) = \alpha(\bar{x}, \bar{z}) + \beta(\bar{y}, \bar{z}) \ \forall \bar{x}, \bar{y}, \bar{z} \in V, \alpha,\beta \in \mathbb{R}$ (линейность по первому аргументу)
		\end{enumerate} 
	\end{definition}
	Из коммутативности выполнена и линейность по второму аргументу, т.е. скалярное произведение - билинейная функция.
	\begin{definition}
		Длиной вектора называется величина $\sqrt{(\bar{x}, \bar{x})}$.
	\end{definition}
	\begin{definition}
		Расстоянием (евклидовым) между точками $A,B \in \mathbb{A}$ называется длина вектора $\overrightarrow{AB}$. Будем обозначать $d(A, B)$ как $|\overrightarrow{AB}|$. 
	\end{definition}
	\begin{remark}
		Зная длины всех векторов, скалярное произведение можно восстановить по формуле $(\bar{x}, \bar{y}) = \frac{1}{2}(|\bar{x} + \bar{y}|^2 - |\bar{x}|^2 - |\bar{y}|^2)$. Это несложно проверить: \\
		$\frac{1}{2}(|\bar{x} + \bar{y}|^2 - |\bar{x}|^2 - |\bar{y}|^2) = \frac{1}{2}((\bar{x} + \bar{y}, \bar{x} + \bar{y}) - (\bar{x}, \bar{x}) - (\bar{y}, \bar{y})) = \frac{1}{2}(2(\bar{x}, \bar{y})) = (\bar{x}, \bar{y})$.
	\end{remark}
	\begin{theorem}[Неравенство Коши-Буняковского]
		$\forall \bar{a}, \bar{b} \in V \ \ (\bar{a}, \bar{b}) \leqslant \sqrt{(\bar{a}, \bar{a})(\bar{b}, \bar{b})}$, причём равенство достигается только при $\bar{a} = \lambda\bar{b}$.
	\end{theorem}
	\begin{proof}
		Рассмотрим выражение $(\bar{a} + t\bar{b}, \bar{a} + t\bar{b})$. Оно равно нулю $\Leftrightarrow \bar{a} = -t\bar{b}$, т.е. может быть равно нулю не более чем при одном $t$. С другой стороны
		$(\bar{a} + t\bar{b}, \bar{a} + t\bar{b})$ = $(\bar{a}, \bar{a}) + 2(\bar{a}, \bar{b})t + (\bar{b}, \bar{b})t^2$ - квадратный трёхчлен относительно $t$. Его дискриминант равен $4(\bar{a}, \bar{b})^2 - 4(\bar{a}, \bar{a})(\bar{b}, \bar{b})$, а из первого рассуждения знаем, что дискриминант $\leqslant 0$, причём равенство достигается только в случае коллинеарности $\bar{a}$ и $\bar{b}$. Отсюда $(\bar{a}, \bar{b}) \leqslant \sqrt{(\bar{a}, \bar{a})(\bar{b}, \bar{b})}$, ч.т.д.
	\end{proof}
	\begin{subtheorem}
		Точки $A, B, C$ лежат на одной прямой $\Leftrightarrow$ одна из них лежит между двумя другими.
	\end{subtheorem}
	\begin{proof} $\\$
		$(\Rightarrow)$ Пусть $A, B, C$ лежат на одной прямой. Тогда $\exists O \in X^2, \bar{v} \in V^2, \alpha,\beta,\gamma \in \mathbb{R}: A = O + \alpha\bar{v}, B = O + \beta\bar{v}, C = O + \gamma\bar{v}$. Пусть без ограничения общности $\alpha \leqslant \beta \leqslant \gamma$. Тогда $d(A, B) = |\overrightarrow{AB}| = (\beta - \alpha)|\bar{v}|, d(B, C) = |\overrightarrow{BC}| = (\gamma - \beta)|\bar{v}|, d(A, C) = |\overrightarrow{AC}| = (\gamma - \alpha)|\bar{v}|$. Отсюда видно, что $d(A, C) = d(A, B) + d(B, C).\\$
		$(\Leftarrow)$ Пусть $d(A, C) = d(A, B) + d(B, C)$. Обозначим $\bar{a} = \overrightarrow{AB}, \bar{b} = \overrightarrow{BC}, \bar{c} = \overrightarrow{AC}$. Тогда $\sqrt{(\bar{c}, \bar{c})} = \sqrt{(\bar{a} + \bar{b}, \bar{a} + \bar{b})} = \sqrt{(\bar{a}, \bar{a})} + \sqrt{(\bar{b}, \bar{b})} \Rightarrow (\bar{a} + \bar{b}, \bar{a} + \bar{b}) = (\bar{a}, \bar{a}) + 2\sqrt{(\bar{a}, \bar{a})(\bar{b}, \bar{b})} + (\bar{b}, \bar{b}) \Rightarrow 2(\bar{a}, \bar{b}) = 2\sqrt{(\bar{a}, \bar{a})(\bar{b}, \bar{b})}$. Из неравенства Коши-Буняковского знаем, что равенство достигается при $\bar{a} = \lambda\bar{b}$, т.е. $\overrightarrow{AB}$ коллинеарен $\overrightarrow{BC}$, то есть $A, B, C$ лежат на одной прямой.  
	\end{proof}
	\begin{definition}
		Величиной угла между ненулевыми векторами $\bar{a}, \bar{b}$  называется число $arccos\frac{(\bar{a}, \bar{b})}{|\bar{a}||\bar{b}|}$ (из н. Коши-Буняковского $|\frac{(\bar{a}, \bar{b})}{|\bar{a}||\bar{b}|}| \leqslant 1$).
	\end{definition}
	\begin{definition}
		Конечномерное аффинное (векторное) пространство вместе со скалярным произведением называется точечно-евклидовым (евклидовым) пространством. Двумерное точечно-евклидово пространство называется евклидовой плоскостью.
	\end{definition}
	\subsection{Проектирование точек и векторов}
	\begin{definition}
		Пусть задано два векторных подпространства $V_{1}, V_{2}$ векторного пространства $V$ такие, что $V_{1} \cap V_{2} = \{\bar{0}\}$ и $V_{1} + V_{2} = V$ (обозначается $V = V_{1} \oplus V_{2}$). Тогда сумма $\bar{x} = \bar{x}_{1} + \bar{x}_{2}$, где $\bar{x} \in V, \bar{x}_{1} \in V_{1}, \bar{x}_{2} \in V_{2}$, определена единственно. (Следует, например, из того, что в любом базисе $V$ каждый его вектор лежит либо в $V_{1}$, либо в $V_{2}$, тогда разложение в эту сумму соответствует единственному разложению по базису). Проекцией вектора $\bar{x} \in V$ на $V_{1}$ параллельно $V_{2}$ называется слагаемое $\bar{x_{1}}$ этой суммы. 
	\end{definition}
	\begin{definition}
		Пусть задано два аффинных подпространства $\mathbb{A}_{1} = (X_{1}, V_{1}, +),\\ \mathbb{A}_{2} = (X_{2}, V_{2}, +)$ аффинного пространства $\mathbb{A} = (X, V, +)$ такие, что $V = V_{1} \oplus V_{2}$. Проекция точки $P \in \mathbb{A}$ на $\mathbb{A}_{1}$ параллельно $\mathbb{A}_{2}$ - точка $P_{1} = A_{1} + \bar{v}$, где $A_{1}$ - произвольная точка из $X_{1}$, а $\bar{v}$ - проекция $\overrightarrow{A_{1}P}$ на $V_{1}$ параллельно $V_{2}$.
		(Очевидно, что от выбора $A_{1}$ расположение проеции не зависит)
	\end{definition}
	\begin{example}
		Рассмотрим координаты точки евклидовой плоскости относительно прямоугольной системы координат. \\
		Найдём проекцию точки $A = (x, y)$ на прямую $Oy$ параллельно прямой $Ox$. По определению это точка (назовём её $A_{y}$), равная $O + \bar{v}$, где $\bar{v}$ - проекция $\overrightarrow{OP}$ на векторное пространство прямой $Oy$ параллельно $Ox$. $\overrightarrow{OP} = \{x, y\} = x\bar{e}_{1} + y\bar{e}_{2}$. Отсюда $\bar{v} = y\bar{e}_{2} = \{0, y\}$, то есть $A_{y} = (0, y)$. Аналогично $A_{x} = (x, 0)$. 
	\end{example}
	\subsection{Ортонормированный базис и прямоугольная система координат}
	\begin{definition}
		Векторы $\bar{a}, \bar{b}$ называются ортогональными (перпендикулярными), если $(\bar{a}, \bar{b}) = 0$.
	\end{definition}
	\begin{definition}
		Базис векторного пространства $V$ со скалярным произведением называется ортонормированным, если все его векторы попарно ортогональны и имеют длину 1.
	\end{definition}
	\begin{definition}
		Система координат в точечно-евклидовом пространстве называется прямоугольной, если её базис ортонормированный.
	\end{definition}
	\begin{subtheorem}
		В точечно-евклидовом пространстве верно следующее выражение скалярного произведения через координаты векторов: если в некотором базисе $ (\bar{e}_{1},...,\bar{e}_{n}) \ \bar{x} = \begin{pmatrix} x_{1} \\ \vdots \\ x_{n} \end{pmatrix}, \bar{y} = \begin{pmatrix} y_{1} \\ \vdots \\ y_{n} \end{pmatrix}$, то $(\bar{x}, \bar{y}) = \sum \limits_{i=1}^n x_{i} \cdot \sum \limits_{j=1}^n y_{j}(\bar{e}_{i}, \bar{e}_{j})$.
	\end{subtheorem}
	\begin{proof}
		$(\bar{x}, \bar{y}) = (x_{1}\bar{e}_{1}+...+x_{n}\bar{e}_{n}, y_{1}\bar{e}_{1}+...+y_{n}\bar{e}_{n}) = \sum \limits_{i=1}^{n}(x_{i}\bar{e}_{i},y_{1}\bar{e}_{1}+...+y_{n}\bar{e}_{n}) = \sum \limits_{i=1}^{n}x_{i}(\bar{e}_{i},y_{1}\bar{e}_{1}+...+y_{n}\bar{e}_{n}) = \sum \limits_{i=1}^n x_{i} \cdot \sum \limits_{j=1}^n y_{j}(\bar{e}_{i}, \bar{e}_{j})$
	\end{proof}
	\begin{remark}
		В случае, когда базис ортонормированный, имеем $(e_{i}, e_{j}) = \delta_{ij}$, т.е. $(\bar{x}, \bar{y}) = x_{1}y_{1}+...+x_{n}y_{n}$. То есть в прямоугольной системе координат длина вектора вычисляется по формуле $|\bar{x}| = \sqrt{x_{1}^2+...+x_{n}^2}$, а расстояние между точками $P = (x_{1},...,x_{n}), Q = (y_{1},...,y_{n})$ выражается как $|PQ| = |\overrightarrow{PQ}| = \sqrt{(y_{1} - x_{1})^2+...+(y_{n}- x_{n})^2}$.
	\end{remark}
	Пусть $\mathbb{A}^n = (X, V^n, +)$ - $n$-мерное точечно-евклидово пространство.
	\begin{subtheorem}
		В $V^n$ любая линейно независимая система из $n$ векторов образует базис.
	\end{subtheorem}
	\begin{proof}
		Предположим, что в $V^n$ существует неполная линейно независимая система из n векторов. Т.к. система не полная, существует вектор из $V^n$, не выражающийся через векторы этой системы, т.е. этот вектор можно добавить в систему без потери линейной независимости. Но по лемме-аналогу ОЛЛЗ линейно независимая система в $V^n$ не может иметь $> n$ векторов. Противоречие, т.е. любая линейно независимая система из $n$ векторов является полной, а значит и базисом, ч.т.д. 
	\end{proof}
	\begin{subtheorem}
		Если $\bar{e}_{1},...,\bar{e}_{n}$ - попарно ортогональные ненулевые векторы в евклидовом пространстве, то $\bar{e}_{1},...,\bar{e}_{n}$ линейно независимы.
	\end{subtheorem}
	\begin{proof}
		Предположим противное. Пусть один из векторов (без ограничения общности $\bar{e}_{n}$) линейно выражается через остальные: $\bar{e}_{n} = \lambda_{1}\bar{e}_{1} +...+ \lambda_{n-1}\bar{e}_{n-1}$. Тогда запишем квадрат его длины:
		$|\bar{e}_{n}|^2 = (\bar{e}_{n}, \bar{e}_{n}) = (\bar{e}_{n}, \lambda_{1}\bar{e}_{1} +...+ \lambda_{n-1}\bar{e}_{n-1}) = \sum \limits_{i=1}^{n-1} \lambda_{i}(\bar{e}_{n}, \bar{e}_{i}) = 0$ (т.к. $\bar{e}_{n}$ ортогонален всем остальным векторам). Отсюда $|\bar{e}_{n}| = 0$, и притом $\bar{e}_{n}$ ненулевой. Противоречие, т.е. никакой вектор системы не выражается через остальные, а значит система линейно независима. ч.т.д.
	\end{proof}
	\begin{theorem}
		В любом евклидовом пространстве существует ортонормированный базис.
	\end{theorem}
	\begin{proof}
		(пер.)
		Индукция по $n$ - размерности пространства:

		База: $n = 1$ - очевидно, что существует вектор длины 1, который составляет ортонормированный базис одномерного пространства;

		Шаг: Пусть в любом $n$-мерном пространстве существует ортонормированный базис. Рассмотрим пространство $V$ размерности $n+1$ и выберем базис какого-то $n$-мерного подпространства $W$ (пусть $(\bar{e}_{1},...,\bar{e}_{n})$). Найдём вектор, ортогональный всем выбранным векторам. Так как базис $W$ не полон в $V$, к нему можно добавить ещё один вектор $x \in V$ без потери линейной независимости $\Rightarrow$ $(\bar{e}_{1},...,\bar{e}_{n}, \bar{x})$ - базис в $V$ (ЛНЗ система из $n+1$ векторов). \\
		Теперь необходимо представить $\bar{x}$ как следующую сумму: $\bar{x} = \lambda_{1}\bar{e}_{1} + ... + \lambda_{n}\bar{e}_{n} + \bar{e}_{n+1}$, где $\bar{e}_{n+1}$ ортогонален $\bar{e}_{1},...,\bar{e}_{n}$. Тогда $\bar{e}_{n+1} = \bar{x} - \lambda_{1}\bar{e}_{1} - ... - \lambda_{n}\bar{e}_{n}$. Рассмотрим $(\bar{e}_{n+1}, \bar{e}_{k}) = (\bar{x} - \lambda_{1}\bar{e}_{1} - ... - \lambda_{n}\bar{e}_{n}, \bar{e}_{k}) = (\bar{x}, \bar{e}_{k}) - \lambda_{1}(\bar{e}_{1}, \bar{e}_{k}) - ... - \lambda_{n}(\bar{e}_{n}, \bar{e}_{k})$. Так как $\bar{e}_{k}$ ортогонально всем этим векторам, кроме $\bar{e}_{k}$ и $\bar{x}$, это выражение равно $(\bar{x}, \bar{e}_{k}) - \lambda_{k}(\bar{e}_{k}, \bar{e}_{k})$. Отсюда при $\lambda_{k} = \frac{(\bar{x}, \bar{e}_{k})}{(\bar{e}_{k}, \bar{e}_{k})}$ векторы $\bar{e}_{n+1}$ и $\bar{e}_{k}$ ортогональны (зависит только от $\lambda_{k}$). Составив таким образом все $\lambda_{1},...,\lambda_{n}$, получим выражение вектора $\bar{e}_{n+1}$, ортогонального всем векторам базиса $W$.
		Таким образом, векторы полученной системы $\bar{e}_{1},...,\bar{e}_{n+1}$ попарно ортогональны (по предположению индукции) $\Rightarrow$ линейно независимы $\Rightarrow$ образуют базис в $V$. Разделив $\bar{e}_{n+1}$ на его длину, получим, что все векторы базиса попарно ортогональны и имеют длину 1 $\Rightarrow V$ имеет ортонормированный базис, ч.т.д.
	\end{proof}
	\begin{consequense}
		Любую систему ортогональных векторов длины 1 в векторном пространстве можно дополнить до ортонормированного базиса.
	\end{consequense}
	\section{Прямые}
	\subsection{Уравнение прямой}
	\begin{definition}
		Уравнением (либо уравнениями) множества точек будем называть уравнение со следующим свойством: точка принадлежит множеству тогда и только тогда, когда её координаты удовлетворяют уравнению. 
	\end{definition}
	\begin{formulas}[\bfseries уравнения прямой\mdseries]


		Пусть $l$ - прямая на плоскости: $l = \{X: \overrightarrow{OX} = \overrightarrow{OM} + t\bar{v}\}$, где $M$ - точка прямой, $\bar{v}$ - её направляющий вектор. Если $M=\begin{pmatrix}x_{0}\\y_{0}\end{pmatrix},\bar{v}=\begin{pmatrix}a\\b\end{pmatrix}$, то из совпадения координат совпадающих векторов $\overrightarrow{OX}$ и $(\overrightarrow{OM}+t\bar{v})$ верно следующее: (\bfseries параметрические уравнения прямой\mdseries) \begin{boxedalign*}X \in l: \begin{cases}x = x_{0} + at \\ y = y_{0} + bt\end{cases}\end{boxedalign*}\\
		Выразим $t$ из первого уравнения и подставим во второе уравнение - получим \bfseries каноническое уравнение прямой\mdseries : \begin{boxedalign*}\frac{x-x_{0}}{a} = \frac{y-y_{0}}{b}\end{boxedalign*}
		(Заметим, что данное выражение не определено при нулевых $a$ или $b$, но очевидно, что они не равны нулю одновременно, а запись, где одна из дробей имеет знаменатель 0, иногда используется, поэтому здесь и далее случай равенства нулю знаменателя может не рассматриваться как отдельный и будет означать, что числитель должен равняться 0)\\
		Если известно, что прямой принадлежат $M = \begin{pmatrix}x_{0}\\y_{0}\end{pmatrix}, N = \begin{pmatrix}x_{1}\\y_{1}\end{pmatrix}$, то $\overrightarrow{MN} = \begin{pmatrix}x_{1} - x_{0}\\y_{1} - y_{0}\end{pmatrix}$ - направляющий вектор, т.е. каноническое уравнение имеет вид \begin{boxedalign*}\frac{x-x_{0}}{x_{1}-x_{0}} = \frac{y-y_{0}}{y_{1}-y_{0}}\end{boxedalign*}(\bfseries уравнение прямой по двум точкам\mdseries).\\
		Домножим каноническое уравнение прямой на знаменатели: $\frac{x-x_{0}}{a} = \frac{y-y_{0}}{b} \Rightarrow bx-bx_{0}=ay-ay_{0} \Rightarrow bx - ay + (ay_{0}-bx_{0}) = 0$. Такое уравнение обычно называют \bfseries общим уравнением прямой\mdseries \ и записывают как \begin{boxedalign*}Ax + By + C = 0\end{boxedalign*}
	\end{formulas}
	\begin{remark}
		Заметим также, что из итоговой формулы вывода общего уравнения ($bx - ay + (ay_{0}-bx_{0}) = 0 \Rightarrow Ax + By + C = 0$) следует, что для прямой $Ax + By +C = 0$ вектор ($B, -A$) (а соответственно и ($-B, A$)) является направляющим.
	\end{remark}
	\begin{subtheorem}
		$Ax + By + C = 0$ является уравнением прямой $\Leftrightarrow A$ и $B$ не равны нулю одновременно.
	\end{subtheorem}
	\begin{proof}
		$\\ \Rightarrow$ Если $Ax + By + C = 0$, то её направляющий вектор ненулевой, а значит вектор $(-B, A)$ ненулевой, то есть одна из его координат $\neq 0$.
		$\\ \Leftarrow$ Пусть без ограничения общности $A \neq 0$. Тогда этому уравнению удовлетворяет точка $(x_{0}, y_{0}) = (-\frac{C}{A}, 0)$, а значит (нетрудно проверить) все удовлетворяющие ему точки имеют вид $(x_{0} + Bt, y_{0} - At)$, что соответствует прямой с такими параметрическими уравнениями.
	\end{proof}
	\subsection{Взаимное расположение прямых}
	\begin{theorem}
		Прямые на плоскости параллельны (или совпадают) $\Leftrightarrow$ их направляющие векторы пропорциональны.
	\end{theorem}
	\begin{proof}
		Пусть $l_{1}: A_{1}x + B_{1}y + C_{1} = 0;\ l_{2}: A_{2}x + B_{2}y + C_{2} = 0$ - данные прямые.
		Рассмотрим систему уравнений, которой удовлетворяют координаты точек, принадлежащих обоим прямым: $\begin{cases}A_{1}x + B_{1}y = -C_{1}\\A_{2}x + B_{2}y = -C_{2}\end{cases}\\$ Из курса алгебры (форумла Крамера) известно, что система не является определённой $\Leftrightarrow det\begin{pmatrix} A_{1} \ B_{1}\\A_{2} \ B_{2}\end{pmatrix} = 0$. Таким образом, прямые параллельны или совпадают $\Leftrightarrow$ имеют 0 или бесконечно много общих точек $\Leftrightarrow A_{1}B_{2} - A_{2}B_{1} = 0 \Leftrightarrow \begin{pmatrix} A_1 \\ B_1 \end{pmatrix}$ пропорционален $\begin{pmatrix} A_2 \\ B_2 \end{pmatrix}$, ч.т.д. 
	\end{proof}
	\begin{remark}
		Из этого также видно, что прямые совпадают $\Leftrightarrow \frac{A_{1}}{A_{2}} = \frac{B_{1}}{B_{2}} = \frac{C_{1}}{C_{2}}$. 
	\end{remark}
	\begin{consequense}
		Прямые $l_{1}: \begin{cases}x = x_{1} + a_{1}t \\ y = y_{1} + b_{1}t\end{cases}; l_{2}: \begin{cases}x = x_{2} + a_{2}t \\ y = y_{2} + b_{2}t\end{cases}$ пересекаются $\Leftrightarrow \frac{a_{1}}{a_{2}} \neq \frac{b_{1}}{b_{2}}$.
		Условие совпадения прямых также можно записать через параметрические уравнения (вектор $(x_{2} - x_{1} \ y_{2} - y_{1}) = \lambda(a, b)$).
	\end{consequense}
	\begin{consequense}
		Через две различные точки проходит ровно одна прямая (все такие прямые совпадают).
	\end{consequense}
	\subsection{Пучки прямых}
	\begin{definition}
		Собственным пучком прямых называется множество всех прямых, проходящих через данную точку, называемую центром пучка.\\
		Несобственным пучком прямых называется множество всех прямых, параллельных данной прямой.
	\end{definition}
	\begin{theorem}
		Пусть прямые $l_{1}: A_{1}x + B_{1}y + C_{1} = 0$ и $l_{2}: A_{2}x + B_{2}y + C_{2} = 0$ задают собственный пучок (т.е. содержатся в нём и не совпадают). Тогда прямая $l$ принадлежит пучку $\Leftrightarrow l$ задаётся уравнением $\lambda(A_{1}x + B_{1}y + C_{1}) + \mu(A_{2}x + B_{2}y + C_{2}) = 0 \ (*)$ для некоторых $\lambda, \mu \in \mathbb{R}$.
	\end{theorem}
	\begin{proof}
		$\\\Leftarrow$ Пусть $l$ задаётся уравнением $(*)$. Тогда, подставив в уравнение $l$ центр пучка ($x_{0}, y_{0}$), получим $\lambda(0) + \mu(0) = 0$ (т.к. центр удовлетворяет уравнениям $l_{1}, l_{2}$).\\
		$\Rightarrow$ Пусть $(x_{0}, y_{0}) \in l$. Возьмём произвольную точку $(x_{1}, y_{1}) \in l, (x_{1}, y_{1}) \neq (x_{0}, y_{0})$. Рассмотрим прямую вида $(*)$ с $\lambda = -(A_{2}x_{1} + B_{2}y_{1} + C_{2}), \ \mu = (A_{1}x_{1} + B_{1}y_{1} + C_{1}) : -(A_{2}x_{1} + B_{2}y_{1} + C_{2})(A_{1}x + B_{1}y + C_{1}) + (A_{1}x_{1} + B_{1}y_{1} + C_{1})(A_{2}x + B_{2}y + C_{2}) = 0$. Заметим, что это уравнение действительно задаёт прямую: в противном случае необходимы условия $\lambda A_{1} + \mu A_{2} = \lambda B_{1} + \mu B_{2} = 0$, но тогда $(A_{1}, B_{1})$ и $(A_{2}, B_{2})$ пропорциональны, а исходные прямые непараллельны. Такой прямой, очевидно, принадлежат точки $(x_{0}, y_{0})$ и $(x_{1}, y_{1})$. Так как через две различные точки проходит ровно одна прямая, любая прямая из собственного пучка имеет вид ($*$), ч.т.д.
	\end{proof}
	\begin{theorem}
		Пусть прямые $l_{1}: A_{1}x + B_{1}y + C_{1} = 0$ и $l_{2}: A_{2}x + B_{2}y + C_{2} = 0$ задают несобственный пучок (т.е. содержатся в нём и не совпадают). Тогда прямая $l$ принадлежит пучку $\Leftrightarrow l$ задаётся уравнением $\lambda(A_{1}x + B_{1}y + C_{1}) + \mu(A_{2}x + B_{2}y + C_{2}) = 0 \ (*)$ для некоторых $\lambda, \mu \in \mathbb{R}$.
	\end{theorem}
	\begin{proof}
		$\\\Leftarrow$ Так как $l_{1} \parallel l_{2}$, $\frac{A_{1}}{A_{2}} = \frac{B_{1}}{B_{2}}$. Тогда если $l$ имеет вид ($*$), то $\frac{\lambda A_{1}+\mu A_{2}}{A_{1}} = \lambda + \frac{\mu A_{2}}{A_{1}} = \lambda + \frac{\mu B_{2}}{B_{1}} = \frac{\lambda B_{1} + \mu B_{2}}{B_{1}} \Rightarrow l \parallel l_{1}$.\\
		$\Rightarrow$ Пусть $l$ принадлежит пучку. Так как направляющие векторы $l, l_{1}$ и $l_{2}$; пропорциональны, можем домножить уравнения на числа так, что коэффициенты перед переменными станут равны: пусть $l_{1}: Ax + By + C_{1}' = 0; \ l_{2} = Ax + By + C_{2}' = 0; \ l = Ax + By + C_{3}' = 0$.Тогда возьмём $\lambda, \mu$ из следующей системы: $\begin{cases}C_{1}'\lambda + C_{2}'\mu = C_{3}'\\\lambda + \mu = 1\end{cases} \Leftrightarrow \begin{cases}\lambda = \frac{C_{3}'-C_{2}'}{C_{1}'-C_{2}'}\\\mu = \frac{C_{1}' - C_{3}'}{C_{1}' - C_{2}'}\end{cases} (C_{1}' \neq C_{2}'$, иначе $l_{1}$ и $l_{2}$ совпадают). Очевидно, что для таких $\lambda, \mu$ уравнение $l$ имеет вид ($*$) (проверяется несложной подстановкой), ч.т.д.
	\end{proof}
	\subsection{Отрезки}
	\begin{definition}
		Пусть $l$ - прямая, $X_{1}(x_{1}, y_{1}), X_{2}(x_{2}, y_{2}) \in l$ и $X_{1} \neq X_{2}$. Отрезком с концами $X_{1}, X_{2}$ на плоскости называется множество всех точек, лежащих между $X_{1}$ и $X_{2}$ (на прямой $l$). Обозначается $[X_{1}, X_{2}]$. 
	\end{definition}
	\begin{formula}[Уравнение отрезка]
		Пусть $X \in [X_{1}, X_{2}]$. Тогда знаем, что $\begin{cases}x = x_{1} + t(x_{2} - x_{1}) \\ y = y_{1} + t(y_{2} - y_{1})\end{cases} (X \in l) \Rightarrow \begin{pmatrix} x-x_{1} \\ y-y_{1} \end{pmatrix} = t\begin{pmatrix} x_{2}-x_{1} \\ y_{2}-y_{1} \end{pmatrix} \Rightarrow \\ t\overrightarrow{X_{1}X_{2}} = \overrightarrow{X_{1}X} \Rightarrow \overrightarrow{XX_{2}} = (1-t)\overrightarrow{X_{1}X_{2}}$. Отсюда видно, что $|\overrightarrow{X_{1}X}|$ и $|\overrightarrow{XX_{2}}| < |\overrightarrow{X_{1}X_{2}}| \Leftrightarrow t \in [0, 1]$. Отсюда \begin{boxedalign*}X \in [X_{1}, X_{2}] \Leftrightarrow \begin{cases}x = x_{1} + t(x_{2} - x_{1}) \\ y = y_{1} + t(y_{2} - y_{1})\\t\in [0, 1]\end{cases}\end{boxedalign*}
	\end{formula}
	\begin{subtheorem}
		На плоскости выполяются вторая и третья аксиомы порядка. 
	\end{subtheorem}
	\begin{proof}
		Формульно зададим множества, на которые точка $(x, y)$ делит прямую $l$ (лучи без начала): $\{X: X(x, y)+\lambda \bar{v}, \lambda > 0\} (\lambda < 0)$, где $\bar{v}$ - произвольный направляющий вектор $l$. Без ограничения общности, пусть у $\bar{v}$ первая координата ненулевая (хотя бы одна ненулевая, т.к. вектор ненулевой). Тогда эти множества можно представить так: $\{X(x_{1}, y_{1}): x_{1} > x\} (x_{1} < x)$. Тогда $(x, y)$ лежит на отрезке между любыми двумя точками из разных множеств (несложно проверить), т.е. вторая аксиома порядка выполнена.\\
		Для третьей аксиомы небходимо в одном из этих множеств найти точку на расстоянии $d$ от $(x, y)$. Тогда из условия на расстояние имеем: $|\lambda\bar{v}| = a$, т.е. $|\lambda| = \frac{a}{|\bar{v}|}$. Так как знак $\lambda$ определяет, какому множеству принадлежит точка, в каждом из множеств такая точка существует и единственная, т.е. третья аксиома выполнена.  
	\end{proof}
	\begin{definition}
		Отрезком в произвольном аффинном пространстве называется множество точек $[X_{1}, X_{2}] = \{X_{1} + t\overrightarrow{X_{1}X_{2}}, t\in [0, 1]\}$
	\end{definition}
	\subsection{Полуплоскости}
	\begin{definition}
		Множество $X$ в произвольном аффинном пространстве называется выпуклым, если $\forall \ X_{1},X_{2} \in X \ \ [X_{1}, X_{2}] \subset X$. 
	\end{definition}
	\begin{definition}
		Пусть в аффинной системе координат $l: Ax + By + C = 0$. Множества $\Pi_{0}^{+} = \{X(x, y): Ax + By + C \geqslant 0\}$ и $\Pi_{0}^{-} = \{X(x, y): Ax + By + C \leqslant 0\}$ называются замкнутыми полуплоскостями, а множества $\Pi^{+} = \{X(x, y): Ax + By + C > 0\}$ и $\Pi^{-} = \{X(x, y): Ax + By + C < 0\}$ - открытыми полуплоскостями.
	\end{definition}
	\begin{theorem}
		Для любой прямой $l: Ax + By + C = 0$ множества $\Pi_{0}^{+}, \Pi_{0}^{-}, \Pi^{+}, \Pi^{-}$ выпуклы.
	\end{theorem}
	\begin{proof}
		Рассмотрим $\Pi_{0}^{+}$ (остальные аналогично).\\
		Пусть $X_{1}(x_{1}, y_{1}), X_{2}(x_{2}, y_{2}) \in \Pi_{0}^{+}$. Знаем, что любая точка $X \in [X_{1}, X_{2}]$ имеет координаты $(tx_{1}+(1-t)x_{2}, ty_{1}+(1-t)y_{2}), 0 \leqslant t \leqslant 1$. Тогда:\\
		$\begin{cases}Ax_{1} + By_{1} + C \geqslant 0 \\ Ax_{2} + By_{2} + C \geqslant 0\end{cases} \Rightarrow \begin{cases}tAx_{1} + tBy_{1} + tC \geqslant 0 \\ (1-t)Ax_{2} + (1-t)By_{2} + (1-t)C \geqslant 0\end{cases} \Rightarrow \\ \Rightarrow A(tx_{1}+(1-t)x_{2}) + B(ty_{1}+(1-t)y_{2}) + C \geqslant 0 \Rightarrow X \in \Pi_{0}^{+}$, ч.т.д.
	\end{proof}
	\begin{consequense}
		На плоскости выполняется четвёртая аксиома порядка.
	\end{consequense}
	\subsection{Угол между двумя прямыми}
	\begin{definition}
		Пусть $l$ - прямая, $O \in l, \bar{v}$ - любой направляющий вектор $l$. Множества $l^{+} = \{O + \lambda\bar{v}, \lambda \geqslant 0\}$ и $l^{-} = \{O + \lambda\bar{v}, \lambda \leqslant 0\}$ называются лучами, на которые $O$ делит $l$. 
	\end{definition}
	\begin{remark}
		Если даны два луча с общим началом (обозначим их $l_{1}^{+}, l_{2}^{+}$), то они являются подмножествами однозначно определённых прямых $l_{1}, l_{2}$, для которых можно выбрать направляющие векторы так, чтобы лучи соответствовали формуле из определения (назовём эти векторы $\bar{v}_{1}, \bar{v}_{2}$).
	\end{remark}
	\begin{definition}
		Углом на плоскости называется объединение двух лучей с общим началом. Величиной угла $l_{1}^{+}\cap l_{2}^{+}$ называется величина угла между векторами $\bar{v}_{1}, \bar{v}_{2}$, определёнными как в замечании. Говорят, что один угол меньше другого, если величина первого угла меньше величины второго.
	\end{definition}
	\begin{definition}
		Угол называется прямым, если его величина равна $\frac{\pi}{2} (\Leftrightarrow$ направляющие векторы лучей ортогональны). Угол называется развёрнутым, если его величина равна $\pi (\Leftrightarrow$ образующие угол лучи дополняют друг друга до прямой).
	\end{definition}
	\begin{remark}
		В дальнейшем будем иногда называть углом величину угла. Также аналогичным образом можно говорить о величине угла между вектором и прямой
	\end{remark}
	\begin{definition}
		Углом между двумя прямыми называется наименьший из углов, образованных лучами с началом в их точке пересечения этих прямых и лежащих на этих прямых, если прямые пересекаются. Если прямые параллельны или совпадают, то угол между ними равен нулю. Обозначается $\angle({l_{1}}{l_{2}})$.
	\end{definition}
	\begin{definition}
		Прямые называются перпендикулярными, если угол между ними равен $\frac{\pi}{2}$.
	\end{definition}
	\begin{definition}
		Перпендикуляром, опущенным из данной точки на данную прямую, называется прямая, проходящая через точку и перпендикулярная прямой, либо отрезок этой прямой с концами в данной точке и точке пересечения прямой с данной. 
	\end{definition}
	\begin{formula}[Угол между прямыми в прямоугольной с.к.]
		Пусть $l_{1}: A_{1}x + B_{1}y + C_{1} = 0, l_{2}: A_{2}x + B_{2}y + C_{2} = 0$ - прямые в прямоугольной системе координат. Тогда их направляющие векторы равны $\begin{pmatrix} -B_{1} \\ A_{1} \end{pmatrix} = \bar{v}_{1}, \begin{pmatrix} -B_{2} \\ A_{2} \end{pmatrix} = \bar{v}_{2}$. Отсюда (выражение скалярного произведения): $\ \cos\angle(l_{1}, l_{2}) = |\cos\angle(\bar{v}_{1}, \bar{v}_{2})| \Rightarrow$ \begin{boxedalign*} \cos\angle(l_{1}, l_{2}) = \frac{|A_{1}A_{2} + B_{1}B_{2}|}{\sqrt{A_{1}^2 + B_{1}^2} + \sqrt{A_{2}^2 + B_{2}^2}} \end{boxedalign*}(модуль в косинусе, а соответственно и в числителе формулы, позволяет сразу взять меньший угол).
	\end{formula}
	\begin{consequense}
		Прямые $l_{1}: A_{1}x + B_{1}y + C_{1} = 0$ и $l_{2}: A_{2}x + B_{2}y + C_{2} = 0$ перпендикулярны $\Leftrightarrow A_{1}A_{2} + B_{1}B_{2} = 0$.
	\end{consequense}
	\subsection{Расстояние от точки до прямой}
	\begin{definition}
		Нетрудно проверить, что в прямоугольной системе координат вектор $\bar{n} = \begin{pmatrix} A \\ B \end{pmatrix}$ перпендикулярен вектору $\begin{pmatrix} -B \\ A \end{pmatrix}$, а значит и прямой $l: Ax + By + C = 0$. Вектор $\bar{n}$ называется нормалью (нормальным вектором) прямой $l$ (в прямоугольной с.к.)
	\end{definition}
	\begin{remark}
		Любой вектор, коллинеарный нормали, также является нормалью, так как уравнения $Ax + By + C = 0$ и $\lambda Ax + \lambda By + \lambda C = 0$ задают одну и ту же прямую.
	\end{remark}
	\begin{definition}
		Пусть $A, B$  - множества в точечно-евклидовом пространстве. Расстоянием от $A$ до $B$ называется число $\inf\{|XY|, X\in A, Y\in B\}$ Расстояние от точки до прямой определяется аналогично. когда $A = \{X\}$ (его часто обозначают $d(X, l)$).
	\end{definition}
	\begin{remark}
		У множества из определения существует верхняя грань, т.к. оно является ограниченным снизу подмножеством действительных чисел (принцип полноты Вейерштрасса из курса математического анализа). 
	\end{remark}
	\begin{theorem}
		Расстояние от точки до прямой равно длине перпендикуляра, опущенного из этой точки на прямую.
	\end{theorem}
	\begin{proof}
		Пусть заданы $l: Ax + By + C = 0$ и $X_{0}$ - произвольные прямая и точка. Выберем на $l$ произвольную точку $X_{1}$. Проведём через $X_{0}$ прямую $l'$ с направляющим вектором $\begin{pmatrix} A \\ B \end{pmatrix}$  - она будет перпендикулярна $l$, а значит имеет с ней единственную общую точку - назовём её $X_{2}$. Имеем: $|\overrightarrow{X_{1}X_{0}}|^2 = (\overrightarrow{X_{1}X_{0}}, \overrightarrow{X_{1}X_{0}}) = (\overrightarrow{X_{1}X_{2}} + \overrightarrow{X_{2}X_{0}}, \overrightarrow{X_{1}X_{2}} + \overrightarrow{X_{2}X_{0}}) = |\overrightarrow{X_{1}X_{2}}|^2 + |\overrightarrow{X_{2}X_{0}}|^2 \geqslant |\overrightarrow{X_{2}X_{0}}|^2$. Отсюда $|\overrightarrow{X_{2}X_{0}}|^2$ - минимум всех расстояний между $X$ и точкой прямой, т.е. он достигается в точке пересечения $l$ и $l'$, ч.т.д.
	\end{proof}
	\begin{remark}
		Расстояние между двумя прямыми на плоскости $\neq 0 \Leftrightarrow$ они параллельны и не совпадают (у них нет общих точек). В этом случае расстояние между ними равно расстоянию от любой точки одной прямой до другой. 
	\end{remark}
	\begin{formula}[Расстояние от точки до прямой в прямоугольной с.к.]
		Пусть $X_{0} = (x_{0}, y_{0}), l: Ax + By + C = 0$. Посчитаем $d(X_{0}, l)$. Проведём через $X_{0}$ перпендикуляр $l'$ к прямой $l$. Направляющий вектор $l'$ - $\begin{pmatrix} A \\ B \end{pmatrix}$, то есть его параметрические уравнения - $\begin{cases} x = x_{0} + At \\ y = y_{0} + Bt \end{cases}$. Найдём $t_{1}$, удовлетворяющее точке пересечения: имеем $A(x_{0} + At_{1}) + B(y_{0} + Bt_{1}) + C = 0 \Rightarrow t_{1} = -\frac{Ax_{0} + By_{0} + C}{A^2+B^2}$.\\
		Отсюда точка пересечения $X_{1}$ имеет координаты $(x_{0} + At_{1}, y_{0} + Bt_{1})$. Тогда $\overrightarrow{X_{0}X_{1}} = \begin{pmatrix} At_{1} \\ Bt_{1} \end{pmatrix} \Rightarrow |\overrightarrow{X_{0}X_{1}}|^2 = A^2t_{1}^2 + B^2t_{1}^2 = (A^2 + B^2)t_{1}^2 = \frac{(Ax_{0} + By_{0} + C)^2}{A^2 + B^2}$, а значит \begin{boxedalign*} d(X, l) = \frac{|Ax_{0} + By_{0} + C|}{\sqrt{A^2 + B^2}} \end{boxedalign*}
	\end{formula}
	\section{Преобразования координат}
	\subsection{Преобразования координат векторов. Матрица перехода}
	\begin{formula}[Матрица перехода]
		Пусть $E = (\bar{e}_{1},...,\bar{e}_{n})$ и $E' = (\bar{e}_{1}',...,\bar{e}_{n}')$ - два базиса в одном и том же евклидовом пространстве $V$. Будем называть $E$ старым базисом, а $E'$ - новым. Получим формулу для координат в старом базисе вектора $\bar{x}\in V$, заданного в новом базисе координатами $(x_{1}',...,x_{n}')$.\\
		Итак, пусть в старом базисе $\bar{x}$ имеет координаты $(x_{1},...,x_{n})$. Установим связь между базисами, выразив векторы нового базиса через старый: пусть \\ $\bar{e}_{i}' = c_{1i}\bar{e}_{1} + ... + c_{ni}\bar{e}_{n}$, т.е. $\bar{e}_{i}' = \begin{pmatrix} c_{1i} \\ \vdots \\ c_{ni} \end{pmatrix}$. Подставляя эти выражения, получим:\\ $\bar{x} = x_{1}'\bar{e}_{1}' + ... + x_{n}'\bar{e}_{n}' = x_{1}'(c_{11}\bar{e}_{1} +...+ c_{n1}\bar{e}_{n}) +...+ x_{n}'(c_{1n}\bar{e}_{1} +...+ c_{nn}\bar{e}_{n}) = \\=(c_{11}x_{1}'+...+c_{1n}x_{n}')\bar{e}_{1} +...+(c_{n1}x_{1}'+...+c_{nn}x_{n}')\bar{e}_{n} \Rightarrow \ $(из равенства координат)  \begin{boxedalign*}\begin{pmatrix} x_{1} \\ \vdots \\ x_{n} \end{pmatrix} = \begin{pmatrix} c_{11} \ ... \ c_{1n} \\ \vdots \ \ \ \ \ \ \ \ \vdots \\ c_{n1} \ ... \ c_{nn} \end{pmatrix} \begin{pmatrix} x_{1}' \\ \vdots \\ x_{n}' \end{pmatrix} \Leftrightarrow \begin{pmatrix} x_{1} \\ \vdots \\ x_{n} \end{pmatrix} = C\begin{pmatrix} x_{1}' \\ \vdots \\ x_{n}' \end{pmatrix}\end{boxedalign*} 
	\end{formula}
	\begin{definition}
		Такая матрица $C$ называется матрицей перехода от базиса $E$ к базису $E'$.
	\end{definition}
	\begin{remark}
		Столбцы матрицы $C$ являются координатами базисных векторов $E'$ в базисе $E \Rightarrow$ столбцы $C$ линейно независимы $\Rightarrow C$ невырожденная (из курса алгебры).
	\end{remark}
	\begin{formula}
		Рассмотрим также скалярное произведение векторов в случае, когда базис $E$ ортонормированный. Если $\bar{x} = \begin{pmatrix} x_{1} \\ \vdots \\ x_{n}  \end{pmatrix}, \bar{y} = \begin{pmatrix} y_{1} \\ \vdots \\ y_{n}  \end{pmatrix}$ (в базисе $E$), имеем: $(\bar{x}, \bar{y}) = \begin{pmatrix} x_{1}&\dots&x_{n} \end{pmatrix} \begin{pmatrix} y_{1} \\ \vdots \\ y_{n} \end{pmatrix} = \begin{pmatrix} x_{1}'&\dots&x_{n}' \end{pmatrix}C^{T}C\begin{pmatrix} y_{1}' \\ \vdots \\ y_{n}' \end{pmatrix}$ (так как $\begin{pmatrix} x_{1}&\dots&x_{n}\end{pmatrix} = \begin{pmatrix} C \begin{pmatrix} x_{1}' \\ \vdots \\x_{n}' \end{pmatrix}\end{pmatrix}^{T}$). Нетрудно видеть, что произведение матриц $C^{T}C$ имеет вид $G = \begin{pmatrix} (\bar{e}_{1}', \bar{e}_{1}')&\dots&(\bar{e}_{1}', \bar{e}_{n}') \\ \vdots&\null&\vdots \\ (\bar{e}_{n}', \bar{e}_{1}') & \dots & (\bar{e}_{n}', \bar{e}_{n}') \end{pmatrix}$ (строки $C^{T}$ и столбцы $C$ - координаты векторов базиса $E'$ в базисе $E$) \\
		Такая матрица называется матрицей Грама (матрицей скалярных произведений) для базиса $E'$. Так как матрица $G$ не зависит от базиса $E$, получаем формулу для скалярного произведения в произвольном базисе: \begin{boxedalign*}(\bar{x}, \bar{y}) = \begin{pmatrix} x_{1}' & \dots & x_{n}' \end{pmatrix}G\begin{pmatrix} y_{1}' \\ \vdots \\ y_{n}' \end{pmatrix} \end{boxedalign*}
	\end{formula}
	\begin{consequense}[1]
		Если $C$ и $C_{1}$ - матрицы перехода от базиса $E$ к $E'$ и от $E'$ к $E''$ соответственно, то матрица перехода от базиса $E$ к $E''$ равна $CC_{1}$
	\end{consequense}
	\begin{consequense}[2]
		Если $C$ - матрица перехода от базиса $E$ к $E'$, то матрица перехода от базиса $E'$ к $E$ равна $C^{-1}$.
	\end{consequense}
	\begin{remark}
		Пусть $\bar{e}_{1},...,\bar{e}_{n}$ - базис векторного пространства $V$. Тогда векторы $\bar{x}_{1},...,\bar{x}_{n} \in V$ линейно независимы $\Leftrightarrow$ столбцы их координат линейно независимы. Это очевидно следует из представления линейной комбинации $\lambda_{1}\bar{x}_{1} + ... + \lambda_{n}\bar{x}_{n} = 0$ через координаты. 
	\end{remark}
	\begin{theorem}
		Для произвольного данного базиса матрица $C$ является матрицей перехода к некоторому другому базису $\Leftrightarrow \det C \neq 0$. 
	\end{theorem}
	\begin{proof}
		Следует из утверждения из курса алгебры о том, что матрица невырожденна $\Leftrightarrow$ её столбцы линейно независимы ($\Leftrightarrow$ векторы-столбцы образуют базис). 
	\end{proof}
	\subsection{Преобразования координат точек}
	\begin{formula}[Координаты точки при перемене с.к.]
		Пусть заданы два репера $O\bar{e}_{1}...\bar{e}_{n}$ и $O'\bar{e}_{1}'...\bar{e}_{n}'$. Для этого необходимо задать новый репер через старый: пусть $C$ - матрица перехода от $\bar{e}_{1},...,\bar{e}_{n}$ к $\bar{e}_{1}'...\bar{e}_{n}'$, а вектор $\overrightarrow{OO'} = \begin{pmatrix} x_{01} \\ \vdots \\ x_{0n} \end{pmatrix}$.
		Пусть $X$ - произвольная точка с координатами $(x_{1},...,x_{n})$ в старой системе координат и $(x_{1}',...,x_{n}')$ в новой. Тогда $\overrightarrow{OX} = \begin{pmatrix} x_{1} \\ \vdots \\ x_{n} \end{pmatrix}$. Также $\overrightarrow{O'X} = \begin{pmatrix} x_{1}' \\ \vdots \\ x_{n}' \end{pmatrix}$ в новой с.к. $\Rightarrow C\begin{pmatrix} x_{1}' \\ \vdots \\ x_{n}' \end{pmatrix}$ в старой. Осталось заметить, что $\overrightarrow{OX} = \overrightarrow{OO'} + \overrightarrow{O'X}$, а отсюда (через старую с.к.)
		\begin{boxedalign*}\begin{pmatrix} x_{1} \\ \vdots \\ x_{n} \end{pmatrix} = \begin{pmatrix} x_{01} \\ \vdots \\ x_{0n} \end{pmatrix} + C \begin{pmatrix} x_{1}' \\ \vdots \\ x_{n}' \end{pmatrix} \end{boxedalign*}
		Выразим новые координаты, умножив слева на $C^{-1}$:  
		\begin{boxedalign*}\begin{pmatrix} x_{1}' \\ \vdots \\ x_{n}' \end{pmatrix} = - C^{-1}\begin{pmatrix} x_{01} \\ \vdots \\ x_{0n} \end{pmatrix} + C^{-1}\begin{pmatrix} x_{1} \\ \vdots \\ x_{n} \end{pmatrix} \end{boxedalign*}
		(отметим также, что $- C^{-1}\begin{pmatrix} x_{01} \\ \vdots \\ x_{0n} \end{pmatrix}$ - новые координаты вектора $O'O$)
	\end{formula}
	\subsection{Ортогональные матрицы}
	\begin{definition}
		Матрица перехода от одного ортонормированного базиса к другому называется ортогональной.
	\end{definition}
	\begin{theorem}
		Для $C = \begin{pmatrix} c_{11}&\dots&c_{1n} \\ \vdots&\null&\vdots \\ c_{n1}&\dots&c_{nn} \end{pmatrix}$ следующие утверждения равносильны:
		\begin{enumerate}
			\item $C$ - ортогональная;
			\item $\sum \limits_{k=1}^{n} c_{ki}c_{kj} = \delta_{ij}$ ("скалярное произведение" столбцов равно $\delta_{ij}$);
			\item $\sum \limits_{k=1}^{n} c_{ik}c_{jk} = \delta_{ij}$ ("скалярное произведение" строк равно $\delta_{ij}$);
			\item $C C^T = E$;
			\item $C^T C = E$;
			\item $C^T = C^{-1}$;
		\end{enumerate}
	\end{theorem}
	\begin{proof}(в конспекте указан "прямой подсчёт")\\
		Для начала заметим равносильность утверждений $\circled{4}, \circled{5}, \circled{6}$: $\circled{4} \Leftrightarrow \circled{5}$ (получаются друг из друга транспонированием обоих частей равенства), а $\circled{6}$ равносильно им по определению обратной матрицы.\\
		$\circled{1} \Rightarrow \circled{2}$: Столбцы матрицы C - координаты векторов ортонормированного базиса $E$ в другом ортонормированном базисе $E'$. Рассмотрим скалярное произведение $(\bar{e}_{i}', \bar{e}_{j}')$: с одной стороны оно равно $\sum \limits_{k=1}^{n} c_{ki}c_{kj}$ (так как базис $E$ ортонормирован, можем применять формулу с матрицей Грама, причём $G = E$ из ортонормированности $E_{1}$), а с другой стороны - $\delta_{ij}$ (из ортонормированности $E_{1}$).\\
		$\circled{1} \Leftarrow \circled{2}$: Из $\circled{2}$ знаем, что векторы с координатами в столбцах попарно ортогональны и имеют длину 1 в базисе, в котором записаны эти координаты. Значит, применив такое преобразование к векторам ортонормированного базиса, получим также попарно ортогональные векторы длины 1, т.е. $C$ - ортогональная.\\
		$\circled{2} \Leftrightarrow \circled{5}$: Оба условия равносильны тому, что элемент $C^T C$ на позиции $ij$ равен $\delta_{ij}$.\\
		Аналогично $\circled{3} \Leftrightarrow \circled{4}$.

		Итого $\circled{1} \Leftrightarrow \circled{2} \Leftrightarrow \circled{5} \Leftrightarrow \circled{6} \Leftrightarrow \circled{4} \Leftrightarrow \circled{3}$.
	\end{proof}
	\begin{consequense}
		Для ортогональной $C: |C^T| = |C|, \ |C^T C| = |E| = 1 \Rightarrow |C| = \pm 1$
	\end{consequense}
	\begin{consequense}(Из определения ортогональной матрицы)\\
		Произведение ортогональных матриц - ортогональная матрица.\\
		Матрица, обратная ортогональной, ортогональна.
	\end{consequense}
	\begin{formula}(Двумерный случай ортогональной матрицы)\\
		$C = \begin{pmatrix} a&b\\c&d \end{pmatrix}$ ортогональна $\Rightarrow a^2 + c^2 = 1 \Rightarrow \exists \phi: a = \cos\phi, c = \sin\phi$.\\
		Из теоремы $a^2 + b^2 = 1, \ c^2 + d^2 = 1 \Rightarrow b = \pm\sin\phi, \ d = \pm\cos\phi$.\\
		Из ортогональности столбцов следует, что $ab + cd = 0$, поэтому остаются следующие случаи: \begin{boxedalign*}C = \begin{pmatrix} \cos\phi&-\sin\phi\\\sin\phi&\cos\phi\end{pmatrix}; \ \ C = \begin{pmatrix} \cos\phi&\sin\phi\\\sin\phi&\-cos\phi\end{pmatrix}\end{boxedalign*}
		т.е. для любой ортогональной $C$ найдётся $\phi$ такой, что $C$ имеет один из видов выше. Несложно также убедиться, что любая матрица такого вида ортогональна: первая производит поворот на угол $\phi$, а вторая - поворот с отражением.
	\end{formula}
	\subsection{Ориентация}
	\begin{definition}
		Два базиса в конечномерном векторном пространстве называются одинаково ориентированными (одноимёнными), если матрица перехода от одного базиса к другому имеет положительный определитель. В противном случае базисы называются противоположно ориентированными (разноимёнными).
	\end{definition}
	\begin{theorem}
		Отношение одноимённости является отношением эквивалентности на множестве базисов.
	\end{theorem}
	\begin{proof}
		По определению отношения эквивалентности:
		\begin{itemize}
			\item рефлексивность: матрица перехода от базиса в себя равна $E$, $|E| = 1$;
			\item симметричность: $C: E \rightarrow E' \Rightarrow C^{-1}: E' \rightarrow E$, причём $|C C^{-1}| = 1$, т.е. $|C|$ и $|C^{-1}|$ одного знака;
			\item транзитивность: $C: E \rightarrow E', \ C': E' \rightarrow E'', C'': E \rightarrow E'' \Rightarrow C'' = CC'$, т.е. из одноимённостей $E$ с $E'$ и $E'$ с $E''$ следует одноимённость $E$ с $E''$.
		\end{itemize}
	\end{proof}
	\begin{definition}
		Ориентацией векторного пространства, а также любого аффинного пространства, с которым оно ассоциировано, называется выбор любого из двух классов эквивалентности по отношению одноимённости и объявления базисов в нём положительными (а остальных - отрицательными).\\ 
		(Достаточно выбрать один базис и объявить его положительным)
	\end{definition}
	\begin{examples} Ориентации различных пространств:
		\begin{enumerate}
			\item \bfseries Прямая\mdseries :
			Достаточно выбрать направляющий вектор (базис), и объявить его положительным. Матрица перехода на прямой - число (любые два вектора коллинеарны), и сонаправленность векторов (одноимённость базисов) зависит от знака этого числа.
			\item  \bfseries Плоскость\mdseries :
			Достаточно выбрать пару неколлинеарных векторов (базис), и объявить его положительным. Заметим, что пары $\bar{a}, \bar{b}$ и $\bar{b}, \bar{a}$ разноимённы ($C = \begin{pmatrix} 0&1\\1&0\end{pmatrix})$.

			Посмотрим, когда пары $\bar{a}, \bar{b}$ и $\bar{a}, \bar{b}'$ одноимённые. Рассмотрим произвольную прямую $l$ с направляющим вектором $\bar{a}$ и любую точку $O \in l$. В репере $O\bar{a}\bar{b}$ уравнение $l$ имеет вид $y = 0$. Подставив в это уравнение координаты $\bar{b} = \begin{pmatrix} 0\\1\end{pmatrix}$, получим $1$.\\
			Пусть $\bar{b}' = \lambda\bar{a} + \mu\bar{b} \Rightarrow$ матрица перехода между парами $C$ равна $\begin{pmatrix} 1&\lambda\\0&\mu\end{pmatrix}$. $|C| > 0 \Leftrightarrow \mu > 0 \Leftrightarrow$ при подстановке координат $\bar{b}'$ в уравнение $l$ результат будет $> 0 \Leftrightarrow$ точки $B, B'$ с радиус-векторами $\bar{b}, \bar{b}'$ лежат в одной полуплоскости относительно $l$.\\
			Получили \bfseries критерий одноимённости пар $\bar{a}, \bar{b}$ и $\bar{a}, \bar{b}'$: они одноимённы $\Leftrightarrow$ точки, полученные прибавлением векторов $\bar{b}$ и $\bar{b}'$ к произвольной точке произвольной прямой с направляющим вектором $\bar{a}$, лежат относительно неё в одной полуплоскости. \mdseries 
		\end{enumerate}
	\end{examples}
	\subsection{Угол от вектора до вектора}
	\begin{definition}
		Углом от вектора $\bar{a}$ до вектора $\bar{b}$ в ориентированном двумерном евклидовом пространстве называется угол между $\bar{a}$ и $\bar{b}$, взятый со знаком плюс, если пара $\bar{a}, \bar{b}$ положительно ориентирована, и со знаком минус иначе.
	\end{definition}
	\begin{definition}
		Положительным углом от $\bar{a}$ до $\bar{b}$ называется угол $\phi$ от $\bar{a}$ до $\bar{b}$, если $\phi > 0$, и угол $2\pi + \phi$, если $\phi < 0$.
	\end{definition}
	\begin{definition}
		Углом от прямой $l_{1}$ до прямой $l_{2}$ в ориентированном двумерном евклидовом пространстве называется наименьший положительный угол от направляющего вектора $l_{1}$ до направляющего вектора $l_{2}$.
	\end{definition}
	\begin{formula}(Известное ранее уравнение прямой. Угол наклона)\\
		Пусть на плоскости задана прямоугольная система координат $O\bar{e}_{1}\bar{e}_{2}$ и в ней прямая $l$ определена уравнением $y = kx + b \Leftrightarrow kx - y + b = 0$. Прямая с направляющим вектором $e_{1}$ называется \bfseries осью абсцисс\mdseries, а угол от этой прямой до прямой $l$ - \bfseries углом наклона \mdseries прямой $l$.\\
		Подсчитаем угол наклона $l$. Направляющие векторы $l$ имеют вид $\begin{pmatrix}\lambda\\ \lambda k\end{pmatrix}$, причём $\phi$, очевидно, зависит только от знака $\lambda$. Угол от $\begin{pmatrix} 1\\0 \end{pmatrix}$ до $\begin{pmatrix} \pm 1\\\pm k\end{pmatrix}$ равен углу от $\begin{pmatrix} -1\\0 \end{pmatrix}$ до $\begin{pmatrix} \mp 1\\\mp k\end{pmatrix}$, поэтому выбирать наименьший можем только из положительных углов от вектора $\begin{pmatrix} 1\\0 \end{pmatrix}$ - наименьшим, как нетрудно видеть, будет угол до $\begin{pmatrix} 1\\k \end{pmatrix}$ при $k>0$ и до $\begin{pmatrix} -1 \\ -k \end{pmatrix}$ иначе. Обозначив нужный вектор $\bar{k}$, имеем: 
		\begin{align*}
			\cos\phi = \cos\angle(\bar{e}_{1}, \bar{k}) = \frac{1}{\sqrt{1 + k^2}} \Rightarrow \sin\angle(\bar{e}_{1}, \bar{k}) = \pm\frac{|k|}{\sqrt{1 + k^2}} \Rightarrow \tg\phi = \pm k 
		\end{align*}
		Заметим, что при $k > 0$ необходимый угол должен иметь тангенс, больший нуля, а при $k < 0$ - меньший нуля. Отсюда видно, что вне зависимости от нужного нам вектора верно: \begin{boxedalign*} \tg\phi = k\end{boxedalign*}
	\end{formula}
	\subsection{Площади фигур на плоскости}
	\begin{definition}
		Фигуры $\Phi_{1}, \Phi_{2}$ называются конгруэнтными, если $\exists$ отображение $f: \Pi \rightarrow \Pi$, сохраняющее расстояния ($|f(A)f(B)| = |AB| \ \forall A, B$) такое, что $f(\Phi_{1}) = \Phi_{2}$.
	\end{definition}
	\begin{definition}
		Площадь(мера) на евклидовой плоскости - численная величина (обозначается $S_{\Phi}, \Phi$ - фигура), соответствующая следующим свойствам:
		\begin{enumerate}
			\item Площадь квадрата со стороной 1 равна 1;
			\item Площади конгруэнтных фигур равны;
			\item Если $\Phi_{1}, \Phi_{2}$ - фигуры, $\exists \ S_{\Phi_{1}}, S_{\Phi_{2}}$ и $\Phi_{1} \cap \Phi_{2} = \varnothing$, то $\exists S_{\Phi_{1}\cup\Phi_{2}} = S_{\Phi_{1}} + S_{\Phi_{2}}$; 
			\item Если $\Phi$ - фигура и существуют такие последовательности фигур $\{\phi_{n}\}_{n=1}^{\infty}$ и $\{\Phi_{n}\}_{n=1}^{\infty}$, что $\phi_{n} \subset \phi_{n+1} \subset \Phi \subset \Phi_{n+1} \subset \Phi{n} \ \forall n \geqslant 1$, причём $\lim \limits_{n\rightarrow\infty}S_{\phi_{n}}$ и $\lim \limits_{n\rightarrow\infty}S_{\Phi_{n}}$ существуют и равны, то $S_{\Phi} = \lim \limits_{n\rightarrow\infty}S_{\phi_{n}}$ = $\lim \limits_{n\rightarrow\infty}S_{\Phi_{n}}$
		\end{enumerate} 
	\end{definition}
	\begin{consequense}(Площади некоторых фигур:)
		\begin{itemize}
			\item Площадь отрезка равна 0.
			\begin{proof}
				Из предельного перехода (Площадь отрезка с длиной 1 равна 0, иначе площадь квадрата со стороной 1 не была бы конечной, а из этого можно получить длину любого отрезка).
			\end{proof}
			\item Площадь квадрата со стороной $a$ равна $a^2$.
			\begin{proof}
				Для $a = \frac{1}{n}$ разбиваем квадрат со стороной 1 на $n^2$ квадратиков, для $a \in \mathbb{Q}$ складываем квадрат со стороной $\frac{m}{n}$ из предыдущих, для $a \notin \mathbb{Q}$ приближаем квадрат сверху и снизу квадратами с рациональными сторонами и переходим к пределам (по 4 пункту определения).
			\end{proof}
			\item Площадь прямоугольника со сторонами $a, b$ равна $ab$.
			\begin{proof}
				Рассмотрим квадрат со стороной $a+b$. Его можно разбить на квадрат со стороной $a$, квадрат со стороной $b$ и два нужных нам прямоугольника. Отсюда $2S = (a+b)^2 - a^2 -b^2 = 2ab \Rightarrow S = ab$.
			\end{proof} 
		\end{itemize}
		Отсюда также выводятся формулы площади параллелограмма (перестановкой треугольника из параллелограммма получается прямоугольник) и треугольника (как половины параллелограмма).
	\end{consequense}
	\begin{definition}
		Говорят. что параллелограмм натянут на векторы $\bar{a}, \bar{b}$, если для одной из вершин $A$ параллелограмма точки $A + \bar{a}$ и $A + \bar{b}$ также являются его вершинами. Площадь параллелограмма, натянутого на векторы $\bar{a}, \bar{b}$, обозначается $S_{\bar{a},\bar{b}}$.
	\end{definition}
	\begin{formula}(Площадь параллелограмма в прямоугольной с.к.)\\
		Нам уже известно, что $S_{\bar{a},\bar{b}} = |\bar{a}||\bar{b}|\sin\phi$ (из выражения высоты параллелограмма через угол). В прямоугольной системе координат:\begin{align*}
			S_{\bar{a},\bar{b}}^2 = |\bar{a}|^2|\bar{b}|^2\sin\phi = |\bar{a}|^2|\bar{b}|^2(1 - \cos^2\phi) = |\bar{a}|^2|\bar{b}|^2 - (\bar{a}, \bar{b})^2
		\end{align*}
		Если $\bar{a} = \begin{pmatrix} x_1 \\ y_1 \end{pmatrix}, \ \bar{b} = \begin{pmatrix} x_2 \\ y_2 \end{pmatrix}$, то $S_{\bar{a},\bar{b}}^2 = (x_1^2 + y_1^2)(x_2^2 + y_2^2) - (x_1x_2 + y_1y_2) = (x_1y_2 - x_2y_1)^2 \Rightarrow$ \begin{boxedalign*}S_{\bar{a},\bar{b}} = |\begin{vmatrix} x_1&y_1\\x_2&y_2 \end{vmatrix}|\end{boxedalign*}
		При этом определитель $>0 \Leftrightarrow \bar{a}, \bar{b}$ одноимённа с базисом.
	\end{formula}
	\begin{definition}
		\ Ориентированной площадью параллелограмма, натянутого на $\bar{a},\bar{b}$, на ориентированной плоскости называется число, равное $S_{\bar{a},\bar{b}}$, если пара $\bar{a},\bar{b}$ положительна, и $-S_{\bar{a},\bar{b}}$ иначе. Обозначается $<\bar{a}, \bar{b}>$.
	\end{definition}
	\begin{consequense}
		В любом положительно ориентированном базисе $<\bar{a}, \bar{b}> = \begin{vmatrix} x_1&y_1\\x_2&y_2 \end{vmatrix}$
	\end{consequense}
	\begin{subtheorem} Свойства ориентированной площади:
		\begin{enumerate}
			\item $<\bar{a}, \bar{b}> = -<\bar{b}, \bar{a}>$
			\item $<\bar{\lambda a}, \bar{b}> = \lambda<\bar{a}, \bar{b}> = <\bar{a}, \bar{\lambda b}>$
			\item $<\bar{a} + \bar{b}, \bar{c}> = <\bar{a}, \bar{c}> + <\bar{b}, \bar{c}>$\\
			($<\bar{a}, \bar{b} + \bar{c}> = <\bar{a}, \bar{b}> + <\bar{a}, \bar{c}>; <\bar{a} - \bar{b}, \bar{c}> = <\bar{a}, \bar{c}>-<\bar{b}, \bar{c}>$)
		\end{enumerate}
	\end{subtheorem}
	\begin{proof}
		Следует из свойств определителя.
	\end{proof}
	\section{Трёхмерное аффинное пространство}
	\subsection{Плоскость в пространстве}
	\begin{definition}
		Плоскость в (трёхмерном) аффинном пространстве - его двумерное аффинное подпространство. 
	\end{definition}
	\begin{formula}[Уравнения плоскости]
		Плоскость $\pi$ - это множество $X_0 + V^2$, где $X_0$ - точка и $V^2$ - двумерное векторное подпространство пространства $V$, ассоциированного с трёхмерным аффинным пространством.
		Так как $\dim V^2 = 2 \Rightarrow$ в нём есть базис $\bar{a}, \bar{b} \Rightarrow \pi = \{X_0 + u\bar{a} + v\bar{b}: u, v\in\mathbb{R}\}$.
		В произвольной системе координат: если $X_0 = (x_0, y_0, z_0), \bar{a} = \begin{pmatrix} a_1 \\ a_2 \\ a_3 \end{pmatrix}, \bar{b} = \begin{pmatrix}  b_1 \\ b_2 \\ b_3 \end{pmatrix}$, то $\pi$ - мн-во точек с координатами: (\bfseries параметрические уравнения плоскости\mdseries)
		\begin{boxedalign*}\begin{cases} x = x_0 + ua_1 + vb_1 \\ y = y_0 + ua_2 + vb_2 \\ z = z_0 + ua_3 + vb_3\end{cases}\end{boxedalign*}
		Векторы $\bar{a}, \bar{b}$ называются направляющими векторами плоскости $\pi$.\\
		Выражая параметры $u, v$, получим \bfseries общее уравнение плоскости\mdseries :
		\begin{boxedalign*}Ax + By + Cz + D = 0 \ \ \ (|A| + |B| + |C| \neq 0)\end{boxedalign*}
	\end{formula}
	\begin{definition}
		Говорят, что вектор $\bar{a}$ параллелен плоскости $\pi$ (обозначается $\bar{a} \parallel \pi$), если он выражается через направляющие векторы этой плоскости ($\Leftrightarrow$ через любой базис ассоциированного с плоскостью векторного пространства).
	\end{definition}
	\begin{remark}
		Ясно, что $X\in \pi \Leftrightarrow \overrightarrow{X_0X} \parallel \pi$ (для любой $X_0 \in \pi$). То же верно и для любой отличной от $X_0$ точки плоскости: $\overrightarrow{X_1X} = \overrightarrow{X_1X_0} + \overrightarrow{X_0X}$, т.е. $\overrightarrow{X_1X}$ выражается через направляющие векторы $\Leftrightarrow \overrightarrow{X_0X}$ выражается через направляющие векторы.
	\end{remark}
	\subsection{Взаимное расположение плоскостей}
	\begin{formula}
		Рассмотрим две плоскости, заданные уравнениями $A_{1}x + B_{1}y + C_{1}z + D_{1} = 0$ и $A_{2}x + B_{2}y + C_{2}z + D_{2} = 0$. Эти плоскости пересекаются $\Leftrightarrow \begin{cases}A_{1}x + B_{1}y + C_{1}z + D_{1} = 0\\A_{2}x + B_{2}y + C_{2}z + D_{2} = 0\end{cases}$ имеет решения. Матрица коэффициентов системы $A = \begin{pmatrix} A_{1}&B_{1}&C_{1}\\A_{2}&B_{2}&C_{2}\end{pmatrix}$, а её расширенная матрица $(A|B)= \begin{pmatrix} A_{1}&B_{1}&C_{1}&\vrule&-D_{1}\\A_{2}&B_{2}&C_{2}&\vrule&-D_{2} \end{pmatrix}$.
		По теореме Кронекера-Капелли из курса алгебры СЛУ совместна $\Leftrightarrow rk A = rk (A|B)$. Так как $1 \leqslant rk A \leqslant 2$, плоскости не пересекаются $\Leftrightarrow$ строки $A$ линейно зависимы ($rk A = 1$), а строки $(A|B)$ - нет. Таким образом, плоскости не пересекаются $\Leftrightarrow$\begin{align*}A_1:A_2 = B_1:B_2 = C_1:C_2 \neq D_1:D_2 \end{align*}
		Очевидно, плоскости совпадают $\Leftarrow$\begin{align*}A_1:A_2 = B_1:B_2 = C_1:C_2 = D_1:D_2\end{align*}
		Докажем, что плоскости совпадают только в этом случае. Пусть плоскости совпадают. Приведём матрицу $(A|B)$ к ступенчатому виду: $\begin{pmatrix} A_{1}&B_{1}&C_{1}&\vrule&-D_{1}\\0&a_{22}&a_{23}&\vrule&a_{24} \end{pmatrix}$ (хотя бы один из коэффициентов первой плоскости $\neq 0$, без ограничения общности - $A_1$). Так как плоскости совпадают, любая точка, принадлежащая первой плоскости, принадлежит и второй. Тогда для $a_{24}$ из принадлежности плоскости $(-\frac{D_1}{A_1}, 0, 0)$ следует $a_{24} = 0$, для $a_{22}$ аналогично при $B \neq 0$, а при $B = 0$ - из принадлежности точки $(-\frac{D_1}{A_1}, 1, 0)$ (для $a_{23}$ и $C$ аналогично). Отсюда вторая строка ступенчатой матрицы нулевая, то есть строки $(A|B)$ пропорциональны, а отсюда плоскости совпадают $\Leftrightarrow$ \begin{align*}A_1:A_2 = B_1:B_2 = C_1:C_2 = D_1:D_2\end{align*}
		Предположим, что плоскости пересекаются, но не совпадают. Тогда $rk A = 2$. Из курса алгебры знаем, что решение такой системы имеет вид $X_0 + V^1$ (частное решение СЛУ + одномерное пространство решений ОСЛУ).
		Таким образом:
		\begin{center}
			Плоскости совпадают $\Leftrightarrow$ их уравнения (со свободными коэффициентами) пропорциональны;

			Плоскости параллельны $\Leftrightarrow$ их уравнения пропорциональны, а свободные члены - нет;

			Плоскости пересекаются и не совпадают (их уравнения не пропорциональны) $\Leftrightarrow$ пересечением плоскостей является множество вида $X_0 + V^1$, где $X_0$ - точка и $V^1 = \{\lambda \bar{a}: \lambda \in \mathbb{R}\}$ ($\bar{a}$ - решение ОСЛУ $AX=0$), т.е. их пересечением является прямая.
		\end{center}
	\end{formula}
	\subsection{Пучки плоскостей}
	\begin{definition}
		Собственным пучком плоскостей называется множество всех плоскостей, проходящих через данную прямую, называемую центром пучка.\\
		Несобственным пучком плоскостей называется множество всех плоскостей, параллельных данной плоскости.
	\end{definition}
	\begin{theorem}
		Пусть плоскости $\pi_{1}: A_{1}x + B_{1}y + C_{1}z + D_1 = 0$ и $\pi_{2}: A_{2}x + B_{2}y + C_{2}z + D_2 = 0$ задают собственный пучок (т.е. содержатся в нём и не совпадают). Тогда плоскость $\pi$ принадлежит пучку $\Leftrightarrow \pi$ задаётся уравнением $\lambda(A_{1}x + B_{1}y + C_{1}z + D_1) + \mu(A_{2}x + B_{2}y + C_{2}z + D_2) = 0 \ (*)$ для некоторых $\lambda, \mu \in \mathbb{R}$.
	\end{theorem}
	\begin{proof}
		$\\\Leftarrow$ Пусть $\pi$ задаётся уравнением $(*)$. Тогда, подставив в уравнение $\pi$ любую точку прямой - центра пучка (назовём её $l$), получим $\lambda(0) + \mu(0) = 0$ (т.к. центр удовлетворяет уравнениям $\pi_{1}, \pi_{2}$).\\
		$\Rightarrow$ Пусть $l \subset \pi$. Возьмём произвольную точку $(x_{1}, y_{1}, z_1) \in \pi, (x_{1}, y_{1}, z_1) \notin l$. Рассмотрим плоскость вида $(*)$ с $\lambda = -(A_{2}x_{1} + B_{2}y_{1} + C_{2}z_1 + D_2), \ \mu = (A_{1}x_{1} + B_{1}y_{1} + C_{1}z_1 + D_1) : -(A_{2}x_{1} + B_{2}y_{1} + C_{2}z_{1} + D_2)(A_{1}x + B_{1}y + C_{1}z + D_1) + (A_{1}x_{1} + B_{1}y_{1} + C_{1}z_1 + D_1)(A_{2}x + B_{2}y + C_{2}z + D_2) = 0$. Заметим, что это уравнение действительно задаёт плоскость: в противном случае необходимы условия $\lambda A_{1} + \mu A_{2} = \lambda B_{1} + \mu B_{2} = \lambda C_{1} + \mu C_{2} = 0$, но тогда $(A_{1}, B_{1}, C_1)$ и $(A_{2}, B_{2}, C_2)$ пропорциональны, а исходные плоскости непараллельны. Такой плоскости, очевидно, принадлежат точки $l$ и $(x_{1}, y_{1}, z_1)$. Так как через прямую и не лежащую на ней точку проходит ровно одна плоскость (она имеет вид $\{X_1 + \bar{v} = X_1 + \lambda\overrightarrow{X_1X_2} + \mu\overrightarrow{X_1X_3}\}$, где $X_2, X_3$ - произвольные точки на $l$), любая плоскость из собственного пучка имеет вид ($*$), ч.т.д.
	\end{proof}
	\begin{theorem}
		Пусть плоскости $\pi_{1}: A_{1}x + B_{1}y + C_{1}z + D_1 = 0$ и $\pi_{2}: A_{2}x + B_{2}y + C_{2}z + D_2 = 0$ задают несобственный пучок (т.е. содержатся в нём и не совпадают). Тогда плоскость $\pi$ принадлежит пучку $\Leftrightarrow \pi$ задаётся уравнением $\lambda(A_{1}x + B_{1}y + C_{1}z + D_1) + \mu(A_{2}x + B_{2}y + C_{2}z + D_2) = 0 \ (*)$ для некоторых $\lambda, \mu \in \mathbb{R}$.
	\end{theorem}
	\begin{proof}
		$\\\Leftarrow$ Так как $\pi_{1} \parallel \pi_{2}$, $\frac{A_{1}}{A_{2}} = \frac{B_{1}}{B_{2}} = \frac{C_1}{C_2}$. Тогда если $\pi$ имеет вид ($*$), то $\frac{\lambda A_{1}+\mu A_{2}}{A_{1}} = \lambda + \frac{\mu A_{2}}{A_{1}} = \lambda + \frac{\mu B_{2}}{B_{1}} = \frac{\lambda B_{1} +\mu B_{2}}{B_{1}} = \lambda + \frac{\mu B_{2}}{B_{1}} = \lambda + \frac{\mu C_{2}}{C_{1}} = \frac{\lambda C_{1} +\mu C_{2}}{C_{1}} \Rightarrow \pi \parallel \pi_{1}$.\\
		$\Rightarrow$ Пусть $\pi$ принадлежит пучку. Так как уравнения $\pi, \pi_{1}$ и $\pi_{2}$; пропорциональны (без свободных коэффициентов), можем домножить их на числа так, что коэффициенты перед переменными станут равны: пусть $\pi_{1}: Ax + By + Cz + D_1' = 0; \ \pi_{2} = Ax + By + Cz + D_2' = 0; \ \pi = Ax + By + Cz + D_3' = 0$.Тогда возьмём $\lambda, \mu$ из следующей системы: $\begin{cases}D_{1}'\lambda + D_{2}'\mu = D_{3}'\\\lambda + \mu = 1\end{cases} \Leftrightarrow \begin{cases}\lambda = \frac{D_{3}'-D_{2}'}{D_{1}'-D_{2}'}\\\mu = \frac{D_{1}' - D_{3}'}{D_{1}' - D_{2}'}\end{cases} (D_{1}' \neq D_{2}'$, иначе $\pi_{1}$ и $\pi_{2}$ совпадают). Очевидно, что для таких $\lambda, \mu$ уравнение $\pi$ имеет вид ($*$) (проверяется несложной подстановкой), ч.т.д.
	\end{proof}
	\subsection{Полупространства}
	\begin{definition}
		Аналогично прямой на плоскости, каждая плоскость $\pi$  в пространстве $V^3$ разбивает множество всех не принадлежащих ей точек пространства на два выпуклых подмножества $V_1, V_2: V_1 \cup \pi \cup V_2 = V^3, V_1 \cap V_2 = V_1 \cap \pi = v_2 \cap \pi = \varnothing$. Такие подмножества называются полупространствами, ограниченными $\pi$ и определяются однозначно с точностью до обозначения.
	\end{definition}
	Также можем определить полупространства следующим образом: возьмём произвольную точку $X \notin \pi$ и скажем, что $V_1 = \{Y \in V^3: [X, Y]\cap\pi = \varnothing\}$, а затем выберем $X' \notin \pi \cup V_1$ (такая точка всегда будет существовать) и определим $V_2 = \{Y' \in V^3: [X', Y']\cap\pi = \varnothing\}$.
	\begin{subtheorem}
		Так определённые множества являются полупространствами и не зависят от выбора точек $X, X'$
	\end{subtheorem}
	\begin{proof}
		(Аналогично случаю прямой) Введём любую систему координат. Тогда полупространства $V^{\pm} = \{X(x, y, z): Ax + By + Cz + D \gtrless 0\}$. Рассмотрим $V^{+}$ (остальные аналогично).\\
		Пусть $X_{1}(x_{1}, y_{1}, z_1), X_{2}(x_{2}, y_{2}, z_2) \in V^{+}$. Знаем, что любая точка $X \in [X_{1}, X_{2}]$ имеет координаты $(tx_{1}+(1-t)x_{2}, ty_{1}+(1-t)y_{2}, tz_1+(1-t)z_2), 0 \leqslant t \leqslant 1$. Тогда:\\
		$\begin{cases}Ax_{1} + By_{1} + Cz_1 + D \geqslant 0 \\ Ax_{2} + By_{2} + Cz_2 + D \geqslant 0\end{cases} \Rightarrow \begin{cases}tAx_{1} + tBy_{1} + tCz_1 + D \geqslant 0 \\ (1-t)(Ax_{2} + By_{2} + Cz_2 + D) \geqslant 0\end{cases} \Rightarrow \\ \Rightarrow A(tx_{1}+(1-t)x_{2}) + B(ty_{1}+(1-t)y_{2}) + C(tz_1+(1-t)z_2) + D \geqslant 0 \Rightarrow X \in V^{+}$.\\
		Таким образом, для $X_1, X_2 \in V^{+} \ \ [X_1, X_2] \subset V^{+}$ из доказанной выше выпуклости, а для точек $X_1 \in V^{+}, X_{2} \in V^{-}$ точку пересечения $[X_1, X_2]$ и $\pi$ можно найти явно, но её существование очевидно.  
	\end{proof}
	\section{Прямая в пространстве}
	\subsection{Уравнения прямой в пространстве}
	\begin{formulas}(Уравнения прямой)
		Пусть $l$ - прямая в пространстве: $l = \{X_0 + t\bar{c}\}$, где $X_0$ - точка прямой, $\bar{c}$ - её направляющий вектор. Если $X_0=\begin{pmatrix}x_{0}\\y_{0}\\z_0\end{pmatrix},\bar{v}=\begin{pmatrix}c_1\\c_2\\c_3\end{pmatrix}$, то координаты точек на прямой выражаются следующим образом: (\bfseries параметрические уравнения прямой\mdseries) \begin{boxedalign*}X \in l: \begin{cases}x = x_{0} + tc_1 \\ y = y_{0} + tc_2 \\ z = z_0 + tc_3\end{cases}\end{boxedalign*}\\
		Выразим $t$ из всех уравнений и приравняем - получим \bfseries каноническое уравнение прямой\mdseries : \begin{boxedalign*}\frac{x-x_{0}}{c_1} = \frac{y-y_{0}}{c_2} = \frac{z-z_0}{c_3}\end{boxedalign*}\\
		Если известно, что прямой принадлежат $X_0 = (x_{0}, y_{0}, z_0), X_1 = (x_{1}, y_{1}, z_1)$, то $\overrightarrow{X_0X_1} = \begin{pmatrix}x_{1} - x_{0}\\y_{1} - y_{0}\\z_1-z_0\end{pmatrix}$ - направляющий вектор, т.е. каноническое уравнение имеет следующий вид (\bfseries уравнение прямой по двум точкам\mdseries)\begin{boxedalign*}\frac{x-x_{0}}{x_{1}-x_{0}} = \frac{y-y_{0}}{y_{1}-y_{0}} = \frac{z-z_{0}}{z_{1}-z_{0}}\end{boxedalign*}
	\end{formulas}
	\begin{consequense}
		Любая прямая является пересечением двух плоскостей. (Видно из канонического уравнения: например, плоскостей $\frac{x-x_{0}}{x_{1}-x_{0}} = \frac{y-y_{0}}{y_{1}-y_{0}}$ и $\frac{y-y_{0}}{y_{1}-y_{0}} = \frac{z-z_{0}}{z_{1}-z_{0}}$) 
	\end{consequense}
	\subsection{Взаимное расположение прямой и плоскости}
	\begin{formulas}(Случаи расположения)
		Пусть $l = \{X_0 + t\bar{c}: t \in \mathbb{R}\}, \pi = \{X_1 + u\bar{a} + v\bar{b}\}$. Уже знаем, что для любых других точек прямой и плоскости верны те же выражения множеств точек. \\
		Предположим, что прямая и плоскость пересекаются хотя бы в двух точках: пусть $X_0, X_1$ - их различные общие точки.Имеем $\overrightarrow{X_0X_0'} = t_0\bar{c} = u_0\bar{a} + v_0\bar{b}$ (для некоторых $t_0 \neq 0, u_0, v_0$, из принадлежности точек и прямой, и плоскости). Отсюда $\bar{c}$ выражается через $\bar{a}. \bar{b}$, а значит для любой $X \in l: X = X_0 + t\bar{c} = t(\frac{u_0}{t_0}\bar{a} + \frac{v_0}{t_0}\bar{b}) \in \pi$. Таким образом, 
		\begin{center}\underline{
			$l,\pi$ имеют две т. пересечения $\Leftrightarrow l \subset \pi \Leftrightarrow l \cap \pi \neq \varnothing$ и $\bar{c}$ выражается через $\bar{a}, \bar{b}$. }
		\end{center}
		Заметим, что $X \in l \cap \pi \Leftrightarrow X = X_0 + t_0\bar{c} = X_1 + u_0\bar{a} + v_0\bar{b} \Rightarrow \overrightarrow{X_0X_1} = u_0\bar{a} + v_0\bar{b} - t_0\bar{c}$. В случае некомпланарности $\bar{a}, \bar{b}, \bar{c}$ вектор $\overrightarrow{X_0X_1}$ выражается через них единственным образом, то есть точка пересечения есть и единственная, т.е.
		\begin{center}\underline{
		$l, \pi$ имеют одну т. пересечения $\Leftrightarrow \bar{a}, \bar{b}, \bar{c}$ линейно независимы}
		\end{center}
		(В случае, если $\bar{c}$ выражается через $\bar{a}, \bar{b}$: $l, \pi$ имеют общую точку $\Rightarrow l \subset \pi$, поэтому т.пересечения одна только в этом случае)
	\end{formulas}
	\begin{consequense}
		\underline{$\bar{c}$ выражается через $\bar{a}, \bar{b} \Leftrightarrow$ либо $l\subset\pi$, либо $l\cup\pi=\varnothing$} (в этом случае говорят, что \bfseries прямая параллельна плоскости\mdseries)\\
		\underline{$\bar{c}$ не выражается через $\bar{a}, \bar{b} \Leftrightarrow$ $l\cap\pi$ - одна точка.}
	\end{consequense}
	\subsection{Взаимное расположение двух прямых}
	\begin{definition}
		Прямые $l_1, l_2$ в пространстве называются параллельными, если они либо совпадают, либо лежат в одной плоскости и не пересекаются. 
	\end{definition}
	\begin{definition}
		Прямые $l_1, l_2$ в пространстве скрещиваются, если они не пересекаются и не параллельны. 
	\end{definition}
	\begin{remark}
		Если прямые $l_1, l_2$ с направляющими векторами $\bar{c}_1, \bar{c}_2$ пересекаются в точке $X_0$ и не совпадают, то $\bar{c}_1 \nparallel \bar{c}_2$ и $l_1, l_2 \subset \pi = \{X_0 + u\bar{c}_1 + v\bar{c}_2: u, v \in\mathbb{R}\}$. Отсюда прямые скрещиваются $\Leftrightarrow$ они не лежат в одной плоскости.\\
		Если направляющие векторы $\bar{c}_1, \bar{c}_2$ прямых $l_1, l_2$ коллинеарны, то прямые либо параллельны, либо совпадают. 
	\end{remark}
	\begin{consequense}
		Прямые $l_1, l_2$ скрещиваются $\Leftrightarrow$ они не пересекаются и $c_1 \nparallel c_2$.\\
		$l_1 \parallel l_2 \Leftrightarrow c_1 \parallel c_2$ (иначе прямые либо скрещиваются, либо пересекаются).
	\end{consequense}
	\bfseries Мораль в том, что дальше очев... \\(из уравнения касательной, которое уже вывел \href{https://github.com/yakovlevki/Calculus/blob/master/Calculus-1/calculus-1.pdf}{yakovlevki}) \mdseries
\end{document}